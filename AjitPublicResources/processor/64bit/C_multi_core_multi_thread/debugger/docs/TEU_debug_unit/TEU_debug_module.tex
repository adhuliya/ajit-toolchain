\documentclass[a4paper, 11pt]{article}
\usepackage[margin=0.75in]{geometry}
\usepackage[english]{babel}
\usepackage[utf8]{inputenc}
\usepackage{amsmath}
\usepackage{graphicx}
\usepackage[colorinlistoftodos]{todonotes}
\usepackage{listings}
\usepackage{epstopdf}
\usepackage{import}
\usepackage[none]{hyphenat}
\usepackage{algorithmicx}
\usepackage{algorithm}% http://ctan.org/pkg/algorithms
\usepackage{algpseudocode}% http://ctan.org/pkg/algorithmicx

\lstset{basicstyle=\ttfamily,breaklines=true}

\title{Debug module in TEU}

\author{Titto Thomas}

\date{\today}

\begin{document}
\maketitle

\section{Introduction}
The debug module inside TEU will be responsible for storing the information about breakpoints and watchpoints and also informing the stream corrector to stop the program execution.

\vspace*{1cm}
% Aa model
\begin{figure}[h!]
	\centering
	\def\svgwidth{\textwidth}
	
	\hspace*{-0.75cm}
	\scalebox{0.5}{\import{Figs/}{block.tex}}
	\caption{TEU debug module architecture}
	\label{BlockFig}
\end{figure}

\section{Expected behaviours}

Expected behaviours of the two blocks are represented in the form of pseudo code in the following sections.

\subsection{CCU interface unit}
The pseudocode is described in the Algorithm \ref{write_daemon} below.
\begin{algorithm}
\caption{ccu interface daemon}\label{write_daemon}
\begin{algorithmic}[1]
	\Function{write\_debug\_registers\_daemon}{}
	\While{1}\Comment{Infinite loop}
	\State $ccu\_debug\_msg\gets ccu\_to\_teu\_debug\_unit\_command$
	\State command, value $\gets$ Decode ($ccu\_debug\_msg$)
	
	\If {(command = update\_breakpoint)}
		\State break\_point\_reg[reg\_id] $\gets $ value
	\ElsIf {(command = update\_watchpoint)}
		\State watch\_point\_reg[reg\_id] $\gets $ value
	\ElsIf {(command = update\_interrupt) and (prev\_interrupt = FALSE)}
		\State currnt\_interrupt $\gets $ TRUE
		\State prev\_interrupt $\gets$ TRUE
	\Else
		\State currnt\_interrupt $\gets $ FALSE
	\EndIf
	
	\For{reg\_id \textbf{in} all breakpoints}
	\State signal\_breakpoint\_reg\_id $\gets$  break\_point\_reg[reg\_id]
	\EndFor
	\For{reg\_id \textbf{in} all watchpoints}
	\State signal\_watchpoint\_reg\_id $\gets$  watch\_point\_reg[reg\_id]
	\EndFor
	\State signal\_interrupt $\gets$  currnt\_interrupt
	
	\If {(command = valid)}
		\State $teu\_debug\_unit\_to\_ccu\_response \gets SUCCESS$
	\EndIf
	\EndWhile	
	\EndFunction
\end{algorithmic}
\end{algorithm}



\subsubsection{Inputs}
The input messages for the daemon will be of the format shown in Fig \ref{write_packet}. The fields could have the values,

\begin{enumerate}
	
	\item interrupt : Interrupt the processor execution flow, if this bit is set (kill request from GDB).
	\item break or watch : 0 is for breakpoint register write and 1 for watchpoint register write.
	\item set or clear : 0 is for clearing the register and 1 is for setting up a new watchpoint/breakpoint.
	\item reg\_id : The 2-bit register id.
	\item watch\_reg\_type : The type of watchpoint register to be set up. Here 1 is for watching just the write access, 2 for read access, 3 is for watching both of them.
	\item reg\_val : The 32-bit address to be stored in the registers.
\end{enumerate}

% Aa model
\begin{figure}[h!]
	\centering
	\def\svgwidth{\textwidth}
	\scalebox{0.75}{\import{Figs/}{write_packet.tex}}
	\caption{ccu\_to\_teu\_debug\_unit\_command message format}
	\label{write_packet}
\end{figure}

\subsubsection{Output}
The teu\_to\_ccu\_break\_watch\_register\_access\_response contains a single bit indicating if the previous command was successfully executed or not.

The block gives out 4 signals corresponding to current watchpoints (35-bit each), 4 breakpoints (33-bit each), and one corresponding to the interrupt. The formats of these signals are shown in Fig. \ref{active_signal_msg}.

\begin{figure}[h!]
	\centering
	\def\svgwidth{\textwidth}
	\scalebox{0.75}{\import{Figs/}{active_bp_signal.tex}}
	\scalebox{0.75}{\import{Figs/}{active_wp_signal.tex}}
	\caption{(a) breakpoint signal format (b) watchpoint signal format}
	\label{active_signal_msg}
\end{figure}

\newpage
\subsection{debug compare unit}
The unit is split into two parts, one which checks the breakpoints \& interrupt and comparing the watchpoints.
The pseudocodes are described in the Algorithm \ref{compare_daemon1} and \ref{compare_daemon2} below.
\begin{algorithm}
	\caption{breakpoint and interrupt unit}\label{compare_daemon1}
	\begin{algorithmic}[1]
		\Function{bp\_and\_intr\_check\_daemon}{}
		\While{1}\Comment{Infinite loop}
		\State $idispatch\_msg\gets idispatch\_to\_teu\_debug\_unit$
		\State $intr\_hit \gets  $ Decode ($idispatch\_msg$)
		\If {(not $intr\_hit$)}
			\State $PC, valid \gets $ Decode ($idispatch\_msg$)
			\State $bp\_hit, reg\_id \gets $ Compare($PC$, valid breakpoint registers)
		\EndIf
		
		\If {($intr\_hit$)}
			\State $teu\_debug\_unit\_to\_stream\_corrector\_intr \gets$ TRUE
		\ElsIf {($bp\_hit$ and $valid$)}
			\State $sc\_msg \gets $ Encode (bp\_hit, reg\_id)
			\State $teu\_debug\_unit\_to\_stream\_corrector\_bp \gets sc\_msg$
		\EndIf
		
		\EndWhile	
		\EndFunction
	\end{algorithmic}
\end{algorithm}

\begin{algorithm}
	\caption{watchpoint unit}\label{compare_daemon2}
	\begin{algorithmic}[1]
		\Function{wp\_check\_daemon}{}
		\While{1}\Comment{Infinite loop}
		\State $iunit\_msg\gets iunit\_to\_teu\_debug\_unit$
		
		\State $address, access\_type \gets  $ Decode ($iunit\_msg$)
		\State $wp\_hit, reg\_id, hit\_type \gets $ Compare($address$, valid breakpoint registers, $access\_type$)
		
		\If {($wp\_hit$)}
		\State $sc\_msg \gets $ Encode (wp\_hit, reg\_id, wp\_hit\_type)
		\State $teu\_debug\_unit\_to\_stream\_corrector\_wp \gets sc\_msg$
		\EndIf
		
		\EndWhile	
		\EndFunction
	\end{algorithmic}
\end{algorithm}

\end{document}
