% Debug solutions for microprocessors

Debugging programs on embedded systems is really different from how its done during traditional software development. The limited resources, memory and processing power on board restrict the presence of a full featured debugger on the target hardware. Another challenge in developing debug solutions for such systems is that they may not even be running on completely verified hardware.

Because of these challenges and restrictions, most of the embedded system debugging is done remotely. It is the process of debugging programs running on a smaller target hardware using a powerful host computer and they communicate through any of the standard interfaces.

\section{Standard debugging techniques}
\subsection{Logic Analyzers and Trace based debugging}
These methods are useful for older, small hardware systems and does not require any external computer for debugging. The main idea is to add tightly coupled hardware in the processor to provide a trace of what is happening inside, for the outside world. After the processor completes executing the program, the trace information can be compared against expected values to find out the errors.

% ICE
\subsection{In-circuit emulators}
In-circuit emulators are available for a few microprocessor architectures, providing a software based debugging solution. Most such emulators are proprietary and support execution of programs on top of the standard microprocessors models. This solution offers both passive and active debugging, gives a non-intrusive view of the program flow, and allows fine control over program execution, CPU state and memory contents. But, they cannot be used for debugging if the actual hardware is involved.

% debug stubs
\subsection{Debug stubs}
Debug stubs are the programs that run on target hardware and communicate with the debug software on host computer. Initially the stub program establishes connection with the host before handing over the control to the main function. During execution, stubs are called whenever exceptions occur. Being a software solution, its unsuitable for early development stages where the initialization code itself has to be debugged. Since this does not require any additional hardware, debug stubs are preferred for systems with a functioning operating system.

% debug stubs
\subsection{Integrated software and hardware solutions}
Integrated debug hardware gives the power of in-circuit emulators with much higher flexibility. The target microprocessors will contain debug functionality in hardware, that could connect to a host computer through a serial communication channel. Using the host computer, user will be able to remotely debug the programs running on target hardware. This is the best possible debug solution, but involves additional hardware and increases design complexity along with a cost overhead.

% Existing solutions for debugging
\section{Existing debug solutions}
Since debugging is a critical part of embedded software development, there are many commercial and open source tools readily available. Most of them are for either x86 or ARM\footnote{http://www.arm.com/} architectures, the commonly used microprocessors in the industry. A few of the existing debug solutions for the architecture of AJIT processor are described in the following sections.

\begin{description}
	\item \textbf{OpenOCD :} The Open On-Chip Debugger (OpenOCD)\footnote{http://www.openocd.org/} is an open source tool for debugging, in-system programming and boundary-scan testing for embedded target devices created by Dominic Rath\cite{openOCDthesis}. The debugging is done with the assistance of a hardware adapter, which is a small module that communicates from the target being debugged. Multiple such adapters have been developed (even as dongles) over the time, and many of them support different communication protocols as well. Currently the project supports ARM7, ARM9, XSCALE, MARVELL\footnote{http://www.marvell.com/} and MIPS processors and use JTAG interface to communicate with the target.
	
	\item \textbf{Advanced on-chip debug support on LEON :} LEON processor is a 32-bit processor core based on the SPARC V8 architecture, developed by Aeroflex Gaisler\footnote{http://www.gaisler.com/}.
	They have a debug monitor named GRMON2\cite{gaisler} that provides  most of the advanced debugging features like read/write access to all registers and memory, breakpoint and watchpoint management, etc.. over a variety of interfaces like PCI, USB, Ethernet, JTAG, UART and SpaceWire\footnote{http://spacewire.esa.int/}. This debugger is part of the supporting applications commercially available for the LEON family of microprocessors.
	
	\item \textbf{Oracle Solaris Studio dbxtool :} An easy-to-use graphical interface combined with the debugging functionality of the Oracle Solaris Studio\footnote{https://www.oracle.com/solaris/studio/} dbx debugger ,the tool \texttt{dbxtool} was created\cite{oracle}. The program allows users to debug programs running on an executable or core file or by attaching to a running process.  The tool also has the ability to save trace output, manage events, detect run time errors, modify source code, and save and re-run a debugging run as well.
\end{description}

