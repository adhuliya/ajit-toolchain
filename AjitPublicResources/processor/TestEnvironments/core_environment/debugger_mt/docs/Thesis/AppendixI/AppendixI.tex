
% Chapter Appendix
\appendix
\chapter{RSP Packets}
\label{appRSP} % Provide a table here.

\begin{longtable}{c |m{6.5cm} |m{3.5cm} |c}
\hline
\hline
\textbf{RSP Command} & \centering \textbf{Description} &  \centering \textbf{Response} & \textbf{Type}\\
\hline
\hline

qSupported  & Return the feature name followed by '+' if it is supported, '-' if it is not supported and value if necessary. As a minimum, just the packet size can be reported & PacketSize=600; multiprocess-; qRelocInsn-; & 2\\
\hline

? & Report why the target halted. The reply should be the POSIX signal ID of the reason , Eg. 05 for BP exception, 10 for Bus error.. & S $signal\:ID$ & 3\\
\hline

qC & Current thread being executed & T0 & 1\\
\hline

Hc*, Hg* & Specifying the threads for which the operations are applicable & OK & 2\\
\hline

qOffsets & The offsets for relocating the downloaded code. As we don't want any offset, they should be set to 0. & Text=0;Data=0; Bss=0; & 2\\
\hline

g & Read all the internal registers of processor. The order for SPARC is G0-G7, O0-O7, L0-L7, I0-I7, F0-F31, Y, PSR, WIM, TBR, PC, NPC, FPSR, CPSR & register contents in order & 3\\
\hline

qSymbol:: & Request any symbol table data & OK & 1\\
\hline

X$add,\:offset\::\:data$ & Load binary $data$ from $add$ to $add+offset$. If fails, GDB will do the same using M ( utilized in the proposed model ) & Empty & 1\\
\hline

M$add,\:count\::\:data$& Write $data$ from $add$ to $add+count$& OK & 3\\
\hline

m$add,\:count$& Read data from memory locations $add$ to $add+count$& memory contents in order & 3\\
\hline

vCont? &  Report if vCont actions are supported. Return vCont followed by 'c' for continue and 't' for stop, if they are supported. & vCont;c;t & 1\\
\hline

s& Single step execution on the target. The response will be the signal (same as for ?) from the hardware after halting execution when it finishes one instruction. & S $signal\:ID$ & 3\\
\hline

c & Continue execution & S $signal\:ID$ & 3\\
\hline

Zn$add$& Set breakpoint or Watch-point at a particular $add$. 'n' can be 0 for software breakpoint and 1 for hardware breakpoint, 2 for write watch point, 3 for read watch-point, 4 for access watch-point & OK & 3\\
\hline

zn$add$& Clear breakpoint at $add$ and 'n' is same as in the previous case & OK & 3\\
\hline

D & Detach from the remote server & OK & 3\\
\hline

P$number=value$ &Write $value$ to the register $number$ in the standard SPARC sequence (given for command `g') & OK & 3\\
\hline

G$values$ & Write $values$ to the registers in the standard SPARC sequence& OK & 3\\
\hline
\hline

\end{longtable}