% Abstract

\chapter*{Abstract}
As the complexity of embedded systems increase, providing debugging support becomes a crucial requisite. AJIT processor project is an attempt to deliver a microprocessor platform that can cater to different embedded applications. The design and implementation of a complete debug infrastructure for this processor is the principal goal of this project.

One of the most powerful and commonly used debuggers, Gnu Debugger (GDB) is chosen as the front end of the system. Provision of a non-intrusive view and the modifiability of processor states are the requisites of this debug system. Software server running on the host computer will work as translator between the GDB and hardware. The processor hardware description is modified to include additional modules and support debug operations. The processor has been implemented as two separate models with in-built debug support. Addition of a hardware server and the modified processor core constitute the debug functionality in the software model. Two debug units are added to the micro-architecture for the same purpose.

The implementation is verified across all the models and the final FPGA prototype is verified with several test programs. This system is mainly being used for the development of the Linux kernel for the AJIT processor. The scope of the design includes supporting a JTAG interface between the computer and processor hardware.
% Page doesn't require footer
\thispagestyle{titlepages}
