\documentclass[a4paper, 11pt]{article}
\usepackage[margin=0.75in]{geometry}
\usepackage[english]{babel}
\usepackage[utf8]{inputenc}
\usepackage{amsmath}
\usepackage{graphicx}
\usepackage[colorinlistoftodos]{todonotes}
\usepackage{listings}
\usepackage{epstopdf}
\usepackage{import}
\usepackage[none]{hyphenat}
\lstset{basicstyle=\ttfamily,breaklines=true}

\title{Debug support on the Aa model}

\author{Titto Thomas}

\date{\today}

\begin{document}
\maketitle


\renewcommand{\thesubsection}{\arabic{subsection}.}

\section{Overview}
\label{sec:overview}

The debug support on the Ajit processor Aa model will be provided through a \texttt{debug\_daemon} block in the hardware and it's connections with \texttt{ccu}.\texttt{debug\_daemon} communicates with with other blocks through AHIR pipes. The processor will communicate with the \texttt{debug\_daemon} only if the processor is running in the \texttt{DEBUG\_MODE}.

\begin{figure}[h!]
	\scalebox{.57}{\import{./figs/}{block.tex}}
	\caption{System architecture}
\end{figure}

The system consists of the two interfaces described below.

\subsection{\texttt{debug\_daemon} to external environment}
The pipes \texttt{ENV\_to\_AJIT\_command} and \texttt{AJIT\_to\_ENV\_response} are used for communicating with the GDB on host PC. The message format on these pipes are same for both the C and micro-architectural model of the processor. They are described in the section 3 of this document.

\subsection{\texttt{debug\_daemon} to \texttt{ccu}}
This interface with \texttt{ccu\_to\_debug} and \texttt{debug\_to\_ccu} pipes will be primarily used for updating the processor state at any instant and to be informed when the thread finishes or stops.

\begin{itemize}
	\item The \texttt{ccu} informs \texttt{debug\_daemon} if any of the following situations occur and then wait for a response. Along with the code for thread finish, the current value of PC, NPC and PSR will also be passed on.
		\begin{itemize}
			\item Connection request (initial)
			\item Breakpoint hit
			\item Watchpoint hit (both read \& write)
			\item Processor is in error mode
			\item Trap occurance
			\item Returning from single step
			\item Kill request executed
		\end{itemize}
	After subsequent operations, when the \texttt{debug\_daemon} wants to continue, it will pass the following details to \texttt{ccu}.
		\begin{itemize}
			\item PC, NPC, PSR
			\item Next processor mode
		\end{itemize}
	 If the host has terminated the connection, then 
	 \texttt{debug\_daemon} will change the processor to normal execution mode.
	 
	\item \texttt{debug\_daemon} will send requests to  \texttt{ccu} for reading/modifying the current processor state. The requests could be 
		\begin{itemize}
			\item Interrupt the processor execution flow when there is a \textit{kill} request from the GDB host.
			\item Read/write the iunit/fpunit/control/co-processor registers
			\item Read/write the fpunit registers
			\item Read/write the memory
			\item Read/write the breakpoint/watchpoint registers
			\item Detach the debugger and continue exeuction in normal mode
		\end{itemize}

\end{itemize}

\vspace*{1cm}

\section{\texttt{debug\_daemon} execution flow}
\label{sec:execution}
\begin{itemize}
\item Wait for a connection request from the \texttt{ccu\_daemon}. Will not continue forward unless the connection request arrives.

\item After receiving the connection request from \texttt{ccu\_daemon}, wait for a similar request from the GDB host to proceed further.

\item Acknowledge the GDB host connect request and execute all the subsequent instructions such as read/write the registers/memory and set/clear the breakpoints/watchpoints.

\item Once the GDB host sends a \textit{continue} message, read the current PC, NPC, PSR values. Pass these values to the \texttt{ccu\_daemon} and start program execution.

\item Read the \texttt{ENV\_to\_AJIT\_command} and \texttt{ccu\_to\_debug} pipes every cycle and respond if there are any valid requests on them.
		\begin{itemize}
		\item \texttt{ccu\_to\_debug} can inform regarding breakpoint hit, watchpoint hit and occurance of trap.
		
				\begin{itemize}
				\item Inform the GDB host regarding the breakpoint hit, watchpoint hit and occurance of trap along with the details. The address of breakpoint/watchpoint hit will be obtained from the local copies of their corresponding registers stored in debugger.
				
				\item Execute all the subsequent instructions such as read/write the registers/memory and set/clear the breakpoints/watchpoints. Update the local copies of PC, NPC, PSR and breakpoints/watchpoint registers with corresponding messages.
				
				\item Once the GDB host sends a \textit{continue} message, pass it to the \texttt{ccu\_daemon} along with PC, NPC, PSR and start program execution.
				
				\item If GDB host sends a \textit{detach} message, pass it to the \texttt{ccu\_daemon} along with PC, NPC, PSR and let the program execution continue in normal mode.
				\end{itemize}
		\item \texttt{ENV\_to\_AJIT\_command} inform when the GDB host wants to kill the program execution.
		
				\begin{itemize}
				\item Send an interrupt signal to the \texttt{ccu\_daemon} and stop the current program execution.
				\end{itemize}
		\end{itemize}

\end{itemize}

\newpage
\section{Packet format on each of the pipes}
\label{sec:packformat}
The message formats on each of the interface pipes are as shown in the following sections.
\setcounter{subsection}{0}
\subsection{\texttt{ENV\_to\_AJIT\_command} (64 bit, non-blocking)}
The messages on this interface will be same for both the micro-architecture and the C model. These 32 bit messages are send through this 64 bit pipe with an additional bit to indicate whether it's valid or not ( shown in fig. \ref{gdb_packet}).
\begin{figure}[h!]
	\centering
	\scalebox{.65}{\import{./figs/}{gdb_packet.tex}}
	\caption{ENV\_to\_AJIT\_command packets}
	\label{gdb_packet}
\end{figure}

The 32 bit GDB messages could either be commands or data for the debugger. These commands have the format shown in fig. \ref{packet1}. 

\begin{figure}[h!]
	\centering
	\scalebox{.65}{\import{./figs/}{packet1.tex}}
	\caption{GDB commands for the debugger}
	\label{packet1}
\end{figure}

The individual messages are as follows.
\begin{itemize}
		\item Read \texttt{iunit} registers : op-code = 1, length = 1, command is interpreted in the following order. If the \textit{reg} or \textit{asr} bit is set, then the \textit{reg\_id} indicates the number of corresponding register. The \textit{win-ptr} field corresponds to the window pointer for the register.
		
		\begin{figure}[h!]
			\centering
			\scalebox{.65}{\import{./figs/}{reg_packet.tex}}
		\end{figure}
		
		\item Write \texttt{iunit} registers : op-code = 2, length = 2, command is interpreted in the same way as the previous one. This command will be followed by the data to be written.
		
		\item Read \texttt{fpunit} registers : op-code = 3, length = 1, command is interpreted as following. If the \textit{fsr} bit is not set, then the register with \textit{reg\_id} will be read.
		
		\begin{figure}[h!]
			\centering
			\scalebox{.65}{\import{./figs/}{fp_packet.tex}}
		\end{figure}
				
		\item Write \texttt{fpunit} registers : op-code = 4, length = 2, command is interpreted in the same way as the read command. This command will be followed by the data to be written.
		
		\item Read from the memory : op-code = 6, length = 1, command is interpreted as follows.
		
		\begin{figure}[h!]
			\centering
			\scalebox{.65}{\import{./figs/}{mem_packet.tex}}
		\end{figure}
		
		\item Write to the memory : op-code = 7, length = 2, command is interpreted in the same way as the previous one. This command will be followed by the data to be written.
		
		\item Set breakpoint : op-code = 8, length = 2, command is interpreted as follows. The \textit{reg\_id} field corresponds to the breakpoint register where the address will be stored. The command is followed by the address to be stored in the register.
		
		\begin{figure}[h!]
			\centering
			\scalebox{.65}{\import{./figs/}{bp_packet.tex}}
		\end{figure}
		
		\item Remove breakpoint : op-code = 9, length = 1, command is interpreted as the previous one.
		
		\item Set watchpoint : op-code = 10, length = 2, command is interpreted as follows. The type indicates the properties of the watchpoint. It could be a write watchpoint (type = 2), read watchpoint (type = 3) or access watchpoint (type = 4). The command is followed by the address to be stored in the register.
		
		\begin{figure}[h!]
					\centering
					\scalebox{.65}{\import{./figs/}{wp_packet.tex}}
				\end{figure}
		
		\item Remove watchpoint : op-code = 11, length = 1, command is interpreted similarly to the breakpoint removal.
		
		\item Read control registers : op-code = 13, length = 1, command is decoded as follows.

		\begin{figure}[h!]
			\centering
			\scalebox{.65}{\import{./figs/}{cntrl_packet.tex}}
		\end{figure}
				
		\item Detach : op-code = 15, length = 1, requests to stop debugger and continue normal execution.
		
		\item Continue : op-code = 16, length = 1, requests to resume program execution.
		
		\item Read co-processor status register : op-code = 17, length = 1, command is unused.
		
		\item Write co-processor status register : op-code = 18, length = 1, command is unused.
		
		\item Kill the current thread : op-code = 19, length = 1, command is unused.
		
\end{itemize}

\subsection{\texttt{AJIT\_to\_ENV\_response} (32 bit, blocking) }
These pipes will use the same message formats of the C model, described the Debugger design document\cite{main_doc}.


\subsection{\texttt{debug\_to\_ccu} (64 bit, non-blocking)}
This pipe has to request the read/write to the general registers/memory/breakpoint registers/watchpoint registers and also allow the \texttt{ccu} to continue after a halt (with new PC, NPC, PSR, and mode).

The message format is almost same as that of the \texttt{ENV\_to\_AJIT\_command} 
commands. The structure in fig. \ref{gdb_packet} is retained and the commands are also same except the following two.
\begin{itemize}
		\item Continue : op-code = 16, length = 4, command will contain the new processor mode ( same values used in the \texttt{ccu} ). Followed by 3 packets with new PC, NPC, PSR values. 
		
		\item Kill the current thread : op-code = 19, length = 4, command will be same as the previous one. Followed by 3 packets with new PC, NPC, PSR values. 
				
\end{itemize}

\subsection{\texttt{ccu\_to\_debug} (64 bit, non-blocking))}
This pipe has to inform the \texttt{debug\_daemon} when it stops and send the register/memory contents when required. The MSB indicates if the pipe has a valid message or not and the rest 32 bits have the following format shown in fig. \ref{packet2}.
\begin{figure}[h!]
	\centering
	\scalebox{.55}{\import{./figs/}{packet2.tex}}
	\caption{ccu\_to\_debug interface packet response}
	\label{packet2}
\end{figure}

\begin{itemize}
		\item Connect request : op-code = 1, length = 1;
		\item Breakpoint hit : op-code = 2, length = 4. Packet is followed by PC, NPC, PSR values.
		\item Watchpoint hit (read) : op-code = 3, length = 4. Packet is followed by PC, NPC, PSR values.
		\item Watchpoint hit (write) : op-code = 4, length = 4. Packet is followed by PC, NPC, PSR values.
		
		\item Thread finished : op-code = 5, length = 4. The reason for the thread being finished is encoded in the command.
		\begin{itemize}
				\item command [0] : Trap occured
				\item command [1] : Returning after single step
				\item command [2] : Interrupt
				\item command [3] : Returning after a kill request
		\end{itemize}
		Multiple bits on the command could be '1' at the same time.
		
		\item Entered error mode : op-code = 6, length = 1.
		
		\item Register/memory content : If the pipe contains a response for one of the register or memory reads, then the then the least significant 32 bits will be the content.
\end{itemize}

\newpage

\begin{thebibliography}{main}
	
	\bibitem{main_doc} Titto Thomas, Ashfaque Ahammed, Prof. Madhav Desai, \textquotedblleft Design Of Debugger For AJIT Processor,\textquotedblright\, 2015.
	
\end{thebibliography}

\end{document}