\documentclass{beamer}
\usepackage{graphics}
\title{Using GDB to debug The ``hello\_world'' example}
\author{Madhav Desai \\ Department of Electrical Engg.\\ IIT Powai\\ Mumbai 400076}

\begin{document}
\maketitle

\frame[containsverbatim]{\frametitle{Recall the hello world example}

\begin{itemize}
\item Print ``hello\_world'' and stop.
\item Single core version (with MMU).
\item Multi  core version (with MMU).
\end{itemize}

}

\frame[containsverbatim]{\frametitle{Blank slide: GDB mechanism to debug a single AJIT thread}
}


\frame[containsverbatim]{\frametitle{Single thread debug}

\begin{itemize}
\item In shell 1, start the GDB client:
\begin{verbatim}
%  sparc-linux-gdb hello_world.elf
\end{verbatim}
\item In shell 2, start the C model:
\begin{verbatim}
%  ajit_C_system_model -m hello_world.mmap.remapped -g -p 8888
\end{verbatim}
\item Go back to shell 1, and specify the target server.
\begin{verbatim}
(gdb) target remote:8888
Remote debugging using :8888
0x00000000 in _start ()
(gdb) 
\end{verbatim}
You will see that the gdb client will stop at the 
\begin{verbatim}
 _start 
\end{verbatim}
routine.
\item Now in shell 2, you can control the execution of the
program using GDB.
\end{itemize}

}

\frame[containsverbatim]{\frametitle{Blank slide: GDB mechanism to simultaneously debug a multiple AJIT thread}
}

\frame[containsverbatim]{\frametitle{Multi thread debug for 4 threads (0,0), (0,1), (1,0), (1,1)}

\begin{itemize}
\item In shells 1,2,3,4 start GDB clients:
\begin{verbatim}
prompt>  sparc-linux-gdb hello_world.elf
\end{verbatim}
\item In shell 5, start the C model:
\begin{verbatim}
prompt>  ajit_C_system_model -m hello_world.mmap.remapped -g -p 8888 -p 8889 -p 8890 -p 8891
\end{verbatim}
This assigns port 8888 to GDB session for (0,0), port 8889 for (0,1), port 8890 for (1,0) and
port 8891 for (1,1).
\item Go back to shells 1,2,3,4 and specify the target server.
\begin{verbatim}
(gdb) target remote:8888
\end{verbatim}
etc. Remember to put the port numbers for the GDB clients (in shell 1, use port 8888, in shell 2,
use 8889 etc.).
You will see that the gdb client will stop at the 
\begin{verbatim}
 _start 
\end{verbatim}
routine in each of the shells 1-4.
\item Now in shells 1-4, you can control the execution of the
program using GDB on thread (0,0), (0,1), (1,0) and (1,1).
\end{itemize}

}

\end{document}
