\documentclass{beamer}
\usepackage{graphics}
\title{The ``hello\_world'' example}
\author{Madhav Desai \\ Department of Electrical Engg.\\ IIT Powai\\ Mumbai 400076}

\begin{document}
\maketitle

\frame[containsverbatim]{\frametitle{Overview}

\begin{itemize}
\item Print ``hello\_world'' and stop.
\item Single core version.
\begin{itemize}
\item Without MMU.
\item With MMU.
\end{itemize}
\item Multi  core version.
\begin{itemize}
\item With MMU.
\end{itemize}
\end{itemize}

}

\frame[containsverbatim]{\frametitle{Blank slide: initialization of a processor core}
}


\frame[containsverbatim]{\frametitle{Single core version: setting up the run-time environment without using the MMU}

\begin{itemize}
\item Set up the stack and frame pointers.
\begin{itemize}
\item We use virtual addresses from $0xffff3000$ onwards for memory mapped I/O.
\item Set the initial stack and frame pointers at 0xffff2ffc since we are not passing
anything to the main program in this example.
\item In general, you can set up the initial stack pointer and frame pointer to
define a stack frame which can be used to pass arguments to your main program.
\end{itemize}
\item Set the processor state register (PSR) to enable interrupts.
\item Set the window invalid mask register to reserve one window for the trap
handler.
\item Set the default cacheable bit in the MMU control register.
\begin{itemize}
\item This marks all accesses as cacheable (other than the bypass ASI accesses).
\end{itemize}
\end{itemize}
}

\frame[containsverbatim]{\frametitle{Blank Slide}
}

\frame[containsverbatim]{\frametitle{Single core version: setting up the run-time environment using the MMU}

\begin{itemize}
\item Same procedure as before, except for changes related to virtual to physical mapping.
\item Write a VMAP file.
\begin{itemize}
\item The script generates an assembly subroutine which sets up the page table in memory.
\end{itemize}
\item Call the page table setup assembly subroutine and set the context table pointer.
\item Enable the MMU.
\item Run with full protection.
\end{itemize}
}

\frame[containsverbatim]{\frametitle{Blank Slide}
}

\frame[containsverbatim]{\frametitle{Multi core version: setting up the run-time environment using the MMU}

\begin{itemize}
\item Set up the stack and frame pointers for each core.
\item Set the processor state register (PSR)  in each core, to enable interrupts.
\item Set the window invalid mask register in each core, to reserve one window for the trap
handler.
\item Write a VMAP file.
\item Call the page table setup assembly subroutine and set the context table pointer.
\item Enable the MMU.
\item Set up locks.
\item Run.
\end{itemize}

}

\frame[containsverbatim]{\frametitle{Blank Slide}
}

\end{document}
