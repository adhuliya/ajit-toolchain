\subsubsection{Shift instructions:}
\label{sec:shift:insn:impl}
The shift  family of instructions  of AJIT  may each be  considered to
have  two versions:  a direct  count version  and a  register indirect
count version.  In the direct count  version the shift count is a part
of the  instruction bits.   In the indirect  count version,  the shift
count is  found on the  register specified by  the bit pattern  in the
instruction  bits.   The direct  count  version  is specified  by  the
14$^{th}$  bit, i.e.  insn[13]  (bit  number 13  in  the  0 based  bit
numbering scheme), being set to 1.  If insn[13] is 0 then the register
indirect version is specified.

Similar to the addition and subtraction instructions, the shift family
of instructions of  SPARC V8 also do  not use bits from 5  to 12 (both
inclusive).  The AJIT processor uses bits  5 and 6.  In particular bit
6 is always 1.   Bit 5 may be used in the direct  version giving a set
of 6 bits  available for specifying the shift count.   The shift count
can have  a maximum  value of  64.  Bit  5 is  unused in  the register
indirect version, and is always 0 in that case.

These instructions  are therefore  worked out  below in  two different
sets: the direct and the register indirect ones.
\begin{enumerate}
\item The direct versions are given by  insn[13] = 1.  The 6 bit shift
  count  is directly  specified  in the  instruction bits.   Therefore
  insn[5:0] specify the shift count.  insn[7:6] = 2, distinguishes the
  AJIT version from the SPARC V8 version.
  \begin{enumerate}
  \item \textbf{SLLD}:\\
    \begin{center}
      \begin{figure}[h]
        \centering
        \epsfxsize=.8\linewidth
        \epsffile{../figs/slld-ajit-insn-32-bit-layout.eps}
        \caption{The AJIT  SLLD instruction.  Shift count  is directly
          specified in the instruction.}
        \label{fig:ajit:slld:insn}
      \end{figure}
    \end{center}
    \begin{itemize}
    \item []\textbf{SLLD}: same as SLL, but with Instr[13]=0 (i=0),
      and Instr[7:6]=2.
    \item []\textbf{Syntax}: ``\texttt{slld SrcReg1, 6BitShiftCnt,
        DestReg}''. \\
      % (\textbf{Note:} In an assembly language program, when the second
      % argument is a number, we have direct mode.  A register number is
      % prefixed with  ``r'', and hence the  syntax itself distinguished
      % between   direct  and   register   indirect   version  of   this
      % instruction.)
    \item []\textbf{Semantics}: rd(pair) $\leftarrow$ rs1(pair) $<<$
      shift count.
    \end{itemize}

    The AJIT/SPARC bit layout of this instruction is:

    \begin{tabular}[h]{lclcl}
      Macro to set   &=&  \verb|F5(x, y, z, b)|     &in& \verb|sparc.h|     \\
      Macro to reset &=&  \verb|F5(~x, ~y, ~z, ~b)| &in& \verb|sparc.h|     \\
      x &=& 0x2      &in& \verb|OP(x) | \\
      y &=& 0x25     &in& \verb|OP3(y) | \\
      z &=& 0x1      &in& \verb|F3I(z) | \\
      a &=& 0x2      &in& \verb|OP_AJIT_BIT_5_AND_6(a) |
    \end{tabular}

  \item \textbf{SRLD}:\\
    \begin{center}
      \begin{figure}[h]
        \centering
        \epsfxsize=.8\linewidth
        \epsffile{../figs/srld-ajit-insn-32-bit-layout.eps}
        \caption{The AJIT  SRLD instruction.  Shift count  is directly
          specified in the instruction.}
        \label{fig:ajit:srld:insn}
      \end{figure}
    \end{center}
    \begin{itemize}
    \item []\textbf{SRLD}: same as SRL, but with Instr[13]=0 (i=0),
      and Instr[7:6]=2.
    \item []\textbf{Syntax}: ``\texttt{sral SrcReg1, 6BitShiftCnt,
        DestReg}''. \\
      % (\textbf{Note:} In an assembly language program, when the second
      % argument is a number, we have direct mode.  A register number is
      % prefixed with  ``r'', and hence the  syntax itself distinguished
      % between   direct  and   register   indirect   version  of   this
      % instruction.)
    \item []\textbf{Semantics}: rd(pair) $\leftarrow$ rs1(pair) $>>$
      shift count.
    \end{itemize}

    The AJIT/SPARC bit layout of this instruction is:

    \begin{tabular}[h]{lclcl}
      Macro to set   &=&  \verb|F5(x, y, z, b)|     &in& \verb|sparc.h|  \\
      Macro to reset &=&  \verb|F5(~x, ~y, ~z, ~b)| &in& \verb|sparc.h|  \\
      x              &=& 0x2                        &in& \verb|OP(x)|    \\
      y              &=& 0x26                       &in& \verb|OP3(y)|   \\
      z              &=& 0x1                        &in& \verb|F3I(z)|   \\
      a              &=& 0x2                        &in& \verb|OP_AJIT_BIT_5_AND_6(a)|
    \end{tabular}
      % x &=& 0x2      &in& \verb|OP(x)                  /* ((x) & 0x3)  $<<$ 30 */| \\
      % y &=& 0x26     &in& \verb|OP3(y)                 /* ((y) & 0x3f) $<<$ 19 */| \\
      % z &=& 0x1      &in& \verb|F3I(z)                 /* ((z) & 0x1)  $<<$ 13 */| \\
      % a &=& 0x2      &in& \verb|OP_AJIT_BIT_5_AND_6(a) /* ((a) & 0x3   $<<$ 6  */|

  \item \textbf{SRAD}:\\
    \begin{center}
      \begin{figure}[h]
        \centering
        \epsfxsize=.8\linewidth
        \epsffile{../figs/srad-ajit-insn-32-bit-layout.eps}
        \caption{The AJIT  SRAD instruction.  Shift count  is directly
          specified in the instruction.}
        \label{fig:ajit:srad:insn}
      \end{figure}
    \end{center}
    \begin{itemize}
    \item []\textbf{SRAD}: same as SRA, but with Instr[13]=0 (i=0),
      and Instr[7:6]=2.
    \item []\textbf{Syntax}: ``\texttt{srad SrcReg1, 6BitShiftCnt,
        DestReg}''. \\
      % (\textbf{Note:} In an assembly language program, when the second
      % argument is a number, we have direct mode.  A register number is
      % prefixed with  ``r'', and hence the  syntax itself distinguished
      % between   direct  and   register   indirect   version  of   this
      % instruction.)
    \item []\textbf{Semantics}: rd(pair) $\leftarrow$ rs1(pair) $>>$
      shift count (with sign extension).
    \end{itemize}

    The AJIT/SPARC bit layout of this instruction is:

    \begin{tabular}[h]{lclcl}
      Macro to set   &=&  \verb|F5(x, y, z, b)|     &in& \verb|sparc.h|  \\
      Macro to reset &=&  \verb|F5(~x, ~y, ~z, ~b)| &in& \verb|sparc.h|  \\
      x              &=& 0x2                        &in& \verb|OP(x)|    \\
      y              &=& 0x27                       &in& \verb|OP3(y)|   \\
      z              &=& 0x1                        &in& \verb|F3I(z)|   \\
      a              &=& 0x2                        &in& \verb|OP_AJIT_BIT_5_AND_6(a)|
    \end{tabular}
  \end{enumerate}
\item The register  indirect versions are given by insn[13]  = 0.  The
  shift count is indirectly specified in the 32 bit register specified
  in instruction bits.  Therefore  insn[4:0] specify the register that
  has the  shift count.  insn[6]  = 1, distinguishes the  AJIT version
  from the SPARC V8 version.  In this case, insn[5] = 0, necessarily.
  \begin{enumerate}
  \item \textbf{SLLD}:\\
    \begin{center}
      \begin{figure}[h]
        \centering
        \epsfxsize=.8\linewidth
        \epsffile{../figs/slld-reg-ajit-insn-32-bit-layout.eps}
        \caption{The AJIT SLLD instruction.   Shift count is specified
          via a register in the instruction.}
        \label{fig:ajit:slldcc:insn}
      \end{figure}
    \end{center}
    \begin{itemize}
    \item []\textbf{SLLD}: same as SLL, but with Instr[13]=0 (i=0),
      and Instr[7:6]=2.
    \item []\textbf{Syntax}: ``\texttt{slld SrcReg1, SrcReg2,
        DestReg}''.
    \item []\textbf{Semantics}: rd(pair) $\leftarrow$ rs1(pair) $<<$
      shift count register rs2.
    \end{itemize}

    \begin{tabular}[h]{lclcl}
      Macro to set   &=&  \verb|F5(x, y, z, b)|     &in& \verb|sparc.h|     \\
      Macro to reset &=&  \verb|F5(~x, ~y, ~z, ~b)| &in& \verb|sparc.h|     \\
      x &=& 0x2      &in& \verb|OP(x)| \\
      y &=& 0x25     &in& \verb|OP3(y)| \\
      z &=& 0x0      &in& \verb|F3I(z)| \\
      a &=& 0x1      &in& \verb|OP_AJIT_BITS_5_AND_6(a)|
    \end{tabular}

  \item \textbf{SRLD}:\\
    \begin{center}
      \begin{figure}[h]
        \centering
        \epsfxsize=.8\linewidth
        \epsffile{../figs/srld-reg-ajit-insn-32-bit-layout.eps}
        \caption{The AJIT SRLD instruction.   Shift count is specified
          via a register in the instruction.}
        \label{fig:ajit:srldcc:insn}
      \end{figure}
    \end{center}
    \begin{itemize}
    \item []\textbf{SRLD}: same as SRL, but with Instr[13]=0 (i=0),
      and Instr[7:6]=2.
    \item []\textbf{Syntax}: ``\texttt{slld SrcReg1, SrcReg2,
        DestReg}''.
    \item []\textbf{Semantics}: rd(pair) $\leftarrow$ rs1(pair) $>>$
      shift count register rs2.
    \end{itemize}

    The AJIT/SPARC bit layout of this instruction is:

    \begin{tabular}[h]{lclcl}
      Macro to set   &=&  \verb|F5(x, y, z, b)|     &in& \verb|sparc.h|     \\
      Macro to reset &=&  \verb|F5(~x, ~y, ~z, ~b)| &in& \verb|sparc.h|     \\
      x &=& 0x2      &in& \verb|OP(x)| \\
      y &=& 0x26     &in& \verb|OP3(y)| \\
      z &=& 0x0      &in& \verb|F3I(z)| \\
      a &=& 0x1      &in& \verb|OP_AJIT_BITS_5_AND_6(a)|
    \end{tabular}

  \item \textbf{SRAD}:\\
    \begin{center}
      \begin{figure}[h]
        \centering
        \epsfxsize=.8\linewidth
        \epsffile{../figs/srad-reg-ajit-insn-32-bit-layout.eps}
        \caption{The AJIT  SRAD instruction.   Shift count is specified
          via a register in the instruction.}
        \label{fig:ajit:slldcc:insn}
      \end{figure}
    \end{center}
    \begin{itemize}
    \item []\textbf{SRAD}: same as SRA, but with Instr[13]=0 (i=0),
      and Instr[7:6]=2.
    \item []\textbf{Syntax}: ``\texttt{slld SrcReg1, SrcReg2,
        DestReg}''.
    \item []\textbf{Semantics}: rd(pair) $\leftarrow$ rs1(pair) $>>$
      shift count register rs2 (with sign extension).
    \end{itemize}

    The AJIT/SPARC bit layout of this instruction is:

    \begin{tabular}[h]{lclcl}
      Macro to set   &=&  \verb|F5(x, y, z, b)|     &in& \verb|sparc.h|     \\
      Macro to reset &=&  \verb|F5(~x, ~y, ~z, ~b)| &in& \verb|sparc.h|     \\
      x &=& 0x2      &in& \verb|OP(x)| \\
      y &=& 0x27     &in& \verb|OP3(y)| \\
      z &=& 0x0      &in& \verb|F3I(z)| \\
      a &=& 0x1      &in& \verb|OP_AJIT_BITS_5_AND_6(a)|
    \end{tabular}
  \end{enumerate}
\end{enumerate}
