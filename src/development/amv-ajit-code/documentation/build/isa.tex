%\documentclass{article}
\documentclass{book}

\usepackage{epsf}
\usepackage{pstricks}
\usepackage{enumitem}
% \usepackage[cm]{fullpage}
\usepackage{fullpage}
% \usepackage{hyperref}
% \usepackage[colorlinks=true,citecolor=brown,pagebackref=true,backref=true,hyperfigures=true,hyperfootnotes=true,hyperindex=true]{hyperref}

\usepackage{hyperref}
% \usepackage{minitoc}
% \usepackage{float}
% \usepackage[algo2e,boxruled,vlined]{algorithm2e}
%\usepackage[algo2e,boxruled,vlined,linesnumbered]{algorithm2e}
%\usepackage[algo2e,algoruled,linesnumbered]{algorithm2e}
%\usepackage[algo2e,tworuled,linesnumbered]{algorithm2e}

\hypersetup{colorlinks=true}
\hypersetup{citecolor=red}
\hypersetup{urlcolor=blue}
\hypersetup{pagebackref=true}
\hypersetup{backref=true}
\hypersetup{hyperfigures=true}
\hypersetup{hyperfootnotes=true}
\hypersetup{hyperindex=true}
% \hypersetup{pdfauthor=Abhijat Vichare}

\newrgbcolor  {darkgreen}    {.533     .8     .533}

%% Define a new floating environment for algorithms. See the LaTeX
%% wikibook.
%% March, 2020.
% \floatstyle{ruled}
% \restylefloat{table}
% \restylefloat{figure}
% \newfloat{algo}{tbph}{alg}
% \floatname{algo}{Algorithm}

\parindent 0in
\parskip 0.1in

% \def\codetoadd#1{%
% \framebox{%
%   \parbox{\textwidth}{%
%     #1%
%   }
% }
% \def\codetoadd#1{%
% \framebox{%
%   \begin{minipage}[h]{1.0\linewidth}
%     #1%    
%   \end{minipage}
% }

\def\note#1{%
  \framebox[\linewidth]{%
    \begin{minipage}[h]{0.95\linewidth}
      {\red \textbf{Note:}}\\%
      {\blue {#1}}
    \end{minipage}
  }
}
\def\attention#1#2{%
  \framebox[\linewidth]{%
    \begin{minipage}[h]{0.95\linewidth}
      {\red \textbf{#1}}\ %\\%
      {\blue {#2}}
    \end{minipage}
  }
}

\setcounter{secnumdepth}{3}
\setcounter{tocdepth}{3}

\begin{document}
\title{64-bit ISA extensions to the AJIT processor}
\author{Madhav Desai}
\maketitle
\newpage
\parskip=0em
\tableofcontents
\newpage
\listoftables
\newpage
\parskip 0.1in

\chapter{The ISA Specification from IITB}
\label{chap:from:mpd:at:iitb}

\begin{center}
  \begin{tabular}[h]{|l|l|}
    \hline
    {\blue \textbf{ISA Version}}&{\red 2}\\
    %{\blue \textbf{Section below}}&{\red \ref{sec:isa:v2}}\\
    {\blue \textbf{Section below}}&{\red \ref{sec:isa:extns}}\\
    {\blue \textbf{ISA Version Date}}&{\red September 2020.}\\
    {\blue \textbf{Updated on}}&{\red December 17, 2020.}\\
    \hline
  \end{tabular}
\end{center}

\section{Overview}
\label{sec:Overview}

The AJIT processor implements the  Sparc-V8 ISA.  We propose to extend
this ISA to provide support for a native 64-bit integer datatype.  The
proposed  extensions use  the  existing instruction  encodings to  the
maximum extent possible.

All  proposed  extensions  are:   {\blue  Register  $\times$  Register
  $\rightarrow$  Register,Condition-codes}   type  instructions.   The
load/store instructions are not modified.

% All proposed extensions are:

% \centerline{Register $\times$ Register $\rightarrow$ Register,Condition-codes}

% type instructions.  The load/store instructions are not modified.

We list  the additional instructions  in the subsequent  sections.  In
each  case,  only the  differences  in  the  encoding relative  to  an
existing Sparc-V8 instruction are provided.

\attention{Note}{This  section records  the ISA  version 2  updates as
  received on September 2020.}

\subsection{Changes relative to Version 1}
\label{sec:changes:wrt:v1}

There has been some rationalization  of the instructions.  Further the
ASR register mappings have been updated.

Notes on instruction naming: V*  means a vector SIMD instruction

\subsubsection{Instruction Modifications}
\label{sec:insn:modified}

\begin{itemize}
\item Some instructions have been removed.

  \texttt{VFDIV VFSQRT}

\item Some instructions have been renamed.

  \begin{itemize}
  \item   \texttt{ADDDBYTER}    replaced   with   \texttt{ADDDREDUCE8}
    instruction encoding modified as shown later.

    Given [a1 a2 a3 ... a8] calculate (a1+a2+...+a8)
  \item    \texttt{ORDBYTER}    replaced   with    \texttt{ORDREDUCE8}
    instruction encoding modified as shown later.
  \item   \texttt{ANDDBYTER}    replaced   with   \texttt{ANDDREDUCE8}
    instruction encoding modified as shown later.
  \item   \texttt{XORDBYTER}    replaced   with   \texttt{XORDREDUCE8}
    instruction encoding modified as shown later.
  \item \texttt{VFADD} replaced with \texttt{VFADD32} opcode modified.

    Given  [x1  x2], [y1  y2]  of  single-precision numbers  calculate
    [(x1+y1) (x2+y2)] this becomes \texttt{VFADD32}

    Added  half-precision [x1  x2  x3  x4], [y1  y2  y3 y4]  calculate
    [(x1+y1) (x2+y2) ... (x4 + y4)] this becomes \texttt{VFADD16}
   \item \texttt{VFSUB} replaced with \texttt{VFSUB32} opcode modified.
   \item \texttt{VFMUL} replaced with \texttt{VFMUL32} opcode modified.
  \end{itemize}

\item Some instructions have been added.  Opcodes have been assigned (see
  below).

  \begin{tabular}[h]{p{.25\linewidth}p{.25\linewidth}p{.25\linewidth}p{.25\linewidth}}
  \texttt{ADDDREDUCE16}&\texttt{ORDREDUCE16}&\texttt{ANDDREDUCE16}&
  \texttt{XORDREDUCE16}\\
  \texttt{VFADD16}&\texttt{VFSUB16}&\texttt{VFMUL16}&~\\
  \texttt{FADDREDUCE16}&~&~&~\\
  \texttt{FSTOH}&\texttt{FHTOS}&~&~\\
  \texttt{VFHTOI16}&\texttt{VFI16TOH}&~&~
  \end{tabular}
  Instruction behaviour is described below.
\end{itemize}

\subsubsection{ASR mappings}
\label{sec:asr:mappings}


\texttt{ASR[31]} and \texttt{ASR[30]}  provide a free-running 64-bit counter
running on the processor clock (same as in AJIT32).

\texttt{ASR[29]} is intialized to processor ID field (writes to this
register are ignored).

The ancillary  state register \texttt{ASR[28]}  is interpreted
as a floating point configuration  register.  The bits of this
register are interpreted as follows:
\begin{tabbing}
  ~~~~~~~~\=~~~~~~~~~~~~~~~~~~~~~~~~~~~~~~~~~~~~~~~~~~~~~~~~ \kill
  31:8 \> unused \\
  7:0  \>  half-precision exponent width
\end{tabbing}
This register is intialized to a value of 5 by default (as
per the IEEE half-precision format).  Valid values of the
exponent width are between 5 and 14.

IEEE half precision = 1 sign + 5 exp + 10 mantissa bits
would like\\ 		
\centerline{1 sign 08 exp 7 mantissa}
\centerline{1 sign 12 exp 3 mantissa}

\newpage
\section{ISA Extensions}
\label{sec:isa:extns}

The extensions to SPArc V8 for AJIT are described in this section.

% \subsection{Integer-Unit Extensions: Arithmetic-Logic Instructions}
\label{sec:integer-unit-extns:arith-logic-insns}



These  instructions provide  64-bit  arithmetic/logic  support in  the
integer unit.  The instructions work  on 64-bit register pairs in most
cases.  Register-pairs are  identified by a 5-bit  even number (lowest
bit     must     be     0).      See     Tables~\ref{tab:arith:insns},
\ref{tab:shift:insns},            \ref{tab:muldiv:insns}           and
\ref{tab:64bit:logical:insns}.

\begin{table}[p]
  \centering
  \begin{tabular}[p]{|l|l|}
    \hline
\multicolumn{2}{|l|}{	\textbf{ADDD}			} \\ 
\hline
 		  same as ADD, but with Instr[13]=0 (i=0), and Instr[5]=1. & 
 		rd(pair) $\leftarrow$ rs1(pair) + rs2(pair)\\
\hline
\hline
\multicolumn{2}{|l|}{	\textbf{ADDDCC}} \\ 
\hline
 		  same as ADDCC, but with Instr[13]=0 (i=0), and Instr[5]=1. & 
 		rd(pair) $\leftarrow$ rs1(pair) + rs2(pair), set Z,N\\
\hline
\hline
\multicolumn{2}{|l|}{	\textbf{SUBD}} \\ 
\hline
 		  same as SUB, but with Instr[13]=0 (i=0), and Instr[5]=1. & 
 		rd(pair) $\leftarrow$ rs1(pair) - rs2(pair)\\
\hline
\hline
\multicolumn{2}{|l|}{	\textbf{SUBDCC}} \\ 
\hline
 		  same as SUBCC, but with Instr[13]=0 (i=0), and Instr[5]=1. & 
 		rd(pair) $\leftarrow$ rs1(pair) - rs2(pair), set Z,N\\
\hline
  \end{tabular}
  \caption{Addition and Subtraction Instructions}
  \label{tab:arith:insns}
\end{table}

\begin{table}[p]
  \centering
  \begin{tabular}[p]{|p{.45\textwidth}|p{.45\textwidth}|}
    \hline
\multicolumn{2}{|l|}{	\textbf{SLLD}} \\ 
 \hline 
    same as SLL, but with Instr[6:5]=2.
    if imm bit (Instr[13]) is 1, then Instr[5:0] is the shift-amount.
    else shift-amount is the lowest 5 bits of rs2. Note that rs2
    is a 32-bit register. & 
    rd(pair) $\leftarrow$  rs1(pair) $<<$ shift-amount\\
\hline
\hline
\multicolumn{2}{|l|}{	\textbf{SRLD}} \\ 
 \hline 
    same as SRL, but with Instr[6:5]=2.
    if imm bit (Instr[13]) is 1, then Instr[5:0] is the shift-amount.
    else shift-amount is the lowest 5 bits of rs2. Note that rs2
    is a 32-bit register. & 
    rd(pair) $\leftarrow$  rs1(pair) $>>$ shift-amount\\
\hline
\hline
\multicolumn{2}{|l|}{	\textbf{SRAD}} \\ 
 \hline 
    same as SRA, but with Instr[6:5]=2.
    if imm bit (Instr[13]) is 1, then Instr[5:0] is the shift-amount.
    else shift-amount is the lowest 5 bits of rs2. Note that rs2
    is a 32-bit register. & 
    rd(pair) $\leftarrow$  rs1(pair) $>>$ shift-amount (with sign extension).\\
\hline
  \end{tabular}
  \caption{Shift instructions}
  \label{tab:shift:insns}
\end{table}

\begin{table}[p]
  \centering
  \begin{tabular}[p]{|p{.45\textwidth}|p{.45\textwidth}|}
    \hline
\multicolumn{2}{|l|}{\textbf{	UMULD}} \\ 
 \hline 
 		  same as UMUL, but with Instr[13]=0 (i=0), and Instr[5]=1. & 
 		rd(pair) $\leftarrow$ rs1(pair) * rs2(pair)\\
\hline
\hline
\multicolumn{2}{|l|}{\textbf{	UMULDCC}} \\ 
 \hline 
 		  same as UMULCC, but with Instr[13]=0 (i=0), and Instr[5]=1. & 
 		rd(pair) $\leftarrow$ rs1(pair) * rs2(pair), sets Z,\\
\hline
\hline
\multicolumn{2}{|l|}{\textbf{	SMULD}} \\ 
 \hline 
 		  same as SMULD, but with Instr[13]=0 (i=0), and Instr[5]=1. & 
 		rd(pair) $\leftarrow$ rs1(pair) * rs2(pair) (signed)\\
\hline
\hline
\multicolumn{2}{|l|}{\textbf{	SMULDCC}} \\ 
 \hline 
 		  same as SMULCC, but with Instr[13]=0 (i=0), and Instr[5]=1. & 
 		\parbox{\linewidth}{rd(pair) $\leftarrow$ rs1(pair) *
                  rs2(pair) (signed)\\	sets condition codes Z,N,Ovflow}\\ 
\hline
\hline
\multicolumn{2}{|l|}{\textbf{	UDIVD}} \\ 
 \hline 
 		  same as UDIV, but with Instr[13]=0 (i=0), and Instr[5]=1. & 
 		\parbox{\linewidth}{rd(pair) $\leftarrow$ rs1(pair) /
                 rs2(pair)\\
    \textbf{Note:} can generate div-by-zero trap.}\\
\hline
\hline
\multicolumn{2}{|l|}{\textbf{	UDIVDCC}} \\ 
 \hline 
 		  same as UDIVCC, but with Instr[13]=0 (i=0), and Instr[5]=1. & 
 		\parbox{\linewidth}{rd(pair) $\leftarrow$ rs1(pair) /
                 rs2(pair),\\ sets condition codes Z,Ovflow \\ \textbf{Note:} can generate div-by-zero trap.}\\
\hline
\hline
\multicolumn{2}{|l|}{\textbf{	SDIVD}} \\ 
 \hline 
 		  same as SDIV, but with Instr[13]=0 (i=0), and Instr[5]=1. & 
 		rd(pair) $\leftarrow$ rs1(pair) / rs2(pair) (signed)\\
\hline
\hline
\multicolumn{2}{|l|}{\textbf{	SDIVDCC}} \\ 
 \hline 
 		  same as SDIVCC, but with Instr[13]=0 (i=0), and Instr[5]=1. & 
 		\parbox{\linewidth}{rd(pair) $\leftarrow$ rs1(pair) /
                  rs2(pair) (signed),\\ sets condition codes
                  Z,N,Ovflow,\\ \textbf{Note:} can generate
                  div-by-zero trap.}\\ 
\hline
  \end{tabular}
  \caption{Multiplication and Division Instructions}
  \label{tab:muldiv:insns}
\end{table}

\begin{table}[p]
  \centering
  \begin{tabular}[p]{|p{.45\textwidth}|p{.45\textwidth}|}
    \hline
\multicolumn{2}{|l|}{\textbf{	ORD}} \\ 
 \hline 
 \parbox{\linewidth}{		  same as OR, but with Instr[13]=0 (i=0), and Instr[5]=1.} & 
 \parbox{\linewidth}{		rd(pair) $\leftarrow$ rs1(pair) $\vert$ rs2(pair)}\\
\hline
\hline
\multicolumn{2}{|l|}{\textbf{	ORDCC}} \\ 
 \hline 
 \parbox{\linewidth}{		  same as ORCC, but with Instr[13]=0 (i=0), and Instr[5]=1.} & 
 \parbox{\linewidth}{		rd(pair) $\leftarrow$ rs1(pair) $\vert$ rs2(pair),\\ sets Z.}\\
\hline
\hline
\multicolumn{2}{|l|}{\textbf{	ORDN}} \\ 
 \hline 
 \parbox{\linewidth}{		  same as ORN, but with Instr[13]=0 (i=0), and Instr[5]=1.} & 
 \parbox{\linewidth}{		rd(pair) $\leftarrow$ rs1(pair) $\vert$ ($\sim$rs2(pair))}\\
\hline
\hline
\multicolumn{2}{|l|}{\textbf{	ORDNCC}} \\ 
 \hline 
 \parbox{\linewidth}{		  same as ORNCC, but with Instr[13]=0 (i=0), and Instr[5]=1.} & 
 \parbox{\linewidth}{		rd(pair) $\leftarrow$ rs1(pair) $\vert$ ($\sim$rs2(pair)),\\ sets Z                 sets Z.}\\
\hline
\hline
\multicolumn{2}{|l|}{\textbf{	XORDCC}} \\ 
 \hline 
 \parbox{\linewidth}{		  same as XORCC, but with Instr[13]=0 (i=0), and Instr[5]=1.} & 
 \parbox{\linewidth}{		rd(pair) $\leftarrow$ rs1(pair) $\hat{}$ rs2(pair), \\sets Z		sets Z.}\\
\hline
\hline
\multicolumn{2}{|l|}{\textbf{	XNORD}} \\ 
 \hline 
 \parbox{\linewidth}{		  same as XNOR, but with Instr[13]=0 (i=0), and Instr[5]=1.} & 
 \parbox{\linewidth}{		rd(pair) $\leftarrow$ rs1(pair) $\hat{}$ rs2(pair)}\\
\hline
\hline
\multicolumn{2}{|l|}{\textbf{	XNORDCC}} \\ 
 \hline 
 \parbox{\linewidth}{		  same as XNORCC, but with Instr[13]=0 (i=0), and Instr[5]=1.} & 
 \parbox{\linewidth}{		rd(pair) $\leftarrow$ rs1(pair) $\hat{}$ rs2(pair),\\ sets Z}\\
\hline
\hline
\multicolumn{2}{|l|}{\textbf{	ANDD}} \\ 
 \hline 
 \parbox{\linewidth}{		  same as AND, but with Instr[13]=0 (i=0), and Instr[5]=1.} & 
 \parbox{\linewidth}{		rd(pair) $\leftarrow$ rs1(pair) . rs2(pair)}\\
\hline
\hline
\multicolumn{2}{|l|}{\textbf{	ANDDCC}} \\ 
 \hline 
 \parbox{\linewidth}{		  same as ANDCC, but with Instr[13]=0 (i=0), and Instr[5]=1.} & 
 \parbox{\linewidth}{		rd(pair) $\leftarrow$ rs1(pair) . rs2(pair),\\ sets Z}\\
\hline
\hline
\multicolumn{2}{|l|}{\textbf{	ANDDN}} \\ 
 \hline 
 \parbox{\linewidth}{		  same as ANDN, but with Instr[13]=0 (i=0), and Instr[5]=1.} & 
 \parbox{\linewidth}{		rd(pair) $\leftarrow$ rs1(pair) . ($\sim$rs2(pair))}\\
\hline
\hline
\multicolumn{2}{|l|}{\textbf{	ANDDNCC}} \\ 
 \hline 
 \parbox{\linewidth}{		  same as ANDNCC, but with Instr[13]=0 (i=0), and Instr[5]=1.} & 
 \parbox{\linewidth}{		rd $\leftarrow$ rs1 . ($\sim$rs2),\\ sets Z}\\
\hline
  \end{tabular}
  \caption{64 bit Logical Instructions}
  \label{tab:64bit:logical:insns}
\end{table}

\subsection{Integer-Unit Extensions: SIMD Instructions}
\label{sec:integer-unit-extns:simd-instructions}

These instructions  are vector instructions  which work on  two source
registers (each a  64 bit register pair), and produce  a 64-bit vector
result.   The   vector  elements  can  be   8-bit/16-bit/32-bit.   See
Table~\ref{tab:simd:insns}.

\begin{table}[p]
  \centering
  \begin{tabular}[p]{|p{.45\textwidth}|p{.45\textwidth}|}
    \hline
\multicolumn{2}{|l|}{\textbf{	VADDD8, VADDD16, VADDD32}} \\ 
 \hline 
 \parbox{\linewidth}{        Same as ADDD, but with Instr[13]=0 (i=0),
    and Instr[6:5]=2. Bits Instr[9:7] are a 3-bit field, which specify
    the data type \\
\begin{tabular}[p]{|l|l|l|}
\hline
  001  &   byte			 & (VADDD8)\\
  010  &   half-word (16-bits)	 & (VADDD16)\\
  100  &   word (32-bits) 		 & (VADDD32)\\
\hline
\end{tabular}\\
} & 
 \parbox{\linewidth}{        Performs a vector operation by considering the 64-bit operands as a vector of objects with specified data-type.}\\
\hline
\hline
\multicolumn{2}{|l|}{\textbf{	VSUBD8, VSUBD16, VSUBD32}} \\ 
 \hline 
 \parbox{\linewidth}{        Same as ADDD, but with Instr[13]=0 (i=0),
    and Instr[6:5]=2.  Bits Instr[9:7] are a 3-bit field, which
    specify the data type\\
\begin{tabular}[p]{|l|l|l|}
\hline
  001  &   byte 			 & (VSUBD8)\\
  010  &   half-word (16-bits)	 & (VSUBD16)\\
  100  &   word (32-bits) 		 & (VSUBD32)\\
\hline
\end{tabular}\\
} & 
 \parbox{\linewidth}{        Performs a vector operation by considering the 64-bit operands as a vector of objects with specified data-type.}\\
\hline
\hline
\multicolumn{2}{|l|}{\textbf{	VUMULD8, VUMULD16, VUMULD32}} \\ 
 \hline 
 \parbox{\linewidth}{        Same as ADDD, but with Instr[13]=0 (i=0),
    and Instr[6:5]=2. Bits Instr[9:7] are a 3-bit field, which specify
    the data type\\
\begin{tabular}[p]{|l|l|l|}
\hline
  001  &   byte			 & (VMULD8)\\
  010  &   half-word (16-bits)	 & (VMULD16)\\
  100  &   word (32-bits) 		 & (VMULD32)\\
\hline
\end{tabular}\\
} & 
 \parbox{\linewidth}{	Performs a vector operation by considering the 64-bit operands as a vector of objects with specified data-type.}\\
\hline
\hline
\multicolumn{2}{|l|}{\textbf{	VSMULD8, VSUMLD16, VSMULD32}} \\ 
 \hline 
 \parbox{\linewidth}{        Same as ADDD, but with Instr[13]=0 (i=0),
    and Instr[6:5]=2. Bits Instr[9:7] are a 3-bit field, which specify
    the data type\\
\begin{tabular}[p]{|l|l|l|}
\hline
  001  &   byte			 & (VSMULD8)\\
  010  &   half-word (16-bits)	 & (VSMULD16)\\
  100  &   word (32-bits) 		 & (VSMULD32)\\
\hline
\end{tabular}\\
} & 
 \parbox{\linewidth}{	Performs a vector operation by considering the 64-bit operands as a vector of objects with specified data-type.}\\
\hline
  \end{tabular}
  \caption{SIMD Instructions}
  \label{tab:simd:insns}
\end{table}

% \newpage
\subsection{Integer-Unit Extensions: SIMD Instructions II}
\label{sec:integer-unit-extns:simd-instructions:2}

These  instructions  are vector  instructions  which  reduce a  source
register to a byte result.  See Table~\ref{tab:simd:2:insns}.

\begin{table}[p]
  \centering
  \begin{tabular}[p]{|p{.45\textwidth}|p{.45\textwidth}|}
    \hline
\multicolumn{2}{|l|}{\textbf{ORDBYTER} (Byte-Reduce OR)} \\ 
 \hline 
 \parbox{\linewidth}{op=2, op3[3:0]=0xe, op3[5:4]=0x2, contents[7:0]
    of rs2 specify a mask.\\

    Instr[31:30] (op) = 0x2\\
    Instr[29:25] (rd)    lowest bit assumed 0.\\
    Instr[24:19] (op3) = 111010\\
    Instr[18:14] (rs1)   lowest bit assumed 0.\\
    Instr[13]    (i)  = 0 (ignored)\\
    Instr[12:5]   (zero)\\
    Instr[4:0]   (rs2)   32-bit register is read.\\
} & 
 \parbox{\linewidth}{rd $\leftarrow$ (rs1\_7.m7 $\vert$ rs1\_6.m6 $\vert$ rs1\_5.m5 ... $\vert$ rs1\_0.m0)}\\
\hline
\hline
\multicolumn{2}{|l|}{\textbf{ANDDBYTER} (Byte-Reduce AND)} \\ 
 \hline 
 \parbox{\linewidth}{op=2, op3[3:0]=0xf, op3[5:4]=0x2, contents[7:0]
    of rs2 specify a mask.\\

    Instr[31:30] (op) = 0x2\\
    Instr[29:25] (rd)    lowest bit assumed 0.\\
    Instr[24:19] (op3) = 111110\\
    Instr[18:14] (rs1)   lowest bit assumed 0.\\
    Instr[13]    (i)  = 0 (ignored)\\
    Instr[12:5]   (zero)\\
    Instr[4:0]   (rs2)   32-bit register is read.\\
} & 
 \parbox{\linewidth}{rd $\leftarrow$ ( (m7 ? rs1\_7 : 0xff) . (m6 ? rs1\_6 : 0xff) \ldots (m0 ? rs1\_0 : 0xff))}\\
\hline
\hline
\multicolumn{2}{|l|}{\textbf{XORDBYTER} (Byte-Reduce XOR)} \\ 
 \hline 
 \parbox{\linewidth}{op=2, op3[3:0]=0xe, op3[5:4]=0x3, contents[7:0]
    of rs2 specify a mask.\\

    Instr[31:30] (op) = 0x2\\
    Instr[29:25] (rd)    lowest bit assumed 0.\\
    Instr[24:19] (op3) = 111011\\
    Instr[18:14] (rs1)   lowest bit assumed 0.\\
    Instr[13]    (i)  = 0 (ignored)\\
    Instr[12:5]   (zero)\\
    Instr[4:0]   (rs2)   32-bit register is read.\\
} & 
 \parbox{\linewidth}{rd $\leftarrow$ (rs1\_7.m7 $\hat{}$ rs1\_6.m6 $\hat{}$ rs1\_5.m5 ... $\hat{}$ rs1\_0.m0)}\\
\hline
\hline
\multicolumn{2}{|l|}{\textbf{ZBYTEDPOS} (Positions-of-Zero-Bytes in D-Word)} \\ 
 \hline 
 \parbox{\linewidth}{op=2, op3[3:0]=0xf, op3[5:4]=0x3, contents[7:0]
    of rs2/imm-value specify a mask.\\

    Instr[31:30] (op) = 0x2\\
    Instr[29:25] (rd)    lowest bit assumed 0.\\
    Instr[24:19] (op3) = 111011\\
    Instr[18:14] (rs1)   lowest bit assumed 0.\\
    Instr[13]    (i)  =  if 0, use rs2, else Instr[7:0]\\
    Instr[12:5]  = 0  (ignored if i=0)\\
    Instr[4:0]   (rs2, if i=0) 32-bit register is read.\\
} & 
 \parbox{\linewidth}{rd $\leftarrow$ [b7\_zero b6\_zero b5\_zero b4\_zero \ldots b0\_zero] (if mask-bit is zero then b$\star$\_zero is zero)}\\
\hline
  \end{tabular}
  \caption{SIMD Instructions II}
  \label{tab:simd:2:insns}
\end{table}

% \newpage
\subsection{Vector Floating Point Instructions}
\label{sec:vector-floating-point-instructions}

These are vector  float operations which work on  two single precision
operand  pairs   to  produce   two  single  precision   results.   See
Table~\ref{tab:simd:float:ops}.

\begin{table}[p]
  \centering
  \begin{tabular}[p]{|l|l|}
    \hline
    \textbf{VFADD} & {op=2, op3=0x34, opf=0x142} \\
    \hline
    \textbf{VFSUB} & {op=2, op3=0x34, opf=0x146} \\
    \hline
    \textbf{VFMUL} & {op=2, op3=0x34, opf=0x14a} \\
    \hline
    \textbf{VFDIV} & {op=2, op3=0x34, opf=0x14e} \\
    \hline
    \textbf{VFSQRT} & {op=2, op3=0x34, opf=0x12a} \\
    \hline      
  \end{tabular}
  \caption[SIMD Floating Point Operations]{SIMD Floating Point
    Operations.  NaN propagated, but no traps. For each of these,
    rs1,rs2,rd are considered even numbers pointing to.
  }
  \label{tab:simd:float:ops}
\end{table}

\subsection{CSWAP instructions}
\label{sec:cswap-instructions}

The Sparc-V8 ISA does not include a compare-and-swap (CAS) instruction
which is very  useful in achieving consensus  among distributed agents
when the number of agents is $>$ 2.  See Table~\ref{tab:cswap:insns}.

We introduce a CSWAP instruction in two flavours:
		% CSWAP64     rs1, rs2-pair/immediate, rd-pair
		% 	op=3
		% 	op3= 10 1111
		% 		(rest of instruction similar to SWAP)
			
		% CSWAP64A    rs1, rs2-pair/immediate, rd-pair, asi
		% 	op=3
		% 	op3= 11 1111
                %         (rest of instruction similar to SWAPA)

                %         // CSWAP64 has no explicit ASI, while CSWAP64A
                %         // does! Any ambiguity issues?

\begin{table}[p]
  \centering
  \begin{tabular}[p]{|p{.45\textwidth}|p{.45\textwidth}|}
    \hline
\multicolumn{2}{|l|}{\textbf{CSWAP64} (effective address in registers
    rs1 and rs2)} \\ 
 \hline 
 \parbox{\linewidth}{op=3, op3=10 1111, i=0.\\

    Instr[31:30] (op) = 0x3\\
    Instr[29:25] (rd)    lowest bit assumed 0.\\
    Instr[24:19] (op3) = 101111\\
    Instr[18:14] (rs1)   lowest bit assumed 0.\\
    Instr[13]    (i)  = 0 (registers based effective address)\\
    Instr[12:5]  (asi) = Address Space Identifier (See: Appendix G of V8)\\
    Instr[4:0]   (rs2)   32-bit register is read.\\
} & 
 \parbox{\linewidth}{~}\\
\hline
    \hline
\multicolumn{2}{|l|}{\textbf{CSWAP64} (immediate effective address)} \\ 
 \hline 
 \parbox{\linewidth}{op=3, op3=10 1111, i=1.\\

    Instr[31:30] (op) = 0x3\\
    Instr[29:25] (rd)    lowest bit assumed 0.\\
    Instr[24:19] (op3) = 101111\\
    Instr[18:14] (rs1)   lowest bit assumed 0.\\
    Instr[13]    (i)  = 1 (immediate effective address)\\
    Instr[12:0]  (simm13) 13-bit immediate address.\\
} & 
 \parbox{\linewidth}{~}\\
\hline
    \hline
\multicolumn{2}{|l|}{\textbf{CSWAP64A} (effective address in registers
    rs1 and rs2)} \\ 
 \hline 
 \parbox{\linewidth}{op=3, op3=10 1111, i=0.\\

    Instr[31:30] (op) = 0x3\\
    Instr[29:25] (rd)    lowest bit assumed 0.\\
    Instr[24:19] (op3) = 111111\\
    Instr[18:14] (rs1)   lowest bit assumed 0.\\
    Instr[13]    (i)  = 0 (registers based effective address)\\
    Instr[12:5]  (asi) = Address Space Identifier (See: Appendix G of V8)\\
    Instr[4:0]   (rs2)   32-bit register is read.\\
} & 
 \parbox{\linewidth}{~}\\
\hline
    \hline
\multicolumn{2}{|l|}{\textbf{CSWAP64A} (immediate effective address)} \\ 
 \hline 
 \parbox{\linewidth}{op=3, op3=10 1111, i=1.\\

    Instr[31:30] (op) = 0x3\\
    Instr[29:25] (rd)    lowest bit assumed 0.\\
    Instr[24:19] (op3) = 111111\\
    Instr[18:14] (rs1)   lowest bit assumed 0.\\
    Instr[13]    (i)  = 1 (immediate effective address)\\
    Instr[12:0]  (simm13) 13-bit immediate address.\\
} & 
 \parbox{\linewidth}{~}\\
\hline
  \end{tabular}
  \caption{CSWAP Instructions}
  \label{tab:cswap:insns}
\end{table}

This has been superseded by version 2 as below.

\section{ISA Version 2}
\label{sec:isa:v2}

\subsection{Integer-Unit Extensions: Arithmetic-Logic Instructions}
\label{sec:integer-unit-extns:arith-logic-insns}



These  instructions provide  64-bit  arithmetic/logic  support in  the
integer unit.  The instructions work  on 64-bit register pairs in most
cases.  Register-pairs are  identified by a 5-bit  even number (lowest
bit     must     be     0).      See     Tables~\ref{tab:arith:insns},
\ref{tab:shift:insns},            \ref{tab:muldiv:insns}           and
\ref{tab:64bit:logical:insns}.

\begin{table}[p]
  \centering
  \begin{tabular}[p]{|l|l|}
    \hline
\multicolumn{2}{|l|}{	\textbf{ADDD}			} \\ 
\hline
 		  same as ADD, but with Instr[13]=0 (i=0), and Instr[5]=1. & 
 		rd(pair) $\leftarrow$ rs1(pair) + rs2(pair)\\
\hline
\hline
\multicolumn{2}{|l|}{	\textbf{ADDDCC}} \\ 
\hline
 		  same as ADDCC, but with Instr[13]=0 (i=0), and Instr[5]=1. & 
 		rd(pair) $\leftarrow$ rs1(pair) + rs2(pair), set Z,N\\
\hline
\hline
\multicolumn{2}{|l|}{	\textbf{SUBD}} \\ 
\hline
 		  same as SUB, but with Instr[13]=0 (i=0), and Instr[5]=1. & 
 		rd(pair) $\leftarrow$ rs1(pair) - rs2(pair)\\
\hline
\hline
\multicolumn{2}{|l|}{	\textbf{SUBDCC}} \\ 
\hline
 		  same as SUBCC, but with Instr[13]=0 (i=0), and Instr[5]=1. & 
 		rd(pair) $\leftarrow$ rs1(pair) - rs2(pair), set Z,N\\
\hline
  \end{tabular}
  \caption{Addition and Subtraction Instructions}
  \label{tab:arith:insns}
\end{table}

\begin{table}[p]
  \centering
  \begin{tabular}[p]{|p{.45\textwidth}|p{.45\textwidth}|}
    \hline
\multicolumn{2}{|l|}{	\textbf{SLLD}} \\ 
 \hline 
    Same as SLL, but with Instr[7:6]=2.
    If imm bit (Instr[13]) is 1, then Instr[5:0] is the shift-amount,
    else shift-amount is the lowest 6 bits of rs2. Note that rs2
    is a 32-bit register. & 
    rd(pair) $\leftarrow$  rs1(pair) $<<$ shift-amount\\
\hline
\hline
\multicolumn{2}{|l|}{	\textbf{SRLD}} \\ 
 \hline 
    Same as SRL, but with Instr[7:6]=2.
    If imm bit (Instr[13]) is 1, then Instr[5:0] is the shift-amount,
    else shift-amount is the lowest 6 bits of rs2. Note that rs2
    is a 32-bit register. & 
    rd(pair) $\leftarrow$  rs1(pair) $>>$ shift-amount\\
\hline
\hline
\multicolumn{2}{|l|}{	\textbf{SRAD}} \\ 
 \hline 
    Same as SRA, but with Instr[7:6]=2.
    If imm bit (Instr[13]) is 1, then Instr[5:0] is the shift-amount,
    else shift-amount is the lowest 6 bits of rs2. Note that rs2
    is a 32-bit register. & 
    rd(pair) $\leftarrow$  rs1(pair) $>>$ shift-amount (with sign extension).\\
\hline
  \end{tabular}
  \caption{Shift instructions}
  \label{tab:shift:insns}
\end{table}

\begin{table}[p]
  \centering
  \begin{tabular}[p]{|p{.45\textwidth}|p{.45\textwidth}|}
    \hline
\multicolumn{2}{|l|}{\textbf{	UMULD}} \\ 
 \hline 
 		  same as UMUL, but with Instr[13]=0 (i=0), and Instr[5]=1. & 
 		rd(pair) $\leftarrow$ rs1(pair) * rs2(pair)\\
\hline
\hline
\multicolumn{2}{|l|}{\textbf{	UMULDCC}} \\ 
 \hline 
 		  same as UMULCC, but with Instr[13]=0 (i=0), and Instr[5]=1. & 
 		rd(pair) $\leftarrow$ rs1(pair) * rs2(pair), sets Z,\\
\hline
\hline
\multicolumn{2}{|l|}{\textbf{	SMULD}} \\ 
 \hline 
 		  same as SMULD, but with Instr[13]=0 (i=0), and Instr[5]=1. & 
 		rd(pair) $\leftarrow$ rs1(pair) * rs2(pair) (signed)\\
\hline
\hline
\multicolumn{2}{|l|}{\textbf{	SMULDCC}} \\ 
 \hline 
 		  same as SMULCC, but with Instr[13]=0 (i=0), and Instr[5]=1. & 
 		\parbox{\linewidth}{rd(pair) $\leftarrow$ rs1(pair) *
                  rs2(pair) (signed)\\	sets condition codes Z,N,Ovflow}\\ 
\hline
\hline
\multicolumn{2}{|l|}{\textbf{	UDIVD}} \\ 
 \hline 
 		  same as UDIV, but with Instr[13]=0 (i=0), and Instr[5]=1. & 
 		\parbox{\linewidth}{rd(pair) $\leftarrow$ rs1(pair) /
                 rs2(pair)\\
    \textbf{Note:} can generate div-by-zero trap.}\\
\hline
\hline
\multicolumn{2}{|l|}{\textbf{	UDIVDCC}} \\ 
 \hline 
 		  same as UDIVCC, but with Instr[13]=0 (i=0), and Instr[5]=1. & 
 		\parbox{\linewidth}{rd(pair) $\leftarrow$ rs1(pair) /
                 rs2(pair),\\ sets condition codes Z,Ovflow \\ \textbf{Note:} can generate div-by-zero trap.}\\
\hline
\hline
\multicolumn{2}{|l|}{\textbf{	SDIVD}} \\ 
 \hline 
 		  same as SDIV, but with Instr[13]=0 (i=0), and Instr[5]=1. & 
 		rd(pair) $\leftarrow$ rs1(pair) / rs2(pair) (signed)\\
\hline
\hline
\multicolumn{2}{|l|}{\textbf{	SDIVDCC}} \\ 
 \hline 
 		  same as SDIVCC, but with Instr[13]=0 (i=0), and Instr[5]=1. & 
 		\parbox{\linewidth}{rd(pair) $\leftarrow$ rs1(pair) /
                  rs2(pair) (signed),\\ sets condition codes
                  Z,N,Ovflow,\\ \textbf{Note:} can generate
                  div-by-zero trap.}\\ 
\hline
  \end{tabular}
  \caption{Multiplication and Division Instructions}
  \label{tab:muldiv:insns}
\end{table}

\begin{table}[p]
  \centering
  \begin{tabular}[p]{|p{.45\textwidth}|p{.45\textwidth}|}
    \hline
\multicolumn{2}{|l|}{\textbf{	ORD}} \\ 
 \hline 
 \parbox{\linewidth}{		  same as OR, but with Instr[13]=0 (i=0), and Instr[5]=1.} & 
 \parbox{\linewidth}{		rd(pair) $\leftarrow$ rs1(pair) $\vert$ rs2(pair)}\\
\hline
\hline
\multicolumn{2}{|l|}{\textbf{	ORDCC}} \\ 
 \hline 
 \parbox{\linewidth}{		  same as ORCC, but with Instr[13]=0 (i=0), and Instr[5]=1.} & 
 \parbox{\linewidth}{		rd(pair) $\leftarrow$ rs1(pair) $\vert$ rs2(pair),\\ sets Z.}\\
\hline
\hline
\multicolumn{2}{|l|}{\textbf{	ORDN}} \\ 
 \hline 
 \parbox{\linewidth}{		  same as ORN, but with Instr[13]=0 (i=0), and Instr[5]=1.} & 
 \parbox{\linewidth}{		rd(pair) $\leftarrow$ rs1(pair) $\vert$ ($\sim$rs2(pair))}\\
\hline
\hline
\multicolumn{2}{|l|}{\textbf{	ORDNCC}} \\ 
 \hline 
 \parbox{\linewidth}{		  same as ORNCC, but with Instr[13]=0 (i=0), and Instr[5]=1.} & 
 \parbox{\linewidth}{		rd(pair) $\leftarrow$ rs1(pair) $\vert$ ($\sim$rs2(pair)),\\ sets Z                 sets Z.}\\
\hline
\hline
\multicolumn{2}{|l|}{\textbf{	XORDCC}} \\ 
 \hline 
 \parbox{\linewidth}{		  same as XORCC, but with Instr[13]=0 (i=0), and Instr[5]=1.} & 
 \parbox{\linewidth}{		rd(pair) $\leftarrow$ rs1(pair) $\hat{}$ rs2(pair), \\sets Z		sets Z.}\\
\hline
\hline
\multicolumn{2}{|l|}{\textbf{	XNORD}} \\ 
 \hline 
 \parbox{\linewidth}{		  same as XNOR, but with Instr[13]=0 (i=0), and Instr[5]=1.} & 
 \parbox{\linewidth}{		rd(pair) $\leftarrow$ rs1(pair) $\hat{}$ rs2(pair)}\\
\hline
\hline
\multicolumn{2}{|l|}{\textbf{	XNORDCC}} \\ 
 \hline 
 \parbox{\linewidth}{		  same as XNORCC, but with Instr[13]=0 (i=0), and Instr[5]=1.} & 
 \parbox{\linewidth}{		rd(pair) $\leftarrow$ rs1(pair) $\hat{}$ rs2(pair),\\ sets Z}\\
\hline
\hline
\multicolumn{2}{|l|}{\textbf{	ANDD}} \\ 
 \hline 
 \parbox{\linewidth}{		  same as AND, but with Instr[13]=0 (i=0), and Instr[5]=1.} & 
 \parbox{\linewidth}{		rd(pair) $\leftarrow$ rs1(pair) . rs2(pair)}\\
\hline
\hline
\multicolumn{2}{|l|}{\textbf{	ANDDCC}} \\ 
 \hline 
 \parbox{\linewidth}{		  same as ANDCC, but with Instr[13]=0 (i=0), and Instr[5]=1.} & 
 \parbox{\linewidth}{		rd(pair) $\leftarrow$ rs1(pair) . rs2(pair),\\ sets Z}\\
\hline
\hline
\multicolumn{2}{|l|}{\textbf{	ANDDN}} \\ 
 \hline 
 \parbox{\linewidth}{		  same as ANDN, but with Instr[13]=0 (i=0), and Instr[5]=1.} & 
 \parbox{\linewidth}{		rd(pair) $\leftarrow$ rs1(pair) . ($\sim$rs2(pair))}\\
\hline
\hline
\multicolumn{2}{|l|}{\textbf{	ANDDNCC}} \\ 
 \hline 
 \parbox{\linewidth}{		  same as ANDNCC, but with Instr[13]=0 (i=0), and Instr[5]=1.} & 
 \parbox{\linewidth}{		rd $\leftarrow$ rs1 . ($\sim$rs2),\\ sets Z}\\
\hline
  \end{tabular}
  \caption{64 bit Logical Instructions}
  \label{tab:64bit:logical:insns}
\end{table}

\subsection{Integer-Unit Extensions: SIMD Instructions}
\label{sec:integer-unit-extns:simd-instructions}

These instructions  are vector instructions  which work on  two source
registers (each a  64 bit register pair), and produce  a 64-bit vector
result.   The   vector  elements  can  be   8-bit/16-bit/32-bit.   See
Table~\ref{tab:simd:insns}.

\begin{table}[p]
  \centering
  \begin{tabular}[p]{|p{.45\textwidth}|p{.45\textwidth}|}
    \hline
\multicolumn{2}{|l|}{\textbf{	VADDD8, VADDD16, VADDD32}} \\ 
 \hline 
 \parbox{\linewidth}{        Same as ADDD, but with Instr[13]=0 (i=0),
    and Instr[6:5]=2. Bits Instr[9:7] are a 3-bit field, which specify
    the data type \\
\begin{tabular}[p]{|l|l|l|}
\hline
  001  &   byte			 & (VADDD8)\\
  010  &   half-word (16-bits)	 & (VADDD16)\\
  100  &   word (32-bits) 	 & (VADDD32)\\
\hline
\end{tabular}\\
} & 
 \parbox{\linewidth}{Performs a vector operation by considering the
    64-bit operands as a vector of objects with specified data-type.
    \vskip \parskip
    \texttt{vaddd8   rs1, rs2, rd}
    \vskip \parskip
    \texttt{vaddd16  rs1, rs2, rd}
    \vskip \parskip
    \texttt{vaddd32  rs1, rs2, rd}
    }\\
\hline
\hline
\multicolumn{2}{|l|}{\textbf{	VSUBD8, VSUBD16, VSUBD32}} \\ 
 \hline 
 \parbox{\linewidth}{Same as SUBD, but with Instr[13]=0 (i=0),
    and Instr[6:5]=2.  Bits Instr[9:7] are a 3-bit field, which
    specify the data type\\
\begin{tabular}[p]{|l|l|l|}
\hline
  001  &   byte 		 & (VSUBD8)\\
  010  &   half-word (16-bits)	 & (VSUBD16)\\
  100  &   word (32-bits) 	 & (VSUBD32)\\
\hline
\end{tabular}\\
} & 
 \parbox{\linewidth}{Performs a vector operation by considering the
    64-bit operands as a vector of objects with specified
    data-type.
    \vskip \parskip
    \texttt{vsubd8   rs1, rs2, rd}
    \vskip \parskip
    \texttt{vsubd16  rs1, rs2, rd}
    \vskip \parskip
    \texttt{vsubd32  rs1, rs2, rd}
    }\\ 
\hline
\hline
\multicolumn{2}{|l|}{\textbf{	VUMULD8, VUMULD16, VUMULD32}} \\ 
 \hline 
 \parbox{\linewidth}{Same as UMULD, but with Instr[13]=0 (i=0),
    and Instr[6:5]=2. Bits Instr[9:7] are a 3-bit field, which specify
    the data type\\
\begin{tabular}[p]{|l|l|l|}
\hline
  001  &   byte			 & (VMULD8)\\
  010  &   half-word (16-bits)	 & (VMULD16)\\
  100  &   word (32-bits) 	 & (VMULD32)\\
\hline
\end{tabular}\\
} & 
 \parbox{\linewidth}{Performs a vector operation by considering the
    64-bit operands as a vector of objects with specified data-type.
    \vskip \parskip
    \texttt{vumuld8   rs1, rs2, rd}
    \vskip \parskip
    \texttt{vumuld16  rs1, rs2, rd}
    \vskip \parskip
    \texttt{vumuld32  rs1, rs2, rd}
    }\\
\hline
\hline
\multicolumn{2}{|l|}{\textbf{	VSMULD8, VSUMLD16, VSMULD32}} \\ 
 \hline 
 \parbox{\linewidth}{        Same as SMULD, but with Instr[13]=0 (i=0),
    and Instr[6:5]=2. Bits Instr[9:7] are a 3-bit field, which specify
    the data type\\
\begin{tabular}[p]{|l|l|l|}
\hline
  001  &   byte			 & (VSMULD8)\\
  010  &   half-word (16-bits)	 & (VSMULD16)\\
  100  &   word (32-bits) 	 & (VSMULD32)\\
\hline
\end{tabular}\\
} & 
 \parbox{\linewidth}{Performs a vector operation by considering the
    64-bit operands as a vector of objects with specified data-type.
    \vskip \parskip
    \texttt{vsmuld8   rs1, rs2, rd}
    \vskip \parskip
    \texttt{vsmuld16  rs1, rs2, rd}
    \vskip \parskip
    \texttt{vsmuld32  rs1, rs2, rd}
    }\\
\hline
  \end{tabular}
  \caption{SIMD Instructions}
  \label{tab:simd:insns}
\end{table}

% \newpage
\subsection{Integer-Unit Extensions: SIMD Instructions II}
\label{sec:integer-unit-extns:simd-instructions:2}

These  instructions are  vector  instructions which  reduce  a 64  bit
source register to a destination  using an associative operation.  See
%Table~\ref{tab:simd:2:insns}.
Tables~\ref{tab:simd:2:insns:1},\ref{tab:simd:2:insns:2},\ref{tab:simd:2:insns:3}.

% \begin{table}[p]
%   \centering
%   \begin{tabular}[p]{|p{.45\textwidth}|p{.45\textwidth}|}
%     \hline
% \multicolumn{2}{|l|}{\textbf{ADDDREDUCE8}} \\ 
%     \hline
%     \parbox{\linewidth}{op=2, op3[3:0]=0xd, op3[5:4]=0x2, contents[7:0] of rs2 specify a mask.\\
%     Instr[31:30] (op) = 0x2\\
%     Instr[29:25] (rd)    32-bit register.\\
%     Instr[24:19] (op3) = 101101\\
%     Instr[18:14] (rs1)   lowest bit assumed 0.\\
%     Instr[13]    (i)  = 0 (ignored)\\
%     Instr[12:10]   (zero)\\
%     Instr[9:7] = 1 for byte reduce
%     contents[7:0] of rs2 specify a mask.\\
%     Instr[6:5]  (zero)\\
%     Instr[4:0]   (rs2)   32-bit register is read.\\
% } & 
%  \parbox{\linewidth}{rd $\leftarrow$ (m7 ? rs1\_7 : 0x0) + (m6 ? rs1\_6
%     : 0x0) + (m5 ? rs1\_5:0)  \ldots + (m0 ? rs1\_0 : 0x0)
%     \vskip \parskip
%     \texttt{adddreduce8 \%rs1, \%rs2,  \%rd}
%     }\\
% \hline
% \hline
% %    
% \multicolumn{2}{|l|}{\textbf{ADDDREDUCE16}} \\ 
%  \hline 
%     \parbox{\linewidth}{op=2, op3[3:0]=0xd, op3[5:4]=0x2, contents[3:0] of rs2 specify a mask.\\
%     Instr[31:30] (op) = 0x2\\
%     Instr[29:25] (rd)    32-bit register.\\
%     Instr[24:19] (op3) = 101101\\
%     Instr[18:14] (rs1)   lowest bit assumed 0.\\
%     Instr[13]    (i)  = 0 (ignored)\\
%     Instr[12:10]   (zero)\\
%     Instr[9:7]   = 2 for half word reduce
%     contents[3:0] of rs2 specify a mask.\\
%     Instr[6:5]  (zero)\\
%     Instr[4:0]   (rs2)   32-bit register is read.\\
% } & 
%  \parbox{\linewidth}{rd $\leftarrow$ (m3 ? rs1\_hw\_3 : 0x0) + (m2 ? rs1\_hw\_2 : 0x0) 
%     + (m1 ? rs1\_hw\_1: 0x0) + (m0 ? rs1\_hw\_0 : 0x0)
%     \vskip \parskip
%     \texttt{adddreduce16 \%rs1, \%rs2,  \%rd}
%     }\\
% \hline
% \hline
% %    
% %    
% \multicolumn{2}{|l|}{\textbf{ORDREDUCE8} (Byte-Reduce OR)} \\ 
%  \hline 
%  \parbox{\linewidth}{op=2, op3[3:0]=0xe, op3[5:4]=0x2, contents[7:0]
%     of rs2 specify a mask.\\

%     Instr[31:30] (op) = 0x2\\
%     Instr[29:25] (rd)    rd is a 32-bit register.\\
%     Instr[24:19] (op3) = 101110\\
%     Instr[18:14] (rs1)   lowest bit assumed 0.\\
%     Instr[13]    (i)  = 0 (ignored)\\
%     Instr[12:10]   (zero)\\
%     Instr[9:7] = 1 for byte reduce
%     contents[7:0] of rs2 specify a mask.\\
%     Instr[6:5]  (zero)\\
%     Instr[4:0]   (rs2)   32-bit register is read.\\
% } & 
%  \parbox{\linewidth}{rd $\leftarrow$ (m7 ? rs1\_7 : 0x0) $\vert$ (m6 ?
%     rs1\_6 : 0x0) $\vert$ (m5 ? rs1\_5:0)  \ldots $\vert$ (m0 ? rs1\_0
%     : 0x0) 
%     \vskip \parskip
%     \texttt{ordreduce8 \%rs1, \%rs2,  \%rd}
%     }\\
% \hline
% \hline
% \multicolumn{2}{|l|}{\textbf{ORDREDUCE16} (Half Word-Reduce OR)} \\ 
%  \hline 
%  \parbox{\linewidth}{op=2, op3[3:0]=0xe, op3[5:4]=0x2, contents[3:0]
%     of rs2 specify a mask.\\
%     Instr[31:30] (op) = 0x2\\
%     Instr[29:25] (rd)    rd is a 32-bit register.\\
%     Instr[24:19] (op3) = 101110\\
%     Instr[18:14] (rs1)   lowest bit assumed 0.\\
%     Instr[13]    (i)  = 0 (ignored)\\
%     Instr[12:10]   (zero)\\
%     Instr[9:7]	= 2 for half-word reduce, contents[3:0] of rs2 specify a mask.\\
%     Instr[6:5]  (zero)\\
%     Instr[4:0]   (rs2)   32-bit register is read.\\
% } & 
%  \parbox{\linewidth}{rd $\leftarrow$ (m3 ? rs1\_3 : 0x0) $\vert$ (m2 ?
%     rs1\_2 : 0x0) $\vert$ (m1 ? rs1\_1 : 0x0) $\vert$ (m0 ? rs1\_0
%     : 0x0) 
%     \vskip \parskip
%     \texttt{ordreduce16 \%rs1, \%rs2,  \%rd}
%     }\\
% \hline
% \hline

% \multicolumn{2}{|l|}{\textbf{ANDDREDUCE8} (Byte-Reduce OR)} \\ 
%  \hline 
%  \parbox{\linewidth}{op=2, op3[3:0]=0xf, op3[5:4]=0x2, contents[7:0]
%     of rs2 specify a mask.\\

%     Instr[31:30] (op) = 0x2\\
%     Instr[29:25] (rd)    rd is a 32-bit register.\\
%     Instr[24:19] (op3) = 101111\\
%     Instr[18:14] (rs1)   lowest bit assumed 0.\\
%     Instr[13]    (i)  = 0 (ignored)\\
%     Instr[12:10]   (zero)\\
%     Instr[9:7] = 1 for byte reduce
%     contents[7:0] of rs2 specify a mask.\\
%     Instr[6:5]  (zero)\\
%     Instr[4:0]   (rs2)   32-bit register is read.\\
% } & 
%  \parbox{\linewidth}{rd $\leftarrow$ (m7 ? rs1\_7 : 0x0) $\vert$ (m6 ?
%     rs1\_6 : 0x0) $\vert$ (m5 ? rs1\_5:0)  \ldots $\vert$ (m0 ? rs1\_0
%     : 0x0) 
%     \vskip \parskip
%     \texttt{anddreduce8 \%rs1, \%rs2,  \%rd}
%     }\\
% \hline
% \hline
% \multicolumn{2}{|l|}{\textbf{ANDDREDUCE16} (Half Word-Reduce OR)} \\ 
%  \hline 
%  \parbox{\linewidth}{op=2, op3[3:0]=0xf, op3[5:4]=0x2, contents[3:0]
%     of rs2 specify a mask.\\
%     Instr[31:30] (op) = 0x2\\
%     Instr[29:25] (rd)    rd is a 32-bit register.\\
%     Instr[24:19] (op3) = 101111\\
%     Instr[18:14] (rs1)   lowest bit assumed 0.\\
%     Instr[13]    (i)  = 0 (ignored)\\
%     Instr[12:10]   (zero)\\
%     Instr[9:7]	= 2 for half-word reduce, contents[3:0] of rs2 specify a mask.\\
%     Instr[6:5]  (zero)\\
%     Instr[4:0]   (rs2)   32-bit register is read.\\
% } & 
%  \parbox{\linewidth}{rd $\leftarrow$ (m3 ? rs1\_3 : 0x0) $\vert$ (m2 ?
%     rs1\_2 : 0x0) $\vert$ (m1 ? rs1\_1 : 0x0) $\vert$ (m0 ? rs1\_0
%     : 0x0) 
%     \vskip \parskip
%     \texttt{anddreduce16 \%rs1, \%rs2,  \%rd}
%     }\\
% \hline
% \hline


% \multicolumn{2}{|l|}{\textbf{XORDREDUCE8} (Byte-Reduce OR)} \\ 
%  \hline 
%  \parbox{\linewidth}{op=2, op3[3:0]=0xe, op3[5:4]=0x3, contents[7:0]
%     of rs2 specify a mask.\\

%     Instr[31:30] (op) = 0x2\\
%     Instr[29:25] (rd)    rd is a 32-bit register.\\
%     Instr[24:19] (op3) = 111110\\
%     Instr[18:14] (rs1)   lowest bit assumed 0.\\
%     Instr[13]    (i)  = 0 (ignored)\\
%     Instr[12:10]   (zero)\\
%     Instr[9:7] = 1 for byte reduce
%     contents[7:0] of rs2 specify a mask.\\
%     Instr[6:5]  (zero)\\
%     Instr[4:0]   (rs2)   32-bit register is read.\\
% } & 
%  \parbox{\linewidth}{rd $\leftarrow$ (m7 ? rs1\_7 : 0x0) $\vert$ (m6 ?
%     rs1\_6 : 0x0) $\vert$ (m5 ? rs1\_5:0)  \ldots $\vert$ (m0 ? rs1\_0
%     : 0x0) 
%     \vskip \parskip
%     \texttt{xordreduce8 \%rs1, \%rs2,  \%rd}
%     }\\
% \hline
% \hline
% \multicolumn{2}{|l|}{\textbf{XORDREDUCE16} (Half Word-Reduce OR)} \\ 
%  \hline 
%  \parbox{\linewidth}{op=2, op3[3:0]=0xe, op3[5:4]=0x3, contents[3:0]
%     of rs2 specify a mask.\\
%     Instr[31:30] (op) = 0x2\\
%     Instr[29:25] (rd)    rd is a 32-bit register.\\
%     Instr[24:19] (op3) = 111110\\
%     Instr[18:14] (rs1)   lowest bit assumed 0.\\
%     Instr[13]    (i)  = 0 (ignored)\\
%     Instr[12:10]   (zero)\\
%     Instr[9:7]	= 2 for half-word reduce, contents[3:0] of rs2 specify a mask.\\
%     Instr[6:5]  (zero)\\
%     Instr[4:0]   (rs2)   32-bit register is read.\\
% } & 
%  \parbox{\linewidth}{rd $\leftarrow$ (m3 ? rs1\_3 : 0x0) $\vert$ (m2 ?
%     rs1\_2 : 0x0) $\vert$ (m1 ? rs1\_1 : 0x0) $\vert$ (m0 ? rs1\_0
%     : 0x0) 
%     \vskip \parskip
%     \texttt{xordreduce16 \%rs1, \%rs2,  \%rd}
%     }\\
% \hline
% \hline
    
% \multicolumn{2}{|l|}{\textbf{ZBYTEDPOS} (Positions-of-Zero-Bytes in D-Word)} \\ 
%  \hline 
%  \parbox{\linewidth}{op=2, op3[3:0]=0xf, op3[5:4]=0x3, contents[7:0]
%     of rs2/imm-value specify a mask.\\

%     Instr[31:30] (op) = 0x2\\
%     Instr[29:25] (rd)    rd is a 32 bit register\\
%     Instr[24:19] (op3) = 111111\\
%     Instr[18:14] (rs1)   lowest bit assumed 0.\\
%     Instr[13]    (i)  =  if 0, use rs2, else Instr[7:0]\\
%     Instr[12:5]  = 0  (ignored if i=0)\\
%     Instr[4:0]   (rs2, if i=0) 32-bit register is read.\\
% } & 
%  \parbox{\linewidth}{rd $\leftarrow$ [b7\_zero b6\_zero b5\_zero
%     b4\_zero \ldots b0\_zero] (if mask-bit is zero then b$\star$\_zero
%     is zero)
%     \vskip \parskip
%     \texttt{zbytedpos \%rs1, \%rs2/imm, \%rd}
%     }\\
% \hline
%   \end{tabular}
%   \caption{SIMD Instructions II}
%   \label{tab:simd:2:insns}
% \end{table}

\begin{table}[p]
  \centering
  \begin{tabular}[p]{|p{.45\textwidth}|p{.45\textwidth}|}
    \hline
\multicolumn{2}{|l|}{\textbf{ADDDREDUCE8}} \\ 
    \hline
    \parbox{\linewidth}{op=2, op3[3:0]=0xd, op3[5:4]=0x2, contents[7:0] of rs2 specify a mask.\\
    Instr[31:30] (op) = 0x2\\
    Instr[29:25] (rd)    32-bit register.\\
    Instr[24:19] (op3) = 101101\\
    Instr[18:14] (rs1)   lowest bit assumed 0.\\
    Instr[13]    (i)  = 0 (ignored)\\
    Instr[12:10]   (zero)\\
    Instr[9:7] = 1 for byte reduce
    contents[7:0] of rs2 specify a mask.\\
    Instr[6:5]  (zero)\\
    Instr[4:0]   (rs2)   32-bit register is read.\\
} & 
 \parbox{\linewidth}{rd $\leftarrow$ (m7 ? rs1\_7 : 0x0) + (m6 ? rs1\_6
    : 0x0) + (m5 ? rs1\_5:0)  \ldots + (m0 ? rs1\_0 : 0x0)
    \vskip \parskip
    \texttt{adddreduce8 \%rs1, \%rs2,  \%rd}
    }\\
\hline
\hline
%    
\multicolumn{2}{|l|}{\textbf{ADDDREDUCE16}} \\ 
 \hline 
    \parbox{\linewidth}{op=2, op3[3:0]=0xd, op3[5:4]=0x2, contents[3:0] of rs2 specify a mask.\\
    Instr[31:30] (op) = 0x2\\
    Instr[29:25] (rd)    32-bit register.\\
    Instr[24:19] (op3) = 101101\\
    Instr[18:14] (rs1)   lowest bit assumed 0.\\
    Instr[13]    (i)  = 0 (ignored)\\
    Instr[12:10]   (zero)\\
    Instr[9:7]   = 2 for half word reduce
    contents[3:0] of rs2 specify a mask.\\
    Instr[6:5]  (zero)\\
    Instr[4:0]   (rs2)   32-bit register is read.\\
} & 
 \parbox{\linewidth}{rd $\leftarrow$ (m3 ? rs1\_hw\_3 : 0x0) + (m2 ? rs1\_hw\_2 : 0x0) 
    + (m1 ? rs1\_hw\_1: 0x0) + (m0 ? rs1\_hw\_0 : 0x0)
    \vskip \parskip
    \texttt{adddreduce16 \%rs1, \%rs2,  \%rd}
    }\\
\hline
\hline
%    
%    
\multicolumn{2}{|l|}{\textbf{ORDREDUCE8} (Byte-Reduce OR)} \\ 
 \hline 
 \parbox{\linewidth}{op=2, op3[3:0]=0xe, op3[5:4]=0x2, contents[7:0]
    of rs2 specify a mask.\\

    Instr[31:30] (op) = 0x2\\
    Instr[29:25] (rd)    rd is a 32-bit register.\\
    Instr[24:19] (op3) = 101110\\
    Instr[18:14] (rs1)   lowest bit assumed 0.\\
    Instr[13]    (i)  = 0 (ignored)\\
    Instr[12:10]   (zero)\\
    Instr[9:7] = 1 for byte reduce
    contents[7:0] of rs2 specify a mask.\\
    Instr[6:5]  (zero)\\
    Instr[4:0]   (rs2)   32-bit register is read.\\
} & 
 \parbox{\linewidth}{rd $\leftarrow$ (m7 ? rs1\_7 : 0x0) $\vert$ (m6 ?
    rs1\_6 : 0x0) $\vert$ (m5 ? rs1\_5:0)  \ldots $\vert$ (m0 ? rs1\_0
    : 0x0) 
    \vskip \parskip
    \texttt{ordreduce8 \%rs1, \%rs2,  \%rd}
    }\\
\hline
\hline
  \end{tabular}
  \caption{SIMD Instructions II - Part 1 of 3}
  \label{tab:simd:2:insns:1}
\end{table}

\begin{table}[p]
  \centering
  \begin{tabular}[p]{|p{.45\textwidth}|p{.45\textwidth}|}
\hline
\multicolumn{2}{|l|}{\textbf{ORDREDUCE16} (Half Word-Reduce OR)} \\ 
 \hline 
 \parbox{\linewidth}{op=2, op3[3:0]=0xe, op3[5:4]=0x2, contents[3:0]
    of rs2 specify a mask.\\
    Instr[31:30] (op) = 0x2\\
    Instr[29:25] (rd)    rd is a 32-bit register.\\
    Instr[24:19] (op3) = 101110\\
    Instr[18:14] (rs1)   lowest bit assumed 0.\\
    Instr[13]    (i)  = 0 (ignored)\\
    Instr[12:10]   (zero)\\
    Instr[9:7]	= 2 for half-word reduce, contents[3:0] of rs2 specify a mask.\\
    Instr[6:5]  (zero)\\
    Instr[4:0]   (rs2)   32-bit register is read.\\
} & 
 \parbox{\linewidth}{rd $\leftarrow$ (m3 ? rs1\_3 : 0x0) $\vert$ (m2 ?
    rs1\_2 : 0x0) $\vert$ (m1 ? rs1\_1 : 0x0) $\vert$ (m0 ? rs1\_0
    : 0x0) 
    \vskip \parskip
    \texttt{ordreduce16 \%rs1, \%rs2,  \%rd}
    }\\
\hline
\hline

\multicolumn{2}{|l|}{\textbf{ANDDREDUCE8} (Byte-Reduce OR)} \\ 
 \hline 
 \parbox{\linewidth}{op=2, op3[3:0]=0xf, op3[5:4]=0x2, contents[7:0]
    of rs2 specify a mask.\\

    Instr[31:30] (op) = 0x2\\
    Instr[29:25] (rd)    rd is a 32-bit register.\\
    Instr[24:19] (op3) = 101111\\
    Instr[18:14] (rs1)   lowest bit assumed 0.\\
    Instr[13]    (i)  = 0 (ignored)\\
    Instr[12:10]   (zero)\\
    Instr[9:7] = 1 for byte reduce
    contents[7:0] of rs2 specify a mask.\\
    Instr[6:5]  (zero)\\
    Instr[4:0]   (rs2)   32-bit register is read.\\
} & 
 \parbox{\linewidth}{rd $\leftarrow$ (m7 ? rs1\_7 : 0x0) $\vert$ (m6 ?
    rs1\_6 : 0x0) $\vert$ (m5 ? rs1\_5:0)  \ldots $\vert$ (m0 ? rs1\_0
    : 0x0) 
    \vskip \parskip
    \texttt{anddreduce8 \%rs1, \%rs2,  \%rd}
    }\\
\hline
\hline
\multicolumn{2}{|l|}{\textbf{ANDDREDUCE16} (Half Word-Reduce OR)} \\ 
 \hline 
 \parbox{\linewidth}{op=2, op3[3:0]=0xf, op3[5:4]=0x2, contents[3:0]
    of rs2 specify a mask.\\
    Instr[31:30] (op) = 0x2\\
    Instr[29:25] (rd)    rd is a 32-bit register.\\
    Instr[24:19] (op3) = 101111\\
    Instr[18:14] (rs1)   lowest bit assumed 0.\\
    Instr[13]    (i)  = 0 (ignored)\\
    Instr[12:10]   (zero)\\
    Instr[9:7]	= 2 for half-word reduce, contents[3:0] of rs2 specify a mask.\\
    Instr[6:5]  (zero)\\
    Instr[4:0]   (rs2)   32-bit register is read.\\
} & 
 \parbox{\linewidth}{rd $\leftarrow$ (m3 ? rs1\_3 : 0x0) $\vert$ (m2 ?
    rs1\_2 : 0x0) $\vert$ (m1 ? rs1\_1 : 0x0) $\vert$ (m0 ? rs1\_0
    : 0x0) 
    \vskip \parskip
    \texttt{anddreduce16 \%rs1, \%rs2,  \%rd}
    }\\
\hline
  \end{tabular}
  \caption{SIMD Instructions II -- Part 2 of 3}
  \label{tab:simd:2:insns:2}
\end{table}

\begin{table}[p]
  \centering
  \begin{tabular}[p]{|p{.45\textwidth}|p{.45\textwidth}|}
    \hline
\multicolumn{2}{|l|}{\textbf{XORDREDUCE8} (Byte-Reduce OR)} \\ 
 \hline 
 \parbox{\linewidth}{op=2, op3[3:0]=0xe, op3[5:4]=0x3, contents[7:0]
    of rs2 specify a mask.\\

    Instr[31:30] (op) = 0x2\\
    Instr[29:25] (rd)    rd is a 32-bit register.\\
    Instr[24:19] (op3) = 111110\\
    Instr[18:14] (rs1)   lowest bit assumed 0.\\
    Instr[13]    (i)  = 0 (ignored)\\
    Instr[12:10]   (zero)\\
    Instr[9:7] = 1 for byte reduce
    contents[7:0] of rs2 specify a mask.\\
    Instr[6:5]  (zero)\\
    Instr[4:0]   (rs2)   32-bit register is read.\\
} & 
 \parbox{\linewidth}{rd $\leftarrow$ (m7 ? rs1\_7 : 0x0) $\vert$ (m6 ?
    rs1\_6 : 0x0) $\vert$ (m5 ? rs1\_5:0)  \ldots $\vert$ (m0 ? rs1\_0
    : 0x0) 
    \vskip \parskip
    \texttt{xordreduce8 \%rs1, \%rs2,  \%rd}
    }\\
\hline
\hline
\multicolumn{2}{|l|}{\textbf{XORDREDUCE16} (Half Word-Reduce OR)} \\ 
 \hline 
 \parbox{\linewidth}{op=2, op3[3:0]=0xe, op3[5:4]=0x3, contents[3:0]
    of rs2 specify a mask.\\
    Instr[31:30] (op) = 0x2\\
    Instr[29:25] (rd)    rd is a 32-bit register.\\
    Instr[24:19] (op3) = 111110\\
    Instr[18:14] (rs1)   lowest bit assumed 0.\\
    Instr[13]    (i)  = 0 (ignored)\\
    Instr[12:10]   (zero)\\
    Instr[9:7]	= 2 for half-word reduce, contents[3:0] of rs2 specify a mask.\\
    Instr[6:5]  (zero)\\
    Instr[4:0]   (rs2)   32-bit register is read.\\
} & 
 \parbox{\linewidth}{rd $\leftarrow$ (m3 ? rs1\_3 : 0x0) $\vert$ (m2 ?
    rs1\_2 : 0x0) $\vert$ (m1 ? rs1\_1 : 0x0) $\vert$ (m0 ? rs1\_0
    : 0x0) 
    \vskip \parskip
    \texttt{xordreduce16 \%rs1, \%rs2,  \%rd}
    }\\
\hline
\hline
    
\multicolumn{2}{|l|}{\textbf{ZBYTEDPOS} (Positions-of-Zero-Bytes in D-Word)} \\ 
 \hline 
 \parbox{\linewidth}{op=2, op3[3:0]=0xf, op3[5:4]=0x3, contents[7:0]
    of rs2/imm-value specify a mask.\\

    Instr[31:30] (op) = 0x2\\
    Instr[29:25] (rd)    rd is a 32 bit register\\
    Instr[24:19] (op3) = 111111\\
    Instr[18:14] (rs1)   lowest bit assumed 0.\\
    Instr[13]    (i)  =  if 0, use rs2, else Instr[7:0]\\
    Instr[12:5]  = 0  (ignored if i=0)\\
    Instr[4:0]   (rs2, if i=0) 32-bit register is read.\\
} & 
 \parbox{\linewidth}{rd $\leftarrow$ [b7\_zero b6\_zero b5\_zero
    b4\_zero \ldots b0\_zero] (if mask-bit is zero then b$\star$\_zero
    is zero)
    \vskip \parskip
    \texttt{zbytedpos \%rs1, \%rs2/imm, \%rd}
    }\\
\hline
  \end{tabular}
  \caption{SIMD Instructions II -- Part 3 of 3}
  \label{tab:simd:2:insns:3}
\end{table}




% \newpage
\subsection{Vector Floating Point Instructions}
\label{sec:vector-floating-point-instructions}

These are vector  float operations which work on  two single precision
operand  pairs   to  produce   two  single  precision   results.   See
Table~\ref{tab:simd:float:ops}.

\begin{table}[p]
  \centering
  \begin{tabular}[p]{|l|l|l|}
    \hline
    \textbf{VFADD32}
    & {op=2, op3=0x34, opf=0x142}
    & \texttt{vfadd32 \%f0, \%f2, \%f4} \\
    \hline
    \textbf{VFADD16}
    & {op=2, op3=0x34, opf=0x143}
    & \texttt{vfadd16 \%f0, \%f2, \%f4} \\
    \hline
    \textbf{VFSUB32}
    & {op=2, op3=0x34, opf=0x144}
    & \texttt{vfadd32 \%f0, \%f2, \%f4} \\
    \hline
    \textbf{VFSUB16}
    & {op=2, op3=0x34, opf=0x145}
    & \texttt{vfadd16 \%f0, \%f2, \%f4} \\
    \hline
    \textbf{VFMUL32}
    & {op=2, op3=0x34, opf=0x146}
    & \texttt{vfadd32 \%f0, \%f2, \%f4} \\
    \hline
    \textbf{VFMUL16}
    & {op=2, op3=0x34, opf=0x147}
    & \texttt{vfadd16 \%f0, \%f2, \%f4} \\
    \hline
    \textbf{VFI16TOH}
    & {op=2, op3=0x34, opf=0x148}
    & \texttt{vfi16toh \%f0, \%f2} \\
    \hline
    \textbf{VFHTOI16}
    & {op=2, op3=0x34, opf=0x149}
    & \texttt{vfhtoi16 \%f0, \%f2} \\
    \hline      
  \end{tabular}
  \caption[SIMD Floating Point Operations]{SIMD Floating Point
    Operations.  NaN propagated, but no traps. For each of these,
    rs1,rs2,rd are considered even numbers pointing to.
  }
  \label{tab:simd:float:ops}
\end{table}
% \begin{table}[p]
%   \centering
%   \begin{tabular}[p]{|l|l|}
%     \hline
%     \textbf{VFADD} & {op=2, op3=0x34, opf=0x142} \\
%     \hline
%     \textbf{VFSUB} & {op=2, op3=0x34, opf=0x146} \\
%     \hline
%     \textbf{VFMUL} & {op=2, op3=0x34, opf=0x14a} \\
%     \hline
%     \textbf{VFDIV} & {op=2, op3=0x34, opf=0x14e} \\
%     \hline
%     \textbf{VFSQRT} & {op=2, op3=0x34, opf=0x12a} \\
%     \hline      
%   \end{tabular}
%   \caption[SIMD Floating Point Operations]{SIMD Floating Point
%     Operations.  NaN propagated, but no traps. For each of these,
%     rs1,rs2,rd are considered even numbers pointing to.
%   }
%   \label{tab:simd:float:ops}
% \end{table}


% \newpage

\subsection{FP Reduce}
\label{sec:fp reduce}

This instruction adds the four half-precision numbers in the 64-bit FP
register pair rs1, and produce a result into the 32-bit FP register.
See Table~\ref{tab:fp:reduce:ops}.

\begin{table}[p]
  \centering
  \begin{tabular}[p]{|l|l|l|}
    \hline
    \textbf{FADDREDUCE16}
    & {op=2, op3=0x34, opf=0x150}
    & \texttt{vfadd32 \%f0, \%f2, \%f4} \\
    \hline
  \end{tabular}
  \caption[SIMD Floating Point Reduce Operations]{SIMD Floating Point
    Reduce Operations.
  }
  \label{tab:fp:reduce:ops}
\end{table}

\subsection{Half Precision Conversion Operations}
\label{sec:half precision conversion ops}

These  instructions  allow   conversion  between  IEEE  half-precision
numbers and IEEE single/double precision numbers and integers.
See Table~\ref{tab:half:precision:conversion:ops}.

\textbf{Note:} the double-to-half  and half-to-double, int-to-half and
half-to-int  instructions are  not provided.   This is  because, these
transformations  are likely  to be  rarer.  Also,  the \texttt{FDTOS},
\texttt{FDTOI},  \texttt{FITOS}, \texttt{FITOD}  instructions together
with  the   added  \texttt{FSTOH},  \texttt{FHTOS}   instructions  are
sufficient (at a minor cost).

\begin{table}[p]
  \centering
  \begin{tabular}[p]{|l|l|l|}
    \hline
    \textbf{FSTOH}
    & {op=2, op3=0x34, opf=0x151}
    & \texttt{fstoh \%f1, \%f2} \\
    \hline
    \textbf{FHTOS}
    & {op=2, op3=0x34, opf=0x152}
    & \texttt{fhtos \%f1, \%f2} \\
    \hline
  \end{tabular}
  \caption[SIMD Floating Point Reduce Operations]{SIMD Floating Point
    Reduce Operations.
  }
  \label{tab:half:precision:conversion:ops}
\end{table}

\subsection{CSWAP instructions}
\label{sec:cswap-instructions}

The Sparc-V8 ISA does not include a compare-and-swap (CAS) instruction
which is very  useful in achieving consensus  among distributed agents
when the number of agents is $>$ 2.

We   introduce   a   CSWAP   instruction   in   two   flavours.    See
Table~\ref{tab:cswap:insns}.

The semantics of the instruction (the entire sequence is atomic)
\begin{verbatim}
TMPVAL = mem[rs1]  (load double, lock system bus)
if <rs2-pair/immediate> == TMPVAL 
        (store double, unlock) mem[rs1] = <rd-pair>  
        <rd-pair>  = TMPVAL
else
        (store double, unlock) mem[rs1] = TMPVAL
\end{verbatim}
The write under  else is redundant but is required  in order to unlock
the bus.

Similar to SWAP, 
\begin{itemize}[noitemsep]
\item  \verb|mem[rs1]| is  left either  with  its value  prior to  the
  instruction or with the value in rd-pair.
\item  \verb|<rd-pair>| is  left either  with its  value prior  to the
  instruction or with the value in mem[rs1].
\end{itemize}
The processor can check rd-pair after execution to confirm if the swap
succeeded.


\begin{table}[p]
  \centering
  \begin{tabular}[p]{|p{.45\textwidth}|p{.45\textwidth}|}
    \hline
\multicolumn{2}{|l|}{\textbf{CSWAP} (effective address in registers
    rs1 and rs2)} \\ 
 \hline 
 \parbox{\linewidth}{op=3, op3=10 1111, i=0.\\

    Instr[31:30] (op) = 0x3\\
    Instr[29:25] (rd)    lowest bit assumed 0.\\
    Instr[24:19] (op3) = 101111\\
    Instr[18:14] (rs1)   lowest bit assumed 0.\\
    Instr[13]    (i)  = 0 (registers based effective address)\\
    Instr[12:5]  (asi) = Address Space Identifier (See: Appendix G of V8)\\
    Instr[4:0]   (rs2)   32-bit register is read.\\
} & 
 \parbox{\linewidth}{\texttt{cswap \%rs1, \%rs2, \%rd} with asi specified.}\\
\hline
    \hline
\multicolumn{2}{|l|}{\textbf{CSWAP} (immediate effective address)} \\ 
 \hline 
 \parbox{\linewidth}{op=3, op3=10 1111, i=1.\\

    Instr[31:30] (op) = 0x3\\
    Instr[29:25] (rd)    lowest bit assumed 0.\\
    Instr[24:19] (op3) = 101111\\
    Instr[18:14] (rs1)   lowest bit assumed 0.\\
    Instr[13]    (i)  = 1 (immediate effective address)\\
    Instr[12:0]  (simm13) 13-bit immediate address.\\
} & 
 \parbox{\linewidth}{\texttt{cswap \%rs1, imm, \%rd}.}\\
\hline
    \hline
\multicolumn{2}{|l|}{\textbf{CSWAPA} (effective address in registers
    rs1 and rs2)} \\ 
 \hline 
 \parbox{\linewidth}{op=3, op3=10 1111, i=0.\\
    Instr[31:30] (op) = 0x3\\
    Instr[29:25] (rd)    lowest bit assumed 0.\\
    Instr[24:19] (op3) = 111111\\
    Instr[18:14] (rs1)   lowest bit assumed 0.\\
    Instr[13]    (i)  = 0 (registers based effective address)\\
    Instr[12:5]  (asi) = Address Space Identifier (See: Appendix G of V8)\\
    Instr[4:0]   (rs2)   32-bit register is read.\\
} & 
 \parbox{\linewidth}{\texttt{cswapa \%rs1, \%rs2, \%rd} with asi specified.}\\
\hline
    \hline
\multicolumn{2}{|l|}{\textbf{CSWAPA} (immediate effective address)} \\ 
 \hline 
 \parbox{\linewidth}{op=3, op3=10 1111, i=1.\\
    Instr[31:30] (op) = 0x3\\
    Instr[29:25] (rd)    lowest bit assumed 0.\\
    Instr[24:19] (op3) = 111111\\
    Instr[18:14] (rs1)   lowest bit assumed 0.\\
    Instr[13]    (i)  = 1 (immediate effective address)\\
    Instr[12:0]  (simm13) 13-bit immediate address.\\
} & 
 \parbox{\linewidth}{\texttt{cswapa \%rs1, imm, \%rd}.}\\
\hline
  \end{tabular}
  \caption{CSWAP Instructions}
  \label{tab:cswap:insns}
\end{table}


\newpage
\chapter{AJIT Support for the GNU Binutils Toolchain}
\label{chap:amv:work}

\section{Towards a GNU Binutils Toolchain}
\label{sec:binutils:support}

This section describes the details  of adding the AJIT instructions to
SPARC v8 part of GNU Binutils 2.22.  We use the SPARC v8 manual to get
the details of  the sparc instruction.  It's bit  pattern is described
\emph{again}, and  the new  bit pattern  required for  AJIT is  set up
alongside.  Bit layouts to determine  the ``match'' etc.  of the sparc
port  are  also  laid  out.    The  SPARC  manual  also  contains  the
``suggested asm syntax''  that we adapt for the  new AJIT instruction.
The sections below follow the sections in chapter~\ref{sec:isa:extns}.
For each  instruction, we  need to  define its  bitfields in  terms of
macros  in  \texttt{\$BINUTILSHOME/include/opcode/sparc.h} and  define
the opcodes table in \texttt{\$BINUTILSHOME/opcodes/sparc-opc.c}.

The AJIT  instructions are  variations of  the corresponding  SPARC V8
instructions.  Please refer to the SPARC V8 manual for details of such
corresponding SPARC instructions. For  example, the \texttt{ADD} insn,
pg. 108 (pg.  130 in PDF  sequence) of the manual.  Other instructions
can be similarly found, and will not be mentioned.

\subsection{General Approach for Developing the Assembler}
\label{sec:general:approach}

The GNU Binutils package is a  collection of low level tools that help
dealing  with binary  files  like program  object  files, object  file
libraries and  program executables  files.  Written  mostly in  C, the
code structure is  typically as used by C programmers  in general, and
GNU community in particular.  Table~\ref{tab:binutils:desc} is a brief
note about  the main contents  of each  top level subdirectory  of the
binutils package.

\begin{table}[h]
  \centering
  \begin{tabular}[h]{|p{.1\linewidth}|p{.65\linewidth}|}
    \hline
    \textbf{Name} & \textbf{Brief description} \\
    \hline
    bfd         & Support for the GNU BFD library. \\
    binutils    & Some tools that do not have their own directory. \\
    config      & Configuration. \\                        
    cpu         & CPU descriptions of some CPUs (See: cpu-gen) \\
    elfcpp      & C++ library for reading and writing ELF information. \\
    etc         & Some miscellaneous files. \\
    gas         & The GNU Assembler.  Parts of the assembler are in other directories. \\
    gold        & The Gold Linker.  This is a new linker. \\
    gprof       & The GNU Profiler. \\
    include     & Most includes are here. \\
    intl        & Internationalisation files. \\
    ld          & The Standard GNU Linker. \\
    libiberty   & Library of subroutines used by various GNU programs. \\
    opcodes     & Per CPU opcodes generator indexed by the mnemonic. \\
    texinfo     & GNU TeXInfo based documentation support files. \\
    \hline
  \end{tabular}
  \caption[Binutils  Brief   Description]{Brief  description   of  the
    directories in the GNU Binutils package.}
  \label{tab:binutils:desc}
\end{table}

We will use the SPARC implementation as a template for developing the
AJIT support within the tools.  The implementation is divided into two
main stages:
\begin{enumerate}
\item \label{stage:1}  \textbf{Stage 1}:  Add AJIT instructions  as an
  ``extension'' to the  SPARC V8 tools.  This implies  adding the AJIT
  opcodes  to  the SPARC  opcodes.   The  tools  that change  are  the
  assembler  \texttt{as}  and  the  disassembler used  by  tools  like
  \texttt{objdump}.  Tools  like the  library archiver  \texttt{ar} do
  \textbf{not}  work   on  individual  opcodes.   They   work  on  the
  structure,  i.e. the  layout,  of the  executable  or object  files.
  Hence they are not affected.

  As of December 2020, this work is completed.
  
  \textbf{This document records the implementation of Stage 1.}

\item  \label{stage:2}  \textbf{Stage  2}: This  implements  the  AJIT
  support  as a  separate individual  processor supported  by the  GNU
  Binutils package.

  As of December 2020, this is ongoing work.
\end{enumerate}

\subsubsection{Opcode Format of SPARC V8}
\label{sec:sparc:v8:opcode:format}

\begin{figure}[h]
  \centering
  \epsfxsize=.8\linewidth
  \epsffile{../figs/sparc-v8-insn-32-bit-layout.eps}
  \caption[Format 3 SPARC V8 Layout.]{The SPARC V8 format 3
    instruction layout.}
  \label{fig:f3:layout}
\end{figure}

The C  preprocessor (CPP) macros  for the SPARC \emph{family}  of CPUs
are in  \texttt{include/opcode/sparc.h}, and are reproduced  below for
convenience.  Only the ones for the SPARC V8 format 3 instructions are
listed below.
{\small
\begin{verbatim}
#define OP(x)           ((unsigned) ((x) & 0x3) << 30) /* Op field of all insns.  */
#define OPF(x)          (((x) & 0x1ff) << 5)           /* Opf field of float insns.  */
#define F3F(x, y, z)    (OP (x) | OP3 (y) | OPF (z))   /* Format3 float insns.  */
#define F3I(x)          (((x) & 0x1) << 13)            /* Immediate field of format 3 insns.  */
#define F3(x, y, z)     (OP (x) | OP3(y) | F3I(z))     /* Format3 insns.  */
#define ASI(x)          (((x) & 0xff) << 5)            /* Asi field of format3 insns.  */
#define RS2(x)          ((x) & 0x1f)                   /* RS2 field.  */
#define SIMM13(x)       ((x) & 0x1fff)                 /* Simm13 field.  */
#define RD(x)           (((x) & 0x1f) << 25)           /* RD Destination register field.  */
#define RS1(x)          (((x) & 0x1f) << 14)           /* RS1 field.  */
\end{verbatim}
}

As  an  illustration of  the  operation  consider the  \texttt{OPF(x)}
macro.   This expands  to: \verb|(((x)  &  0x1ff) <<  5)|.  The  inner
expression, \verb|(x)  & 0x1FF| isolates  the 9 bits required  for the
\texttt{OPF} field  using the constant \texttt{0x1FF}.   The result is
left shifted by  5 to position these  9 bits at the  desired offset in
the 32 bit instruction.  Similarly, the \verb|F3I(x)| macro expands to
\verb|(((x)  & 0x1)  << 13)|.   This  sets the  ``\texttt{i}'' bit  at
offset 13 to the value of ``\texttt{x}''.  If \texttt{x = 0}, then the
``\texttt{i}'' bit is 0 giving the \emph{non-immediate} variant of the
instruction.  When  \texttt{x = 1},  then the ``\texttt{i}'' bit  is 1
giving the \emph{immediate} variant of the instruction.

Finally, the \verb|F3(x,  y, z)| macro uses such  macros whose results
are bitwise  OR'd to obtain  the set of  bits in the  instruction that
\emph{uniquely} identify  the instruction.  The  implementation refers
to these bits as the  ``\textbf{match}'' bits.  It also constructs the
bit inverses of each field and  then bitwise OR's the result to obtain
the so called ``\textbf{lose}'' bits.  In the following we discuss the
construction of  the match  bits only.  However,  we do  implement the
match as well as the lose bits.

\subsubsection{Illustrating SPARC V8 Opcode Implementation}
\label{sec:sparc:v8:opcode:implementation}

The  \texttt{include/opcode/sparc.h} also  defines  the  layout of  an
entry  that describes  one instruction.   It is  reproduced below  for
convenience.
{\small 
\begin{verbatim}
/* Structure of an opcode table entry as used in GNU Binutils 2.22 */

typedef struct sparc_opcode
{
  const char *name;
  unsigned long match;  /* Match bits that must be set. */
  unsigned long lose;   /* Lose bits. */
  const char *args;
  unsigned int flags;
  short architecture;   /* Bitmask of sparc_opcode_arch_val's. */
} sparc_opcode;
\end{verbatim}
}

All  the instructions  are  listed as  a  table, \texttt{const  struct
  sparc\_opcode      sparc\_opcodes[]},      in      \texttt{opcodes}-
\texttt{/sparc-opc.c}.   Each table  entry  is an  instance  of the  C
structure  described  above.   The   entry  for  the  ``\texttt{add}''
instruction for SPARC V8 looks like: {\small
\begin{verbatim}
{ "add",        F3(2, 0x00, 0), F3(~2, ~0x00, ~0)|ASI(~0),      "1,2,d", 0, v6 },
{ "add",        F3(2, 0x00, 1), F3(~2, ~0x00, ~1),              "1,i,d", 0, v6 },
\end{verbatim}
}

Note that  the third argument  of the \texttt{F3(...)} sets  or resets
the  ``\texttt{i}''  bit  (at  offset   13).   When  1,  we  have  the
\emph{immediate} value variant of  the \texttt{add} instruction.  When
0, we have  the variant that has its arguments  in the registers, with
the \texttt{asi} field specifying the alternate space index.

\subsubsection{Opcode Format of AJIT}
\label{sec:ajit:opcode:format}

The AJIT architecture  augments the SPARC V8 instruction  set with its
own  instructions.   These  are  format  3  style  instructions  which
primarily focus on arithmetic operations.

{\small
\begin{verbatim}
/* AJIT Additions */
/* Bit setters */
#define OP_AJIT_BIT_5(x)          (((x) & 0x1) << 5)      /* Set the bit 5 (6th bit) for AJIT */
#define OP_AJIT_BIT_5_AND_6(x)    (((x) & 0x3) << 5)      /* Set the bits 5 and 6 for AJIT */
#define OP_AJIT_BIT_7_THRU_9(x)   (((x) & 0x7) << 7)      /* Set bits 7 through 9 for AJIT */

/* Bit setters for full instructions */
#define OP_AJIT_BITS_30_TO_31(x)    (((x) & 0x03) << 30)  /* op, match */
#define OP_AJIT_BITS_25_TO_29(x)    (((x) & 0x1F) << 25)  /* rd */
#define OP_AJIT_BITS_19_TO_24(x)    (((x) & 0x3F) << 19)  /* op3, match */
#define OP_AJIT_BITS_14_TO_18(x)    (((x) & 0x1F) << 14)  /* rs1 */
#define OP_AJIT_BITS_13_TO_13(x)    (((x) & 0x1) << 13)   /* i */
#define OP_AJIT_BITS_05_TO_12(x)    (((x) & 0xFF) << 05)  /* for UNUSED, set to zero */
#define OP_AJIT_BITS_00_TO_04(x)    (((x) & 0x1F) << 00)  /* rs2 */
#define OP_AJIT_BITS_05_TO_13(x)    (((x) & 0x1FF) << 05) /* opf */
#define OP_AJIT_BITS_08_TO_12(x)    (((x) & 0x1F) << 8)
#define OP_AJIT_BITS_00_TO_07(x)    (((x) & 0xFF) << 00)

#define SET13   OP_AJIT_BITS_13_TO_13(1)

/* Most arithmetic instructions */
#define F4(x, y, z, b)        (F3(x, y, z) | OP_AJIT_BIT_5(b))            /* Format 3 with bit 5 */
#define F5(x, y, z, b)        (F3(x, y, z) | OP_AJIT_BIT_5_AND_6 (b))     /* Format 3 with bits 5 and 6 */
#define F6(x, y, z, b, a)     (F5 (x, y, z, b) | OP_AJIT_BIT_7_THRU_9(a)) /* Format 3 with bits 5-6 and 7-9 */

/* For SIMD II instructions */
#define F7(a, b, c, d)            (OP(a) | OP3(b) | F3I(c))
#define F10(a, b, c, d)           (OP_AJIT_BITS_30_TO_31(a) | \
                                   OP_AJIT_BITS_19_TO_24(b) | \
                                   OP_AJIT_BITS_13_TO_13(c) | \
                                   OP_AJIT_BITS_08_TO_12(0))

/* For SIMD Floating point ops */
#define F8(a, b, c)               (OP_AJIT_BITS_30_TO_31(a) | \
                                   OP_AJIT_BITS_19_TO_24(b) | \
                                   OP_AJIT_BITS_05_TO_13(c))
/* For CSWAP non immediate ops */
#define F9(a, b, c)               (OP_AJIT_BITS_30_TO_31(a) | \
                                   OP_AJIT_BITS_19_TO_24(b) | \
                                   OP_AJIT_BITS_13_TO_13(c))
/* For CSWAP immediate ops */
#define F9d(a, b, c)              (OP_AJIT_BITS_30_TO_31(a) | \
                                   OP_AJIT_BITS_19_TO_24(b) | \
                                   OP_AJIT_BITS_13_TO_13(1) | \
                                   SIMM13(c))

/* End of AJIT specific additions */
\end{verbatim}
}

\subsubsection{Illustrating AJIT Opcode Implementation}
\label{sec:ajit:opcode:implementation}

For AJIT we use the layout  of an entry that describes one instruction
that is identical to SPARC V8.

The entry for  the ``\texttt{addd}'' instruction for  AJIT looks like:
{\small
\begin{verbatim}
{ "addd",       F4(2, 0x00, 0, 1), F4(~2, ~0x00, ~0, ~1),       "1,2,d", 0, v8 }, /* AJIT */
\end{verbatim}
}

The ``\texttt{addd}'' AJIT instruction has no immediate value variant.
It  uses  the  \texttt{F4(...)}   macro  to set  the  match  bits;  in
particular  it  sets  the  bit  at  offset  5.   Hence  the  macro  is
instantiated to describe the  \texttt{addd} instruction as: \texttt{F4
  (2,   0x00,  0,   1)}.   It   is   implemented  in   terms  of   the
\texttt{F3(...)}   macro  whose  result   is  bitwise  OR'd  with  the
\texttt{OP\_AJIT\_BIT\_5(x)} macro  that either sets (\texttt{x  = 1})
the bit at the 5$^{\mathrm{th}}$ offset (i.e. bit number 6), or resets
(\texttt{x = 0}) it.

\subsubsection{AJIT Implementation Notes}
\label{sec:ajit:implementation:notes}

The AJIT  specific additions to  the SPARC V8 implementation,  and the
illustrations of the previous section  show the basic approach used to
obtain the AJIT implementation.  In this section we note a few general
aspects  that  describe  the   AJIT  implementation.   The  subsequent
sections of this  chapter only record the details.   The complete AJIT
instructions as  implemented in Stage 1  (See: page \pageref{stage:1})
are  below.  Note  that  as required  by  the assembler  architecture,
variants  of an  instruction need  to be  grouped together.   So these
instructions  are collected  together  from  their implementations  in
\texttt{opcodes/sparc-opc.c}.

\hrulefill
% {\small
% \begin{verbatim}
% {"slld",         F5(2, 0x25, 1, 0x1),      F5(~2, ~0x25, ~1, ~0x1),       "1,Y,d",    0,       v6}, /* AJIT */
% {"slld",         F5(2, 0x25, 0, 0x1),      F5(~2, ~0x25, ~0, ~0x1),       "1,2,d",    0,       v6}, /* AJIT */
% {"srad",         F5(2, 0x27, 1, 0x1),      F5(~2, ~0x27, ~1, ~0x1),       "1,Y,d",    0,       v6}, /* AJIT */
% {"srad",         F5(2, 0x27, 0, 0x1),      F5(~2, ~0x27, ~0, ~0x1),       "1,2,d",    0,       v6}, /* AJIT */
% {"srld",         F5(2, 0x26, 1, 0x1),      F5(~2, ~0x26, ~1, ~0x1),       "1,Y,d",    0,       v6}, /* AJIT */
% {"srld",         F5(2, 0x26, 0, 0x1),      F5(~2, ~0x26, ~0, ~0x1),       "1,2,d",    0,       v6}, /* AJIT */
% {"ord",          F4(2, 0x02, 0, 1),        F4(~2, ~0x02, ~0, ~1),         "1,2,d",    0,       v6}, /* AJIT */
% {"ordcc",        F4(2, 0x12, 0, 1),        F4(~2, ~0x12, ~0, ~1),         "1,2,d",    0,       v6}, /* AJIT */
% {"ordn",         F4(2, 0x06, 0, 1),        F4(~2, ~0x06, ~0, ~1),         "1,2,d",    0,       v6}, /* AJIT */
% {"ordncc",       F4(2, 0x16, 0, 1),        F4(~2, ~0x16, ~0, ~1),         "1,2,d",    0,       v6}, /* AJIT */
% {"anddn",        F4(2, 0x05, 0, 1),        F4(~2, ~0x05, ~0, ~1),         "1,2,d",    0,       v6}, /* AJIT */
% {"anddncc",      F4(2, 0x15, 0, 1),        F4(~2, ~0x15, ~0, ~1),         "1,2,d",    0,       v6}, /* AJIT */
% {"subd",         F4(2, 0x04, 0, 1),        F4(~2, ~0x04, ~0, ~1),         "1,2,d",    0,       v8}, /* AJIT */
% {"subdcc",       F4(2, 0x14, 0, 1),        F4(~2, ~0x14, ~0, ~1),         "1,2,d",    0,       v8}, /* AJIT */
% {"vsubd8",       F6(2, 0x04, 0, 2, 1),     F6(~2, ~0x04, ~0, ~2, ~1),     "1,2,d",    0,       v8}, /* AJIT */
% {"vsubd16",      F6(2, 0x04, 0, 2, 2),     F6(~2, ~0x04, ~0, ~2, ~3),     "1,2,d",    0,       v8}, /* AJIT */
% {"vsubd32",      F6(2, 0x04, 0, 2, 4),     F6(~2, ~0x04, ~0, ~2, ~4),     "1,2,d",    0,       v8}, /* AJIT */
% {"andd",         F4(2, 0x01, 0, 1),        F4(~2, ~0x01, ~0, ~1),         "1,2,d",    0,       v6}, /* AJIT */
% {"anddcc",       F4(2, 0x11, 0, 1),        F4(~2, ~0x11, ~0, ~1),         "1,2,d",    0,       v6}, /* AJIT */
% {"addd",         F4(2, 0x00, 0, 1),        F4(~2, ~0x00, ~0, ~1),         "1,2,d",    0,       v8}, /* AJIT */
% {"adddcc",       F4(2, 0x10, 0,1),         F4(~2, ~0x10, ~0, ~1),         "1,2,d",    0,       v8}, /* AJIT */
% {"vaddd8",       F6(2, 0x00, 0, 2, 1),     F6(~2, ~0x00, ~0, ~2, ~1),     "1,2,d",    0,       v8}, /* AJIT */
% {"vaddd16",      F6(2, 0x00, 0, 2, 2),     F6(~2, ~0x00, ~0, ~2, ~2),     "1,2,d",    0,       v8}, /* AJIT */
% {"vaddd32",      F6(2, 0x00, 0, 2, 4),     F6(~2, ~0x00, ~0, ~2, ~4),     "1,2,d",    0,       v8}, /* AJIT */
% {"smuld",        F4(2, 0x0b, 0, 1),        F4(~2, ~0x0b, ~0, ~1),         "1,2,d",    F_MUL32, v8}, /* AJIT */
% {"smuldcc",      F4(2, 0x1b, 0, 1),        F4(~2, ~0x1b, ~0, ~1),         "1,2,d",    F_MUL32, v8}, /* AJIT */
% {"umuld",        F4(2, 0x0a, 0, 1),        F4(~2, ~0x0a, ~0, ~1),         "1,2,d",    F_MUL32, v8}, /* AJIT */
% {"umuldcc",      F4(2, 0x1a, 0, 1),        F4(~2, ~0x1a, ~0, ~1),         "1,2,d",    F_MUL32, v8}, /* AJIT */
% {"vumuld8",      F6(2, 0x0a, 0, 2, 1),     F6(~2, ~0x0a, ~0, ~2, ~1),     "1,2,d",    0,       v8}, /* AJIT */
% {"vumuld16",     F6(2, 0x0a, 0, 2, 2),     F6(~2, ~0x0a, ~0, ~2, ~3),     "1,2,d",    0,       v8}, /* AJIT */
% {"vumuld32",     F6(2, 0x0a, 0, 2, 4),     F6(~2, ~0x0a, ~0, ~2, ~4),     "1,2,d",    0,       v8}, /* AJIT */
% {"vsmuld8",      F6(2, 0x1a, 0, 2, 1),     F6(~2, ~0x1a, ~0, ~2, ~1),     "1,2,d",    0,       v8}, /* AJIT */
% {"vsmuld16",     F6(2, 0x1a, 0, 2, 2),     F6(~2, ~0x1a, ~0, ~2, ~3),     "1,2,d",    0,       v8}, /* AJIT */
% {"vsmuld32",     F6(2, 0x1a, 0, 2, 4),     F6(~2, ~0x1a, ~0, ~2, ~4),     "1,2,d",    0,       v8}, /* AJIT */
% {"sdivd",        F4(2, 0x0f, 0, 1),        F4(~2, ~0x0f, ~0, ~1),         "1,2,d",    F_DIV32, v8}, /* AJIT */
% {"sdivdcc",      F4(2, 0x1f, 0, 1),        F4(~2, ~0x1f, ~0, ~1),         "1,2,d",    F_DIV32, v8}, /* AJIT */
% {"udivdcc",      F4(2, 0x1e, 0, 1),        F4(~2, ~0x1e, ~0, ~1),         "1,2,d",    F_DIV32, v8}, /* AJIT */
% {"udivd",        F4(2, 0x0e, 0, 1),        F4(~2, ~0x0e, ~0, ~1),         "1,2,d",    F_DIV32, v8}, /* AJIT */
% {"cswap",        F3(3, 0x2f, 0),           F3(~3, ~0x2f, ~0),             "[1+2]A,d", 0,       v8}, /* AJIT */
% {"cswap",        F3(3, 0x2f, 1),           F3(~3, ~0x2f, ~1),             "[1+i],d",  0,       v8}, /* AJIT */
% {"cswapa",       F3(3, 0x3f, 0),           F3(~3, ~0x3f, ~0),             "[1+2]A,d", 0,       v8}, /* AJIT */
% {"cswapa",       F3(3, 0x3f, 1),           F3(~3, ~0x3f, ~1),             "[1+i],d",  0,       v8}, /* AJIT */
% {"xnord",        F4(2, 0x07, 0, 1),        F4(~2, ~0x07, ~0, ~1),         "1,2,d",    0,       v6}, /* AJIT */
% {"xnordcc",      F4(2, 0x17, 0, 1),        F4(~2, ~0x17, ~0, ~1),         "1,2,d",    0,       v6}, /* AJIT */
% {"xordcc",       F4(2, 0x13, 0, 1),        F4(~2, ~0x13, ~0, ~1),         "1,2,d",    0,       v6}, /* AJIT */
% {"adddreduce8",  F6(2, 0x2d, 0, 0x0, 0x1), F6(~2, ~0x2d, ~0, ~0x0, ~0x1), "1,2,d",    0,       v8}, /* AJIT */
% {"adddreduce16", F6(2, 0x2d, 0, 0x0, 0x2), F6(~2, ~0x2d, ~0, ~0x0, ~0x2), "1,2,d",    0,       v8}, /* AJIT */
% {"ordreduce8",   F6(2, 0x2e, 0, 0x0, 0x1), F6(~2, ~0x2e, ~0, ~0x0, ~0x1), "1,2,d",    0,       v8}, /* AJIT */
% {"ordreduce16",  F6(2, 0x2e, 0, 0x0, 0x2), F6(~2, ~0x2e, ~0, ~0x0, ~0x2), "1,2,d",    0,       v8}, /* AJIT */
% {"anddreduce8",  F6(2, 0x2f, 0, 0x0, 0x1), F6(~2, ~0x2f, ~0, ~0x0, ~0x1), "1,2,d",    0,       v8}, /* AJIT */
% {"anddreduce16", F6(2, 0x2f, 0, 0x0, 0x2), F6(~2, ~0x2f, ~0, ~0x0, ~0x2), "1,2,d",    0,       v8}, /* AJIT */
% {"xordreduce8",  F6(2, 0x3e, 0, 0x0, 0x1), F6(~2, ~0x3e, ~0, ~0x0, ~0x1), "1,2,d",    0,       v8}, /* AJIT */
% {"xordreduce16", F6(2, 0x3e, 0, 0x0, 0x2), F6(~2, ~0x3e, ~0, ~0x0, ~0x2), "1,2,d",    0,       v8}, /* AJIT */
% {"zbytedpos",    F7(2, 0x3f, 0x0, 0x0),    F7(~2, ~0x3f, ~0x0, ~0x0),     "1,2,d",    0,       v8}, /* AJIT */
% {"zbytedpos",    F7(2, 0x3f, 0x1, 0x0),    F7(~2, ~0x3f, ~0x1, ~0x0),     "1,i,d",    0,       v8}, /* AJIT */
% {"vfadd32",      F3F(2, 0x34, 0x142),      F3F(~2, ~0x34, ~0x142),        "v,B,H",    F_FLOAT, v8}, /* AJIT */
% {"vfadd16",      F3F(2, 0x34, 0x143),      F3F(~2, ~0x34, ~0x143),        "v,B,H",    F_FLOAT, v8}, /* AJIT */
% {"vfsub32",      F3F(2, 0x34, 0x144),      F3F(~2, ~0x34, ~0x144),        "v,B,H",    F_FLOAT, v8}, /* AJIT */
% {"vfsub16",      F3F(2, 0x34, 0x145),      F3F(~2, ~0x34, ~0x145),        "v,B,H",    F_FLOAT, v8}, /* AJIT */
% {"vfmul32",      F3F(2, 0x34, 0x146),      F3F(~2, ~0x34, ~0x146),        "v,B,H",    F_FLOAT, v8}, /* AJIT */
% {"vfmul16",      F3F(2, 0x34, 0x147),      F3F(~2, ~0x34, ~0x147),        "v,B,H",    F_FLOAT, v8}, /* AJIT */
% {"vfi16toh",     F3F(2, 0x34, 0x148),      F3F(~2, ~0x34, ~0x148),        "v,B,H",    F_FLOAT, v8}, /* AJIT */
% {"vfhtoi16",     F3F(2, 0x34, 0x149),      F3F(~2, ~0x34, ~0x149),        "v,B,H",    F_FLOAT, v8}, /* AJIT */
% {"faddreduce16", F3F(2, 0x34, 0x150),      F3F(~2, ~0x34, ~0x150),        "v,g",      F_FLOAT, v8}, /* AJIT */
% {"fstoh",        F3F(2, 0x34, 0x151),      F3F(~2, ~0x34, ~0x151),        "e,g",      F_FLOAT, v8}, /* AJIT */
% {"fhtos",        F3F(2, 0x34, 0x152),      F3F(~2, ~0x34, ~0x152),        "e,g",      F_FLOAT, v8}, /* AJIT */
% \end{verbatim}
% }
{\small
\begin{verbatim}
{"slld",         F5(2, 0x25, 1, 0x2),    F5(~2, ~0x25, ~1, ~0x2),      "1,Y,d",    0,       v6},
{"slld",         F5(2, 0x25, 0, 0x2),    F5(~2, ~0x25, ~0, ~0x2),      "1,2,d",    0,       v6},
{"srad",         F5(2, 0x27, 1, 0x2),    F5(~2, ~0x27, ~1, ~0x2),      "1,Y,d",    0,       v6},
{"srad",         F5(2, 0x27, 0, 0x2),    F5(~2, ~0x27, ~0, ~0x2),      "1,2,d",    0,       v6},
{"srld",         F5(2, 0x26, 1, 0x2),    F5(~2, ~0x26, ~1, ~0x2),      "1,Y,d",    0,       v6},
{"srld",         F5(2, 0x26, 0, 0x2),    F5(~2, ~0x26, ~0, ~0x2),      "1,2,d",    0,       v6},
{"ord",          F4(2, 0x02, 0, 1),      F4(~2, ~0x02, ~0, ~1),        "1,2,d",    0,       v6},
{"ordcc",        F4(2, 0x12, 0, 1),      F4(~2, ~0x12, ~0, ~1),        "1,2,d",    0,       v6},
{"ordn",         F4(2, 0x06, 0, 1),      F4(~2, ~0x06, ~0, ~1),        "1,2,d",    0,       v6},
{"ordncc",       F4(2, 0x16, 0, 1),      F4(~2, ~0x16, ~0, ~1),        "1,2,d",    0,       v6},
{"anddn",        F4(2, 0x05, 0, 1),      F4(~2, ~0x05, ~0, ~1),        "1,2,d",    0,       v6},
{"anddncc",      F4(2, 0x15, 0, 1),      F4(~2, ~0x15, ~0, ~1),        "1,2,d",    0,       v6},
{"subd",         F4(2, 0x04, 0, 1),      F4(~2, ~0x04, ~0, ~1),        "1,2,d",    0,       v8},
{"subdcc",       F4(2, 0x14, 0, 1),      F4(~2, ~0x14, ~0, ~1),        "1,2,d",    0,       v8},
{"vsubd8",       F6(2, 0x04, 0, 2, 1),   F6(~2, ~0x04, ~0, ~2, ~1),    "1,2,d",    0,       v8},
{"vsubd16",      F6(2, 0x04, 0, 2, 2),   F6(~2, ~0x04, ~0, ~2, ~3),    "1,2,d",    0,       v8},
{"vsubd32",      F6(2, 0x04, 0, 2, 4),   F6(~2, ~0x04, ~0, ~2, ~4),    "1,2,d",    0,       v8},
{"andd",         F4(2, 0x01, 0, 1),      F4(~2, ~0x01, ~0, ~1),        "1,2,d",    0,       v6},
{"anddcc",       F4(2, 0x11, 0, 1),      F4(~2, ~0x11, ~0, ~1),        "1,2,d",    0,       v6},
{"addd",         F4(2, 0x00, 0, 1),      F4(~2, ~0x00, ~0, ~1),        "1,2,d",    0,       v8},
{"adddcc",       F4(2, 0x10, 0,1),       F4(~2, ~0x10, ~0, ~1),        "1,2,d",    0,       v8},
{"vaddd8",       F6(2, 0x00, 0, 2, 1),   F6(~2, ~0x00, ~0, ~2, ~1),    "1,2,d",    0,       v8},
{"vaddd16",      F6(2, 0x00, 0, 2, 2),   F6(~2, ~0x00, ~0, ~2, ~2),    "1,2,d",    0,       v8},
{"vaddd32",      F6(2, 0x00, 0, 2, 4),   F6(~2, ~0x00, ~0, ~2, ~4),    "1,2,d",    0,       v8},
{"smuld",        F4(2, 0x0b, 0, 1),      F4(~2, ~0x0b, ~0, ~1),        "1,2,d",    F_MUL32, v8},
{"smuldcc",      F4(2, 0x1b, 0, 1),      F4(~2, ~0x1b, ~0, ~1),        "1,2,d",    F_MUL32, v8},
{"umuld",        F4(2, 0x0a, 0, 1),      F4(~2, ~0x0a, ~0, ~1),        "1,2,d",    F_MUL32, v8},
{"umuldcc",      F4(2, 0x1a, 0, 1),      F4(~2, ~0x1a, ~0, ~1),        "1,2,d",    F_MUL32, v8},
{"vumuld8",      F6(2, 0x0a, 0, 2, 1),   F6(~2, ~0x0a, ~0, ~2, ~1),    "1,2,d",    0,       v8},
{"vumuld16",     F6(2, 0x0a, 0, 2, 2),   F6(~2, ~0x0a, ~0, ~2, ~3),    "1,2,d",    0,       v8},
{"vumuld32",     F6(2, 0x0a, 0, 2, 4),   F6(~2, ~0x0a, ~0, ~2, ~4),    "1,2,d",    0,       v8},
{"vsmuld8",      F6(2, 0x0b, 0, 2, 1),   F6(~2, ~0x0b, ~0, ~2, ~1),    "1,2,d",    0,       v8},
{"vsmuld16",     F6(2, 0x0b, 0, 2, 2),   F6(~2, ~0x0b, ~0, ~2, ~3),    "1,2,d",    0,       v8},
{"vsmuld32",     F6(2, 0x0b, 0, 2, 4),   F6(~2, ~0x0b, ~0, ~2, ~4),    "1,2,d",    0,       v8},
{"sdivd",        F4(2, 0x0f, 0, 1),      F4(~2, ~0x0f, ~0, ~1),        "1,2,d",    F_DIV32, v8},
{"sdivdcc",      F4(2, 0x1f, 0, 1),      F4(~2, ~0x1f, ~0, ~1),        "1,2,d",    F_DIV32, v8},
{"udivdcc",      F4(2, 0x1e, 0, 1),      F4(~2, ~0x1e, ~0, ~1),        "1,2,d",    F_DIV32, v8},
{"udivd",        F4(2, 0x0e, 0, 1),      F4(~2, ~0x0e, ~0, ~1),        "1,2,d",    F_DIV32, v8},
{"cswap",        F3(3, 0x2f, 0),         F3(~3, ~0x2f, ~0),            "[1+2]A,d", 0,       v8},
{"cswap",        F3(3, 0x2f, 1),         F3(~3, ~0x2f, ~1),            "[1+i],d",  0,       v8},
{"cswapa",       F3(3, 0x3f, 0),         F3(~3, ~0x3f, ~0),            "[1+2]A,d", 0,       v8},
{"cswapa",       F3(3, 0x3f, 1),         F3(~3, ~0x3f, ~1),            "[1+i],d",  0,       v8},
{"xnord",        F4(2, 0x07, 0, 1),      F4(~2, ~0x07, ~0, ~1),        "1,2,d",    0,       v6},
{"xnordcc",      F4(2, 0x17, 0, 1),      F4(~2, ~0x17, ~0, ~1),        "1,2,d",    0,       v6},
{"xordcc",       F4(2, 0x13, 0, 1),      F4(~2, ~0x13, ~0, ~1),        "1,2,d",    0,       v6},
{"adddreduce8",  F8(2, 0x2d, 0x0, 0x1),  F8(~2, ~0x2d, ~0x0, ~0x1),    "1,2,d",    0,       v8},
{"ordreduce8",   F8(2, 0x2e, 0x0, 0x1),  F8(~2, ~0x2e, ~0x0, ~0x1),    "1,2,d",    0,       v8},
{"anddreduce8",  F8(2, 0x2f, 0x0, 0x1),  F8(~2, ~0x2f, ~0x0, ~0x1),    "1,2,d",    0,       v8},
{"xordreduce8",  F8(2, 0x3e, 0x0, 0x1),  F8(~2, ~0x3e, ~0x0, ~0x1),    "1,2,d",    0,       v8},
{"adddreduce16", F8(2, 0x2d, 0x0, 0x2),  F8(~2, ~0x2d, ~0x0, ~0x2),    "1,2,d",    0,       v8},
{"ordreduce16",  F8(2, 0x2e, 0x0, 0x2),  F8(~2, ~0x2e, ~0x0, ~0x2),    "1,2,d",    0,       v8},
{"anddreduce16", F8(2, 0x2f, 0x0, 0x2),  F8(~2, ~0x2f, ~0x0, ~0x2),    "1,2,d",    0,       v8},
{"xordreduce16", F8(2, 0x3e, 0x0, 0x2),  F8(~2, ~0x3e, ~0x0, ~0x2),    "1,2,d",    0,       v8},
{"zbytedpos",    F8(2, 0x3f, 0x0, 0x0),  F8(~2, ~0x3f, ~0x0, ~0x0),    "1,2,d",    0,       v8},
{"zbytedpos",    F8I(2, 0x3f, 0x1, 0x0), F8I(~2, ~0x3f, ~0x1, ~0x0),   "1,i,d",    0,       v8},
{"vfadd32",      F3F(2, 0x34, 0x142),    F3F(~2, ~0x34, ~0x142),       "v,B,H",    F_FLOAT, v8},
{"vfadd16",      F3F(2, 0x34, 0x143),    F3F(~2, ~0x34, ~0x143),       "v,B,H",    F_FLOAT, v8},
{"vfsub32",      F3F(2, 0x34, 0x144),    F3F(~2, ~0x34, ~0x144),       "v,B,H",    F_FLOAT, v8},
{"vfsub16",      F3F(2, 0x34, 0x145),    F3F(~2, ~0x34, ~0x145),       "v,B,H",    F_FLOAT, v8},
{"vfmul32",      F3F(2, 0x34, 0x146),    F3F(~2, ~0x34, ~0x146),       "v,B,H",    F_FLOAT, v8},
{"vfmul16",      F3F(2, 0x34, 0x147),    F3F(~2, ~0x34, ~0x147),       "v,B,H",    F_FLOAT, v8},
{"vfi16toh",     F3F(2, 0x34, 0x148),    F3F(~2, ~0x34, ~0x148),       "f,H",      F_FLOAT, v8},
{"vfhtoi16",     F3F(2, 0x34, 0x149),    F3F(~2, ~0x34, ~0x149),       "f,H",      F_FLOAT, v8},
{"faddreduce16", F3F(2, 0x34, 0x150),    F3F(~2, ~0x34, ~0x150),       "f,H",      F_FLOAT, v8},
{"fstoh",        F3F(2, 0x34, 0x151),    F3F(~2, ~0x34, ~0x151),       "f,H",      F_FLOAT, v8},
{"fhtos",        F3F(2, 0x34, 0x152),    F3F(~2, ~0x34, ~0x152),       "f,H",      F_FLOAT, v8},
\end{verbatim}
}
\vskip -.25in
\hrulefill


\subsection{Integer-Unit Extensions: Arithmetic-Logic Instructions}
\label{sec:integer-unit-extns:arith-logic-insns:impl}

The  integer  unit extensions  of  AJIT  are  based  on the  SPARC  V8
instructions.    See:  SPArc   v8  architecture   manual.   SPARC   v8
instructions  are  32   bits  long.   The  GNU   Binutils  2.22  SPARC
implementation defines a  set of macros to capture the  bits set by an
instruction.  These are the so called ``match'' masks.  Please see the
code     in     \texttt{\$BINUTILSHOME/include/opcode/sparc.h}     and
\texttt{\$BINUTILSHOME/opcodes/sparc-opc.c}.

\subsubsection{Addition and subtraction instructions:}
\label{sec:add:sub:insn:impl}
\begin{enumerate}
\item \textbf{ADDD}:\\
  \begin{center}
    \begin{tabular}[p]{|c|c|l|l|l|}
      \hline
      \textbf{Start} & \textbf{End} & \textbf{Range} & \textbf{Meaning} &
                                                                          \textbf{New Meaning}\\
      \hline
      0 & 4 & 32 & Source register 2, rs2 & No change \\
      5 & 12 & -- & \textbf{unused} & \textbf{Set bit 5 to ``1''} \\
      13 & 13 & 0,1 & The \textbf{i} bit & \textbf{Set i to ``0''} \\
      14 & 18 & 32 & Source register 1, rs1 & No change \\
      19 & 24 & 000000 & ``\textbf{op3}'' & No change \\
      25 & 29 & 32 & Destination register, rd & No change \\
      30 & 31 & 4 & Always ``10'' & No change \\
      \hline
    \end{tabular}
  \end{center}
  \begin{itemize}
  \item []\textbf{ADDD}: same as ADD, but with Instr[13]=0 (i=0), and
    Instr[5]=1.
  \item []\textbf{Syntax}: ``\texttt{addd  SrcReg1, SrcReg2, DestReg}''.
  \item []\textbf{Semantics}: rd(pair) $\leftarrow$ rs1(pair) + rs2(pair).
  \end{itemize}
  Bits layout:
\begin{verbatim}
    Offsets      : 31       24 23       16  15        8   7        0
    Bit layout   :  XXXX  XXXX  XXXX  XXXX   XXXX  XXXX   XXXX  XXXX
    Insn Bits    :  10       0  0000  0        0            1       
    Destination  :    DD  DDD                                       
    Source 1     :                     SSS   SS
    Source 2     :                                           S  SSSS
    Unused (0)   :                              U  UUUU   UU        
    Final layout :  10DD  DDD0  0000  0SSS   SS0U  UUUU   UU1S  SSSS
\end{verbatim}

  Hence the SPARC bit layout of this instruction is:

  \begin{tabular}[h]{lclcl}
    Macro to set  &=& \texttt{F4(x, y, z)} &in& \texttt{sparc.h}     \\
    Macro to reset  &=& \texttt{INVF4(x, y, z)} &in& \texttt{sparc.h}     \\
    x &=& 0x2      &in& \texttt{OP(x)  /* ((x) \& 0x3)  $<<$ 30 */} \\
    y &=& 0x00     &in& \texttt{OP3(y) /* ((y) \& 0x3f) $<<$ 19 */} \\
    z &=& 0x0      &in& \texttt{F3I(z) /* ((z) \& 0x1)  $<<$ 13 */} \\
    a &=& 0x1      &in& \texttt{OP\_AJIT\_BIT(a) /* ((a) \& 0x1)  $<<$ 5 */}
  \end{tabular}

  The AJIT bit  (insn[5]) is set internally by  \texttt{F4}, and hence
  there are only three arguments.

\item \textbf{ADDDCC}:\\
  \begin{center}
    \begin{tabular}[p]{|c|c|l|l|l|}
      \hline
      \textbf{Start} & \textbf{End} & \textbf{Range} & \textbf{Meaning} &
                                                                          \textbf{New Meaning}\\
      \hline
      0 & 4 & 32 & Source register 2, rs2 & No change \\
      5 & 12 & -- & \textbf{unused} & \textbf{Set bit 5 to ``1''} \\
      13 & 13 & 0,1 & The \textbf{i} bit & \textbf{Set i to ``0''} \\
      14 & 18 & 32 & Source register 1, rs1 & No change \\
      19 & 24 & 010000 & ``\textbf{op3}'' & No change \\
      25 & 29 & 32 & Destination register, rd & No change \\
      30 & 31 & 4 & Always ``10'' & No change \\
      \hline
    \end{tabular}
  \end{center}
  New addition:
  \begin{itemize}
  \item []\textbf{ADDDCC}: same as ADDCC, but with Instr[13]=0 (i=0), and
    Instr[5]=1.
  \item []\textbf{Syntax}: ``\texttt{adddcc  SrcReg1, SrcReg2, DestReg}''.
  \item []\textbf{Semantics}: rd(pair) $\leftarrow$ rs1(pair) + rs2(pair), set Z,N
  \end{itemize}
  Bits layout:
\begin{verbatim}
    Offsets      : 31       24 23       16  15        8   7        0
    Bit layout   :  XXXX  XXXX  XXXX  XXXX   XXXX  XXXX   XXXX  XXXX
    Insn Bits    :  10       0  1000  0        0            1       
    Destination  :    DD  DDD                                       
    Source 1     :                     SSS   SS
    Source 2     :                                           S  SSSS
    Unused (0)   :                              U  UUUU   UU        
    Final layout :  10DD  DDD0  1000  0SSS   SS0U  UUUU   UU1S  SSSS
\end{verbatim}

  Hence the SPARC bit layout of this instruction is:

  \begin{tabular}[h]{lclcl}
    Macro to set  &=& \texttt{F4(x, y, z)} &in& \texttt{sparc.h}     \\
    Macro to reset  &=& \texttt{INVF4(x, y, z)} &in& \texttt{sparc.h}     \\
    x &=& 0x2      &in& \texttt{OP(x)  /* ((x) \& 0x3)  $<<$ 30 */} \\
    y &=& 0x10     &in& \texttt{OP3(y) /* ((y) \& 0x3f) $<<$ 19 */} \\
    z &=& 0x0      &in& \texttt{F3I(z) /* ((z) \& 0x1)  $<<$ 13 */} \\
    a &=& 0x1      &in& \texttt{OP\_AJIT\_BIT(a) /* ((a) \& 0x1)  $<<$ 5 */}
  \end{tabular}

  The AJIT bit  (insn[5]) is set internally by  \texttt{F4}, and hence
  there are only three arguments.

\item \textbf{SUBD}:\\
  \begin{center}
    \begin{tabular}[p]{|c|c|l|l|l|}
      \hline
      \textbf{Start} & \textbf{End} & \textbf{Range} & \textbf{Meaning} &
                                                                          \textbf{New Meaning}\\
      \hline
      0 & 4 & 32 & Source register 2, rs2 & No change \\
      5 & 12 & -- & \textbf{unused} & \textbf{Set bit 5 to ``1''} \\
      13 & 13 & 0,1 & The \textbf{i} bit & \textbf{Set i to ``0''} \\
      14 & 18 & 32 & Source register 1, rs1 & No change \\
      19 & 24 & 000100 & ``\textbf{op3}'' & No change \\
      25 & 29 & 32 & Destination register, rd & No change \\
      30 & 31 & 4 & Always ``10'' & No change \\
      \hline
    \end{tabular}
  \end{center}
  New addition:
  \begin{itemize}
  \item []\textbf{SUBD}: same as SUB, but with Instr[13]=0 (i=0), and
    Instr[5]=1.
  \item []\textbf{Syntax}: ``\texttt{subd  SrcReg1, SrcReg2, DestReg}''.
  \item []\textbf{Semantics}: rd(pair) $\leftarrow$ rs1(pair) - rs2(pair).
  \end{itemize}
  Bits layout:
\begin{verbatim}
    Offsets      : 31       24 23       16  15        8   7        0
    Bit layout   :  XXXX  XXXX  XXXX  XXXX   XXXX  XXXX   XXXX  XXXX
    Insn Bits    :  10       0  0010  0        0            1       
    Destination  :    DD  DDD                                       
    Source 1     :                     SSS   SS
    Source 2     :                                           S  SSSS
    Unused (0)   :                              U  UUUU   UU        
    Final layout :  10DD  DDD0  0010  0SSS   SS0U  UUUU   UU1S  SSSS
\end{verbatim}

  Hence the SPARC bit layout of this instruction is:

  \begin{tabular}[h]{lclcl}
    Macro to set  &=& \texttt{F4(x, y, z)} &in& \texttt{sparc.h}     \\
    Macro to reset  &=& \texttt{INVF4(x, y, z)} &in& \texttt{sparc.h}     \\
    x &=& 0x2      &in& \texttt{OP(x)  /* ((x) \& 0x3)  $<<$ 30 */} \\
    y &=& 0x04     &in& \texttt{OP3(y) /* ((y) \& 0x3f) $<<$ 19 */} \\
    z &=& 0x0      &in& \texttt{F3I(z) /* ((z) \& 0x1)  $<<$ 13 */} \\
    a &=& 0x1      &in& \texttt{OP\_AJIT\_BIT(a) /* ((a) \& 0x1)  $<<$ 5 */}
  \end{tabular}

  The AJIT bit  (insn[5]) is set internally by  \texttt{F4}, and hence
  there are only three arguments.

\item \textbf{SUBDCC}:\\
  \begin{center}
    \begin{tabular}[p]{|c|c|l|l|l|}
      \hline
      \textbf{Start} & \textbf{End} & \textbf{Range} & \textbf{Meaning} &
                                                                          \textbf{New Meaning}\\
      \hline
      0 & 4 & 32 & Source register 2, rs2 & No change \\
      5 & 12 & -- & \textbf{unused} & \textbf{Set bit 5 to ``1''} \\
      13 & 13 & 0,1 & The \textbf{i} bit & \textbf{Set i to ``0''} \\
      14 & 18 & 32 & Source register 1, rs1 & No change \\
      19 & 24 & 010100 & ``\textbf{op3}'' & No change \\
      25 & 29 & 32 & Destination register, rd & No change \\
      30 & 31 & 4 & Always ``10'' & No change \\
      \hline
    \end{tabular}
  \end{center}
  New addition:
  \begin{itemize}
  \item []\textbf{SUBDCC}: same as SUBCC, but with Instr[13]=0 (i=0), and
    Instr[5]=1.
  \item []\textbf{Syntax}: ``\texttt{subdcc  SrcReg1, SrcReg2, DestReg}''.
  \item []\textbf{Semantics}: rd(pair) $\leftarrow$ rs1(pair) - rs2(pair), set Z,N
  \end{itemize}
  Bits layout:
\begin{verbatim}
    Offsets      : 31       24 23       16  15        8   7        0
    Bit layout   :  XXXX  XXXX  XXXX  XXXX   XXXX  XXXX   XXXX  XXXX
    Insn Bits    :  10       0  1010  0        0            1       
    Destination  :    DD  DDD                                       
    Source 1     :                     SSS   SS
    Source 2     :                                           S  SSSS
    Unused (0)   :                              U  UUUU   UU        
    Final layout :  10DD  DDD0  1010  0SSS   SS0U  UUUU   UU1S  SSSS
\end{verbatim}

  Hence the SPARC bit layout of this instruction is:

  \begin{tabular}[h]{lclcl}
    Macro to set  &=& \texttt{F4(x, y, z)} &in& \texttt{sparc.h}     \\
    Macro to reset  &=& \texttt{INVF4(x, y, z)} &in& \texttt{sparc.h}     \\
    x &=& 0x2      &in& \texttt{OP(x)  /* ((x) \& 0x3)  $<<$ 30 */} \\
    y &=& 0x14     &in& \texttt{OP3(y) /* ((y) \& 0x3f) $<<$ 19 */} \\
    z &=& 0x0      &in& \texttt{F3I(z) /* ((z) \& 0x1)  $<<$ 13 */} \\
    a &=& 0x1      &in& \texttt{OP\_AJIT\_BIT(a) /* ((a) \& 0x1)  $<<$ 5 */}
  \end{tabular}

  The AJIT bit  (insn[5]) is set internally by  \texttt{F4}, and hence
  there are only three arguments.
\end{enumerate}

\subsubsection{Shift instructions:}
\label{sec:shift:insn:impl}
The shift  family of instructions  of AJIT  may each be  considered to
have  two versions:  a direct  count version  and a  register indirect
count version.  In the direct count  version the shift count is a part
of the  instruction bits.   In the indirect  count version,  the shift
count is  found on the  register specified by  the bit pattern  in the
instruction  bits.   The direct  count  version  is specified  by  the
14$^{th}$  bit, i.e.  insn[13]  (bit  number 13  in  the  0 based  bit
numbering scheme), being set to 1.  If insn[13] is 0 then the register
indirect version is specified.

Similar to the addition and subtraction instructions, the shift family
of instructions of  SPARC V8 also do  not use bits from 5  to 12 (both
inclusive).  The AJIT processor uses bits  5 and 6.  In particular bit
6 is always 1.   Bit 5 may be used in the direct  version giving a set
of 6 bits  available for specifying the shift count.   The shift count
can have  a maximum  value of  64.  Bit  5 is  unused in  the register
indirect version, and is always 0 in that case.

These instructions  are therefore  worked out  below in  two different
sets: the direct and the register indirect ones.
\begin{enumerate}
\item The direct versions  are given by insn[13] = 1.  The 6 bit shift
  count  is directly  specified  in the  instruction bits.   Therefore
  insn[5:0] specify the  shift count.  insn[6] =  1, distinguishes the
  AJIT version from the SPARC V8 version.
  \begin{enumerate}
  \item \textbf{SLLD}:\\
    \begin{center}
      \begin{tabular}[p]{|c|c|l|p{.25\textwidth}|p{.3\textwidth}|}
        \hline
        \textbf{Start} & \textbf{End} & \textbf{Range} & \textbf{Meaning} & \textbf{New Meaning}\\
        \hline
        0 & 4 & 32 & Source register 2, rs2 & Lowest 5 bits of shift count \\
        \hline
        5 & 12 & -- & \textbf{Unused. Set to 0 by software.} &
                                        \begin{minipage}[h]{1.0\linewidth}
                                          \begin{itemize}
                                          \item \textbf{Use bit 5
                                              to specify the msb of
                                              shift count.}
                                          \item \textbf{Use bit 6 to
                                              distinguish AJIT from
                                              SPARC V8.}
                                          \item \textbf{Set bits 7:12
                                              to 0.}
                                          \end{itemize}
                                        \end{minipage}
        \\
        \hline
        13 & 13 & 0,1 & The \textbf{i} bit & \textbf{Set i to ``1''} \\
        14 & 18 & 32 & Source register 1, rs1 & No change \\
        19 & 24 & 100101 & ``\textbf{op3}'' & No change \\
        25 & 29 & 32 & Destination register, rd & No change \\
        30 & 31 & 4 & Always ``10'' & No change \\
        \hline
      \end{tabular}
    \end{center}
    \begin{itemize}
    \item []\textbf{SLLD}: same as SLL, but with Instr[13]=0 (i=0),
      and Instr[5]=1.
    \item []\textbf{Syntax}: ``\texttt{slld SrcReg1, 6BitShiftCnt,
        DestReg}''. \\
      (\textbf{Note:} In an assembly language program, when the second
      argument is a number, we have direct mode.  A register number is
      prefixed with  ``r'', and hence the  syntax itself distinguished
      between   direct  and   register   indirect   version  of   this
      instruction.)
    \item []\textbf{Semantics}: rd(pair) $\leftarrow$ rs1(pair) $<<$
      shift count.
    \end{itemize}
    Bits layout:
\begin{verbatim}
    Offsets      : 31       24 23       16  15        8   7        0
    Bit layout   :  XXXX  XXXX  XXXX  XXXX   XXXX  XXXX   XXXX  XXXX
    Insn Bits    :  10       1  0010  1        1           1        
    Destination  :    DD  DDD                                       
    Source 1     :                     SSS   SS
    Source 2     :                                           S  SSSS
    Unused (0)   :                              U  UUUU   UU        
    Final layout :  10DD  DDD1  0010  1SSS   SS1U  UUUU   U1II  IIII
\end{verbatim}

    This will need another macro that sets bits 5 and 6. Let's call it
    \texttt{OP\_AJIT\_BITS\_5\_AND\_6}.   Hence the  SPARC bit  layout of  this
    instruction is:

    \begin{tabular}[h]{lclcl}
      Macro to set  &=& \texttt{F5(x, y, z)} &in& \texttt{sparc.h}     \\
      Macro to reset  &=& \texttt{INVF5(x, y, z)} &in& \texttt{sparc.h}     \\
      x &=& 0x2      &in& \texttt{OP(x)  /* ((x) \& 0x3)  $<<$ 30 */} \\
      y &=& 0x25     &in& \texttt{OP3(y) /* ((y) \& 0x3f) $<<$ 19 */} \\
      z &=& 0x1      &in& \texttt{F3I(z) /* ((z) \& 0x1)  $<<$ 13 */} \\
      a &=& 0x2      &in& \texttt{OP\_AJIT\_BITS\_5\_AND\_6(a) /* ((a) \& 0x3  $<<$ 6 */}
    \end{tabular}

    The AJIT bits (insn[6:5]) is  set or reset internally by \texttt{F5}
    (just  like  in  \texttt{F4}),  and   hence  there  are  only  three
    arguments.

  \item \textbf{SRLD}:\\
    \begin{center}
      \begin{tabular}[p]{|c|c|l|l|p{.35\textwidth}|}
        \hline
        \textbf{Start} & \textbf{End} & \textbf{Range} & \textbf{Meaning} & \textbf{New Meaning}\\
        \hline
        0 & 4 & 32 & Source register 2, rs2 & Lowest 5 bits of shift count \\
        \hline
        5 & 12 & -- & \textbf{unused} &
                                        \begin{minipage}[h]{1.0\linewidth}
                                          \begin{itemize}
                                          \item \textbf{Use bit 5
                                              to specify the msb of
                                              shift count.}
                                          \item \textbf{Use bit 6 to
                                              distinguish AJIT from
                                              SPARC V8.}
                                          \end{itemize}
                                        \end{minipage}
        \\
        \hline
        13 & 13 & 0,1 & The \textbf{i} bit & \textbf{Set i to ``1''} \\
        14 & 18 & 32 & Source register 1, rs1 & No change \\
        19 & 24 & 100110 & ``\textbf{op3}'' & No change \\
        25 & 29 & 32 & Destination register, rd & No change \\
        30 & 31 & 4 & Always ``10'' & No change \\
        \hline
      \end{tabular}
    \end{center}
    \begin{itemize}
    \item []\textbf{SRLD}: same as SRL, but with Instr[13]=0 (i=0),
      and Instr[5]=1.
    \item []\textbf{Syntax}: ``\texttt{sral SrcReg1, 6BitShiftCnt,
        DestReg}''. \\
      (\textbf{Note:} In an assembly language program, when the second
      argument is a number, we have direct mode.  A register number is
      prefixed with  ``r'', and hence the  syntax itself distinguished
      between   direct  and   register   indirect   version  of   this
      instruction.)
    \item []\textbf{Semantics}: rd(pair) $\leftarrow$ rs1(pair) $>>$
      shift count.
    \end{itemize}
    Bits layout:
\begin{verbatim}
    Offsets      : 31       24 23       16  15        8   7        0
    Bit layout   :  XXXX  XXXX  XXXX  XXXX   XXXX  XXXX   XXXX  XXXX
    Insn Bits    :  10       1  0011  0        1           1        
    Destination  :    DD  DDD                                       
    Source 1     :                     SSS   SS
    Source 2     :                                           S  SSSS
    Unused (0)   :                              U  UUUU   UU        
    Final layout :  10DD  DDD1  0011  0SSS   SS1U  UUUU   U1II  IIII
\end{verbatim}

    This will need another macro that sets bits 5 and 6. Let's call it
    \texttt{OP\_AJIT\_BITS\_5\_AND\_6}.   Hence the  SPARC bit  layout of  this
    instruction is:

    \begin{tabular}[h]{lclcl}
      Macro to set  &=& \texttt{F5(x, y, z)} &in& \texttt{sparc.h}     \\
      Macro to reset  &=& \texttt{INVF5(x, y, z)} &in& \texttt{sparc.h}     \\
      x &=& 0x2      &in& \texttt{OP(x)  /* ((x) \& 0x3)  $<<$ 30 */} \\
      y &=& 0x26     &in& \texttt{OP3(y) /* ((y) \& 0x3f) $<<$ 19 */} \\
      z &=& 0x1      &in& \texttt{F3I(z) /* ((z) \& 0x1)  $<<$ 13 */} \\
      a &=& 0x2      &in& \texttt{OP\_AJIT\_BITS\_5\_AND\_6(a) /* ((a) \& 0x3  $<<$ 6 */}
    \end{tabular}

    The AJIT bits (insn[6:5]) is  set or reset internally by \texttt{F5}
    (just  like  in  \texttt{F4}),  and   hence  there  are  only  three
    arguments.
    
  \item \textbf{SRAD}:\\
    \begin{center}
      \begin{tabular}[p]{|c|c|l|l|p{.35\textwidth}|}
        \hline
        \textbf{Start} & \textbf{End} & \textbf{Range} & \textbf{Meaning} & \textbf{New Meaning}\\
        \hline
        0 & 4 & 32 & Source register 2, rs2 & Lowest 5 bits of shift count \\
        \hline
        5 & 12 & -- & \textbf{unused} &
                                        \begin{minipage}[h]{1.0\linewidth}
                                          \begin{itemize}
                                          \item \textbf{Use bit 5
                                              to specify the msb of
                                              shift count.}
                                          \item \textbf{Use bit 6 to
                                              distinguish AJIT from
                                              SPARC V8.}
                                          \end{itemize}
                                        \end{minipage}
        \\
        \hline
        13 & 13 & 0,1 & The \textbf{i} bit & \textbf{Set i to ``1''} \\
        14 & 18 & 32 & Source register 1, rs1 & No change \\
        19 & 24 & 100111 & ``\textbf{op3}'' & No change \\
        25 & 29 & 32 & Destination register, rd & No change \\
        30 & 31 & 4 & Always ``10'' & No change \\
        \hline
      \end{tabular}
    \end{center}
    \begin{itemize}
    \item []\textbf{SRAD}: same as SRA, but with Instr[13]=0 (i=0),
      and Instr[5]=1.
    \item []\textbf{Syntax}: ``\texttt{srad SrcReg1, 6BitShiftCnt,
        DestReg}''. \\
      (\textbf{Note:} In an assembly language program, when the second
      argument is a number, we have direct mode.  A register number is
      prefixed with  ``r'', and hence the  syntax itself distinguished
      between   direct  and   register   indirect   version  of   this
      instruction.)
    \item []\textbf{Semantics}: rd(pair) $\leftarrow$ rs1(pair) $>>$
      shift count (with sign extension).
    \end{itemize}
    Bits layout:
\begin{verbatim}
    Offsets      : 31       24 23       16  15        8   7        0
    Bit layout   :  XXXX  XXXX  XXXX  XXXX   XXXX  XXXX   XXXX  XXXX
    Insn Bits    :  10       1  0011  1        1           1        
    Destination  :    DD  DDD                                       
    Source 1     :                     SSS   SS
    Source 2     :                                           S  SSSS
    Unused (0)   :                              U  UUUU   UU        
    Final layout :  10DD  DDD1  0011  1SSS   SS1U  UUUU   U1II  IIII
\end{verbatim}

    This will need another macro that sets bits 5 and 6. Let's call it
    \texttt{OP\_AJIT\_BITS\_5\_AND\_6}.   Hence the  SPARC bit  layout of  this
    instruction is:

    \begin{tabular}[h]{lclcl}
      Macro to set  &=& \texttt{F5(x, y, z)} &in& \texttt{sparc.h}     \\
      Macro to reset  &=& \texttt{INVF5(x, y, z)} &in& \texttt{sparc.h}     \\
      x &=& 0x2      &in& \texttt{OP(x)  /* ((x) \& 0x3)  $<<$ 30 */} \\
      y &=& 0x27     &in& \texttt{OP3(y) /* ((y) \& 0x3f) $<<$ 19 */} \\
      z &=& 0x1      &in& \texttt{F3I(z) /* ((z) \& 0x1)  $<<$ 13 */} \\
      a &=& 0x2      &in& \texttt{OP\_AJIT\_BITS\_5\_AND\_6(a) /* ((a) \& 0x3  $<<$ 6 */}
    \end{tabular}

    The AJIT bits (insn[6:5]) is  set or reset internally by \texttt{F5}
    (just  like  in  \texttt{F4}),  and   hence  there  are  only  three
    arguments.

  \end{enumerate}
\item The register  indirect versions are given by insn[13]  = 0.  The
  shift count is indirectly specified in the 32 bit register specified
  in instruction bits.  Therefore  insn[4:0] specify the register that
  has the  shift count.  insn[6]  = 1, distinguishes the  AJIT version
  from the SPARC V8 version.  In this case, insn[5] = 0, necessarily.
  \begin{enumerate}
  \item \textbf{SLLD}:\\
    \begin{center}
      \begin{tabular}[p]{|c|c|l|l|p{.35\textwidth}|}
        \hline
        \textbf{Start} & \textbf{End} & \textbf{Range} & \textbf{Meaning} &
                                                                            \textbf{New Meaning}\\
        \hline
        0 & 4 & 32 & Source register 2, rs2 & Register number \\
        \hline
        5 & 12 & -- & \textbf{unused} &
                                        \begin{minipage}[h]{1.0\linewidth}
                                          \begin{itemize}
                                          \item \textbf{Set bit 5 to 0.}
                                          \item \textbf{Use bit 6 to
                                              distinguish AJIT from
                                              SPARC V8.}
                                          \end{itemize}
                                        \end{minipage}
        \\
        \hline
        13 & 13 & 0,1 & The \textbf{i} bit & \textbf{Set i to ``0''} \\
        14 & 18 & 32 & Source register 1, rs1 & No change \\
        19 & 24 & 100101 & ``\textbf{op3}'' & No change \\
        25 & 29 & 32 & Destination register, rd & No change \\
        30 & 31 & 4 & Always ``10'' & No change \\
        \hline
      \end{tabular}
    \end{center}
    \begin{itemize}
    \item []\textbf{SLLD}: same as SLL, but with Instr[13]=0 (i=0),
      and Instr[5]=1.
    \item []\textbf{Syntax}: ``\texttt{slld SrcReg1, SrcReg2,
        DestReg}''.
    \item []\textbf{Semantics}: rd(pair) $\leftarrow$ rs1(pair) $<<$
      shift count register rs2.
    \end{itemize}
    Bits layout:
\begin{verbatim}
    Offsets      : 31       24 23       16  15        8   7        0
    Bit layout   :  XXXX  XXXX  XXXX  XXXX   XXXX  XXXX   XXXX  XXXX
    Insn Bits    :  10       1  0010  1        0           10        
    Destination  :    DD  DDD                                       
    Source 1     :                     SSS   SS
    Source 2     :                                           S  SSSS
    Unused (0)   :                              U  UUUU   UU        
    Final layout :  10DD  DDD1  0010  1SSS   SS0U  UUUU   U10I  IIII
\end{verbatim}

    This will need another macro that sets bits 5 and 6. Let's call it
    \texttt{OP\_AJIT\_BITS\_5\_AND\_6}.   Hence the  SPARC bit  layout of  this
    instruction is:

    \begin{tabular}[h]{lclcl}
      Macro to set  &=& \texttt{F5(x, y, z)} &in& \texttt{sparc.h}     \\
      Macro to reset  &=& \texttt{INVF5(x, y, z)} &in& \texttt{sparc.h}     \\
      x &=& 0x2      &in& \texttt{OP(x)  /* ((x) \& 0x3)  $<<$ 30 */} \\
      y &=& 0x25     &in& \texttt{OP3(y) /* ((y) \& 0x3f) $<<$ 19 */} \\
      z &=& 0x0      &in& \texttt{F3I(z) /* ((z) \& 0x1)  $<<$ 13 */} \\
      a &=& 0x2      &in& \texttt{OP\_AJIT\_BITS\_5\_AND\_6(a) /* ((a) \& 0x3  $<<$ 6 */}
    \end{tabular}

    The AJIT bits (insn[6:5]) is  set or reset internally by \texttt{F5}
    (just  like  in  \texttt{F4}),  and   hence  there  are  only  three
    arguments.

  \item \textbf{SRLD}:\\
    \begin{center}
      \begin{tabular}[p]{|c|c|l|l|p{.35\textwidth}|}
        \hline
        \textbf{Start} & \textbf{End} & \textbf{Range} & \textbf{Meaning} &
                                                                            \textbf{New Meaning}\\
        \hline
        0 & 4 & 32 & Source register 2, rs2 & Register number \\
        \hline
        5 & 12 & -- & \textbf{unused} &
                                        \begin{minipage}[h]{1.0\linewidth}
                                          \begin{itemize}
                                          \item \textbf{Set bit 5 to 0.}
                                          \item \textbf{Use bit 6 to
                                              distinguish AJIT from
                                              SPARC V8.}
                                          \end{itemize}
                                        \end{minipage}
        \\
        \hline
        13 & 13 & 0,1 & The \textbf{i} bit & \textbf{Set i to ``0''} \\
        14 & 18 & 32 & Source register 1, rs1 & No change \\
        19 & 24 & 100110 & ``\textbf{op3}'' & No change \\
        25 & 29 & 32 & Destination register, rd & No change \\
        30 & 31 & 4 & Always ``10'' & No change \\
        \hline
      \end{tabular}
    \end{center}
    \begin{itemize}
    \item []\textbf{SRLD}: same as SRL, but with Instr[13]=0 (i=0),
      and Instr[5]=1.
    \item []\textbf{Syntax}: ``\texttt{slld SrcReg1, SrcReg2,
        DestReg}''.
    \item []\textbf{Semantics}: rd(pair) $\leftarrow$ rs1(pair) $>>$
      shift count register rs2.
    \end{itemize}
    Bits layout:
\begin{verbatim}
    Offsets      : 31       24 23       16  15        8   7        0
    Bit layout   :  XXXX  XXXX  XXXX  XXXX   XXXX  XXXX   XXXX  XXXX
    Insn Bits    :  10       1  0011  0        0           10        
    Destination  :    DD  DDD                                       
    Source 1     :                     SSS   SS
    Source 2     :                                           S  SSSS
    Unused (0)   :                              U  UUUU   UU        
    Final layout :  10DD  DDD1  0011  0SSS   SS0U  UUUU   U10I  IIII
\end{verbatim}

    This will need another macro that sets bits 5 and 6. Let's call it
    \texttt{OP\_AJIT\_BITS\_5\_AND\_6}.   Hence the  SPARC bit  layout of  this
    instruction is:

    \begin{tabular}[h]{lclcl}
      Macro to set  &=& \texttt{F5(x, y, z)} &in& \texttt{sparc.h}     \\
      Macro to reset  &=& \texttt{INVF5(x, y, z)} &in& \texttt{sparc.h}     \\
      x &=& 0x2      &in& \texttt{OP(x)  /* ((x) \& 0x3)  $<<$ 30 */} \\
      y &=& 0x26     &in& \texttt{OP3(y) /* ((y) \& 0x3f) $<<$ 19 */} \\
      z &=& 0x0      &in& \texttt{F3I(z) /* ((z) \& 0x1)  $<<$ 13 */} \\
      a &=& 0x2      &in& \texttt{OP\_AJIT\_BITS\_5\_AND\_6(a) /* ((a) \& 0x3  $<<$ 6 */}
    \end{tabular}

    The AJIT bits (insn[6:5]) is  set or reset internally by \texttt{F5}
    (just  like  in  \texttt{F4}),  and   hence  there  are  only  three
    arguments.

  \item \textbf{SRAD}:\\
    \begin{center}
      \begin{tabular}[p]{|c|c|l|l|p{.35\textwidth}|}
        \hline
        \textbf{Start} & \textbf{End} & \textbf{Range} & \textbf{Meaning} &
                                                                            \textbf{New Meaning}\\
        \hline
        0 & 4 & 32 & Source register 2, rs2 & Register number \\
        \hline
        5 & 12 & -- & \textbf{unused} &
                                        \begin{minipage}[h]{1.0\linewidth}
                                          \begin{itemize}
                                          \item \textbf{Set bit 5 to 0.}
                                          \item \textbf{Use bit 6 to
                                              distinguish AJIT from
                                              SPARC V8.}
                                          \end{itemize}
                                        \end{minipage}
        \\
        \hline
        13 & 13 & 0,1 & The \textbf{i} bit & \textbf{Set i to ``0''} \\
        14 & 18 & 32 & Source register 1, rs1 & No change \\
        19 & 24 & 100101 & ``\textbf{op3}'' & No change \\
        25 & 29 & 32 & Destination register, rd & No change \\
        30 & 31 & 4 & Always ``10'' & No change \\
        \hline
      \end{tabular}
    \end{center}
    \begin{itemize}
    \item []\textbf{SRAD}: same as SRA, but with Instr[13]=0 (i=0),
      and Instr[5]=1.
    \item []\textbf{Syntax}: ``\texttt{slld SrcReg1, SrcReg2,
        DestReg}''.
    \item []\textbf{Semantics}: rd(pair) $\leftarrow$ rs1(pair) $>>$
      shift count register rs2 (with sign extension).
    \end{itemize}
    Bits layout:
\begin{verbatim}
    Offsets      : 31       24 23       16  15        8   7        0
    Bit layout   :  XXXX  XXXX  XXXX  XXXX   XXXX  XXXX   XXXX  XXXX
    Insn Bits    :  10       1  0011  1        0           10        
    Destination  :    DD  DDD                                       
    Source 1     :                     SSS   SS
    Source 2     :                                           S  SSSS
    Unused (0)   :                              U  UUUU   UU        
    Final layout :  10DD  DDD1  0011  1SSS   SS0U  UUUU   U10I  IIII
\end{verbatim}

    This will need another macro that sets bits 5 and 6. Let's call it
    \texttt{OP\_AJIT\_BITS\_5\_AND\_6}.   Hence the  SPARC bit  layout of  this
    instruction is:

    \begin{tabular}[h]{lclcl}
      Macro to set  &=& \texttt{F5(x, y, z)} &in& \texttt{sparc.h}     \\
      Macro to reset  &=& \texttt{INVF5(x, y, z)} &in& \texttt{sparc.h}     \\
      x &=& 0x2      &in& \texttt{OP(x)  /* ((x) \& 0x3)  $<<$ 30 */} \\
      y &=& 0x27     &in& \texttt{OP3(y) /* ((y) \& 0x3f) $<<$ 19 */} \\
      z &=& 0x0      &in& \texttt{F3I(z) /* ((z) \& 0x1)  $<<$ 13 */} \\
      a &=& 0x2      &in& \texttt{OP\_AJIT\_BITS\_5\_AND\_6(a) /* ((a) \& 0x3  $<<$ 6 */}
    \end{tabular}

    The AJIT bits (insn[6:5]) is  set or reset internally by \texttt{F5}
    (just  like  in  \texttt{F4}),  and   hence  there  are  only  three
    arguments.
  \end{enumerate}
\end{enumerate}

\subsubsection{Multiplication and division instructions:}
\label{sec:mul:div:insn:impl}
\begin{enumerate}
\item \textbf{UMULD}: Unsigned Integer Multiply AJIT, no immediate
  version (i.e. i is always 0).\\
	\textbf{NOTE:} The \emph{suggested} mnemonic ``umuld'' conflicts with a mnemonic of the same name for another sparc architecture (other than v8).   Hence we change it to: ``\textbf{umuldaj}'' in the implementation, but not in the documentation below.

 This conflict occurs despite forcing the GNU assembler to assemble for v8 only via the command line switch ``-Av8''! It appears that forcing the assembler to use v8 is not universally applied throughout the assembler code. 
  \begin{center}
    \begin{tabular}[p]{|c|c|l|l|l|}
      \hline
      \textbf{Start} & \textbf{End} & \textbf{Range} & \textbf{Meaning} &
                                                                          \textbf{New Meaning}\\
      \hline
      0 & 4 & 32 & Source register 2, rs2 & No change \\
      5 & 12 & -- & \textbf{unused} & \textbf{Set bit 5 to ``1''} \\
      13 & 13 & 0,1 & The \textbf{i} bit & \textbf{Set i to ``0''} \\
      14 & 18 & 32 & Source register 1, rs1 & No change \\
      19 & 24 & 001010 & ``\textbf{op3}'' & No change \\
      25 & 29 & 32 & Destination register, rd & No change \\
      30 & 31 & 4 & Always ``10'' & No change \\
      \hline
    \end{tabular}
  \end{center}
  \begin{itemize}
  \item []\textbf{UMULD}: same as UMUL, but with Instr[13]=0 (i=0), and
    Instr[5]=1.
  \item []\textbf{Syntax}: ``\texttt{umuld  SrcReg1, SrcReg2, DestReg}''.
  \item []\textbf{Semantics}: rd(pair) $\leftarrow$ rs1(pair) * rs2(pair).
  \end{itemize}
  Bits layout:
\begin{verbatim}
    Offsets      : 31       24 23       16  15        8   7        0
    Bit layout   :  XXXX  XXXX  XXXX  XXXX   XXXX  XXXX   XXXX  XXXX
    Insn Bits    :  10       0  0101  0        0            1       
    Destination  :    DD  DDD                                       
    Source 1     :                     SSS   SS
    Source 2     :                                           S  SSSS
    Unused (0)   :                              U  UUUU   UU        
    Final layout :  10DD  DDD0  0101  0SSS   SS0U  UUUU   UU1S  SSSS
\end{verbatim}

  Hence the SPARC bit layout of this instruction is:

  \begin{tabular}[h]{lclcl}
    Macro to set  &=& \texttt{F4(x, y, z)} &in& \texttt{sparc.h}     \\
    Macro to reset  &=& \texttt{INVF4(x, y, z)} &in& \texttt{sparc.h}     \\
    x &=& 0x2      &in& \texttt{OP(x)  /* ((x) \& 0x3)  $<<$ 30 */} \\
    y &=& 0x0A     &in& \texttt{OP3(y) /* ((y) \& 0x3f) $<<$ 19 */} \\
    z &=& 0x0      &in& \texttt{F3I(z) /* ((z) \& 0x1)  $<<$ 13 */} \\
    a &=& 0x1      &in& \texttt{OP\_AJIT\_BIT(a) /* ((a) \& 0x1)  $<<$ 5 */}
  \end{tabular}

  The AJIT bit  (insn[5]) is set internally by  \texttt{F4}, and hence
  there are only three arguments.

\item \textbf{UMULDCC}:\\
  \begin{center}
    \begin{tabular}[p]{|c|c|l|l|l|}
      \hline
      \textbf{Start} & \textbf{End} & \textbf{Range} & \textbf{Meaning} &
                                                                          \textbf{New Meaning}\\
      \hline
      0 & 4 & 32 & Source register 2, rs2 & No change \\
      5 & 12 & -- & \textbf{unused} & \textbf{Set bit 5 to ``1''} \\
      13 & 13 & 0,1 & The \textbf{i} bit & \textbf{Set i to ``0''} \\
      14 & 18 & 32 & Source register 1, rs1 & No change \\
      19 & 24 & 011010 & ``\textbf{op3}'' & No change \\
      25 & 29 & 32 & Destination register, rd & No change \\
      30 & 31 & 4 & Always ``10'' & No change \\
      \hline
    \end{tabular}
  \end{center}
  New addition:
  \begin{itemize}
  \item []\textbf{UMULDCC}: same as UMULCC, but with Instr[13]=0 (i=0), and
    Instr[5]=1.
  \item []\textbf{Syntax}: ``\texttt{umuldcc  SrcReg1, SrcReg2, DestReg}''.
  \item []\textbf{Semantics}: rd(pair) $\leftarrow$ rs1(pair) * rs2(pair), set Z
  \end{itemize}
  Bits layout:
\begin{verbatim}
    Offsets      : 31       24 23       16  15        8   7        0
    Bit layout   :  XXXX  XXXX  XXXX  XXXX   XXXX  XXXX   XXXX  XXXX
    Insn Bits    :  10       0  1101  0        0            1       
    Destination  :    DD  DDD                                       
    Source 1     :                     SSS   SS
    Source 2     :                                           S  SSSS
    Unused (0)   :                              U  UUUU   UU        
    Final layout :  10DD  DDD0  1101  0SSS   SS0U  UUUU   UU1S  SSSS
\end{verbatim}

  Hence the SPARC bit layout of this instruction is:

  \begin{tabular}[h]{lclcl}
    Macro to set  &=& \texttt{F4(x, y, z)} &in& \texttt{sparc.h}     \\
    Macro to reset  &=& \texttt{INVF4(x, y, z)} &in& \texttt{sparc.h}     \\
    x &=& 0x2      &in& \texttt{OP(x)  /* ((x) \& 0x3)  $<<$ 30 */} \\
    y &=& 0x1A     &in& \texttt{OP3(y) /* ((y) \& 0x3f) $<<$ 19 */} \\
    z &=& 0x0      &in& \texttt{F3I(z) /* ((z) \& 0x1)  $<<$ 13 */} \\
    a &=& 0x1      &in& \texttt{OP\_AJIT\_BIT(a) /* ((a) \& 0x1)  $<<$ 5 */}
  \end{tabular}

  The AJIT bit  (insn[5]) is set internally by  \texttt{F4}, and hence
  there are only three arguments.

\item \textbf{SMULD}: Unsigned Integer Multiply AJIT, no immediate
  version (i.e. i is always 0).\\
  \begin{center}
    \begin{tabular}[p]{|c|c|l|l|l|}
      \hline
      \textbf{Start} & \textbf{End} & \textbf{Range} & \textbf{Meaning} &
                                                                          \textbf{New Meaning}\\
      \hline
      0 & 4 & 32 & Source register 2, rs2 & No change \\
      5 & 12 & -- & \textbf{unused} & \textbf{Set bit 5 to ``1''} \\
      13 & 13 & 0,1 & The \textbf{i} bit & \textbf{Set i to ``0''} \\
      14 & 18 & 32 & Source register 1, rs1 & No change \\
      19 & 24 & 001011 & ``\textbf{op3}'' & No change \\
      25 & 29 & 32 & Destination register, rd & No change \\
      30 & 31 & 4 & Always ``10'' & No change \\
      \hline
    \end{tabular}
  \end{center}
  \begin{itemize}
  \item []\textbf{SMULD}: same as SMUL, but with Instr[13]=0 (i=0), and
    Instr[5]=1.
  \item []\textbf{Syntax}: ``\texttt{smuld  SrcReg1, SrcReg2, DestReg}''.
  \item []\textbf{Semantics}: rd(pair) $\leftarrow$ rs1(pair) *
    rs2(pair) (signed).
  \end{itemize}
  Bits layout:
\begin{verbatim}
    Offsets      : 31       24 23       16  15        8   7        0
    Bit layout   :  XXXX  XXXX  XXXX  XXXX   XXXX  XXXX   XXXX  XXXX
    Insn Bits    :  10       0  0101  1        0            1       
    Destination  :    DD  DDD                                       
    Source 1     :                     SSS   SS
    Source 2     :                                           S  SSSS
    Unused (0)   :                              U  UUUU   UU        
    Final layout :  10DD  DDD0  0101  1SSS   SS0U  UUUU   UU1S  SSSS
\end{verbatim}

  Hence the SPARC bit layout of this instruction is:

  \begin{tabular}[h]{lclcl}
    Macro to set  &=& \texttt{F4(x, y, z)} &in& \texttt{sparc.h}     \\
    Macro to reset  &=& \texttt{INVF4(x, y, z)} &in& \texttt{sparc.h}     \\
    x &=& 0x2      &in& \texttt{OP(x)  /* ((x) \& 0x3)  $<<$ 30 */} \\
    y &=& 0x0B     &in& \texttt{OP3(y) /* ((y) \& 0x3f) $<<$ 19 */} \\
    z &=& 0x0      &in& \texttt{F3I(z) /* ((z) \& 0x1)  $<<$ 13 */} \\
    a &=& 0x1      &in& \texttt{OP\_AJIT\_BIT(a) /* ((a) \& 0x1)  $<<$ 5 */}
  \end{tabular}

  The AJIT bit  (insn[5]) is set internally by  \texttt{F4}, and hence
  there are only three arguments.

\item \textbf{SMULDCC}:\\
  \begin{center}
    \begin{tabular}[p]{|c|c|l|l|l|}
      \hline
      \textbf{Start} & \textbf{End} & \textbf{Range} & \textbf{Meaning} &
                                                                          \textbf{New Meaning}\\
      \hline
      0 & 4 & 32 & Source register 2, rs2 & No change \\
      5 & 12 & -- & \textbf{unused} & \textbf{Set bit 5 to ``1''} \\
      13 & 13 & 0,1 & The \textbf{i} bit & \textbf{Set i to ``0''} \\
      14 & 18 & 32 & Source register 1, rs1 & No change \\
      19 & 24 & 011011 & ``\textbf{op3}'' & No change \\
      25 & 29 & 32 & Destination register, rd & No change \\
      30 & 31 & 4 & Always ``10'' & No change \\
      \hline
    \end{tabular}
  \end{center}
  New addition:
  \begin{itemize}
  \item []\textbf{SMULDCC}: same as SMULCC, but with Instr[13]=0 (i=0), and
    Instr[5]=1.
  \item []\textbf{Syntax}: ``\texttt{smuldcc  SrcReg1, SrcReg2, DestReg}''.
  \item []\textbf{Semantics}: rd(pair) $\leftarrow$ rs1(pair) *
    rs2(pair) (signed), set Z,N,O
  \end{itemize}
  Bits layout:
\begin{verbatim}
    Offsets      : 31       24 23       16  15        8   7        0
    Bit layout   :  XXXX  XXXX  XXXX  XXXX   XXXX  XXXX   XXXX  XXXX
    Insn Bits    :  10       0  1101  1        0            1       
    Destination  :    DD  DDD                                       
    Source 1     :                     SSS   SS
    Source 2     :                                           S  SSSS
    Unused (0)   :                              U  UUUU   UU        
    Final layout :  10DD  DDD0  1101  1SSS   SS0U  UUUU   UU1S  SSSS
\end{verbatim}

  Hence the SPARC bit layout of this instruction is:

  \begin{tabular}[h]{lclcl}
    Macro to set  &=& \texttt{F4(x, y, z)} &in& \texttt{sparc.h}     \\
    Macro to reset  &=& \texttt{INVF4(x, y, z)} &in& \texttt{sparc.h}     \\
    x &=& 0x2      &in& \texttt{OP(x)  /* ((x) \& 0x3)  $<<$ 30 */} \\
    y &=& 0x1B     &in& \texttt{OP3(y) /* ((y) \& 0x3f) $<<$ 19 */} \\
    z &=& 0x0      &in& \texttt{F3I(z) /* ((z) \& 0x1)  $<<$ 13 */} \\
    a &=& 0x1      &in& \texttt{OP\_AJIT\_BIT(a) /* ((a) \& 0x1)  $<<$ 5 */}
  \end{tabular}

  The AJIT bit  (insn[5]) is set internally by  \texttt{F4}, and hence
  there are only three arguments.

\item \textbf{UDIVD}:\\
  \begin{center}
    \begin{tabular}[p]{|c|c|l|l|l|}
      \hline
      \textbf{Start} & \textbf{End} & \textbf{Range} & \textbf{Meaning} &
                                                                          \textbf{New Meaning}\\
      \hline
      0 & 4 & 32 & Source register 2, rs2 & No change \\
      5 & 12 & -- & \textbf{unused} & \textbf{Set bit 5 to ``1''} \\
      13 & 13 & 0,1 & The \textbf{i} bit & \textbf{Set i to ``0''} \\
      14 & 18 & 32 & Source register 1, rs1 & No change \\
      19 & 24 & 001110 & ``\textbf{op3}'' & No change \\
      25 & 29 & 32 & Destination register, rd & No change \\
      30 & 31 & 4 & Always ``10'' & No change \\
      \hline
    \end{tabular}
  \end{center}
  New addition:
  \begin{itemize}
  \item []\textbf{UDIVD}: same as UDIV, but with Instr[13]=0 (i=0), and
    Instr[5]=1.
  \item []\textbf{Syntax}: ``\texttt{udivd  SrcReg1, SrcReg2, DestReg}''.
  \item []\textbf{Semantics}: rd(pair) $\leftarrow$ rs1(pair) / rs2(pair).
  \end{itemize}
  Bits layout:
\begin{verbatim}
    Offsets      : 31       24 23       16  15        8   7        0
    Bit layout   :  XXXX  XXXX  XXXX  XXXX   XXXX  XXXX   XXXX  XXXX
    Insn Bits    :  10       0  0111  0        0            1       
    Destination  :    DD  DDD                                       
    Source 1     :                     SSS   SS
    Source 2     :                                           S  SSSS
    Unused (0)   :                              U  UUUU   UU        
    Final layout :  10DD  DDD0  0111  0SSS   SS0U  UUUU   UU1S  SSSS
\end{verbatim}

  Hence the SPARC bit layout of this instruction is:

  \begin{tabular}[h]{lclcl}
    Macro to set  &=& \texttt{F4(x, y, z)} &in& \texttt{sparc.h}     \\
    Macro to reset  &=& \texttt{INVF4(x, y, z)} &in& \texttt{sparc.h}     \\
    x &=& 0x2      &in& \texttt{OP(x)  /* ((x) \& 0x3)  $<<$ 30 */} \\
    y &=& 0x0E     &in& \texttt{OP3(y) /* ((y) \& 0x3f) $<<$ 19 */} \\
    z &=& 0x0      &in& \texttt{F3I(z) /* ((z) \& 0x1)  $<<$ 13 */} \\
    a &=& 0x1      &in& \texttt{OP\_AJIT\_BIT(a) /* ((a) \& 0x1)  $<<$ 5 */}
  \end{tabular}

  The AJIT bit  (insn[5]) is set internally by  \texttt{F4}, and hence
  there are only three arguments.

\item \textbf{UDIVDCC}:\\
  \begin{center}
    \begin{tabular}[p]{|c|c|l|l|l|}
      \hline
      \textbf{Start} & \textbf{End} & \textbf{Range} & \textbf{Meaning} &
                                                                          \textbf{New Meaning}\\
      \hline
      0 & 4 & 32 & Source register 2, rs2 & No change \\
      5 & 12 & -- & \textbf{unused} & \textbf{Set bit 5 to ``1''} \\
      13 & 13 & 0,1 & The \textbf{i} bit & \textbf{Set i to ``0''} \\
      14 & 18 & 32 & Source register 1, rs1 & No change \\
      19 & 24 & 011110 & ``\textbf{op3}'' & No change \\
      25 & 29 & 32 & Destination register, rd & No change \\
      30 & 31 & 4 & Always ``10'' & No change \\
      \hline
    \end{tabular}
  \end{center}
  New addition:
  \begin{itemize}
  \item []\textbf{UDIVDCC}: same as UDIVCC, but with Instr[13]=0 (i=0), and
    Instr[5]=1.
  \item []\textbf{Syntax}: ``\texttt{udivdcc  SrcReg1, SrcReg2, DestReg}''.
  \item []\textbf{Semantics}: rd(pair) $\leftarrow$ rs1(pair) / rs2(pair), set Z,O
  \end{itemize}
  Bits layout:
\begin{verbatim}
    Offsets      : 31       24 23       16  15        8   7        0
    Bit layout   :  XXXX  XXXX  XXXX  XXXX   XXXX  XXXX   XXXX  XXXX
    Insn Bits    :  10       0  1111  0        0            1       
    Destination  :    DD  DDD                                       
    Source 1     :                     SSS   SS
    Source 2     :                                           S  SSSS
    Unused (0)   :                              U  UUUU   UU        
    Final layout :  10DD  DDD0  1111  0SSS   SS0U  UUUU   UU1S  SSSS
\end{verbatim}

  Hence the SPARC bit layout of this instruction is:

  \begin{tabular}[h]{lclcl}
    Macro to set  &=& \texttt{F4(x, y, z)} &in& \texttt{sparc.h}     \\
    Macro to reset  &=& \texttt{INVF4(x, y, z)} &in& \texttt{sparc.h}     \\
    x &=& 0x2      &in& \texttt{OP(x)  /* ((x) \& 0x3)  $<<$ 30 */} \\
    y &=& 0x1E     &in& \texttt{OP3(y) /* ((y) \& 0x3f) $<<$ 19 */} \\
    z &=& 0x0      &in& \texttt{F3I(z) /* ((z) \& 0x1)  $<<$ 13 */} \\
    a &=& 0x1      &in& \texttt{OP\_AJIT\_BIT(a) /* ((a) \& 0x1)  $<<$ 5 */}
  \end{tabular}

  The AJIT bit  (insn[5]) is set internally by  \texttt{F4}, and hence
  there are only three arguments.

\item \textbf{SDIVD}:\\
  \begin{center}
    \begin{tabular}[p]{|c|c|l|l|l|}
      \hline
      \textbf{Start} & \textbf{End} & \textbf{Range} & \textbf{Meaning} &
                                                                          \textbf{New Meaning}\\
      \hline
      0 & 4 & 32 & Source register 2, rs2 & No change \\
      5 & 12 & -- & \textbf{unused} & \textbf{Set bit 5 to ``1''} \\
      13 & 13 & 0,1 & The \textbf{i} bit & \textbf{Set i to ``0''} \\
      14 & 18 & 32 & Source register 1, rs1 & No change \\
      19 & 24 & 001111 & ``\textbf{op3}'' & No change \\
      25 & 29 & 32 & Destination register, rd & No change \\
      30 & 31 & 4 & Always ``10'' & No change \\
      \hline
    \end{tabular}
  \end{center}
  New addition:
  \begin{itemize}
  \item []\textbf{SDIVD}: same as SDIV, but with Instr[13]=0 (i=0), and
    Instr[5]=1.
  \item []\textbf{Syntax}: ``\texttt{sdivd  SrcReg1, SrcReg2, DestReg}''.
  \item []\textbf{Semantics}: rd(pair) $\leftarrow$ rs1(pair) /
    rs2(pair) (signed).
  \end{itemize}
  Bits layout:
\begin{verbatim}
    Offsets      : 31       24 23       16  15        8   7        0
    Bit layout   :  XXXX  XXXX  XXXX  XXXX   XXXX  XXXX   XXXX  XXXX
    Insn Bits    :  10       0  0111  1        0            1       
    Destination  :    DD  DDD                                       
    Source 1     :                     SSS   SS
    Source 2     :                                           S  SSSS
    Unused (0)   :                              U  UUUU   UU        
    Final layout :  10DD  DDD0  0111  1SSS   SS0U  UUUU   UU1S  SSSS
\end{verbatim}

  Hence the SPARC bit layout of this instruction is:

  \begin{tabular}[h]{lclcl}
    Macro to set  &=& \texttt{F4(x, y, z)} &in& \texttt{sparc.h}     \\
    Macro to reset  &=& \texttt{INVF4(x, y, z)} &in& \texttt{sparc.h}     \\
    x &=& 0x2      &in& \texttt{OP(x)  /* ((x) \& 0x3)  $<<$ 30 */} \\
    y &=& 0x0F     &in& \texttt{OP3(y) /* ((y) \& 0x3f) $<<$ 19 */} \\
    z &=& 0x0      &in& \texttt{F3I(z) /* ((z) \& 0x1)  $<<$ 13 */} \\
    a &=& 0x1      &in& \texttt{OP\_AJIT\_BIT(a) /* ((a) \& 0x1)  $<<$ 5 */}
  \end{tabular}

  The AJIT bit  (insn[5]) is set internally by  \texttt{F4}, and hence
  there are only three arguments.

\item \textbf{SDIVDCC}:\\
  \begin{center}
    \begin{tabular}[p]{|c|c|l|l|l|}
      \hline
      \textbf{Start} & \textbf{End} & \textbf{Range} & \textbf{Meaning} &
                                                                          \textbf{New Meaning}\\
      \hline
      0 & 4 & 32 & Source register 2, rs2 & No change \\
      5 & 12 & -- & \textbf{unused} & \textbf{Set bit 5 to ``1''} \\
      13 & 13 & 0,1 & The \textbf{i} bit & \textbf{Set i to ``0''} \\
      14 & 18 & 32 & Source register 1, rs1 & No change \\
      19 & 24 & 011111 & ``\textbf{op3}'' & No change \\
      25 & 29 & 32 & Destination register, rd & No change \\
      30 & 31 & 4 & Always ``10'' & No change \\
      \hline
    \end{tabular}
  \end{center}
  New addition:
  \begin{itemize}
  \item []\textbf{SDIVDCC}: same as SDIVCC, but with Instr[13]=0 (i=0), and
    Instr[5]=1.
  \item []\textbf{Syntax}: ``\texttt{sdivdcc  SrcReg1, SrcReg2, DestReg}''.
  \item []\textbf{Semantics}: rd(pair) $\leftarrow$ rs1(pair) /
    rs2(pair) (signed), set Z,N,O
  \end{itemize}
  Bits layout:
\begin{verbatim}
    Offsets      : 31       24 23       16  15        8   7        0
    Bit layout   :  XXXX  XXXX  XXXX  XXXX   XXXX  XXXX   XXXX  XXXX
    Insn Bits    :  10       0  1111  1        0            1       
    Destination  :    DD  DDD                                       
    Source 1     :                     SSS   SS
    Source 2     :                                           S  SSSS
    Unused (0)   :                              U  UUUU   UU        
    Final layout :  10DD  DDD0  1111  1SSS   SS0U  UUUU   UU1S  SSSS
\end{verbatim}

  Hence the SPARC bit layout of this instruction is:

  \begin{tabular}[h]{lclcl}
    Macro to set  &=& \texttt{F4(x, y, z)} &in& \texttt{sparc.h}     \\
    Macro to reset  &=& \texttt{INVF4(x, y, z)} &in& \texttt{sparc.h}     \\
    x &=& 0x2      &in& \texttt{OP(x)  /* ((x) \& 0x3)  $<<$ 30 */} \\
    y &=& 0x1F     &in& \texttt{OP3(y) /* ((y) \& 0x3f) $<<$ 19 */} \\
    z &=& 0x0      &in& \texttt{F3I(z) /* ((z) \& 0x1)  $<<$ 13 */} \\
    a &=& 0x1      &in& \texttt{OP\_AJIT\_BIT(a) /* ((a) \& 0x1)  $<<$ 5 */}
  \end{tabular}

  The AJIT bit  (insn[5]) is set internally by  \texttt{F4}, and hence
  there are only three arguments.
\end{enumerate}

%%% Local Variables:
%%% mode: latex
%%% TeX-master: t
%%% End:

\subsubsection{64 Bit Logical Instructions:}
\label{sec:64:bit:logical:insn:impl}

No immediate mode, i.e. insn[5] $\equiv$ i = 0, always.

\begin{enumerate}
\item \textbf{ORD}:\\
  \begin{center}
    \begin{figure}[h]
      \centering
      \epsfxsize=.8\linewidth
      \epsffile{../figs/ord-ajit-insn-32-bit-layout.eps}
      \caption{The AJIT ORD instruction  with register operands.}
      \label{fig:ajit:ord:insn}
    \end{figure}
  \end{center}
  \begin{itemize}
  \item []\textbf{ORD}: same as OR, but with Instr[13]=0 (i=0), and
    Instr[5]=1.
  \item []\textbf{Syntax}: ``\texttt{ord  SrcReg1, SrcReg2, DestReg}''.
  \item []\textbf{Semantics}: rd(pair) $\leftarrow$ rs1(pair) $\vert$ rs2(pair).
  \end{itemize}

  Hence the SPARC bit layout of this instruction is:

  \begin{tabular}[h]{lclcl}
    Macro to set   &=&  \verb|F4(x, y, z, b)|     &in& \verb|sparc.h|     \\
    Macro to reset &=&  \verb|F4(~x, ~y, ~z, ~b)| &in& \verb|sparc.h|     \\
    x              &=& 0x2                        &in& \verb|OP(x) | \\
    y              &=& 0x02                       &in& \verb|OP3(y) | \\
    z              &=& 0x0                        &in& \verb|F3I(z) | \\
    b              &=& 0x1                        &in& \verb|OP_AJIT_BIT_5(a) |
  \end{tabular}

\item \textbf{ORDCC}:\\
  \begin{center}
    \begin{figure}[h]
      \centering
      \epsfxsize=.8\linewidth
      \epsffile{../figs/ordcc-ajit-insn-32-bit-layout.eps}
      \caption{The AJIT ORDCC instruction  with register operands.}
      \label{fig:ajit:ordcc:insn}
    \end{figure}
  \end{center}
  \begin{itemize}
  \item []\textbf{ORDCC}: same as ORCC, but with Instr[13]=0 (i=0), and
    Instr[5]=1.
  \item []\textbf{Syntax}: ``\texttt{ordcc  SrcReg1, SrcReg2, DestReg}''.
  \item []\textbf{Semantics}: rd(pair) $\leftarrow$ rs1(pair) $\vert$
    rs2(pair), sets Z.
  \end{itemize}

  Hence the SPARC bit layout of this instruction is:

  \begin{tabular}[h]{lclcl}
    Macro to set   &=&  \verb|F4(x, y, z, b)|     &in& \verb|sparc.h|     \\
    Macro to reset &=&  \verb|F4(~x, ~y, ~z, ~b)| &in& \verb|sparc.h|     \\
    x              &=& 0x2                        &in& \verb|OP(x) | \\
    y              &=& 0x12                       &in& \verb|OP3(y) | \\
    z              &=& 0x0                        &in& \verb|F3I(z) | \\
    b              &=& 0x1                        &in& \verb|OP_AJIT_BIT_5(a) |
  \end{tabular}

\item \textbf{ORDN}:\\
  \begin{center}
    \begin{figure}[h]
      \centering
      \epsfxsize=.8\linewidth
      \epsffile{../figs/ordn-ajit-insn-32-bit-layout.eps}
      \caption{The AJIT ORDN instruction  with register operands.}
      \label{fig:ajit:ordn:insn}
    \end{figure}
  \end{center}
  \begin{itemize}
  \item []\textbf{ORDN}: same as ORN, but with Instr[13]=0 (i=0), and
    Instr[5]=1.
  \item []\textbf{Syntax}: ``\texttt{ordn  SrcReg1, SrcReg2, DestReg}''.
  \item []\textbf{Semantics}: rd(pair) $\leftarrow$ rs1(pair) $\vert$ ($\sim$rs2(pair)).
  \end{itemize}

  Hence the SPARC bit layout of this instruction is:

  \begin{tabular}[h]{lclcl}
    Macro to set   &=&  \verb|F4(x, y, z, b)|     &in& \verb|sparc.h|     \\
    Macro to reset &=&  \verb|F4(~x, ~y, ~z, ~b)| &in& \verb|sparc.h|     \\
    x              &=& 0x2                        &in& \verb|OP(x) | \\
    y              &=& 0x06                       &in& \verb|OP3(y) | \\
    z              &=& 0x0                        &in& \verb|F3I(z) | \\
    b              &=& 0x1                        &in& \verb|OP_AJIT_BIT_5(a) |
  \end{tabular}

\item \textbf{ORDNCC}:\\
  \begin{center}
    \begin{figure}[h]
      \centering
      \epsfxsize=.8\linewidth
      \epsffile{../figs/ordncc-ajit-insn-32-bit-layout.eps}
      \caption{The AJIT ORDNCC instruction  with register operands.}
      \label{fig:ajit:ordncc:insn}
    \end{figure}
  \end{center}
  \begin{itemize}
  \item []\textbf{ORDNCC}: same as ORN, but with Instr[13]=0 (i=0), and
    Instr[5]=1.
  \item []\textbf{Syntax}: ``\texttt{ordncc  SrcReg1, SrcReg2, DestReg}''.
  \item []\textbf{Semantics}: rd(pair) $\leftarrow$ rs1(pair) $\vert$
    ($\sim$rs2(pair)), sets Z.
  \end{itemize}

  Hence the SPARC bit layout of this instruction is:

  \begin{tabular}[h]{lclcl}
    Macro to set   &=&  \verb|F4(x, y, z, b)|     &in& \verb|sparc.h|     \\
    Macro to reset &=&  \verb|F4(~x, ~y, ~z, ~b)| &in& \verb|sparc.h|     \\
    x              &=& 0x2                        &in& \verb|OP(x) | \\
    y              &=& 0x16                       &in& \verb|OP3(y) | \\
    z              &=& 0x0                        &in& \verb|F3I(z) | \\
    b              &=& 0x1                        &in& \verb|OP_AJIT_BIT_5(a) |
  \end{tabular}

\item \textbf{XORDCC}:\\
  \begin{center}
    \begin{figure}[h]
      \centering
      \epsfxsize=.8\linewidth
      \epsffile{../figs/xordcc-ajit-insn-32-bit-layout.eps}
      \caption{The AJIT XORDCC instruction  with register operands.}
      \label{fig:ajit:xordcc:insn}
    \end{figure}
  \end{center}
  \begin{itemize}
  \item []\textbf{XORDCC}: same as XORCC, but with Instr[13]=0 (i=0), and
    Instr[5]=1.
  \item []\textbf{Syntax}: ``\texttt{xordcc  SrcReg1, SrcReg2, DestReg}''.
  \item []\textbf{Semantics}: rd(pair) $\leftarrow$ rs1(pair) $\hat{~}$
    rs2(pair), sets Z.
  \end{itemize}

  Hence the SPARC bit layout of this instruction is:

  \begin{tabular}[h]{lclcl}
    Macro to set   &=&  \verb|F4(x, y, z, b)|     &in& \verb|sparc.h|     \\
    Macro to reset &=&  \verb|F4(~x, ~y, ~z, ~b)| &in& \verb|sparc.h|     \\
    x              &=& 0x2                        &in& \verb|OP(x) | \\
    y              &=& 0x13                       &in& \verb|OP3(y) | \\
    z              &=& 0x0                        &in& \verb|F3I(z) | \\
    b              &=& 0x1                        &in& \verb|OP_AJIT_BIT_5(a) |
  \end{tabular}

  The AJIT bit  (insn[5]) is set internally by  \texttt{F4}, and hence
  there are only three arguments.

\item \textbf{XNORD}:\\
  \begin{center}
    \begin{figure}[h]
      \centering
      \epsfxsize=.8\linewidth
      \epsffile{../figs/xnord-ajit-insn-32-bit-layout.eps}
      \caption{The AJIT XNORD instruction  with register operands.}
      \label{fig:ajit:xnord:insn}
    \end{figure}
  \end{center}
  \begin{itemize}
  \item []\textbf{XNORD}: same as XNOR, but with Instr[13]=0 (i=0), and
    Instr[5]=1.
  \item []\textbf{Syntax}: ``\texttt{xnordcc  SrcReg1, SrcReg2, DestReg}''.
  \item []\textbf{Semantics}: rd(pair) $\leftarrow$ rs1(pair) $\hat{~}$
    rs2(pair).
  \end{itemize}

  Hence the SPARC bit layout of this instruction is:

  \begin{tabular}[h]{lclcl}
    Macro to set   &=&  \verb|F4(x, y, z, b)|     &in& \verb|sparc.h|     \\
    Macro to reset &=&  \verb|F4(~x, ~y, ~z, ~b)| &in& \verb|sparc.h|     \\
    x              &=& 0x2                        &in& \verb|OP(x) | \\
    y              &=& 0x07                       &in& \verb|OP3(y) | \\
    z              &=& 0x0                        &in& \verb|F3I(z) | \\
    b              &=& 0x1                        &in& \verb|OP_AJIT_BIT_5(a) |
  \end{tabular}

\item \textbf{XNORDCC}:\\
  \begin{center}
    \begin{figure}[h]
      \centering
      \epsfxsize=.8\linewidth
      \epsffile{../figs/xnordcc-ajit-insn-32-bit-layout.eps}
      \caption{The AJIT XNORDCC instruction  with register operands.}
      \label{fig:ajit:xnordcc:insn}
    \end{figure}
  \end{center}
  \begin{itemize}
  \item []\textbf{XNORDCC}: same as XNORD, but with Instr[13]=0 (i=0), and
    Instr[5]=1.
  \item []\textbf{Syntax}: ``\texttt{xnordcc  SrcReg1, SrcReg2, DestReg}''.
  \item []\textbf{Semantics}: rd(pair) $\leftarrow$ rs1(pair) $\hat{~}$
    rs2(pair), sets Z.
  \end{itemize}

  Hence the SPARC bit layout of this instruction is:

  \begin{tabular}[h]{lclcl}
    Macro to set   &=&  \verb|F4(x, y, z, b)|     &in& \verb|sparc.h|     \\
    Macro to reset &=&  \verb|F4(~x, ~y, ~z, ~b)| &in& \verb|sparc.h|     \\
    x              &=& 0x2                        &in& \verb|OP(x) | \\
    y              &=& 0x17                       &in& \verb|OP3(y) | \\
    z              &=& 0x0                        &in& \verb|F3I(z) | \\
    b              &=& 0x1                        &in& \verb|OP_AJIT_BIT_5(a) |
  \end{tabular}

\item \textbf{ANDD}:\\
  \begin{center}
    \begin{figure}[h]
      \centering
      \epsfxsize=.8\linewidth
      \epsffile{../figs/andd-ajit-insn-32-bit-layout.eps}
      \caption{The AJIT ANDD instruction  with register operands.}
      \label{fig:ajit:andd:insn}
    \end{figure}
  \end{center}
  \begin{itemize}
  \item []\textbf{ANDD}: same as AND, but with Instr[13]=0 (i=0), and
    Instr[5]=1.
  \item []\textbf{Syntax}: ``\texttt{andd  SrcReg1, SrcReg2, DestReg}''.
  \item []\textbf{Semantics}: rd(pair) $\leftarrow$ rs1(pair) $\cdot$ rs2(pair).
  \end{itemize}

  Hence the SPARC bit layout of this instruction is:

  \begin{tabular}[h]{lclcl}
    Macro to set   &=&  \verb|F4(x, y, z, b)|     &in& \verb|sparc.h|     \\
    Macro to reset &=&  \verb|F4(~x, ~y, ~z, ~b)| &in& \verb|sparc.h|     \\
    x              &=& 0x2                        &in& \verb|OP(x) | \\
    y              &=& 0x01                       &in& \verb|OP3(y) | \\
    z              &=& 0x0                        &in& \verb|F3I(z) | \\
    b              &=& 0x1                        &in& \verb|OP_AJIT_BIT_5(a) |
  \end{tabular}

\item \textbf{ANDDCC}:\\
  \begin{center}
    \begin{figure}[h]
      \centering
      \epsfxsize=.8\linewidth
      \epsffile{../figs/anddcc-ajit-insn-32-bit-layout.eps}
      \caption{The AJIT ANDDCC instruction  with register operands.}
      \label{fig:ajit:anddcc:insn}
    \end{figure}
  \end{center}
  \begin{itemize}
  \item []\textbf{ANDDCC}: same as ANDCC, but with Instr[13]=0 (i=0), and
    Instr[5]=1.
  \item []\textbf{Syntax}: ``\texttt{anddcc  SrcReg1, SrcReg2, DestReg}''.
  \item []\textbf{Semantics}: rd(pair) $\leftarrow$ rs1(pair) $\cdot$
    rs2(pair), sets Z.
  \end{itemize}

  Hence the SPARC bit layout of this instruction is:

  \begin{tabular}[h]{lclcl}
    Macro to set   &=&  \verb|F4(x, y, z, b)|     &in& \verb|sparc.h|     \\
    Macro to reset &=&  \verb|F4(~x, ~y, ~z, ~b)| &in& \verb|sparc.h|     \\
    x              &=& 0x2                        &in& \verb|OP(x) | \\
    y              &=& 0x11                       &in& \verb|OP3(y) | \\
    z              &=& 0x0                        &in& \verb|F3I(z) | \\
    b              &=& 0x1                        &in& \verb|OP_AJIT_BIT_5(a) |
  \end{tabular}

\item \textbf{ANDDN}:\\
  \begin{center}
    \begin{figure}[h]
      \centering
      \epsfxsize=.8\linewidth
      \epsffile{../figs/anddn-ajit-insn-32-bit-layout.eps}
      \caption{The AJIT ANDDN instruction  with register operands.}
      \label{fig:ajit:anddn:insn}
    \end{figure}
  \end{center}
  \begin{itemize}
  \item []\textbf{ANDDN}: same as ANDN, but with Instr[13]=0 (i=0), and
    Instr[5]=1.
  \item []\textbf{Syntax}: ``\texttt{anddn  SrcReg1, SrcReg2, DestReg}''.
  \item []\textbf{Semantics}: rd(pair) $\leftarrow$ rs1(pair) $\cdot$ ($\sim$rs2(pair)).
  \end{itemize}

  Hence the SPARC bit layout of this instruction is:

  \begin{tabular}[h]{lclcl}
    Macro to set   &=&  \verb|F4(x, y, z, b)|     &in& \verb|sparc.h|     \\
    Macro to reset &=&  \verb|F4(~x, ~y, ~z, ~b)| &in& \verb|sparc.h|     \\
    x              &=& 0x2                        &in& \verb|OP(x) | \\
    y              &=& 0x05                       &in& \verb|OP3(y) | \\
    z              &=& 0x0                        &in& \verb|F3I(z) | \\
    b              &=& 0x1                        &in& \verb|OP_AJIT_BIT_5(a) |
  \end{tabular}

\item \textbf{ANDDNCC}:\\
  \begin{center}
    \begin{figure}[h]
      \centering
      \epsfxsize=.8\linewidth
      \epsffile{../figs/anddncc-ajit-insn-32-bit-layout.eps}
      \caption{The AJIT ANDDNCC instruction  with register operands.}
      \label{fig:ajit:anddncc:insn}
    \end{figure}
  \end{center}
  \begin{itemize}
  \item []\textbf{ANDDNCC}: same as ANDN, but with Instr[13]=0 (i=0), and
    Instr[5]=1.
  \item []\textbf{Syntax}: ``\texttt{anddncc  SrcReg1, SrcReg2, DestReg}''.
  \item []\textbf{Semantics}: rd(pair) $\leftarrow$ rs1(pair) $\cdot$
    ($\sim$rs2(pair)), sets Z.
  \end{itemize}

  Hence the SPARC bit layout of this instruction is:

  \begin{tabular}[h]{lclcl}
    Macro to set   &=&  \verb|F4(x, y, z, b)|     &in& \verb|sparc.h|     \\
    Macro to reset &=&  \verb|F4(~x, ~y, ~z, ~b)| &in& \verb|sparc.h|     \\
    x              &=& 0x2                        &in& \verb|OP(x) | \\
    y              &=& 0x15                       &in& \verb|OP3(y) | \\
    z              &=& 0x0                        &in& \verb|F3I(z) | \\
    b              &=& 0x1                        &in& \verb|OP_AJIT_BIT_5(a) |
  \end{tabular}

  The AJIT bit  (insn[5]) is set internally by  \texttt{F4}, and hence
  there are only three arguments.

\end{enumerate}

%%% Local Variables:
%%% mode: latex
%%% TeX-master: t
%%% End:

%\subsubsection{Integer Unit Extensions Summary}
\label{sec:int:unit:extns:summary}

\begin{itemize}
  % \subsubsection{Addition and subtraction instructions:}
\label{sec:add:sub:insn:impl}
\begin{enumerate}
\item \textbf{ADDD}:\\
  \begin{center}
    \begin{tabular}[p]{|c|c|l|l|l|}
      \hline
      \textbf{Start} & \textbf{End} & \textbf{Range} & \textbf{Meaning} &
                                                                          \textbf{New Meaning}\\
      \hline
      0 & 4 & 32 & Source register 2, rs2 & No change \\
      5 & 12 & -- & \textbf{unused} & \textbf{Set bit 5 to ``1''} \\
      13 & 13 & 0,1 & The \textbf{i} bit & \textbf{Set i to ``0''} \\
      14 & 18 & 32 & Source register 1, rs1 & No change \\
      19 & 24 & 000000 & ``\textbf{op3}'' & No change \\
      25 & 29 & 32 & Destination register, rd & No change \\
      30 & 31 & 4 & Always ``10'' & No change \\
      \hline
    \end{tabular}
  \end{center}
  \begin{itemize}
  \item []\textbf{ADDD}: same as ADD, but with Instr[13]=0 (i=0), and
    Instr[5]=1.
  \item []\textbf{Syntax}: ``\texttt{addd  SrcReg1, SrcReg2, DestReg}''.
  \item []\textbf{Semantics}: rd(pair) $\leftarrow$ rs1(pair) + rs2(pair).
  \end{itemize}
  Bits layout:
\begin{verbatim}
    Offsets      : 31       24 23       16  15        8   7        0
    Bit layout   :  XXXX  XXXX  XXXX  XXXX   XXXX  XXXX   XXXX  XXXX
    Insn Bits    :  10       0  0000  0        0            1       
    Destination  :    DD  DDD                                       
    Source 1     :                     SSS   SS
    Source 2     :                                           S  SSSS
    Unused (0)   :                              U  UUUU   UU        
    Final layout :  10DD  DDD0  0000  0SSS   SS0U  UUUU   UU1S  SSSS
\end{verbatim}

  Hence the SPARC bit layout of this instruction is:

  \begin{tabular}[h]{lclcl}
    Macro to set  &=& \texttt{F4(x, y, z)} &in& \texttt{sparc.h}     \\
    Macro to reset  &=& \texttt{INVF4(x, y, z)} &in& \texttt{sparc.h}     \\
    x &=& 0x2      &in& \texttt{OP(x)  /* ((x) \& 0x3)  $<<$ 30 */} \\
    y &=& 0x00     &in& \texttt{OP3(y) /* ((y) \& 0x3f) $<<$ 19 */} \\
    z &=& 0x0      &in& \texttt{F3I(z) /* ((z) \& 0x1)  $<<$ 13 */} \\
    a &=& 0x1      &in& \texttt{OP\_AJIT\_BIT(a) /* ((a) \& 0x1)  $<<$ 5 */}
  \end{tabular}

  The AJIT bit  (insn[5]) is set internally by  \texttt{F4}, and hence
  there are only three arguments.

\item \textbf{ADDDCC}:\\
  \begin{center}
    \begin{tabular}[p]{|c|c|l|l|l|}
      \hline
      \textbf{Start} & \textbf{End} & \textbf{Range} & \textbf{Meaning} &
                                                                          \textbf{New Meaning}\\
      \hline
      0 & 4 & 32 & Source register 2, rs2 & No change \\
      5 & 12 & -- & \textbf{unused} & \textbf{Set bit 5 to ``1''} \\
      13 & 13 & 0,1 & The \textbf{i} bit & \textbf{Set i to ``0''} \\
      14 & 18 & 32 & Source register 1, rs1 & No change \\
      19 & 24 & 010000 & ``\textbf{op3}'' & No change \\
      25 & 29 & 32 & Destination register, rd & No change \\
      30 & 31 & 4 & Always ``10'' & No change \\
      \hline
    \end{tabular}
  \end{center}
  New addition:
  \begin{itemize}
  \item []\textbf{ADDDCC}: same as ADDCC, but with Instr[13]=0 (i=0), and
    Instr[5]=1.
  \item []\textbf{Syntax}: ``\texttt{adddcc  SrcReg1, SrcReg2, DestReg}''.
  \item []\textbf{Semantics}: rd(pair) $\leftarrow$ rs1(pair) + rs2(pair), set Z,N
  \end{itemize}
  Bits layout:
\begin{verbatim}
    Offsets      : 31       24 23       16  15        8   7        0
    Bit layout   :  XXXX  XXXX  XXXX  XXXX   XXXX  XXXX   XXXX  XXXX
    Insn Bits    :  10       0  1000  0        0            1       
    Destination  :    DD  DDD                                       
    Source 1     :                     SSS   SS
    Source 2     :                                           S  SSSS
    Unused (0)   :                              U  UUUU   UU        
    Final layout :  10DD  DDD0  1000  0SSS   SS0U  UUUU   UU1S  SSSS
\end{verbatim}

  Hence the SPARC bit layout of this instruction is:

  \begin{tabular}[h]{lclcl}
    Macro to set  &=& \texttt{F4(x, y, z)} &in& \texttt{sparc.h}     \\
    Macro to reset  &=& \texttt{INVF4(x, y, z)} &in& \texttt{sparc.h}     \\
    x &=& 0x2      &in& \texttt{OP(x)  /* ((x) \& 0x3)  $<<$ 30 */} \\
    y &=& 0x10     &in& \texttt{OP3(y) /* ((y) \& 0x3f) $<<$ 19 */} \\
    z &=& 0x0      &in& \texttt{F3I(z) /* ((z) \& 0x1)  $<<$ 13 */} \\
    a &=& 0x1      &in& \texttt{OP\_AJIT\_BIT(a) /* ((a) \& 0x1)  $<<$ 5 */}
  \end{tabular}

  The AJIT bit  (insn[5]) is set internally by  \texttt{F4}, and hence
  there are only three arguments.

\item \textbf{SUBD}:\\
  \begin{center}
    \begin{tabular}[p]{|c|c|l|l|l|}
      \hline
      \textbf{Start} & \textbf{End} & \textbf{Range} & \textbf{Meaning} &
                                                                          \textbf{New Meaning}\\
      \hline
      0 & 4 & 32 & Source register 2, rs2 & No change \\
      5 & 12 & -- & \textbf{unused} & \textbf{Set bit 5 to ``1''} \\
      13 & 13 & 0,1 & The \textbf{i} bit & \textbf{Set i to ``0''} \\
      14 & 18 & 32 & Source register 1, rs1 & No change \\
      19 & 24 & 000100 & ``\textbf{op3}'' & No change \\
      25 & 29 & 32 & Destination register, rd & No change \\
      30 & 31 & 4 & Always ``10'' & No change \\
      \hline
    \end{tabular}
  \end{center}
  New addition:
  \begin{itemize}
  \item []\textbf{SUBD}: same as SUB, but with Instr[13]=0 (i=0), and
    Instr[5]=1.
  \item []\textbf{Syntax}: ``\texttt{subd  SrcReg1, SrcReg2, DestReg}''.
  \item []\textbf{Semantics}: rd(pair) $\leftarrow$ rs1(pair) - rs2(pair).
  \end{itemize}
  Bits layout:
\begin{verbatim}
    Offsets      : 31       24 23       16  15        8   7        0
    Bit layout   :  XXXX  XXXX  XXXX  XXXX   XXXX  XXXX   XXXX  XXXX
    Insn Bits    :  10       0  0010  0        0            1       
    Destination  :    DD  DDD                                       
    Source 1     :                     SSS   SS
    Source 2     :                                           S  SSSS
    Unused (0)   :                              U  UUUU   UU        
    Final layout :  10DD  DDD0  0010  0SSS   SS0U  UUUU   UU1S  SSSS
\end{verbatim}

  Hence the SPARC bit layout of this instruction is:

  \begin{tabular}[h]{lclcl}
    Macro to set  &=& \texttt{F4(x, y, z)} &in& \texttt{sparc.h}     \\
    Macro to reset  &=& \texttt{INVF4(x, y, z)} &in& \texttt{sparc.h}     \\
    x &=& 0x2      &in& \texttt{OP(x)  /* ((x) \& 0x3)  $<<$ 30 */} \\
    y &=& 0x04     &in& \texttt{OP3(y) /* ((y) \& 0x3f) $<<$ 19 */} \\
    z &=& 0x0      &in& \texttt{F3I(z) /* ((z) \& 0x1)  $<<$ 13 */} \\
    a &=& 0x1      &in& \texttt{OP\_AJIT\_BIT(a) /* ((a) \& 0x1)  $<<$ 5 */}
  \end{tabular}

  The AJIT bit  (insn[5]) is set internally by  \texttt{F4}, and hence
  there are only three arguments.

\item \textbf{SUBDCC}:\\
  \begin{center}
    \begin{tabular}[p]{|c|c|l|l|l|}
      \hline
      \textbf{Start} & \textbf{End} & \textbf{Range} & \textbf{Meaning} &
                                                                          \textbf{New Meaning}\\
      \hline
      0 & 4 & 32 & Source register 2, rs2 & No change \\
      5 & 12 & -- & \textbf{unused} & \textbf{Set bit 5 to ``1''} \\
      13 & 13 & 0,1 & The \textbf{i} bit & \textbf{Set i to ``0''} \\
      14 & 18 & 32 & Source register 1, rs1 & No change \\
      19 & 24 & 010100 & ``\textbf{op3}'' & No change \\
      25 & 29 & 32 & Destination register, rd & No change \\
      30 & 31 & 4 & Always ``10'' & No change \\
      \hline
    \end{tabular}
  \end{center}
  New addition:
  \begin{itemize}
  \item []\textbf{SUBDCC}: same as SUBCC, but with Instr[13]=0 (i=0), and
    Instr[5]=1.
  \item []\textbf{Syntax}: ``\texttt{subdcc  SrcReg1, SrcReg2, DestReg}''.
  \item []\textbf{Semantics}: rd(pair) $\leftarrow$ rs1(pair) - rs2(pair), set Z,N
  \end{itemize}
  Bits layout:
\begin{verbatim}
    Offsets      : 31       24 23       16  15        8   7        0
    Bit layout   :  XXXX  XXXX  XXXX  XXXX   XXXX  XXXX   XXXX  XXXX
    Insn Bits    :  10       0  1010  0        0            1       
    Destination  :    DD  DDD                                       
    Source 1     :                     SSS   SS
    Source 2     :                                           S  SSSS
    Unused (0)   :                              U  UUUU   UU        
    Final layout :  10DD  DDD0  1010  0SSS   SS0U  UUUU   UU1S  SSSS
\end{verbatim}

  Hence the SPARC bit layout of this instruction is:

  \begin{tabular}[h]{lclcl}
    Macro to set  &=& \texttt{F4(x, y, z)} &in& \texttt{sparc.h}     \\
    Macro to reset  &=& \texttt{INVF4(x, y, z)} &in& \texttt{sparc.h}     \\
    x &=& 0x2      &in& \texttt{OP(x)  /* ((x) \& 0x3)  $<<$ 30 */} \\
    y &=& 0x14     &in& \texttt{OP3(y) /* ((y) \& 0x3f) $<<$ 19 */} \\
    z &=& 0x0      &in& \texttt{F3I(z) /* ((z) \& 0x1)  $<<$ 13 */} \\
    a &=& 0x1      &in& \texttt{OP\_AJIT\_BIT(a) /* ((a) \& 0x1)  $<<$ 5 */}
  \end{tabular}

  The AJIT bit  (insn[5]) is set internally by  \texttt{F4}, and hence
  there are only three arguments.
\end{enumerate}

\item {Addition and subtraction instructions:}\\
  \begin{enumerate}
  \item \textbf{ADDD}:\\
    \begin{tabular}[h]{lclcl}
      Macro to set  &=& \texttt{F4(x, y, z)} &in& \texttt{sparc.h}     \\
      Macro to reset  &=& \texttt{INVF4(x, y, z)} &in& \texttt{sparc.h}     \\
      x &=& 0x2      &in& \texttt{OP(x)  /* ((x) \& 0x3)  $<<$ 30 */} \\
      y &=& 0x00     &in& \texttt{OP3(y) /* ((y) \& 0x3f) $<<$ 19 */} \\
      z &=& 0x0      &in& \texttt{F3I(z) /* ((z) \& 0x1)  $<<$ 13 */} \\
      a &=& 0x1      &in& \texttt{OP\_AJIT\_BIT(a) /* ((a) \& 0x1)  $<<$ 5 */}
    \end{tabular}

    The AJIT bit  (insn[5]) is set internally by  \texttt{F4}, and hence
    there are only three arguments.

  \item \textbf{ADDDCC}:\\
    \begin{tabular}[h]{lclcl}
      Macro to set  &=& \texttt{F4(x, y, z)} &in& \texttt{sparc.h}     \\
      Macro to reset  &=& \texttt{INVF4(x, y, z)} &in& \texttt{sparc.h}     \\
      x &=& 0x2      &in& \texttt{OP(x)  /* ((x) \& 0x3)  $<<$ 30 */} \\
      y &=& 0x10     &in& \texttt{OP3(y) /* ((y) \& 0x3f) $<<$ 19 */} \\
      z &=& 0x0      &in& \texttt{F3I(z) /* ((z) \& 0x1)  $<<$ 13 */} \\
      a &=& 0x1      &in& \texttt{OP\_AJIT\_BIT(a) /* ((a) \& 0x1)  $<<$ 5 */}
    \end{tabular}

    The AJIT bit  (insn[5]) is set internally by  \texttt{F4}, and hence
    there are only three arguments.

  \item \textbf{SUBD}:\\
    \begin{tabular}[h]{lclcl}
      Macro to set  &=& \texttt{F4(x, y, z)} &in& \texttt{sparc.h}     \\
      Macro to reset  &=& \texttt{INVF4(x, y, z)} &in& \texttt{sparc.h}     \\
      x &=& 0x2      &in& \texttt{OP(x)  /* ((x) \& 0x3)  $<<$ 30 */} \\
      y &=& 0x04     &in& \texttt{OP3(y) /* ((y) \& 0x3f) $<<$ 19 */} \\
      z &=& 0x0      &in& \texttt{F3I(z) /* ((z) \& 0x1)  $<<$ 13 */} \\
      a &=& 0x1      &in& \texttt{OP\_AJIT\_BIT(a) /* ((a) \& 0x1)  $<<$ 5 */}
    \end{tabular}

    The AJIT bit  (insn[5]) is set internally by  \texttt{F4}, and hence
    there are only three arguments.

  \item \textbf{SUBDCC}:\\
    \begin{tabular}[h]{lclcl}
      Macro to set  &=& \texttt{F4(x, y, z)} &in& \texttt{sparc.h}     \\
      Macro to reset  &=& \texttt{INVF4(x, y, z)} &in& \texttt{sparc.h}     \\
      x &=& 0x2      &in& \texttt{OP(x)  /* ((x) \& 0x3)  $<<$ 30 */} \\
      y &=& 0x14     &in& \texttt{OP3(y) /* ((y) \& 0x3f) $<<$ 19 */} \\
      z &=& 0x0      &in& \texttt{F3I(z) /* ((z) \& 0x1)  $<<$ 13 */} \\
      a &=& 0x1      &in& \texttt{OP\_AJIT\_BIT(a) /* ((a) \& 0x1)  $<<$ 5 */}
    \end{tabular}

    The AJIT bit  (insn[5]) is set internally by  \texttt{F4}, and hence
    there are only three arguments.
  \end{enumerate}

  % \subsubsection{Multiplication and division instructions:}
\label{sec:mul:div:insn:impl}
\begin{enumerate}
\item \textbf{UMULD}: Unsigned Integer Multiply AJIT, no immediate
  version (i.e. i is always 0).\\
	\textbf{NOTE:} The \emph{suggested} mnemonic ``umuld'' conflicts with a mnemonic of the same name for another sparc architecture (other than v8).   Hence we change it to: ``\textbf{umuldaj}'' in the implementation, but not in the documentation below.

 This conflict occurs despite forcing the GNU assembler to assemble for v8 only via the command line switch ``-Av8''! It appears that forcing the assembler to use v8 is not universally applied throughout the assembler code. 
  \begin{center}
    \begin{tabular}[p]{|c|c|l|l|l|}
      \hline
      \textbf{Start} & \textbf{End} & \textbf{Range} & \textbf{Meaning} &
                                                                          \textbf{New Meaning}\\
      \hline
      0 & 4 & 32 & Source register 2, rs2 & No change \\
      5 & 12 & -- & \textbf{unused} & \textbf{Set bit 5 to ``1''} \\
      13 & 13 & 0,1 & The \textbf{i} bit & \textbf{Set i to ``0''} \\
      14 & 18 & 32 & Source register 1, rs1 & No change \\
      19 & 24 & 001010 & ``\textbf{op3}'' & No change \\
      25 & 29 & 32 & Destination register, rd & No change \\
      30 & 31 & 4 & Always ``10'' & No change \\
      \hline
    \end{tabular}
  \end{center}
  \begin{itemize}
  \item []\textbf{UMULD}: same as UMUL, but with Instr[13]=0 (i=0), and
    Instr[5]=1.
  \item []\textbf{Syntax}: ``\texttt{umuld  SrcReg1, SrcReg2, DestReg}''.
  \item []\textbf{Semantics}: rd(pair) $\leftarrow$ rs1(pair) * rs2(pair).
  \end{itemize}
  Bits layout:
\begin{verbatim}
    Offsets      : 31       24 23       16  15        8   7        0
    Bit layout   :  XXXX  XXXX  XXXX  XXXX   XXXX  XXXX   XXXX  XXXX
    Insn Bits    :  10       0  0101  0        0            1       
    Destination  :    DD  DDD                                       
    Source 1     :                     SSS   SS
    Source 2     :                                           S  SSSS
    Unused (0)   :                              U  UUUU   UU        
    Final layout :  10DD  DDD0  0101  0SSS   SS0U  UUUU   UU1S  SSSS
\end{verbatim}

  Hence the SPARC bit layout of this instruction is:

  \begin{tabular}[h]{lclcl}
    Macro to set  &=& \texttt{F4(x, y, z)} &in& \texttt{sparc.h}     \\
    Macro to reset  &=& \texttt{INVF4(x, y, z)} &in& \texttt{sparc.h}     \\
    x &=& 0x2      &in& \texttt{OP(x)  /* ((x) \& 0x3)  $<<$ 30 */} \\
    y &=& 0x0A     &in& \texttt{OP3(y) /* ((y) \& 0x3f) $<<$ 19 */} \\
    z &=& 0x0      &in& \texttt{F3I(z) /* ((z) \& 0x1)  $<<$ 13 */} \\
    a &=& 0x1      &in& \texttt{OP\_AJIT\_BIT(a) /* ((a) \& 0x1)  $<<$ 5 */}
  \end{tabular}

  The AJIT bit  (insn[5]) is set internally by  \texttt{F4}, and hence
  there are only three arguments.

\item \textbf{UMULDCC}:\\
  \begin{center}
    \begin{tabular}[p]{|c|c|l|l|l|}
      \hline
      \textbf{Start} & \textbf{End} & \textbf{Range} & \textbf{Meaning} &
                                                                          \textbf{New Meaning}\\
      \hline
      0 & 4 & 32 & Source register 2, rs2 & No change \\
      5 & 12 & -- & \textbf{unused} & \textbf{Set bit 5 to ``1''} \\
      13 & 13 & 0,1 & The \textbf{i} bit & \textbf{Set i to ``0''} \\
      14 & 18 & 32 & Source register 1, rs1 & No change \\
      19 & 24 & 011010 & ``\textbf{op3}'' & No change \\
      25 & 29 & 32 & Destination register, rd & No change \\
      30 & 31 & 4 & Always ``10'' & No change \\
      \hline
    \end{tabular}
  \end{center}
  New addition:
  \begin{itemize}
  \item []\textbf{UMULDCC}: same as UMULCC, but with Instr[13]=0 (i=0), and
    Instr[5]=1.
  \item []\textbf{Syntax}: ``\texttt{umuldcc  SrcReg1, SrcReg2, DestReg}''.
  \item []\textbf{Semantics}: rd(pair) $\leftarrow$ rs1(pair) * rs2(pair), set Z
  \end{itemize}
  Bits layout:
\begin{verbatim}
    Offsets      : 31       24 23       16  15        8   7        0
    Bit layout   :  XXXX  XXXX  XXXX  XXXX   XXXX  XXXX   XXXX  XXXX
    Insn Bits    :  10       0  1101  0        0            1       
    Destination  :    DD  DDD                                       
    Source 1     :                     SSS   SS
    Source 2     :                                           S  SSSS
    Unused (0)   :                              U  UUUU   UU        
    Final layout :  10DD  DDD0  1101  0SSS   SS0U  UUUU   UU1S  SSSS
\end{verbatim}

  Hence the SPARC bit layout of this instruction is:

  \begin{tabular}[h]{lclcl}
    Macro to set  &=& \texttt{F4(x, y, z)} &in& \texttt{sparc.h}     \\
    Macro to reset  &=& \texttt{INVF4(x, y, z)} &in& \texttt{sparc.h}     \\
    x &=& 0x2      &in& \texttt{OP(x)  /* ((x) \& 0x3)  $<<$ 30 */} \\
    y &=& 0x1A     &in& \texttt{OP3(y) /* ((y) \& 0x3f) $<<$ 19 */} \\
    z &=& 0x0      &in& \texttt{F3I(z) /* ((z) \& 0x1)  $<<$ 13 */} \\
    a &=& 0x1      &in& \texttt{OP\_AJIT\_BIT(a) /* ((a) \& 0x1)  $<<$ 5 */}
  \end{tabular}

  The AJIT bit  (insn[5]) is set internally by  \texttt{F4}, and hence
  there are only three arguments.

\item \textbf{SMULD}: Unsigned Integer Multiply AJIT, no immediate
  version (i.e. i is always 0).\\
  \begin{center}
    \begin{tabular}[p]{|c|c|l|l|l|}
      \hline
      \textbf{Start} & \textbf{End} & \textbf{Range} & \textbf{Meaning} &
                                                                          \textbf{New Meaning}\\
      \hline
      0 & 4 & 32 & Source register 2, rs2 & No change \\
      5 & 12 & -- & \textbf{unused} & \textbf{Set bit 5 to ``1''} \\
      13 & 13 & 0,1 & The \textbf{i} bit & \textbf{Set i to ``0''} \\
      14 & 18 & 32 & Source register 1, rs1 & No change \\
      19 & 24 & 001011 & ``\textbf{op3}'' & No change \\
      25 & 29 & 32 & Destination register, rd & No change \\
      30 & 31 & 4 & Always ``10'' & No change \\
      \hline
    \end{tabular}
  \end{center}
  \begin{itemize}
  \item []\textbf{SMULD}: same as SMUL, but with Instr[13]=0 (i=0), and
    Instr[5]=1.
  \item []\textbf{Syntax}: ``\texttt{smuld  SrcReg1, SrcReg2, DestReg}''.
  \item []\textbf{Semantics}: rd(pair) $\leftarrow$ rs1(pair) *
    rs2(pair) (signed).
  \end{itemize}
  Bits layout:
\begin{verbatim}
    Offsets      : 31       24 23       16  15        8   7        0
    Bit layout   :  XXXX  XXXX  XXXX  XXXX   XXXX  XXXX   XXXX  XXXX
    Insn Bits    :  10       0  0101  1        0            1       
    Destination  :    DD  DDD                                       
    Source 1     :                     SSS   SS
    Source 2     :                                           S  SSSS
    Unused (0)   :                              U  UUUU   UU        
    Final layout :  10DD  DDD0  0101  1SSS   SS0U  UUUU   UU1S  SSSS
\end{verbatim}

  Hence the SPARC bit layout of this instruction is:

  \begin{tabular}[h]{lclcl}
    Macro to set  &=& \texttt{F4(x, y, z)} &in& \texttt{sparc.h}     \\
    Macro to reset  &=& \texttt{INVF4(x, y, z)} &in& \texttt{sparc.h}     \\
    x &=& 0x2      &in& \texttt{OP(x)  /* ((x) \& 0x3)  $<<$ 30 */} \\
    y &=& 0x0B     &in& \texttt{OP3(y) /* ((y) \& 0x3f) $<<$ 19 */} \\
    z &=& 0x0      &in& \texttt{F3I(z) /* ((z) \& 0x1)  $<<$ 13 */} \\
    a &=& 0x1      &in& \texttt{OP\_AJIT\_BIT(a) /* ((a) \& 0x1)  $<<$ 5 */}
  \end{tabular}

  The AJIT bit  (insn[5]) is set internally by  \texttt{F4}, and hence
  there are only three arguments.

\item \textbf{SMULDCC}:\\
  \begin{center}
    \begin{tabular}[p]{|c|c|l|l|l|}
      \hline
      \textbf{Start} & \textbf{End} & \textbf{Range} & \textbf{Meaning} &
                                                                          \textbf{New Meaning}\\
      \hline
      0 & 4 & 32 & Source register 2, rs2 & No change \\
      5 & 12 & -- & \textbf{unused} & \textbf{Set bit 5 to ``1''} \\
      13 & 13 & 0,1 & The \textbf{i} bit & \textbf{Set i to ``0''} \\
      14 & 18 & 32 & Source register 1, rs1 & No change \\
      19 & 24 & 011011 & ``\textbf{op3}'' & No change \\
      25 & 29 & 32 & Destination register, rd & No change \\
      30 & 31 & 4 & Always ``10'' & No change \\
      \hline
    \end{tabular}
  \end{center}
  New addition:
  \begin{itemize}
  \item []\textbf{SMULDCC}: same as SMULCC, but with Instr[13]=0 (i=0), and
    Instr[5]=1.
  \item []\textbf{Syntax}: ``\texttt{smuldcc  SrcReg1, SrcReg2, DestReg}''.
  \item []\textbf{Semantics}: rd(pair) $\leftarrow$ rs1(pair) *
    rs2(pair) (signed), set Z,N,O
  \end{itemize}
  Bits layout:
\begin{verbatim}
    Offsets      : 31       24 23       16  15        8   7        0
    Bit layout   :  XXXX  XXXX  XXXX  XXXX   XXXX  XXXX   XXXX  XXXX
    Insn Bits    :  10       0  1101  1        0            1       
    Destination  :    DD  DDD                                       
    Source 1     :                     SSS   SS
    Source 2     :                                           S  SSSS
    Unused (0)   :                              U  UUUU   UU        
    Final layout :  10DD  DDD0  1101  1SSS   SS0U  UUUU   UU1S  SSSS
\end{verbatim}

  Hence the SPARC bit layout of this instruction is:

  \begin{tabular}[h]{lclcl}
    Macro to set  &=& \texttt{F4(x, y, z)} &in& \texttt{sparc.h}     \\
    Macro to reset  &=& \texttt{INVF4(x, y, z)} &in& \texttt{sparc.h}     \\
    x &=& 0x2      &in& \texttt{OP(x)  /* ((x) \& 0x3)  $<<$ 30 */} \\
    y &=& 0x1B     &in& \texttt{OP3(y) /* ((y) \& 0x3f) $<<$ 19 */} \\
    z &=& 0x0      &in& \texttt{F3I(z) /* ((z) \& 0x1)  $<<$ 13 */} \\
    a &=& 0x1      &in& \texttt{OP\_AJIT\_BIT(a) /* ((a) \& 0x1)  $<<$ 5 */}
  \end{tabular}

  The AJIT bit  (insn[5]) is set internally by  \texttt{F4}, and hence
  there are only three arguments.

\item \textbf{UDIVD}:\\
  \begin{center}
    \begin{tabular}[p]{|c|c|l|l|l|}
      \hline
      \textbf{Start} & \textbf{End} & \textbf{Range} & \textbf{Meaning} &
                                                                          \textbf{New Meaning}\\
      \hline
      0 & 4 & 32 & Source register 2, rs2 & No change \\
      5 & 12 & -- & \textbf{unused} & \textbf{Set bit 5 to ``1''} \\
      13 & 13 & 0,1 & The \textbf{i} bit & \textbf{Set i to ``0''} \\
      14 & 18 & 32 & Source register 1, rs1 & No change \\
      19 & 24 & 001110 & ``\textbf{op3}'' & No change \\
      25 & 29 & 32 & Destination register, rd & No change \\
      30 & 31 & 4 & Always ``10'' & No change \\
      \hline
    \end{tabular}
  \end{center}
  New addition:
  \begin{itemize}
  \item []\textbf{UDIVD}: same as UDIV, but with Instr[13]=0 (i=0), and
    Instr[5]=1.
  \item []\textbf{Syntax}: ``\texttt{udivd  SrcReg1, SrcReg2, DestReg}''.
  \item []\textbf{Semantics}: rd(pair) $\leftarrow$ rs1(pair) / rs2(pair).
  \end{itemize}
  Bits layout:
\begin{verbatim}
    Offsets      : 31       24 23       16  15        8   7        0
    Bit layout   :  XXXX  XXXX  XXXX  XXXX   XXXX  XXXX   XXXX  XXXX
    Insn Bits    :  10       0  0111  0        0            1       
    Destination  :    DD  DDD                                       
    Source 1     :                     SSS   SS
    Source 2     :                                           S  SSSS
    Unused (0)   :                              U  UUUU   UU        
    Final layout :  10DD  DDD0  0111  0SSS   SS0U  UUUU   UU1S  SSSS
\end{verbatim}

  Hence the SPARC bit layout of this instruction is:

  \begin{tabular}[h]{lclcl}
    Macro to set  &=& \texttt{F4(x, y, z)} &in& \texttt{sparc.h}     \\
    Macro to reset  &=& \texttt{INVF4(x, y, z)} &in& \texttt{sparc.h}     \\
    x &=& 0x2      &in& \texttt{OP(x)  /* ((x) \& 0x3)  $<<$ 30 */} \\
    y &=& 0x0E     &in& \texttt{OP3(y) /* ((y) \& 0x3f) $<<$ 19 */} \\
    z &=& 0x0      &in& \texttt{F3I(z) /* ((z) \& 0x1)  $<<$ 13 */} \\
    a &=& 0x1      &in& \texttt{OP\_AJIT\_BIT(a) /* ((a) \& 0x1)  $<<$ 5 */}
  \end{tabular}

  The AJIT bit  (insn[5]) is set internally by  \texttt{F4}, and hence
  there are only three arguments.

\item \textbf{UDIVDCC}:\\
  \begin{center}
    \begin{tabular}[p]{|c|c|l|l|l|}
      \hline
      \textbf{Start} & \textbf{End} & \textbf{Range} & \textbf{Meaning} &
                                                                          \textbf{New Meaning}\\
      \hline
      0 & 4 & 32 & Source register 2, rs2 & No change \\
      5 & 12 & -- & \textbf{unused} & \textbf{Set bit 5 to ``1''} \\
      13 & 13 & 0,1 & The \textbf{i} bit & \textbf{Set i to ``0''} \\
      14 & 18 & 32 & Source register 1, rs1 & No change \\
      19 & 24 & 011110 & ``\textbf{op3}'' & No change \\
      25 & 29 & 32 & Destination register, rd & No change \\
      30 & 31 & 4 & Always ``10'' & No change \\
      \hline
    \end{tabular}
  \end{center}
  New addition:
  \begin{itemize}
  \item []\textbf{UDIVDCC}: same as UDIVCC, but with Instr[13]=0 (i=0), and
    Instr[5]=1.
  \item []\textbf{Syntax}: ``\texttt{udivdcc  SrcReg1, SrcReg2, DestReg}''.
  \item []\textbf{Semantics}: rd(pair) $\leftarrow$ rs1(pair) / rs2(pair), set Z,O
  \end{itemize}
  Bits layout:
\begin{verbatim}
    Offsets      : 31       24 23       16  15        8   7        0
    Bit layout   :  XXXX  XXXX  XXXX  XXXX   XXXX  XXXX   XXXX  XXXX
    Insn Bits    :  10       0  1111  0        0            1       
    Destination  :    DD  DDD                                       
    Source 1     :                     SSS   SS
    Source 2     :                                           S  SSSS
    Unused (0)   :                              U  UUUU   UU        
    Final layout :  10DD  DDD0  1111  0SSS   SS0U  UUUU   UU1S  SSSS
\end{verbatim}

  Hence the SPARC bit layout of this instruction is:

  \begin{tabular}[h]{lclcl}
    Macro to set  &=& \texttt{F4(x, y, z)} &in& \texttt{sparc.h}     \\
    Macro to reset  &=& \texttt{INVF4(x, y, z)} &in& \texttt{sparc.h}     \\
    x &=& 0x2      &in& \texttt{OP(x)  /* ((x) \& 0x3)  $<<$ 30 */} \\
    y &=& 0x1E     &in& \texttt{OP3(y) /* ((y) \& 0x3f) $<<$ 19 */} \\
    z &=& 0x0      &in& \texttt{F3I(z) /* ((z) \& 0x1)  $<<$ 13 */} \\
    a &=& 0x1      &in& \texttt{OP\_AJIT\_BIT(a) /* ((a) \& 0x1)  $<<$ 5 */}
  \end{tabular}

  The AJIT bit  (insn[5]) is set internally by  \texttt{F4}, and hence
  there are only three arguments.

\item \textbf{SDIVD}:\\
  \begin{center}
    \begin{tabular}[p]{|c|c|l|l|l|}
      \hline
      \textbf{Start} & \textbf{End} & \textbf{Range} & \textbf{Meaning} &
                                                                          \textbf{New Meaning}\\
      \hline
      0 & 4 & 32 & Source register 2, rs2 & No change \\
      5 & 12 & -- & \textbf{unused} & \textbf{Set bit 5 to ``1''} \\
      13 & 13 & 0,1 & The \textbf{i} bit & \textbf{Set i to ``0''} \\
      14 & 18 & 32 & Source register 1, rs1 & No change \\
      19 & 24 & 001111 & ``\textbf{op3}'' & No change \\
      25 & 29 & 32 & Destination register, rd & No change \\
      30 & 31 & 4 & Always ``10'' & No change \\
      \hline
    \end{tabular}
  \end{center}
  New addition:
  \begin{itemize}
  \item []\textbf{SDIVD}: same as SDIV, but with Instr[13]=0 (i=0), and
    Instr[5]=1.
  \item []\textbf{Syntax}: ``\texttt{sdivd  SrcReg1, SrcReg2, DestReg}''.
  \item []\textbf{Semantics}: rd(pair) $\leftarrow$ rs1(pair) /
    rs2(pair) (signed).
  \end{itemize}
  Bits layout:
\begin{verbatim}
    Offsets      : 31       24 23       16  15        8   7        0
    Bit layout   :  XXXX  XXXX  XXXX  XXXX   XXXX  XXXX   XXXX  XXXX
    Insn Bits    :  10       0  0111  1        0            1       
    Destination  :    DD  DDD                                       
    Source 1     :                     SSS   SS
    Source 2     :                                           S  SSSS
    Unused (0)   :                              U  UUUU   UU        
    Final layout :  10DD  DDD0  0111  1SSS   SS0U  UUUU   UU1S  SSSS
\end{verbatim}

  Hence the SPARC bit layout of this instruction is:

  \begin{tabular}[h]{lclcl}
    Macro to set  &=& \texttt{F4(x, y, z)} &in& \texttt{sparc.h}     \\
    Macro to reset  &=& \texttt{INVF4(x, y, z)} &in& \texttt{sparc.h}     \\
    x &=& 0x2      &in& \texttt{OP(x)  /* ((x) \& 0x3)  $<<$ 30 */} \\
    y &=& 0x0F     &in& \texttt{OP3(y) /* ((y) \& 0x3f) $<<$ 19 */} \\
    z &=& 0x0      &in& \texttt{F3I(z) /* ((z) \& 0x1)  $<<$ 13 */} \\
    a &=& 0x1      &in& \texttt{OP\_AJIT\_BIT(a) /* ((a) \& 0x1)  $<<$ 5 */}
  \end{tabular}

  The AJIT bit  (insn[5]) is set internally by  \texttt{F4}, and hence
  there are only three arguments.

\item \textbf{SDIVDCC}:\\
  \begin{center}
    \begin{tabular}[p]{|c|c|l|l|l|}
      \hline
      \textbf{Start} & \textbf{End} & \textbf{Range} & \textbf{Meaning} &
                                                                          \textbf{New Meaning}\\
      \hline
      0 & 4 & 32 & Source register 2, rs2 & No change \\
      5 & 12 & -- & \textbf{unused} & \textbf{Set bit 5 to ``1''} \\
      13 & 13 & 0,1 & The \textbf{i} bit & \textbf{Set i to ``0''} \\
      14 & 18 & 32 & Source register 1, rs1 & No change \\
      19 & 24 & 011111 & ``\textbf{op3}'' & No change \\
      25 & 29 & 32 & Destination register, rd & No change \\
      30 & 31 & 4 & Always ``10'' & No change \\
      \hline
    \end{tabular}
  \end{center}
  New addition:
  \begin{itemize}
  \item []\textbf{SDIVDCC}: same as SDIVCC, but with Instr[13]=0 (i=0), and
    Instr[5]=1.
  \item []\textbf{Syntax}: ``\texttt{sdivdcc  SrcReg1, SrcReg2, DestReg}''.
  \item []\textbf{Semantics}: rd(pair) $\leftarrow$ rs1(pair) /
    rs2(pair) (signed), set Z,N,O
  \end{itemize}
  Bits layout:
\begin{verbatim}
    Offsets      : 31       24 23       16  15        8   7        0
    Bit layout   :  XXXX  XXXX  XXXX  XXXX   XXXX  XXXX   XXXX  XXXX
    Insn Bits    :  10       0  1111  1        0            1       
    Destination  :    DD  DDD                                       
    Source 1     :                     SSS   SS
    Source 2     :                                           S  SSSS
    Unused (0)   :                              U  UUUU   UU        
    Final layout :  10DD  DDD0  1111  1SSS   SS0U  UUUU   UU1S  SSSS
\end{verbatim}

  Hence the SPARC bit layout of this instruction is:

  \begin{tabular}[h]{lclcl}
    Macro to set  &=& \texttt{F4(x, y, z)} &in& \texttt{sparc.h}     \\
    Macro to reset  &=& \texttt{INVF4(x, y, z)} &in& \texttt{sparc.h}     \\
    x &=& 0x2      &in& \texttt{OP(x)  /* ((x) \& 0x3)  $<<$ 30 */} \\
    y &=& 0x1F     &in& \texttt{OP3(y) /* ((y) \& 0x3f) $<<$ 19 */} \\
    z &=& 0x0      &in& \texttt{F3I(z) /* ((z) \& 0x1)  $<<$ 13 */} \\
    a &=& 0x1      &in& \texttt{OP\_AJIT\_BIT(a) /* ((a) \& 0x1)  $<<$ 5 */}
  \end{tabular}

  The AJIT bit  (insn[5]) is set internally by  \texttt{F4}, and hence
  there are only three arguments.
\end{enumerate}

%%% Local Variables:
%%% mode: latex
%%% TeX-master: t
%%% End:

\item {Multiplication and division instructions:} \\
  \begin{enumerate}
  \item \textbf{UMULD}: Unsigned Integer Multiply AJIT, no immediate
    version (i.e. i is always 0).\\
    \begin{tabular}[h]{lclcl}
      Macro to set  &=& \texttt{F4(x, y, z)} &in& \texttt{sparc.h}     \\
      Macro to reset  &=& \texttt{INVF4(x, y, z)} &in& \texttt{sparc.h}     \\
      x &=& 0x2      &in& \texttt{OP(x)  /* ((x) \& 0x3)  $<<$ 30 */} \\
      y &=& 0x0A     &in& \texttt{OP3(y) /* ((y) \& 0x3f) $<<$ 19 */} \\
      z &=& 0x0      &in& \texttt{F3I(z) /* ((z) \& 0x1)  $<<$ 13 */} \\
      a &=& 0x1      &in& \texttt{OP\_AJIT\_BIT(a) /* ((a) \& 0x1)  $<<$ 5 */}
    \end{tabular}

    The AJIT bit  (insn[5]) is set internally by  \texttt{F4}, and hence
    there are only three arguments.

  \item \textbf{UMULDCC}:\\
    \begin{tabular}[h]{lclcl}
      Macro to set  &=& \texttt{F4(x, y, z)} &in& \texttt{sparc.h}     \\
      Macro to reset  &=& \texttt{INVF4(x, y, z)} &in& \texttt{sparc.h}     \\
      x &=& 0x2      &in& \texttt{OP(x)  /* ((x) \& 0x3)  $<<$ 30 */} \\
      y &=& 0x1A     &in& \texttt{OP3(y) /* ((y) \& 0x3f) $<<$ 19 */} \\
      z &=& 0x0      &in& \texttt{F3I(z) /* ((z) \& 0x1)  $<<$ 13 */} \\
      a &=& 0x1      &in& \texttt{OP\_AJIT\_BIT(a) /* ((a) \& 0x1)  $<<$ 5 */}
    \end{tabular}

    The AJIT bit  (insn[5]) is set internally by  \texttt{F4}, and hence
    there are only three arguments.

  \item \textbf{SMULD}: Unsigned Integer Multiply AJIT, no immediate
    version (i.e. i is always 0).\\
    \begin{tabular}[h]{lclcl}
      Macro to set  &=& \texttt{F4(x, y, z)} &in& \texttt{sparc.h}     \\
      Macro to reset  &=& \texttt{INVF4(x, y, z)} &in& \texttt{sparc.h}     \\
      x &=& 0x2      &in& \texttt{OP(x)  /* ((x) \& 0x3)  $<<$ 30 */} \\
      y &=& 0x0B     &in& \texttt{OP3(y) /* ((y) \& 0x3f) $<<$ 19 */} \\
      z &=& 0x0      &in& \texttt{F3I(z) /* ((z) \& 0x1)  $<<$ 13 */} \\
      a &=& 0x1      &in& \texttt{OP\_AJIT\_BIT(a) /* ((a) \& 0x1)  $<<$ 5 */}
    \end{tabular}

    The AJIT bit  (insn[5]) is set internally by  \texttt{F4}, and hence
    there are only three arguments.

  \item \textbf{SMULDCC}:\\
    \begin{tabular}[h]{lclcl}
      Macro to set  &=& \texttt{F4(x, y, z)} &in& \texttt{sparc.h}     \\
      Macro to reset  &=& \texttt{INVF4(x, y, z)} &in& \texttt{sparc.h}     \\
      x &=& 0x2      &in& \texttt{OP(x)  /* ((x) \& 0x3)  $<<$ 30 */} \\
      y &=& 0x1B     &in& \texttt{OP3(y) /* ((y) \& 0x3f) $<<$ 19 */} \\
      z &=& 0x0      &in& \texttt{F3I(z) /* ((z) \& 0x1)  $<<$ 13 */} \\
      a &=& 0x1      &in& \texttt{OP\_AJIT\_BIT(a) /* ((a) \& 0x1)  $<<$ 5 */}
    \end{tabular}

    The AJIT bit  (insn[5]) is set internally by  \texttt{F4}, and hence
    there are only three arguments.

  \item \textbf{UDIVD}:\\
    \begin{tabular}[h]{lclcl}
      Macro to set  &=& \texttt{F4(x, y, z)} &in& \texttt{sparc.h}     \\
      Macro to reset  &=& \texttt{INVF4(x, y, z)} &in& \texttt{sparc.h}     \\
      x &=& 0x2      &in& \texttt{OP(x)  /* ((x) \& 0x3)  $<<$ 30 */} \\
      y &=& 0x0E     &in& \texttt{OP3(y) /* ((y) \& 0x3f) $<<$ 19 */} \\
      z &=& 0x0      &in& \texttt{F3I(z) /* ((z) \& 0x1)  $<<$ 13 */} \\
      a &=& 0x1      &in& \texttt{OP\_AJIT\_BIT(a) /* ((a) \& 0x1)  $<<$ 5 */}
    \end{tabular}

    The AJIT bit  (insn[5]) is set internally by  \texttt{F4}, and hence
    there are only three arguments.

  \item \textbf{UDIVDCC}:\\
    \begin{tabular}[h]{lclcl}
      Macro to set  &=& \texttt{F4(x, y, z)} &in& \texttt{sparc.h}     \\
      Macro to reset  &=& \texttt{INVF4(x, y, z)} &in& \texttt{sparc.h}     \\
      x &=& 0x2      &in& \texttt{OP(x)  /* ((x) \& 0x3)  $<<$ 30 */} \\
      y &=& 0x1E     &in& \texttt{OP3(y) /* ((y) \& 0x3f) $<<$ 19 */} \\
      z &=& 0x0      &in& \texttt{F3I(z) /* ((z) \& 0x1)  $<<$ 13 */} \\
      a &=& 0x1      &in& \texttt{OP\_AJIT\_BIT(a) /* ((a) \& 0x1)  $<<$ 5 */}
    \end{tabular}

    The AJIT bit  (insn[5]) is set internally by  \texttt{F4}, and hence
    there are only three arguments.

  \item \textbf{SDIVD}:\\
    \begin{tabular}[h]{lclcl}
      Macro to set  &=& \texttt{F4(x, y, z)} &in& \texttt{sparc.h}     \\
      Macro to reset  &=& \texttt{INVF4(x, y, z)} &in& \texttt{sparc.h}     \\
      x &=& 0x2      &in& \texttt{OP(x)  /* ((x) \& 0x3)  $<<$ 30 */} \\
      y &=& 0x0F     &in& \texttt{OP3(y) /* ((y) \& 0x3f) $<<$ 19 */} \\
      z &=& 0x0      &in& \texttt{F3I(z) /* ((z) \& 0x1)  $<<$ 13 */} \\
      a &=& 0x1      &in& \texttt{OP\_AJIT\_BIT(a) /* ((a) \& 0x1)  $<<$ 5 */}
    \end{tabular}

    The AJIT bit  (insn[5]) is set internally by  \texttt{F4}, and hence
    there are only three arguments.

  \item \textbf{SDIVDCC}:\\
    \begin{tabular}[h]{lclcl}
      Macro to set  &=& \texttt{F4(x, y, z)} &in& \texttt{sparc.h}     \\
      Macro to reset  &=& \texttt{INVF4(x, y, z)} &in& \texttt{sparc.h}     \\
      x &=& 0x2      &in& \texttt{OP(x)  /* ((x) \& 0x3)  $<<$ 30 */} \\
      y &=& 0x1F     &in& \texttt{OP3(y) /* ((y) \& 0x3f) $<<$ 19 */} \\
      z &=& 0x0      &in& \texttt{F3I(z) /* ((z) \& 0x1)  $<<$ 13 */} \\
      a &=& 0x1      &in& \texttt{OP\_AJIT\_BIT(a) /* ((a) \& 0x1)  $<<$ 5 */}
    \end{tabular}

    The AJIT bit  (insn[5]) is set internally by  \texttt{F4}, and hence
    there are only three arguments.
  \end{enumerate}

  % \subsubsection{64 Bit Logical Instructions:}
\label{sec:64:bit:logical:insn:impl}

No immediate mode, i.e. insn[5] $\equiv$ i = 0, always.

\begin{enumerate}
\item \textbf{ORD}:\\
  \begin{center}
    \begin{figure}[h]
      \centering
      \epsfxsize=.8\linewidth
      \epsffile{../figs/ord-ajit-insn-32-bit-layout.eps}
      \caption{The AJIT ORD instruction  with register operands.}
      \label{fig:ajit:ord:insn}
    \end{figure}
  \end{center}
  \begin{itemize}
  \item []\textbf{ORD}: same as OR, but with Instr[13]=0 (i=0), and
    Instr[5]=1.
  \item []\textbf{Syntax}: ``\texttt{ord  SrcReg1, SrcReg2, DestReg}''.
  \item []\textbf{Semantics}: rd(pair) $\leftarrow$ rs1(pair) $\vert$ rs2(pair).
  \end{itemize}

  Hence the SPARC bit layout of this instruction is:

  \begin{tabular}[h]{lclcl}
    Macro to set   &=&  \verb|F4(x, y, z, b)|     &in& \verb|sparc.h|     \\
    Macro to reset &=&  \verb|F4(~x, ~y, ~z, ~b)| &in& \verb|sparc.h|     \\
    x              &=& 0x2                        &in& \verb|OP(x) | \\
    y              &=& 0x02                       &in& \verb|OP3(y) | \\
    z              &=& 0x0                        &in& \verb|F3I(z) | \\
    b              &=& 0x1                        &in& \verb|OP_AJIT_BIT_5(a) |
  \end{tabular}

\item \textbf{ORDCC}:\\
  \begin{center}
    \begin{figure}[h]
      \centering
      \epsfxsize=.8\linewidth
      \epsffile{../figs/ordcc-ajit-insn-32-bit-layout.eps}
      \caption{The AJIT ORDCC instruction  with register operands.}
      \label{fig:ajit:ordcc:insn}
    \end{figure}
  \end{center}
  \begin{itemize}
  \item []\textbf{ORDCC}: same as ORCC, but with Instr[13]=0 (i=0), and
    Instr[5]=1.
  \item []\textbf{Syntax}: ``\texttt{ordcc  SrcReg1, SrcReg2, DestReg}''.
  \item []\textbf{Semantics}: rd(pair) $\leftarrow$ rs1(pair) $\vert$
    rs2(pair), sets Z.
  \end{itemize}

  Hence the SPARC bit layout of this instruction is:

  \begin{tabular}[h]{lclcl}
    Macro to set   &=&  \verb|F4(x, y, z, b)|     &in& \verb|sparc.h|     \\
    Macro to reset &=&  \verb|F4(~x, ~y, ~z, ~b)| &in& \verb|sparc.h|     \\
    x              &=& 0x2                        &in& \verb|OP(x) | \\
    y              &=& 0x12                       &in& \verb|OP3(y) | \\
    z              &=& 0x0                        &in& \verb|F3I(z) | \\
    b              &=& 0x1                        &in& \verb|OP_AJIT_BIT_5(a) |
  \end{tabular}

\item \textbf{ORDN}:\\
  \begin{center}
    \begin{figure}[h]
      \centering
      \epsfxsize=.8\linewidth
      \epsffile{../figs/ordn-ajit-insn-32-bit-layout.eps}
      \caption{The AJIT ORDN instruction  with register operands.}
      \label{fig:ajit:ordn:insn}
    \end{figure}
  \end{center}
  \begin{itemize}
  \item []\textbf{ORDN}: same as ORN, but with Instr[13]=0 (i=0), and
    Instr[5]=1.
  \item []\textbf{Syntax}: ``\texttt{ordn  SrcReg1, SrcReg2, DestReg}''.
  \item []\textbf{Semantics}: rd(pair) $\leftarrow$ rs1(pair) $\vert$ ($\sim$rs2(pair)).
  \end{itemize}

  Hence the SPARC bit layout of this instruction is:

  \begin{tabular}[h]{lclcl}
    Macro to set   &=&  \verb|F4(x, y, z, b)|     &in& \verb|sparc.h|     \\
    Macro to reset &=&  \verb|F4(~x, ~y, ~z, ~b)| &in& \verb|sparc.h|     \\
    x              &=& 0x2                        &in& \verb|OP(x) | \\
    y              &=& 0x06                       &in& \verb|OP3(y) | \\
    z              &=& 0x0                        &in& \verb|F3I(z) | \\
    b              &=& 0x1                        &in& \verb|OP_AJIT_BIT_5(a) |
  \end{tabular}

\item \textbf{ORDNCC}:\\
  \begin{center}
    \begin{figure}[h]
      \centering
      \epsfxsize=.8\linewidth
      \epsffile{../figs/ordncc-ajit-insn-32-bit-layout.eps}
      \caption{The AJIT ORDNCC instruction  with register operands.}
      \label{fig:ajit:ordncc:insn}
    \end{figure}
  \end{center}
  \begin{itemize}
  \item []\textbf{ORDNCC}: same as ORN, but with Instr[13]=0 (i=0), and
    Instr[5]=1.
  \item []\textbf{Syntax}: ``\texttt{ordncc  SrcReg1, SrcReg2, DestReg}''.
  \item []\textbf{Semantics}: rd(pair) $\leftarrow$ rs1(pair) $\vert$
    ($\sim$rs2(pair)), sets Z.
  \end{itemize}

  Hence the SPARC bit layout of this instruction is:

  \begin{tabular}[h]{lclcl}
    Macro to set   &=&  \verb|F4(x, y, z, b)|     &in& \verb|sparc.h|     \\
    Macro to reset &=&  \verb|F4(~x, ~y, ~z, ~b)| &in& \verb|sparc.h|     \\
    x              &=& 0x2                        &in& \verb|OP(x) | \\
    y              &=& 0x16                       &in& \verb|OP3(y) | \\
    z              &=& 0x0                        &in& \verb|F3I(z) | \\
    b              &=& 0x1                        &in& \verb|OP_AJIT_BIT_5(a) |
  \end{tabular}

\item \textbf{XORDCC}:\\
  \begin{center}
    \begin{figure}[h]
      \centering
      \epsfxsize=.8\linewidth
      \epsffile{../figs/xordcc-ajit-insn-32-bit-layout.eps}
      \caption{The AJIT XORDCC instruction  with register operands.}
      \label{fig:ajit:xordcc:insn}
    \end{figure}
  \end{center}
  \begin{itemize}
  \item []\textbf{XORDCC}: same as XORCC, but with Instr[13]=0 (i=0), and
    Instr[5]=1.
  \item []\textbf{Syntax}: ``\texttt{xordcc  SrcReg1, SrcReg2, DestReg}''.
  \item []\textbf{Semantics}: rd(pair) $\leftarrow$ rs1(pair) $\hat{~}$
    rs2(pair), sets Z.
  \end{itemize}

  Hence the SPARC bit layout of this instruction is:

  \begin{tabular}[h]{lclcl}
    Macro to set   &=&  \verb|F4(x, y, z, b)|     &in& \verb|sparc.h|     \\
    Macro to reset &=&  \verb|F4(~x, ~y, ~z, ~b)| &in& \verb|sparc.h|     \\
    x              &=& 0x2                        &in& \verb|OP(x) | \\
    y              &=& 0x13                       &in& \verb|OP3(y) | \\
    z              &=& 0x0                        &in& \verb|F3I(z) | \\
    b              &=& 0x1                        &in& \verb|OP_AJIT_BIT_5(a) |
  \end{tabular}

  The AJIT bit  (insn[5]) is set internally by  \texttt{F4}, and hence
  there are only three arguments.

\item \textbf{XNORD}:\\
  \begin{center}
    \begin{figure}[h]
      \centering
      \epsfxsize=.8\linewidth
      \epsffile{../figs/xnord-ajit-insn-32-bit-layout.eps}
      \caption{The AJIT XNORD instruction  with register operands.}
      \label{fig:ajit:xnord:insn}
    \end{figure}
  \end{center}
  \begin{itemize}
  \item []\textbf{XNORD}: same as XNOR, but with Instr[13]=0 (i=0), and
    Instr[5]=1.
  \item []\textbf{Syntax}: ``\texttt{xnordcc  SrcReg1, SrcReg2, DestReg}''.
  \item []\textbf{Semantics}: rd(pair) $\leftarrow$ rs1(pair) $\hat{~}$
    rs2(pair).
  \end{itemize}

  Hence the SPARC bit layout of this instruction is:

  \begin{tabular}[h]{lclcl}
    Macro to set   &=&  \verb|F4(x, y, z, b)|     &in& \verb|sparc.h|     \\
    Macro to reset &=&  \verb|F4(~x, ~y, ~z, ~b)| &in& \verb|sparc.h|     \\
    x              &=& 0x2                        &in& \verb|OP(x) | \\
    y              &=& 0x07                       &in& \verb|OP3(y) | \\
    z              &=& 0x0                        &in& \verb|F3I(z) | \\
    b              &=& 0x1                        &in& \verb|OP_AJIT_BIT_5(a) |
  \end{tabular}

\item \textbf{XNORDCC}:\\
  \begin{center}
    \begin{figure}[h]
      \centering
      \epsfxsize=.8\linewidth
      \epsffile{../figs/xnordcc-ajit-insn-32-bit-layout.eps}
      \caption{The AJIT XNORDCC instruction  with register operands.}
      \label{fig:ajit:xnordcc:insn}
    \end{figure}
  \end{center}
  \begin{itemize}
  \item []\textbf{XNORDCC}: same as XNORD, but with Instr[13]=0 (i=0), and
    Instr[5]=1.
  \item []\textbf{Syntax}: ``\texttt{xnordcc  SrcReg1, SrcReg2, DestReg}''.
  \item []\textbf{Semantics}: rd(pair) $\leftarrow$ rs1(pair) $\hat{~}$
    rs2(pair), sets Z.
  \end{itemize}

  Hence the SPARC bit layout of this instruction is:

  \begin{tabular}[h]{lclcl}
    Macro to set   &=&  \verb|F4(x, y, z, b)|     &in& \verb|sparc.h|     \\
    Macro to reset &=&  \verb|F4(~x, ~y, ~z, ~b)| &in& \verb|sparc.h|     \\
    x              &=& 0x2                        &in& \verb|OP(x) | \\
    y              &=& 0x17                       &in& \verb|OP3(y) | \\
    z              &=& 0x0                        &in& \verb|F3I(z) | \\
    b              &=& 0x1                        &in& \verb|OP_AJIT_BIT_5(a) |
  \end{tabular}

\item \textbf{ANDD}:\\
  \begin{center}
    \begin{figure}[h]
      \centering
      \epsfxsize=.8\linewidth
      \epsffile{../figs/andd-ajit-insn-32-bit-layout.eps}
      \caption{The AJIT ANDD instruction  with register operands.}
      \label{fig:ajit:andd:insn}
    \end{figure}
  \end{center}
  \begin{itemize}
  \item []\textbf{ANDD}: same as AND, but with Instr[13]=0 (i=0), and
    Instr[5]=1.
  \item []\textbf{Syntax}: ``\texttt{andd  SrcReg1, SrcReg2, DestReg}''.
  \item []\textbf{Semantics}: rd(pair) $\leftarrow$ rs1(pair) $\cdot$ rs2(pair).
  \end{itemize}

  Hence the SPARC bit layout of this instruction is:

  \begin{tabular}[h]{lclcl}
    Macro to set   &=&  \verb|F4(x, y, z, b)|     &in& \verb|sparc.h|     \\
    Macro to reset &=&  \verb|F4(~x, ~y, ~z, ~b)| &in& \verb|sparc.h|     \\
    x              &=& 0x2                        &in& \verb|OP(x) | \\
    y              &=& 0x01                       &in& \verb|OP3(y) | \\
    z              &=& 0x0                        &in& \verb|F3I(z) | \\
    b              &=& 0x1                        &in& \verb|OP_AJIT_BIT_5(a) |
  \end{tabular}

\item \textbf{ANDDCC}:\\
  \begin{center}
    \begin{figure}[h]
      \centering
      \epsfxsize=.8\linewidth
      \epsffile{../figs/anddcc-ajit-insn-32-bit-layout.eps}
      \caption{The AJIT ANDDCC instruction  with register operands.}
      \label{fig:ajit:anddcc:insn}
    \end{figure}
  \end{center}
  \begin{itemize}
  \item []\textbf{ANDDCC}: same as ANDCC, but with Instr[13]=0 (i=0), and
    Instr[5]=1.
  \item []\textbf{Syntax}: ``\texttt{anddcc  SrcReg1, SrcReg2, DestReg}''.
  \item []\textbf{Semantics}: rd(pair) $\leftarrow$ rs1(pair) $\cdot$
    rs2(pair), sets Z.
  \end{itemize}

  Hence the SPARC bit layout of this instruction is:

  \begin{tabular}[h]{lclcl}
    Macro to set   &=&  \verb|F4(x, y, z, b)|     &in& \verb|sparc.h|     \\
    Macro to reset &=&  \verb|F4(~x, ~y, ~z, ~b)| &in& \verb|sparc.h|     \\
    x              &=& 0x2                        &in& \verb|OP(x) | \\
    y              &=& 0x11                       &in& \verb|OP3(y) | \\
    z              &=& 0x0                        &in& \verb|F3I(z) | \\
    b              &=& 0x1                        &in& \verb|OP_AJIT_BIT_5(a) |
  \end{tabular}

\item \textbf{ANDDN}:\\
  \begin{center}
    \begin{figure}[h]
      \centering
      \epsfxsize=.8\linewidth
      \epsffile{../figs/anddn-ajit-insn-32-bit-layout.eps}
      \caption{The AJIT ANDDN instruction  with register operands.}
      \label{fig:ajit:anddn:insn}
    \end{figure}
  \end{center}
  \begin{itemize}
  \item []\textbf{ANDDN}: same as ANDN, but with Instr[13]=0 (i=0), and
    Instr[5]=1.
  \item []\textbf{Syntax}: ``\texttt{anddn  SrcReg1, SrcReg2, DestReg}''.
  \item []\textbf{Semantics}: rd(pair) $\leftarrow$ rs1(pair) $\cdot$ ($\sim$rs2(pair)).
  \end{itemize}

  Hence the SPARC bit layout of this instruction is:

  \begin{tabular}[h]{lclcl}
    Macro to set   &=&  \verb|F4(x, y, z, b)|     &in& \verb|sparc.h|     \\
    Macro to reset &=&  \verb|F4(~x, ~y, ~z, ~b)| &in& \verb|sparc.h|     \\
    x              &=& 0x2                        &in& \verb|OP(x) | \\
    y              &=& 0x05                       &in& \verb|OP3(y) | \\
    z              &=& 0x0                        &in& \verb|F3I(z) | \\
    b              &=& 0x1                        &in& \verb|OP_AJIT_BIT_5(a) |
  \end{tabular}

\item \textbf{ANDDNCC}:\\
  \begin{center}
    \begin{figure}[h]
      \centering
      \epsfxsize=.8\linewidth
      \epsffile{../figs/anddncc-ajit-insn-32-bit-layout.eps}
      \caption{The AJIT ANDDNCC instruction  with register operands.}
      \label{fig:ajit:anddncc:insn}
    \end{figure}
  \end{center}
  \begin{itemize}
  \item []\textbf{ANDDNCC}: same as ANDN, but with Instr[13]=0 (i=0), and
    Instr[5]=1.
  \item []\textbf{Syntax}: ``\texttt{anddncc  SrcReg1, SrcReg2, DestReg}''.
  \item []\textbf{Semantics}: rd(pair) $\leftarrow$ rs1(pair) $\cdot$
    ($\sim$rs2(pair)), sets Z.
  \end{itemize}

  Hence the SPARC bit layout of this instruction is:

  \begin{tabular}[h]{lclcl}
    Macro to set   &=&  \verb|F4(x, y, z, b)|     &in& \verb|sparc.h|     \\
    Macro to reset &=&  \verb|F4(~x, ~y, ~z, ~b)| &in& \verb|sparc.h|     \\
    x              &=& 0x2                        &in& \verb|OP(x) | \\
    y              &=& 0x15                       &in& \verb|OP3(y) | \\
    z              &=& 0x0                        &in& \verb|F3I(z) | \\
    b              &=& 0x1                        &in& \verb|OP_AJIT_BIT_5(a) |
  \end{tabular}

  The AJIT bit  (insn[5]) is set internally by  \texttt{F4}, and hence
  there are only three arguments.

\end{enumerate}

%%% Local Variables:
%%% mode: latex
%%% TeX-master: t
%%% End:

\item {64 Bit Logical Instructions:}\\

  No immediate mode, i.e. insn[5] $\equiv$ i = 0, always.

  \begin{enumerate}
  \item \textbf{ORD}:\\
    \begin{tabular}[h]{lclcl}
      Macro to set  &=& \texttt{F4(x, y, z)} &in& \texttt{sparc.h}     \\
      Macro to reset  &=& \texttt{INVF4(x, y, z)} &in& \texttt{sparc.h}     \\
      x &=& 0x2      &in& \texttt{OP(x)  /* ((x) \& 0x3)  $<<$ 30 */} \\
      y &=& 0x02     &in& \texttt{OP3(y) /* ((y) \& 0x3f) $<<$ 19 */} \\
      z &=& 0x0      &in& \texttt{F3I(z) /* ((z) \& 0x1)  $<<$ 13 */} \\
      a &=& 0x1      &in& \texttt{OP\_AJIT\_BIT(a) /* ((a) \& 0x1)  $<<$ 5 */}
    \end{tabular}

    The AJIT bit  (insn[5]) is set internally by  \texttt{F4}, and hence
    there are only three arguments.

  \item \textbf{ORDCC}:\\
    \begin{tabular}[h]{lclcl}
      Macro to set  &=& \texttt{F4(x, y, z)} &in& \texttt{sparc.h}     \\
      Macro to reset  &=& \texttt{INVF4(x, y, z)} &in& \texttt{sparc.h}     \\
      x &=& 0x2      &in& \texttt{OP(x)  /* ((x) \& 0x3)  $<<$ 30 */} \\
      y &=& 0x12     &in& \texttt{OP3(y) /* ((y) \& 0x3f) $<<$ 19 */} \\
      z &=& 0x0      &in& \texttt{F3I(z) /* ((z) \& 0x1)  $<<$ 13 */} \\
      a &=& 0x1      &in& \texttt{OP\_AJIT\_BIT(a) /* ((a) \& 0x1)  $<<$ 5 */}
    \end{tabular}

    The AJIT bit  (insn[5]) is set internally by  \texttt{F4}, and hence
    there are only three arguments.

  \item \textbf{ORDN}:\\
    \begin{tabular}[h]{lclcl}
      Macro to set  &=& \texttt{F4(x, y, z)} &in& \texttt{sparc.h}     \\
      Macro to reset  &=& \texttt{INVF4(x, y, z)} &in& \texttt{sparc.h}     \\
      x &=& 0x2      &in& \texttt{OP(x)  /* ((x) \& 0x3)  $<<$ 30 */} \\
      y &=& 0x06     &in& \texttt{OP3(y) /* ((y) \& 0x3f) $<<$ 19 */} \\
      z &=& 0x0      &in& \texttt{F3I(z) /* ((z) \& 0x1)  $<<$ 13 */} \\
      a &=& 0x1      &in& \texttt{OP\_AJIT\_BIT(a) /* ((a) \& 0x1)  $<<$ 5 */}
    \end{tabular}

    The AJIT bit  (insn[5]) is set internally by  \texttt{F4}, and hence
    there are only three arguments.

  \item \textbf{ORDNCC}:\\
    \begin{tabular}[h]{lclcl}
      Macro to set  &=& \texttt{F4(x, y, z)} &in& \texttt{sparc.h}     \\
      Macro to reset  &=& \texttt{INVF4(x, y, z)} &in& \texttt{sparc.h}     \\
      x &=& 0x2      &in& \texttt{OP(x)  /* ((x) \& 0x3)  $<<$ 30 */} \\
      y &=& 0x16     &in& \texttt{OP3(y) /* ((y) \& 0x3f) $<<$ 19 */} \\
      z &=& 0x0      &in& \texttt{F3I(z) /* ((z) \& 0x1)  $<<$ 13 */} \\
      a &=& 0x1      &in& \texttt{OP\_AJIT\_BIT(a) /* ((a) \& 0x1)  $<<$ 5 */}
    \end{tabular}

    The AJIT bit  (insn[5]) is set internally by  \texttt{F4}, and hence
    there are only three arguments.

  \item \textbf{XORDCC}:\\
    \begin{tabular}[h]{lclcl}
      Macro to set  &=& \texttt{F4(x, y, z)} &in& \texttt{sparc.h}     \\
      Macro to reset  &=& \texttt{INVF4(x, y, z)} &in& \texttt{sparc.h}     \\
      x &=& 0x2      &in& \texttt{OP(x)  /* ((x) \& 0x3)  $<<$ 30 */} \\
      y &=& 0x13     &in& \texttt{OP3(y) /* ((y) \& 0x3f) $<<$ 19 */} \\
      z &=& 0x0      &in& \texttt{F3I(z) /* ((z) \& 0x1)  $<<$ 13 */} \\
      a &=& 0x1      &in& \texttt{OP\_AJIT\_BIT(a) /* ((a) \& 0x1)  $<<$ 5 */}
    \end{tabular}

    The AJIT bit  (insn[5]) is set internally by  \texttt{F4}, and hence
    there are only three arguments.

  \item \textbf{XNORD}:\\
    \begin{tabular}[h]{lclcl}
      Macro to set  &=& \texttt{F4(x, y, z)} &in& \texttt{sparc.h}     \\
      Macro to reset  &=& \texttt{INVF4(x, y, z)} &in& \texttt{sparc.h}     \\
      x &=& 0x2      &in& \texttt{OP(x)  /* ((x) \& 0x3)  $<<$ 30 */} \\
      y &=& 0x07     &in& \texttt{OP3(y) /* ((y) \& 0x3f) $<<$ 19 */} \\
      z &=& 0x0      &in& \texttt{F3I(z) /* ((z) \& 0x1)  $<<$ 13 */} \\
      a &=& 0x1      &in& \texttt{OP\_AJIT\_BIT(a) /* ((a) \& 0x1)  $<<$ 5 */}
    \end{tabular}

    The AJIT bit  (insn[5]) is set internally by  \texttt{F4}, and hence
    there are only three arguments.
    
  \item \textbf{XNORDCC}:\\
    \begin{tabular}[h]{lclcl}
      Macro to set  &=& \texttt{F4(x, y, z)} &in& \texttt{sparc.h}     \\
      Macro to reset  &=& \texttt{INVF4(x, y, z)} &in& \texttt{sparc.h}     \\
      x &=& 0x2      &in& \texttt{OP(x)  /* ((x) \& 0x3)  $<<$ 30 */} \\
      y &=& 0x07     &in& \texttt{OP3(y) /* ((y) \& 0x3f) $<<$ 19 */} \\
      z &=& 0x0      &in& \texttt{F3I(z) /* ((z) \& 0x1)  $<<$ 13 */} \\
      a &=& 0x1      &in& \texttt{OP\_AJIT\_BIT(a) /* ((a) \& 0x1)  $<<$ 5 */}
    \end{tabular}

    The AJIT bit  (insn[5]) is set internally by  \texttt{F4}, and hence
    there are only three arguments.
    
  \item \textbf{ANDD}:\\
    \begin{tabular}[h]{lclcl}
      Macro to set  &=& \texttt{F4(x, y, z)} &in& \texttt{sparc.h}     \\
      Macro to reset  &=& \texttt{INVF4(x, y, z)} &in& \texttt{sparc.h}     \\
      x &=& 0x2      &in& \texttt{OP(x)  /* ((x) \& 0x3)  $<<$ 30 */} \\
      y &=& 0x01     &in& \texttt{OP3(y) /* ((y) \& 0x3f) $<<$ 19 */} \\
      z &=& 0x0      &in& \texttt{F3I(z) /* ((z) \& 0x1)  $<<$ 13 */} \\
      a &=& 0x1      &in& \texttt{OP\_AJIT\_BIT(a) /* ((a) \& 0x1)  $<<$ 5 */}
    \end{tabular}

    The AJIT bit  (insn[5]) is set internally by  \texttt{F4}, and hence
    there are only three arguments.

  \item \textbf{ANDDCC}:\\
    \begin{tabular}[h]{lclcl}
      Macro to set  &=& \texttt{F4(x, y, z)} &in& \texttt{sparc.h}     \\
      Macro to reset  &=& \texttt{INVF4(x, y, z)} &in& \texttt{sparc.h}     \\
      x &=& 0x2      &in& \texttt{OP(x)  /* ((x) \& 0x3)  $<<$ 30 */} \\
      y &=& 0x11     &in& \texttt{OP3(y) /* ((y) \& 0x3f) $<<$ 19 */} \\
      z &=& 0x0      &in& \texttt{F3I(z) /* ((z) \& 0x1)  $<<$ 13 */} \\
      a &=& 0x1      &in& \texttt{OP\_AJIT\_BIT(a) /* ((a) \& 0x1)  $<<$ 5 */}
    \end{tabular}

    The AJIT bit  (insn[5]) is set internally by  \texttt{F4}, and hence
    there are only three arguments.

  \item \textbf{ANDDN}:\\
    \begin{tabular}[h]{lclcl}
      Macro to set  &=& \texttt{F4(x, y, z)} &in& \texttt{sparc.h}     \\
      Macro to reset  &=& \texttt{INVF4(x, y, z)} &in& \texttt{sparc.h}     \\
      x &=& 0x2      &in& \texttt{OP(x)  /* ((x) \& 0x3)  $<<$ 30 */} \\
      y &=& 0x05     &in& \texttt{OP3(y) /* ((y) \& 0x3f) $<<$ 19 */} \\
      z &=& 0x0      &in& \texttt{F3I(z) /* ((z) \& 0x1)  $<<$ 13 */} \\
      a &=& 0x1      &in& \texttt{OP\_AJIT\_BIT(a) /* ((a) \& 0x1)  $<<$ 5 */}
    \end{tabular}

    The AJIT bit  (insn[5]) is set internally by  \texttt{F4}, and hence
    there are only three arguments.

  \item \textbf{ANDDNCC}:\\
    \begin{tabular}[h]{lclcl}
      Macro to set  &=& \texttt{F4(x, y, z)} &in& \texttt{sparc.h}     \\
      Macro to reset  &=& \texttt{INVF4(x, y, z)} &in& \texttt{sparc.h}     \\
      x &=& 0x2      &in& \texttt{OP(x)  /* ((x) \& 0x3)  $<<$ 30 */} \\
      y &=& 0x15     &in& \texttt{OP3(y) /* ((y) \& 0x3f) $<<$ 19 */} \\
      z &=& 0x0      &in& \texttt{F3I(z) /* ((z) \& 0x1)  $<<$ 13 */} \\
      a &=& 0x1      &in& \texttt{OP\_AJIT\_BIT(a) /* ((a) \& 0x1)  $<<$ 5 */}
    \end{tabular}

    The AJIT bit  (insn[5]) is set internally by  \texttt{F4}, and hence
    there are only three arguments.

  \end{enumerate}
% \subsubsection{Shift instructions:}
\label{sec:shift:insn:impl}
The shift  family of instructions  of AJIT  may each be  considered to
have  two versions:  a direct  count version  and a  register indirect
count version.  In the direct count  version the shift count is a part
of the  instruction bits.   In the indirect  count version,  the shift
count is  found on the  register specified by  the bit pattern  in the
instruction  bits.   The direct  count  version  is specified  by  the
14$^{th}$  bit, i.e.  insn[13]  (bit  number 13  in  the  0 based  bit
numbering scheme), being set to 1.  If insn[13] is 0 then the register
indirect version is specified.

Similar to the addition and subtraction instructions, the shift family
of instructions of  SPARC V8 also do  not use bits from 5  to 12 (both
inclusive).  The AJIT processor uses bits  5 and 6.  In particular bit
6 is always 1.   Bit 5 may be used in the direct  version giving a set
of 6 bits  available for specifying the shift count.   The shift count
can have  a maximum  value of  64.  Bit  5 is  unused in  the register
indirect version, and is always 0 in that case.

These instructions  are therefore  worked out  below in  two different
sets: the direct and the register indirect ones.
\begin{enumerate}
\item The direct versions  are given by insn[13] = 1.  The 6 bit shift
  count  is directly  specified  in the  instruction bits.   Therefore
  insn[5:0] specify the  shift count.  insn[6] =  1, distinguishes the
  AJIT version from the SPARC V8 version.
  \begin{enumerate}
  \item \textbf{SLLD}:\\
    \begin{center}
      \begin{tabular}[p]{|c|c|l|p{.25\textwidth}|p{.3\textwidth}|}
        \hline
        \textbf{Start} & \textbf{End} & \textbf{Range} & \textbf{Meaning} & \textbf{New Meaning}\\
        \hline
        0 & 4 & 32 & Source register 2, rs2 & Lowest 5 bits of shift count \\
        \hline
        5 & 12 & -- & \textbf{Unused. Set to 0 by software.} &
                                        \begin{minipage}[h]{1.0\linewidth}
                                          \begin{itemize}
                                          \item \textbf{Use bit 5
                                              to specify the msb of
                                              shift count.}
                                          \item \textbf{Use bit 6 to
                                              distinguish AJIT from
                                              SPARC V8.}
                                          \item \textbf{Set bits 7:12
                                              to 0.}
                                          \end{itemize}
                                        \end{minipage}
        \\
        \hline
        13 & 13 & 0,1 & The \textbf{i} bit & \textbf{Set i to ``1''} \\
        14 & 18 & 32 & Source register 1, rs1 & No change \\
        19 & 24 & 100101 & ``\textbf{op3}'' & No change \\
        25 & 29 & 32 & Destination register, rd & No change \\
        30 & 31 & 4 & Always ``10'' & No change \\
        \hline
      \end{tabular}
    \end{center}
    \begin{itemize}
    \item []\textbf{SLLD}: same as SLL, but with Instr[13]=0 (i=0),
      and Instr[5]=1.
    \item []\textbf{Syntax}: ``\texttt{slld SrcReg1, 6BitShiftCnt,
        DestReg}''. \\
      (\textbf{Note:} In an assembly language program, when the second
      argument is a number, we have direct mode.  A register number is
      prefixed with  ``r'', and hence the  syntax itself distinguished
      between   direct  and   register   indirect   version  of   this
      instruction.)
    \item []\textbf{Semantics}: rd(pair) $\leftarrow$ rs1(pair) $<<$
      shift count.
    \end{itemize}
    Bits layout:
\begin{verbatim}
    Offsets      : 31       24 23       16  15        8   7        0
    Bit layout   :  XXXX  XXXX  XXXX  XXXX   XXXX  XXXX   XXXX  XXXX
    Insn Bits    :  10       1  0010  1        1           1        
    Destination  :    DD  DDD                                       
    Source 1     :                     SSS   SS
    Source 2     :                                           S  SSSS
    Unused (0)   :                              U  UUUU   UU        
    Final layout :  10DD  DDD1  0010  1SSS   SS1U  UUUU   U1II  IIII
\end{verbatim}

    This will need another macro that sets bits 5 and 6. Let's call it
    \texttt{OP\_AJIT\_BITS\_5\_AND\_6}.   Hence the  SPARC bit  layout of  this
    instruction is:

    \begin{tabular}[h]{lclcl}
      Macro to set  &=& \texttt{F5(x, y, z)} &in& \texttt{sparc.h}     \\
      Macro to reset  &=& \texttt{INVF5(x, y, z)} &in& \texttt{sparc.h}     \\
      x &=& 0x2      &in& \texttt{OP(x)  /* ((x) \& 0x3)  $<<$ 30 */} \\
      y &=& 0x25     &in& \texttt{OP3(y) /* ((y) \& 0x3f) $<<$ 19 */} \\
      z &=& 0x1      &in& \texttt{F3I(z) /* ((z) \& 0x1)  $<<$ 13 */} \\
      a &=& 0x2      &in& \texttt{OP\_AJIT\_BITS\_5\_AND\_6(a) /* ((a) \& 0x3  $<<$ 6 */}
    \end{tabular}

    The AJIT bits (insn[6:5]) is  set or reset internally by \texttt{F5}
    (just  like  in  \texttt{F4}),  and   hence  there  are  only  three
    arguments.

  \item \textbf{SRLD}:\\
    \begin{center}
      \begin{tabular}[p]{|c|c|l|l|p{.35\textwidth}|}
        \hline
        \textbf{Start} & \textbf{End} & \textbf{Range} & \textbf{Meaning} & \textbf{New Meaning}\\
        \hline
        0 & 4 & 32 & Source register 2, rs2 & Lowest 5 bits of shift count \\
        \hline
        5 & 12 & -- & \textbf{unused} &
                                        \begin{minipage}[h]{1.0\linewidth}
                                          \begin{itemize}
                                          \item \textbf{Use bit 5
                                              to specify the msb of
                                              shift count.}
                                          \item \textbf{Use bit 6 to
                                              distinguish AJIT from
                                              SPARC V8.}
                                          \end{itemize}
                                        \end{minipage}
        \\
        \hline
        13 & 13 & 0,1 & The \textbf{i} bit & \textbf{Set i to ``1''} \\
        14 & 18 & 32 & Source register 1, rs1 & No change \\
        19 & 24 & 100110 & ``\textbf{op3}'' & No change \\
        25 & 29 & 32 & Destination register, rd & No change \\
        30 & 31 & 4 & Always ``10'' & No change \\
        \hline
      \end{tabular}
    \end{center}
    \begin{itemize}
    \item []\textbf{SRLD}: same as SRL, but with Instr[13]=0 (i=0),
      and Instr[5]=1.
    \item []\textbf{Syntax}: ``\texttt{sral SrcReg1, 6BitShiftCnt,
        DestReg}''. \\
      (\textbf{Note:} In an assembly language program, when the second
      argument is a number, we have direct mode.  A register number is
      prefixed with  ``r'', and hence the  syntax itself distinguished
      between   direct  and   register   indirect   version  of   this
      instruction.)
    \item []\textbf{Semantics}: rd(pair) $\leftarrow$ rs1(pair) $>>$
      shift count.
    \end{itemize}
    Bits layout:
\begin{verbatim}
    Offsets      : 31       24 23       16  15        8   7        0
    Bit layout   :  XXXX  XXXX  XXXX  XXXX   XXXX  XXXX   XXXX  XXXX
    Insn Bits    :  10       1  0011  0        1           1        
    Destination  :    DD  DDD                                       
    Source 1     :                     SSS   SS
    Source 2     :                                           S  SSSS
    Unused (0)   :                              U  UUUU   UU        
    Final layout :  10DD  DDD1  0011  0SSS   SS1U  UUUU   U1II  IIII
\end{verbatim}

    This will need another macro that sets bits 5 and 6. Let's call it
    \texttt{OP\_AJIT\_BITS\_5\_AND\_6}.   Hence the  SPARC bit  layout of  this
    instruction is:

    \begin{tabular}[h]{lclcl}
      Macro to set  &=& \texttt{F5(x, y, z)} &in& \texttt{sparc.h}     \\
      Macro to reset  &=& \texttt{INVF5(x, y, z)} &in& \texttt{sparc.h}     \\
      x &=& 0x2      &in& \texttt{OP(x)  /* ((x) \& 0x3)  $<<$ 30 */} \\
      y &=& 0x26     &in& \texttt{OP3(y) /* ((y) \& 0x3f) $<<$ 19 */} \\
      z &=& 0x1      &in& \texttt{F3I(z) /* ((z) \& 0x1)  $<<$ 13 */} \\
      a &=& 0x2      &in& \texttt{OP\_AJIT\_BITS\_5\_AND\_6(a) /* ((a) \& 0x3  $<<$ 6 */}
    \end{tabular}

    The AJIT bits (insn[6:5]) is  set or reset internally by \texttt{F5}
    (just  like  in  \texttt{F4}),  and   hence  there  are  only  three
    arguments.
    
  \item \textbf{SRAD}:\\
    \begin{center}
      \begin{tabular}[p]{|c|c|l|l|p{.35\textwidth}|}
        \hline
        \textbf{Start} & \textbf{End} & \textbf{Range} & \textbf{Meaning} & \textbf{New Meaning}\\
        \hline
        0 & 4 & 32 & Source register 2, rs2 & Lowest 5 bits of shift count \\
        \hline
        5 & 12 & -- & \textbf{unused} &
                                        \begin{minipage}[h]{1.0\linewidth}
                                          \begin{itemize}
                                          \item \textbf{Use bit 5
                                              to specify the msb of
                                              shift count.}
                                          \item \textbf{Use bit 6 to
                                              distinguish AJIT from
                                              SPARC V8.}
                                          \end{itemize}
                                        \end{minipage}
        \\
        \hline
        13 & 13 & 0,1 & The \textbf{i} bit & \textbf{Set i to ``1''} \\
        14 & 18 & 32 & Source register 1, rs1 & No change \\
        19 & 24 & 100111 & ``\textbf{op3}'' & No change \\
        25 & 29 & 32 & Destination register, rd & No change \\
        30 & 31 & 4 & Always ``10'' & No change \\
        \hline
      \end{tabular}
    \end{center}
    \begin{itemize}
    \item []\textbf{SRAD}: same as SRA, but with Instr[13]=0 (i=0),
      and Instr[5]=1.
    \item []\textbf{Syntax}: ``\texttt{srad SrcReg1, 6BitShiftCnt,
        DestReg}''. \\
      (\textbf{Note:} In an assembly language program, when the second
      argument is a number, we have direct mode.  A register number is
      prefixed with  ``r'', and hence the  syntax itself distinguished
      between   direct  and   register   indirect   version  of   this
      instruction.)
    \item []\textbf{Semantics}: rd(pair) $\leftarrow$ rs1(pair) $>>$
      shift count (with sign extension).
    \end{itemize}
    Bits layout:
\begin{verbatim}
    Offsets      : 31       24 23       16  15        8   7        0
    Bit layout   :  XXXX  XXXX  XXXX  XXXX   XXXX  XXXX   XXXX  XXXX
    Insn Bits    :  10       1  0011  1        1           1        
    Destination  :    DD  DDD                                       
    Source 1     :                     SSS   SS
    Source 2     :                                           S  SSSS
    Unused (0)   :                              U  UUUU   UU        
    Final layout :  10DD  DDD1  0011  1SSS   SS1U  UUUU   U1II  IIII
\end{verbatim}

    This will need another macro that sets bits 5 and 6. Let's call it
    \texttt{OP\_AJIT\_BITS\_5\_AND\_6}.   Hence the  SPARC bit  layout of  this
    instruction is:

    \begin{tabular}[h]{lclcl}
      Macro to set  &=& \texttt{F5(x, y, z)} &in& \texttt{sparc.h}     \\
      Macro to reset  &=& \texttt{INVF5(x, y, z)} &in& \texttt{sparc.h}     \\
      x &=& 0x2      &in& \texttt{OP(x)  /* ((x) \& 0x3)  $<<$ 30 */} \\
      y &=& 0x27     &in& \texttt{OP3(y) /* ((y) \& 0x3f) $<<$ 19 */} \\
      z &=& 0x1      &in& \texttt{F3I(z) /* ((z) \& 0x1)  $<<$ 13 */} \\
      a &=& 0x2      &in& \texttt{OP\_AJIT\_BITS\_5\_AND\_6(a) /* ((a) \& 0x3  $<<$ 6 */}
    \end{tabular}

    The AJIT bits (insn[6:5]) is  set or reset internally by \texttt{F5}
    (just  like  in  \texttt{F4}),  and   hence  there  are  only  three
    arguments.

  \end{enumerate}
\item The register  indirect versions are given by insn[13]  = 0.  The
  shift count is indirectly specified in the 32 bit register specified
  in instruction bits.  Therefore  insn[4:0] specify the register that
  has the  shift count.  insn[6]  = 1, distinguishes the  AJIT version
  from the SPARC V8 version.  In this case, insn[5] = 0, necessarily.
  \begin{enumerate}
  \item \textbf{SLLD}:\\
    \begin{center}
      \begin{tabular}[p]{|c|c|l|l|p{.35\textwidth}|}
        \hline
        \textbf{Start} & \textbf{End} & \textbf{Range} & \textbf{Meaning} &
                                                                            \textbf{New Meaning}\\
        \hline
        0 & 4 & 32 & Source register 2, rs2 & Register number \\
        \hline
        5 & 12 & -- & \textbf{unused} &
                                        \begin{minipage}[h]{1.0\linewidth}
                                          \begin{itemize}
                                          \item \textbf{Set bit 5 to 0.}
                                          \item \textbf{Use bit 6 to
                                              distinguish AJIT from
                                              SPARC V8.}
                                          \end{itemize}
                                        \end{minipage}
        \\
        \hline
        13 & 13 & 0,1 & The \textbf{i} bit & \textbf{Set i to ``0''} \\
        14 & 18 & 32 & Source register 1, rs1 & No change \\
        19 & 24 & 100101 & ``\textbf{op3}'' & No change \\
        25 & 29 & 32 & Destination register, rd & No change \\
        30 & 31 & 4 & Always ``10'' & No change \\
        \hline
      \end{tabular}
    \end{center}
    \begin{itemize}
    \item []\textbf{SLLD}: same as SLL, but with Instr[13]=0 (i=0),
      and Instr[5]=1.
    \item []\textbf{Syntax}: ``\texttt{slld SrcReg1, SrcReg2,
        DestReg}''.
    \item []\textbf{Semantics}: rd(pair) $\leftarrow$ rs1(pair) $<<$
      shift count register rs2.
    \end{itemize}
    Bits layout:
\begin{verbatim}
    Offsets      : 31       24 23       16  15        8   7        0
    Bit layout   :  XXXX  XXXX  XXXX  XXXX   XXXX  XXXX   XXXX  XXXX
    Insn Bits    :  10       1  0010  1        0           10        
    Destination  :    DD  DDD                                       
    Source 1     :                     SSS   SS
    Source 2     :                                           S  SSSS
    Unused (0)   :                              U  UUUU   UU        
    Final layout :  10DD  DDD1  0010  1SSS   SS0U  UUUU   U10I  IIII
\end{verbatim}

    This will need another macro that sets bits 5 and 6. Let's call it
    \texttt{OP\_AJIT\_BITS\_5\_AND\_6}.   Hence the  SPARC bit  layout of  this
    instruction is:

    \begin{tabular}[h]{lclcl}
      Macro to set  &=& \texttt{F5(x, y, z)} &in& \texttt{sparc.h}     \\
      Macro to reset  &=& \texttt{INVF5(x, y, z)} &in& \texttt{sparc.h}     \\
      x &=& 0x2      &in& \texttt{OP(x)  /* ((x) \& 0x3)  $<<$ 30 */} \\
      y &=& 0x25     &in& \texttt{OP3(y) /* ((y) \& 0x3f) $<<$ 19 */} \\
      z &=& 0x0      &in& \texttt{F3I(z) /* ((z) \& 0x1)  $<<$ 13 */} \\
      a &=& 0x2      &in& \texttt{OP\_AJIT\_BITS\_5\_AND\_6(a) /* ((a) \& 0x3  $<<$ 6 */}
    \end{tabular}

    The AJIT bits (insn[6:5]) is  set or reset internally by \texttt{F5}
    (just  like  in  \texttt{F4}),  and   hence  there  are  only  three
    arguments.

  \item \textbf{SRLD}:\\
    \begin{center}
      \begin{tabular}[p]{|c|c|l|l|p{.35\textwidth}|}
        \hline
        \textbf{Start} & \textbf{End} & \textbf{Range} & \textbf{Meaning} &
                                                                            \textbf{New Meaning}\\
        \hline
        0 & 4 & 32 & Source register 2, rs2 & Register number \\
        \hline
        5 & 12 & -- & \textbf{unused} &
                                        \begin{minipage}[h]{1.0\linewidth}
                                          \begin{itemize}
                                          \item \textbf{Set bit 5 to 0.}
                                          \item \textbf{Use bit 6 to
                                              distinguish AJIT from
                                              SPARC V8.}
                                          \end{itemize}
                                        \end{minipage}
        \\
        \hline
        13 & 13 & 0,1 & The \textbf{i} bit & \textbf{Set i to ``0''} \\
        14 & 18 & 32 & Source register 1, rs1 & No change \\
        19 & 24 & 100110 & ``\textbf{op3}'' & No change \\
        25 & 29 & 32 & Destination register, rd & No change \\
        30 & 31 & 4 & Always ``10'' & No change \\
        \hline
      \end{tabular}
    \end{center}
    \begin{itemize}
    \item []\textbf{SRLD}: same as SRL, but with Instr[13]=0 (i=0),
      and Instr[5]=1.
    \item []\textbf{Syntax}: ``\texttt{slld SrcReg1, SrcReg2,
        DestReg}''.
    \item []\textbf{Semantics}: rd(pair) $\leftarrow$ rs1(pair) $>>$
      shift count register rs2.
    \end{itemize}
    Bits layout:
\begin{verbatim}
    Offsets      : 31       24 23       16  15        8   7        0
    Bit layout   :  XXXX  XXXX  XXXX  XXXX   XXXX  XXXX   XXXX  XXXX
    Insn Bits    :  10       1  0011  0        0           10        
    Destination  :    DD  DDD                                       
    Source 1     :                     SSS   SS
    Source 2     :                                           S  SSSS
    Unused (0)   :                              U  UUUU   UU        
    Final layout :  10DD  DDD1  0011  0SSS   SS0U  UUUU   U10I  IIII
\end{verbatim}

    This will need another macro that sets bits 5 and 6. Let's call it
    \texttt{OP\_AJIT\_BITS\_5\_AND\_6}.   Hence the  SPARC bit  layout of  this
    instruction is:

    \begin{tabular}[h]{lclcl}
      Macro to set  &=& \texttt{F5(x, y, z)} &in& \texttt{sparc.h}     \\
      Macro to reset  &=& \texttt{INVF5(x, y, z)} &in& \texttt{sparc.h}     \\
      x &=& 0x2      &in& \texttt{OP(x)  /* ((x) \& 0x3)  $<<$ 30 */} \\
      y &=& 0x26     &in& \texttt{OP3(y) /* ((y) \& 0x3f) $<<$ 19 */} \\
      z &=& 0x0      &in& \texttt{F3I(z) /* ((z) \& 0x1)  $<<$ 13 */} \\
      a &=& 0x2      &in& \texttt{OP\_AJIT\_BITS\_5\_AND\_6(a) /* ((a) \& 0x3  $<<$ 6 */}
    \end{tabular}

    The AJIT bits (insn[6:5]) is  set or reset internally by \texttt{F5}
    (just  like  in  \texttt{F4}),  and   hence  there  are  only  three
    arguments.

  \item \textbf{SRAD}:\\
    \begin{center}
      \begin{tabular}[p]{|c|c|l|l|p{.35\textwidth}|}
        \hline
        \textbf{Start} & \textbf{End} & \textbf{Range} & \textbf{Meaning} &
                                                                            \textbf{New Meaning}\\
        \hline
        0 & 4 & 32 & Source register 2, rs2 & Register number \\
        \hline
        5 & 12 & -- & \textbf{unused} &
                                        \begin{minipage}[h]{1.0\linewidth}
                                          \begin{itemize}
                                          \item \textbf{Set bit 5 to 0.}
                                          \item \textbf{Use bit 6 to
                                              distinguish AJIT from
                                              SPARC V8.}
                                          \end{itemize}
                                        \end{minipage}
        \\
        \hline
        13 & 13 & 0,1 & The \textbf{i} bit & \textbf{Set i to ``0''} \\
        14 & 18 & 32 & Source register 1, rs1 & No change \\
        19 & 24 & 100101 & ``\textbf{op3}'' & No change \\
        25 & 29 & 32 & Destination register, rd & No change \\
        30 & 31 & 4 & Always ``10'' & No change \\
        \hline
      \end{tabular}
    \end{center}
    \begin{itemize}
    \item []\textbf{SRAD}: same as SRA, but with Instr[13]=0 (i=0),
      and Instr[5]=1.
    \item []\textbf{Syntax}: ``\texttt{slld SrcReg1, SrcReg2,
        DestReg}''.
    \item []\textbf{Semantics}: rd(pair) $\leftarrow$ rs1(pair) $>>$
      shift count register rs2 (with sign extension).
    \end{itemize}
    Bits layout:
\begin{verbatim}
    Offsets      : 31       24 23       16  15        8   7        0
    Bit layout   :  XXXX  XXXX  XXXX  XXXX   XXXX  XXXX   XXXX  XXXX
    Insn Bits    :  10       1  0011  1        0           10        
    Destination  :    DD  DDD                                       
    Source 1     :                     SSS   SS
    Source 2     :                                           S  SSSS
    Unused (0)   :                              U  UUUU   UU        
    Final layout :  10DD  DDD1  0011  1SSS   SS0U  UUUU   U10I  IIII
\end{verbatim}

    This will need another macro that sets bits 5 and 6. Let's call it
    \texttt{OP\_AJIT\_BITS\_5\_AND\_6}.   Hence the  SPARC bit  layout of  this
    instruction is:

    \begin{tabular}[h]{lclcl}
      Macro to set  &=& \texttt{F5(x, y, z)} &in& \texttt{sparc.h}     \\
      Macro to reset  &=& \texttt{INVF5(x, y, z)} &in& \texttt{sparc.h}     \\
      x &=& 0x2      &in& \texttt{OP(x)  /* ((x) \& 0x3)  $<<$ 30 */} \\
      y &=& 0x27     &in& \texttt{OP3(y) /* ((y) \& 0x3f) $<<$ 19 */} \\
      z &=& 0x0      &in& \texttt{F3I(z) /* ((z) \& 0x1)  $<<$ 13 */} \\
      a &=& 0x2      &in& \texttt{OP\_AJIT\_BITS\_5\_AND\_6(a) /* ((a) \& 0x3  $<<$ 6 */}
    \end{tabular}

    The AJIT bits (insn[6:5]) is  set or reset internally by \texttt{F5}
    (just  like  in  \texttt{F4}),  and   hence  there  are  only  three
    arguments.
  \end{enumerate}
\end{enumerate}

\item {Shift instructions:} \\

  The shift  family of instructions  of AJIT  may each be  considered to
  have  two versions:  a direct  count version  and a  register indirect
  count version.  In the direct count  version the shift count is a part
  of the  instruction bits.   In the indirect  count version,  the shift
  count is  found on the  register specified by  the bit pattern  in the
  instruction  bits.   The direct  count  version  is specified  by  the
  14$^{th}$  bit, i.e.  insn[13]  (bit  number 13  in  the  0 based  bit
  numbering scheme), being set to 1.  If insn[13] is 0 then the register
  indirect version is specified.

  Similar to the addition and subtraction instructions, the shift family
  of instructions of  SPARC V8 also do  not use bits from 5  to 12 (both
  inclusive).  The AJIT processor uses bits  5 and 6.  In particular bit
  6 is always 1.   Bit 5 may be used in the direct  version giving a set
  of 6 bits  available for specifying the shift count.   The shift count
  can have  a maximum  value of  64.  Bit  5 is  unused in  the register
  indirect version, and is always 0 in that case.

  These instructions  are therefore  worked out  below in  two different
  sets: the direct and the register indirect ones.
  \begin{enumerate}
  \item The direct versions  are given by insn[13] = 1.  The 6 bit shift
    count  is directly  specified  in the  instruction bits.   Therefore
    insn[5:0] specify the  shift count.  insn[6] =  1, distinguishes the
    AJIT version from the SPARC V8 version.
    \begin{enumerate}
    \item \textbf{SLLD}:\\
      This will need another macro that sets bits 5 and 6. Let's call it
      \texttt{OP\_AJIT\_BIT\_2}.   Hence the  SPARC bit  layout of  this
      instruction is:

      \begin{tabular}[h]{lclcl}
        Macro to set  &=& \texttt{F5(x, y, z)} &in& \texttt{sparc.h}     \\
        Macro to reset  &=& \texttt{INVF5(x, y, z)} &in& \texttt{sparc.h}     \\
        x &=& 0x2      &in& \texttt{OP(x)  /* ((x) \& 0x3)  $<<$ 30 */} \\
        y &=& 0x25     &in& \texttt{OP3(y) /* ((y) \& 0x3f) $<<$ 19 */} \\
        z &=& 0x1      &in& \texttt{F3I(z) /* ((z) \& 0x1)  $<<$ 13 */} \\
        a &=& 0x2      &in& \texttt{OP\_AJIT\_BIT\_2(a) /* ((a) \& 0x3  $<<$ 6 */}
      \end{tabular}

      The AJIT bits (insn[6:5]) is  set or reset internally by \texttt{F5}
      (just  like  in  \texttt{F4}),  and   hence  there  are  only  three
      arguments.

    \item \textbf{SRLD}:\\
      This will need another macro that sets bits 5 and 6. Let's call it
      \texttt{OP\_AJIT\_BIT\_2}.   Hence the  SPARC bit  layout of  this
      instruction is:

      \begin{tabular}[h]{lclcl}
        Macro to set  &=& \texttt{F5(x, y, z)} &in& \texttt{sparc.h}     \\
        Macro to reset  &=& \texttt{INVF5(x, y, z)} &in& \texttt{sparc.h}     \\
        x &=& 0x2      &in& \texttt{OP(x)  /* ((x) \& 0x3)  $<<$ 30 */} \\
        y &=& 0x26     &in& \texttt{OP3(y) /* ((y) \& 0x3f) $<<$ 19 */} \\
        z &=& 0x1      &in& \texttt{F3I(z) /* ((z) \& 0x1)  $<<$ 13 */} \\
        a &=& 0x2      &in& \texttt{OP\_AJIT\_BIT\_2(a) /* ((a) \& 0x3  $<<$ 6 */}
      \end{tabular}

      The AJIT bits (insn[6:5]) is  set or reset internally by \texttt{F5}
      (just  like  in  \texttt{F4}),  and   hence  there  are  only  three
      arguments.
      
    \item \textbf{SRAD}:\\
      This will need another macro that sets bits 5 and 6. Let's call it
      \texttt{OP\_AJIT\_BIT\_2}.   Hence the  SPARC bit  layout of  this
      instruction is:

      \begin{tabular}[h]{lclcl}
        Macro to set  &=& \texttt{F5(x, y, z)} &in& \texttt{sparc.h}     \\
        Macro to reset  &=& \texttt{INVF5(x, y, z)} &in& \texttt{sparc.h}     \\
        x &=& 0x2      &in& \texttt{OP(x)  /* ((x) \& 0x3)  $<<$ 30 */} \\
        y &=& 0x27     &in& \texttt{OP3(y) /* ((y) \& 0x3f) $<<$ 19 */} \\
        z &=& 0x1      &in& \texttt{F3I(z) /* ((z) \& 0x1)  $<<$ 13 */} \\
        a &=& 0x2      &in& \texttt{OP\_AJIT\_BIT\_2(a) /* ((a) \& 0x3  $<<$ 6 */}
      \end{tabular}

      The AJIT bits (insn[6:5]) is  set or reset internally by \texttt{F5}
      (just  like  in  \texttt{F4}),  and   hence  there  are  only  three
      arguments.

    \end{enumerate}
  \item The register  indirect versions are given by insn[13]  = 0.  The
    shift count is indirectly specified in the 32 bit register specified
    in instruction bits.  Therefore  insn[4:0] specify the register that
    has the  shift count.  insn[6]  = 1, distinguishes the  AJIT version
    from the SPARC V8 version.  In this case, insn[5] = 0, necessarily.
    \begin{enumerate}
    \item \textbf{SLLD}:\\
      This will need another macro that sets bits 5 and 6. Let's call it
      \texttt{OP\_AJIT\_BIT\_2}.   Hence the  SPARC bit  layout of  this
      instruction is:

      \begin{tabular}[h]{lclcl}
        Macro to set  &=& \texttt{F5(x, y, z)} &in& \texttt{sparc.h}     \\
        Macro to reset  &=& \texttt{INVF5(x, y, z)} &in& \texttt{sparc.h}     \\
        x &=& 0x2      &in& \texttt{OP(x)  /* ((x) \& 0x3)  $<<$ 30 */} \\
        y &=& 0x25     &in& \texttt{OP3(y) /* ((y) \& 0x3f) $<<$ 19 */} \\
        z &=& 0x0      &in& \texttt{F3I(z) /* ((z) \& 0x1)  $<<$ 13 */} \\
        a &=& 0x2      &in& \texttt{OP\_AJIT\_BIT\_2(a) /* ((a) \& 0x3  $<<$ 6 */}
      \end{tabular}

      The AJIT bits (insn[6:5]) is  set or reset internally by \texttt{F5}
      (just  like  in  \texttt{F4}),  and   hence  there  are  only  three
      arguments.

    \item \textbf{SRLD}:\\
      This will need another macro that sets bits 5 and 6. Let's call it
      \texttt{OP\_AJIT\_BIT\_2}.   Hence the  SPARC bit  layout of  this
      instruction is:

      \begin{tabular}[h]{lclcl}
        Macro to set  &=& \texttt{F5(x, y, z)} &in& \texttt{sparc.h}     \\
        Macro to reset  &=& \texttt{INVF5(x, y, z)} &in& \texttt{sparc.h}     \\
        x &=& 0x2      &in& \texttt{OP(x)  /* ((x) \& 0x3)  $<<$ 30 */} \\
        y &=& 0x26     &in& \texttt{OP3(y) /* ((y) \& 0x3f) $<<$ 19 */} \\
        z &=& 0x0      &in& \texttt{F3I(z) /* ((z) \& 0x1)  $<<$ 13 */} \\
        a &=& 0x2      &in& \texttt{OP\_AJIT\_BIT\_2(a) /* ((a) \& 0x3  $<<$ 6 */}
      \end{tabular}

      The AJIT bits (insn[6:5]) is  set or reset internally by \texttt{F5}
      (just  like  in  \texttt{F4}),  and   hence  there  are  only  three
      arguments.

    \item \textbf{SRAD}:\\
      This will need another macro that sets bits 5 and 6. Let's call it
      \texttt{OP\_AJIT\_BIT\_2}.   Hence the  SPARC bit  layout of  this
      instruction is:

      \begin{tabular}[h]{lclcl}
        Macro to set  &=& \texttt{F5(x, y, z)} &in& \texttt{sparc.h}     \\
        Macro to reset  &=& \texttt{INVF5(x, y, z)} &in& \texttt{sparc.h}     \\
        x &=& 0x2      &in& \texttt{OP(x)  /* ((x) \& 0x3)  $<<$ 30 */} \\
        y &=& 0x27     &in& \texttt{OP3(y) /* ((y) \& 0x3f) $<<$ 19 */} \\
        z &=& 0x0      &in& \texttt{F3I(z) /* ((z) \& 0x1)  $<<$ 13 */} \\
        a &=& 0x2      &in& \texttt{OP\_AJIT\_BIT\_2(a) /* ((a) \& 0x3  $<<$ 6 */}
      \end{tabular}

      The AJIT bits (insn[6:5]) is  set or reset internally by \texttt{F5}
      (just  like  in  \texttt{F4}),  and   hence  there  are  only  three
      arguments.
    \end{enumerate}
  \end{enumerate}
\end{itemize}

%%% Local Variables:
%%% mode: latex
%%% TeX-master: t
%%% End:


\subsection{Integer-Unit Extensions: SIMD Instructions}
\label{sec:integer-unit-extns:simd-instructions:impl}

\subsubsection{SIMD I instructions:}
\label{sec:simd:1:insn:impl}

The  first   set  of  SIMD  instructions  are   the  three  arithmetic
instructions: add,  sub, and mul.  The ``mul''  instruction has signed
and unsigned variations.  Each of the three instructions have 8 bit (1
byte),  16 bit  (1 half  word) and  32 bit  (1 word)  versions.  These
versions  are encoded  as  shown in  table~\ref{tab:types:for:simd:1},
where the first column denotes the  bit numbers.  We list all the SIMD
I instructions version wise below.
\begin{table}[h]
  \centering
  \begin{tabular}[p]{|l|l|l|}
  \hline
  \textbf{987} & \textbf{Type} & \textbf{Example}\\
  \hline
  001 & Byte & e.g. VADDD8\\
  010 & Half-word (16-bits) & e.g. VADDD16\\
  100 & Word (32-bits) & e.g. VADDD32\\
  \hline
\end{tabular}
\caption{Data type encoding for SIMD I instructions.}
\label{tab:types:for:simd:1}
\end{table}
\begin{enumerate}
\item \textbf{8 bit} (\textbf{1 Byte})
  \begin{enumerate}
  \item \textbf{VADDD8}:\\
    \begin{center}
      \begin{tabular}[p]{|c|c|l|l|}
        \hline
        \textbf{Start} & \textbf{End} & \textbf{Range} & \textbf{Meaning} \\
        \hline
        0 & 4 & 32 & Source register 2, rs2 \\
        5 & 6 & 4 & \emph{Always} 2, i.e. insn[6:5] = 10$_b$ \\
        7 & 9 & 8 & \textbf{Data type} specifier:  \emph{Always} 0x1\\
        10 & 12 & -- & \textbf{unused} \\
        13 & 13 & 0,1 & The \textbf{i} bit. \emph{Always} 0. \\
        14 & 18 & 32 & Source register 1, rs1 \\
        19 & 24 & 000000 & ``\textbf{op3}'' \\
        25 & 29 & 32 & Destination register, rd \\
        30 & 31 & 4 & Always ``10'' \\
        \hline
      \end{tabular}
    \end{center}
    \begin{itemize}
    \item []\textbf{VADDD8}: same as  ADD, but with Instr[13]=0 (i=0),
      and  Instr[6:5]=2.  Bits Instr[9:7]  are  a  3-bit field,  which
      specify the data type
    \item []\textbf{Syntax}: ``\texttt{vaddd8 SrcReg1, SrcReg2,
        DestReg}''.
    \item []\textbf{Semantics}: \emph{not given}
    \end{itemize}
    Bits layout:
\begin{verbatim}
    Offsets      : 31       24 23       16  15        8   7        0
    Bit layout   :  XXXX  XXXX  XXXX  XXXX   XXXX  XXXX   XXXX  XXXX
    Insn Bits    :  10       0  0000  0        0     00   110       
    Destination  :    DD  DDD                                       
    Source 1     :                     SSS   SS
    Source 2     :                                           S  SSSS
    Unused (0)   :                              U  UU               
    Final layout :  10DD  DDD0  0000  0SSS   SS0U  UU00   110S  SSSS
    To match     :  ^^       ^  ^^^^  ^        ^     ^^   ^^^
    Bitfield name:  OP          OP3            i     9-   765
\end{verbatim}

    To  set  up  bits  5  and  6, we  use  an  already  defined  macro
    \texttt{OP\_AJIT\_BIT\_5\_AND\_6}.  The  value to be set  in these
    two bits is 0x2.   To set bits 7 through 9, we  define a new macro
    \texttt{OP\_AJIT\_BIT\_7\_THRU\_9}.  The value  set in these three
    bits  decides the  \emph{type}, byte,  half word  or word,  of the
    instruction.  For  \textbf{vaddd8} instruction,  bits 7  through 9
    are  set  to the  value  0x1.   Both  these macros  influence  the
    \emph{unused} bits of  the SPARC V8 architecture.  So  we define a
    macro \texttt{OP\_AJIT\_SET\_UNUSED} that uses the previous two to
    set these bits unused by the SPARC V8, but used by AJIT.

    \verb|#define OP_AJIT_BIT_7_THRU_9(x)   ((x) << 0x7)|

    \verb+#define OP_AJIT_SET_UNUSED        (OP_AJIT_BIT_5_AND_6(0x2) | \\+

    \verb+                                   OP_AJIT_BIT_7_THRU_9(0x1))+

    We can  now define the final  macro \texttt{F6(x, y, z,  b, a)} to
    set the match bits for this instruction.
\begin{verbatim}
#define OP_AJIT_BIT_5(x)          (((x) & 0x1) << 5)
#define F4(x, y, z, b)            (F3(x, y, z) | OP_AJIT_BIT_5(b))
#define OP_AJIT_BIT_5_AND_6(x)    (((x) & 0x3) << 6)
#define F5(x, y, z, b)            (F3(x, y, z) | OP_AJIT_BIT_5_AND_6 (b))
#define OP_AJIT_BIT_7_THRU_9(x)   (((x) & 0x3) << 9)
#define F6(x, y, z, b, a)         (F5 (x, y, z, b) | OP_AJIT_BIT_7_THRU_9(a))
\end{verbatim}
    Hence the SPARC bit layout of this instruction is:

  \begin{tabular}[h]{lclcl}
    Macro to set  &=& \texttt{F4(x, y, z)} &in& \texttt{sparc.h}     \\
    Macro to reset  &=& \texttt{INVF4(x, y, z)} &in& \texttt{sparc.h}     \\
    x &=& 0x2      &in& \texttt{OP(x)  /* ((x) \& 0x3)  $<<$ 30 */} \\
    y &=& 0x00     &in& \texttt{OP3(y) /* ((y) \& 0x3f) $<<$ 19 */} \\
    z &=& 0x0      &in& \texttt{F3I(z) /* ((z) \& 0x1)  $<<$ 13 */} \\
    a &=& 0x1      &in& \texttt{OP\_AJIT\_BIT(a) /* ((a) \& 0x1)  $<<$ 5 */}
  \end{tabular}

  The AJIT bit (insn[5]) is set internally by \texttt{F4}, and hence
  there are only three arguments.
\end{enumerate}
\item \textbf{1 Half word} (\textbf{16 bit})
\item \textbf{1 Word} (\textbf{32 bit})
\end{enumerate}



\subsection{Integer-Unit Extensions: SIMD Instructions II}
\label{sec:integer-unit-extns:simd-instructions:2:impl}

\subsection{Vector Floating Point Instructions}
\label{sec:vector-floating-point-instructions:impl}

\subsection{CSWAP instructions}
\label{sec:cswap-instructions:impl}


\chapter{Towards Assembler Extraction}
\label{chap:asm:extraction}

\section{Succinct ISA Descriptions}
\label{sec:sisad}

\textbf{A. M. Vichare}

ISA description languages seem to be  at least 20 years old problem as
of 2018.   Attempts like  MIMOLA or  LISA have  been made  to describe
processors and  generate system software through  them.  This document
records my  attempts to develop  such a  language afresh, but  for the
AJIT processor of IIT Bombay.  The benefit of hindsight should ideally
be employed in this design process.  I shall try to bring that in as a
parallel activity along side the attempts to a practical design.

\subsection{Instruction Set Design Study}
\label{sec:isa:design:study}

This is the  background work mainly of conceptual ideas,  and study of
some known examples.

\subsubsection{Basic Concepts of Instruction Set Design}
\label{sec:basic:isa:design:info}

From: Henn-Patt, CA-Quant.Approach. Ed.5, App.A:

\begin{itemize}
\item \textbf{Type of internal storage}:
  \begin{itemize}
  \item Stack: Operands are on the stack, and hence \emph{implicit} in
    the instruction.
  \item Accumulator: One of the  operands is in the \emph{accumulator}
    register, and hence implicit in the instruction.
  \item  Register-Memory:   Memory  \emph{can   be}  a  part   of  the
    instruction.
  \item  Register-Register: Memory  is  \textbf{never} a  part of  the
    instruction,   except   for    the   \emph{load-store}   pair   of
    instructions.
  \item Memory-Memory:  All operands  are in  the memory  and directly
    addressed as a  part of the instruction.  This is  an old style is
    not often found today ($\sim$ 2018).
  \item  Variations:  Dedicating  some   registers  for  some  special
    purposes  --  \textbf{extended   accumulator}  or  \textbf{special
      purpose registers}.
  \item  Number of  operands: This  depends  on the  type of  internal
    storage,  and  a  design   choice.   An  binary  instruction  (aka
    \emph{operation}) may explicitly take two data source operands and
    one result destination operand.  Or it may take only two operands,
    with one  of them being \textbf{both}  a data source and  a result
    destination operand.
  \end{itemize}
\item \textbf{Memory layout addressing}:
  \begin{itemize}
  \item Byte ordering: There are two ways to order a set of bytes of a
    multi-byte object  (e.g. 32 bit,  i.e. 4 byte integer).
    \begin{itemize}
    \item \textbf{Little Endian}: The  byte with the least significant
      bit can  be stored at  the smallest byte  address, or 
    \item \textbf{Big Endian}: The byte with the least significant bit
      can be stored at the largest byte address.
    \end{itemize}
  \item Alignment  needs: For  multibyte objects, an  architecture may
    need the components to be  aligned on suitable address boundaries.
    Or it may not need them to be so aligned!  If $k$ is the number of
    bytes of a  multibyte object, $a$ is the address  of the byte with
    the  least  significant  bit,  then  the  object  is  aligned  if:
    $a = n \times k$, where $n$  is a natural number.  The address $a$
    is an integral multiple of the object size $k$.
  \item Shifting needs: Consider  reading a \emph{single} byte aligned
    at a word address into a  \emph{64 bit} register.  A single 64 bit
    read, i.e. a  double word read, would be performed  on double word
    aligned address.  If  the word aligned byte  would \textbf{not} be
    double word aligned,  then the byte that is read  would not occupy
    the least  significant position in  the 64 bit register.   In such
    cases for correct  alignment, we will need to shift  the byte read
    in by 3  positions (calculate this ``3'') so that  it occupies the
    correct position in a 64 bit register.
  \end{itemize}
\item \textbf{Addressing Modes}: \\
  How do we address the primary memory?
  \begin{itemize}
  \item  \emph{Immediate}:  No  addressing  at  all.   The  argument/s
    (i.e. operand/s) is/are given as a part of the instruction.  There
    is a finite size, finite number of bits, and layout norms.
  \item \emph{Register Direct}: The  operand/s is/are available in one
    or more  registers.  Instead of  being placed in  the instruction,
    the operands are available in the register.
    \begin{itemize}
    \item \emph{PC Relative}: A  variant of register direct addressing
      where the  register to be used  is fixed as the  program counter
      (i.e.  the instruction pointer).
    \end{itemize}
  \item  \emph{Direct} or  \emph{Absolute}:  The  address is  provided
    directly as an  argument.  There could be  finite size definitions
    that could  be same as or  different from the size  of the address
    bus.
  \item \emph{Register Indirect}: The operand location is given in one
    or more registers.   The register size is expected to  be the same
    as the size of the address bus.
    \begin{itemize}
    \item \emph{Auto  Increment or  Decrement}: A variant  of register
      indirect where the  indirection value in the  register is either
      automatically incremented  or decremented.   Autoincrementing is
      useful for array  traversals with the base address  of the array
      in the  register, and  the array element  size as  the increment
      value.    Autodecrementing  is   similarly   useful  for   stack
      operations.
    \item   \emph{Displacement}:  A   variant  of   register  indirect
      addressing   mode,  the   operand  location   is  given   as  an
      \emph{offset}   (i.e.   \emph{displacement}),   relative  to   a
      register  indirect address.   The  memory location  is thus  the
      offset relative to the location given in a register.
    \item  \emph{Indexed}:   Another  variant  of   register  indirect
      addressing  where  the  operand   location  is  a  well  defined
      algebraic  relation of  values in  a few  registers.  Thus,  for
      example, the location  might be given as a  \emph{sum} of values
      in two registers where one  register has the ``base'' value, and
      the other has an ``index'' (i.e. an offset) relative to the base
      value.
    \item  \emph{Scaled}: Yet  another  variant  of register  indirect
      addressing where  the operand location  is again a  well defined
      algebraic relation of values in  a few registers.  The algebraic
      relation  is  a  displacement  relative to  a  ``base''  in  one
      register  and  an integral  scale  up  of ``index''  in  another
      register.
    \end{itemize}
  \item \emph{Memory  Indirect}: Adding one more  level of indirection
    to the register  indirect mode yields this mode.   The location of
    the operand is  now available at the memory location  given by the
    register indirect mode.
  \end{itemize}
  The immediate, displacement, and  register indirect addressing modes
  are predominantly used (about 75\% to 99\% of modes used).
\item \textbf{Types and Size of Operands}:
  \begin{itemize}
  \item  Some specifications  of  size have  standardized (e.g.   IEEE
    floating point), some have become conventional (e.g. 8 bit byte, 2
    byte half words, 4 byte words etc.), some are optionally supported
    by  the  processor  architecture   (e.g.   strings,  binary  coded
    decimal, packed  decimal).  Representation  is either  tagged (not
    used  much  today  $\sim$  2018), or  encoded  within  the  opcode
    (preferred method today).
  \item \emph{Standardised}: IEEE Floating  point -- single and double
    precision.  Single precision is 4 bytes, and double precision is 8
    bytes.
  \item \emph{Conventional}:
    \begin{center}
      \begin{tabular}[h]{|c|c|c|c|c|c|}
        \hline
        Quad Word   & Double word & Word & Half word & Byte & Bits \\
        \hline
        -- & -- & -- & -- & 1 & 8 \\
        -- & -- & -- & 1 & 2 & 16 \\
        -- & -- & 1 & 2 & 4 & 32 \\
        -- & 1 & 2 & 4 & 8 & 64 \\
        1 & 2 & 4 & 8 & 16 & 128 \\
        \hline
      \end{tabular}
    \end{center}
  \end{itemize}
\item \textbf{Operations in the Instruction Set}:\\
  Thumb rule: Simplest instructions are the most widely executed ones.
  \begin{center}
    \begin{tabular}[h]{|l|l|}
      \hline
      \textbf{Type} & \textbf{Description or Examples} \\
      \hline
      Arithmetic & Arithmetic operations on numbers: +, -, *, / etc. \\
      Logical & Logical: AND, OR, NOT \\
      Data Transfer & Load, Store, Move \\
      Control Flow & Branch, Loop, Jump, Procedure call and return,
                     Trap \\
      System & OS System call, Virtual memory management \\
      Floating point & Floating point +, -, *, / etc. \\
      Decimal & Decimal  +, -, *, / etc. \\
      String & String move, compare, search \\
      Graphics & Pixel and vertex operation, compression \&
                 decompression \\
      Signal Processing & FFT, MAC \\
      \hline
    \end{tabular}
  \end{center}

  It might be useful to classify at a little more higher level: 
  \begin{center}
    \begin{tabular}[h]{|l|l|}
      \hline
      \textbf{Class} & \textbf{Description or Examples} \\
      \hline
      Data Type based & Arithmetic, Logical, Floating point, Signal
                        Processing, Graphics, Decimal, String \\
      Data Transfer & All I/O \\
      System Control & Control flow, System management \\
      \hline
    \end{tabular}
  \end{center}
\item \textbf{Instructions for Control Flow}:\\
  \begin{itemize}
  \item No well defined convention for  naming, but we will follow the
    text referred at the beginning  of this section. Four main control
    flow instructions are usually offered.
    \begin{itemize}
    \item Jump: These are unconditional.
    \item Branch: These are conditional.
    \item Procedure call.
    \item Procedure return.
    \end{itemize}
  \item  It is  useful to  use \emph{PC-Relative}  addressing mode  to
    specify  the destination  address of  a control  flow instruction.
    This allows running the code independent  of where it is loaded --
    a   property   called  \emph{position   independence}.    Position
    independence may not always be  possible, especially if the target
    of control flow cannot be computed at compile time.  In such cases
    other addressing modes are  used.  Register indirect addressing is
    useful for:
    \begin{enumerate}
    \item Case analysis as in \emph{switch} statements.
    \item Virtual functions or methods,
    \item Higher order functions or function pointers, and
    \item Dynamically shared libraries.
    \end{enumerate}
  \item Condition code techniques: Three methods have been used --
    \begin{itemize}
    \item Condition codes register (aka  the flags register): A set of
      reserved special bits each  indicating some defined condition is
      set or  reset during  an operation.   The subsequent  branch can
      test  these  bits.   Typically, a  separate  branch  instruction
      exists for each condition code bit.
    \item  Condition  register:  No dedicated  register.   Instead  an
      arbitrary register can be designated as the ``flags'' register.
    \item Compare and  Branch: The comparison is a part  of the branch
      instruction itself.
    \end{itemize}
  \end{itemize}
\item \textbf{Encoding an Instruction Set}:
  \begin{itemize}
  \item Variable sized.
  \item Fixed width.
  \item Hybrid: Some size varying part and some fixed part.
  \end{itemize}
\end{itemize}

\subsubsection{Some Examples of Instruction Set Design Languages}
\label{sec:isa:design:lang:eg}

We will look at MIMOLA and LISA.

\subsection{Instruction Set Description and Generation}
\label{sec:describe:generate:isa}

We use  an ``engineering'' approach  to design and development  of the
language and its  processors for describing an ISA  and generating the
processing software.

\subsubsection{Basic Elements  of the Structure of  an Instruction Set
  Language}
\label{sec:isa:lang:struct}

\begin{itemize}
\item  Mnemonic:  A string  of  ``word''  characters.   A ``word''  is
  understood intuitively, and from the context.
\item Class: ISAs frequently  group instructions into \emph{groups} or
  \emph{classes} typically  based on the semantics.  Thus  we can have
  logical  instructions,  integer  arithmetic  instructions,  etc.  We
  capture the class in this field.
\item Bit pattern:  An instruction is expressed using  a set of binary
  digits, aka bits.  The key attributes are:
  \begin{itemize}
  \item Length: The total number of bits that make up the instruction.
    For  our architecture this  is a  constant with  value \textbf{32}
    bits.
  \item  Composition: An  instruction  bit pattern  is  composed of  a
    subsets of bits that describe  components of the bit pattern.  The
    various \emph{kinds} of subsets that may be needed are:
    \begin{itemize}
    \item 
    \end{itemize}
  \end{itemize}
\end{itemize}

\subsection{Instruction Set Generation}
\label{sec:generate:isa}

\subsubsection{Basic Elements of the ``Language'' to Describe the
  Instruction}
\label{sec:describe:isa:lang}

\begin{itemize}
\item   ``insn-mnemonic''  denotes   the   \textbf{mnemonic}  of   the
  instruction.
\item ``insn-bit-pattern'' denotes the  top level composite of the bit
  pattern of the given instruction.
  \begin{itemize}
  \item  ``length'' is a  field of  the bit  pattern that  records the
    total number of bits that make up the instruction.
  \item ``composition''  is a variable  length field that  records the
    composition of the bits pattern.
  \end{itemize}
\end{itemize}

\newpage
\chapter{Packaging AJIT Within BuildRoot System}
\label{chap:amv:packaging:work}

\section{List and Sequence of Files}
\label{sec:packaging:config:list:seq}

Basic directory structure:

\begin{tabular}[h]{|p{.3\textwidth}|p{.7\textwidth}|}
  \hline
  \textbf{VARIABLE NAME} & \textbf{DESCRIPTION} \\
  \hline
  \texttt{TOP} & Some top level directory of buildroot software.\\
  \texttt{BUILDROOT\_VERSION} & \texttt{2014.08}\\
  \texttt{BUILDROOT\_TOP} & \texttt{\$\{TOP\}/buildroot-\$\{BUILDROOT\_VERSION\}}\\
  \texttt{OUTPUT} & \texttt{\$\{BUILDROOT\_TOP\}/output}\\
  \texttt{BUILD} & \texttt{\$\{OUTPUT\}/build}\\
  \hline
\end{tabular}

\begin{enumerate}
\item File  \texttt{\$\{BUILDROOT\_TOP\}/arch/Config.in}: This file is
  used  to add  a  new \emph{architecture}  to the  \texttt{buildroot}
  system.  Add the AJIT processor as follows:
% \codetoadd{
% \begin{verbatim}
% config BR2_ajit
%         bool "AJIT (IIT Bombay)"
%         help
%           Synopsys' IIT Bombay designed SPARC V8 like processor that is
%           targetted for netblazers. Little endian.
% \end{verbatim}
% }
% \framebox{
%   \begin{minipage}[h]{\linewidth}
% \begin{verbatim}
% config BR2_ajit
%         bool "AJIT (IIT Bombay)"
%         help
%           Synopsys' IIT Bombay designed SPARC V8 like processor that is
%           targetted for netblazers. Little endian.
% \end{verbatim}    
%   \end{minipage}
% }

\hrulefill
\begin{verbatim}
config BR2_ajit
        bool "AJIT (IIT Bombay)"
        help
          Synopsys' IIT Bombay designed SPARC V8 like processor that is
          targetted for netblazers. Little endian.
\end{verbatim}    
\hrulefill

The indentation  uses the TAB  character.  It appears to  be mandatory
as every ``\texttt{*/*Config.in}'' file I looked into uses it.

\item \texttt{binutils/gas/configure.tgt}: This is  the file where the
  \texttt{gas}  tool sets  the target  CPU files  given the  usual GNU
  triad (or quad, sometimes).
\end{enumerate}

We  first  focus on  binutils-2.22  for  buildroot-2014.08 only.   The
gdb-7.6.2 port would be similar, and dealt with later.

\section{List and Sequence of Files Processing}
\label{sec:packaging:config:list:seq:processing}

Since  AJIT  is based  on  Sparc  V8, we  first  search  for files  in
\texttt{binutils}  that contain  the  string ``\texttt{sparc}''.   The
search is case insensitive.  We first  focus on adding ajit to the GNU
BFD   system   in   the    binutils.    Hence   our   search   is   in
\texttt{binutils/bfd}.  This yields the following list of files in
tables \ref{tab:bfd:sparc:files:1} and \ref{tab:bfd:sparc:files:2}.

\begin{table}[ht]
  \centering
  \begin{tabular}[h]{|p{.3\linewidth}|p{.7\linewidth}|}
\hline
bfd/aoutf1.h          &  A.out  ``format 1'' file  handling code  for BFD. \\
bfd/aout-sparcle.c    &  BFD backend for sparc little-endian aout binaries. \\
bfd/aoutx.h           &  BFD semi-generic back-end for a.out binaries. \\
bfd/archures.c        &  BFD library support routines for architectures. \\
    \hline
bfd/bfd-in2.h         & 
\begin{minipage}[h]{\linewidth}
  Main  header file  for the  bfd library:  portable access  to object
  files.  This file is automatically generated from

  \begin{tabular}[h]{|l|l|l|}
    \hline 
    ``bfd-in.h'' &    ``init.c'' &    ``opncls.c'' \\
    ``libbfd.c'' &    ``bfdio.c'' &   ``bfdwin.c'' \\
    ``section.c'' &   ``archures.c'' &``reloc.c''  \\
    ``syms.c'' &      ``bfd.c'' &     ``archive.c'' \\
    ``corefile.c'' &  ``targets.c'' & ``format.c'' \\
    ``linker.c'' &    ``simple.c''  &  ``compress.c''\\
    \hline 
  \end{tabular}

  Run ``\texttt{make headers}'' in your build \texttt{bfd/} to regenerate.
\end{minipage} \\
    \hline
bfd/bfd-in.h          &  Main header file for the bfd library:  portable access   to   object   files. \\
bfd/bfd.m4            &  This file was derived from acinclude.m4. \\
bfd/cf-sparclynx.c    &  BFD back-end for Sparc COFF LynxOS files. \\
bfd/coffcode.h        &  Support for the generic parts of most COFF variants, for BFD. \\
bfd/coff-sparc.c      &  BFD back-end for Sparc COFF files. \\
bfd/coff-tic4x.c      &  BFD back-end for TMS320C4X coff binaries. \\
bfd/coff-tic54x.c     &  BFD back-end for TMS320C54X coff binaries. \\
    \hline
bfd/config.bfd        &  
\begin{minipage}[h]{\linewidth}
  Convert  a canonical  host type  into a  BFD host  type.   Set shell
  variable \texttt{targ} to canonical target
  name, and run using ``\texttt{. config.bfd}''. \\
  Sets the following shell variables: \\
  \begin{tabular}[h]{|l|l|}
    \hline
  \texttt{targ\_defvec} &	Default vector for this target \\
  \texttt{targ\_selvecs} &	Vectors to build for this target \\
  \texttt{targ64\_selvecs} &	\parbox{.65\linewidth}{Vectors     to
    build    if \texttt{--enable-64-bit-bfd}  is  given or if host is
    64 bit.} \\ 
  \texttt{targ\_archs} &	Architectures for this target \\
  \texttt{targ\_cflags} &	\texttt{\$(CFLAGS)} for this target (FIXME: pretty bogus) \\
  \texttt{targ\_underscore} &	Whether underscores are used: yes or no \\
    \hline
\end{tabular}
\end{minipage}\\
    \hline
bfd/configure         &  Guess values for system-dependent variables and create Makefiles. \\
bfd/configure.in      &  Process this file with autoconf to produce a configure script. \\
bfd/cpu-sparc.c       &  BFD support for the SPARC architecture. \\
bfd/elf32-cris.c      &  CRIS-specific support for 32-bit ELF.  In comment. \\
bfd/elf32-m68hc1x.h   &  Motorola 68HC11/68HC12-specific support for 32-bit ELF. In comment. \\
bfd/elf32-sparc.c     &  SPARC-specific support for 32-bit ELF. \\
bfd/elf64-ajit.c      &  SPARC-specific support for 64-bit ELF \\
bfd/elf64-sparc.c     &  SPARC-specific support for 64-bit ELF \\
bfd/elf-bfd.h         &  BFD back-end data structures for ELF files. \\
bfd/elf.c             &  ELF executable support for BFD. \\
bfd/elfcode.h         &  ELF executable support for BFD. \\
bfd/elfxx-sparc.c     &  SPARC-specific support for ELF. \\
bfd/elfxx-sparc.h     &  SPARC ELF specific backend routines. \\
bfd/freebsd.h         &  BFD back-end definitions used by all FreeBSD targets. \\
bfd/libaout.h         &  BFD back-end data structures for a.out (and similar) files. \\
  \hline
bfd/libbfd.h          &  
\begin{minipage}[h]{\linewidth}
  Declarations  used by  bfd library  *implementation*.  (This include
  file is not  for users of the library.)   This file is automatically
  generated from
  \begin{tabular}[h]{|l|l|l|}
  \hline
  ``libbfd-in.h'' &    ``init.c'' &     ``libbfd.c'' \\
  ``bfdio.c'' &        ``bfdwin.c'' &   ``cache.c''  \\
  ``reloc.c'' &        ``archures.c'' &  ``elf.c'' \\
  \hline
\end{tabular}
  Run  ``\texttt{make  headers}''  in  your  build  \texttt{bfd/}  to
  regenerate.
\end{minipage} \\
\hline
\end{tabular}
\caption[List of files 1]{List of files in \texttt{binutils/bfd} that
  contain the word ``\texttt{sparc}''.  Continued in
  Table~\ref{tab:bfd:sparc:files:2}.} 
\label{tab:bfd:sparc:files:1}
\end{table}
\begin{table}[ht]
  \centering
  \begin{tabular}[h]{|p{.3\linewidth}|p{.7\linewidth}|}
\hline
bfd/lynx-core.c       &  BFD back end for Lynx core files \\
bfd/mach-o.c          &  Mach-O support for BFD. \\
bfd/Makefile.am       &  Process this file with automake to generate Makefile.in \\
bfd/Makefile.in       &  Makefile.in generated by automake 1.11.1 from Makefile.am. \\
bfd/mipsbsd.c         &  BFD backend for MIPS BSD (a.out) binaries. \\
bfd/netbsd-core.c     &  BFD back end for NetBSD style core files \\
bfd/nlm32-sparc.c     &  Support for 32-bit SPARC NLM (NetWare Loadable Module) \\
bfd/pdp11.c           &  BFD back-end for PDP-11 a.out binaries. \\
bfd/reloc.c           &  BFD support for handling relocation entries. \\
bfd/sparclinux.c      &  BFD back-end for linux flavored sparc a.out binaries. \\
bfd/sparclynx.c       &  BFD support for Sparc binaries under LynxOS. \\
bfd/sparcnetbsd.c     &  BFD back-end for NetBSD/sparc a.out-ish binaries. \\
bfd/sunos.c           &  BFD backend for SunOS binaries. \\
bfd/targets.c         &  Generic target-file-type support for the BFD library. \\
\hline
\end{tabular}
\caption[List of files 2]{Continued from
  Table~\ref{tab:bfd:sparc:files:1}. List of files in
  \texttt{binutils/bfd} that contain the word ``\texttt{sparc}''.}
\label{tab:bfd:sparc:files:2}
\end{table}

We can  eliminate files that are not  related to ELF in  any way.  For
example they may  be dealing with the COFF format.  Or  they may be 64
bit;  AJIT is  a 32  bit system  as of  date.\footnote{2020.}   Of the
remaining, some would  most certainly be candidate files  for the AJIT
port  and some  would probably  be.  Table~\ref{tab:bfd:sparc:files:3}
lists these files  with a tag ``yes'' if the file  is most certainly a
candidate for AJIT port or a tag ``maybe''.
\begin{table}[ht]
  \centering
  \begin{tabular}[h]{|l|l|}
\hline
bfd/archures.c   & yes \\
bfd/config.bfd   & yes \\
bfd/cpu-sparc.c   & yes \\
bfd/elf32-sparc.c   & yes \\
bfd/elf-bfd.h   & yes \\
bfd/elf.c   & yes \\
bfd/elfcode.h   & yes \\
bfd/elfxx-sparc.c   & yes \\
bfd/elfxx-sparc.h   & yes \\
bfd/targets.c   & yes \\
\hline
bfd/bfd-in2.h   & maybe \\
bfd/bfd-in.h   & maybe \\
bfd/bfd.m4   & maybe \\
bfd/configure   & maybe \\
bfd/configure.in   & maybe \\
bfd/elf64-ajit.c   & maybe \\
bfd/elf64-sparc.c   & maybe \\
bfd/freebsd.h   & maybe \\
bfd/libbfd.h   & maybe \\
bfd/Makefile.am   & maybe \\
bfd/Makefile.in   & maybe \\
bfd/nlm32-sparc.c   & maybe \\
bfd/reloc.c   & maybe \\
\hline
\end{tabular}
\caption[List of files 3]{List of files in \texttt{binutils/bfd} that
  contain the word ``\texttt{sparc}'' and are possible candidate files
  for the AJIT port.} 
\label{tab:bfd:sparc:files:3}
\end{table}
\begin{table}[ht]
  \centering
  \begin{tabular}[h]{|p{.3\linewidth}|p{.7\linewidth}|}
\hline
bfd/archures.c (y)       &  BFD library support routines for architectures. \\
    \hline
bfd/bfd-in2.h (m)     & 
\begin{minipage}[h]{\linewidth}
  Main  header file  for the  bfd library:  portable access  to object
  files.  This file is automatically generated from

  \begin{tabular}[h]{|l|l|l|}
    \hline 
    ``bfd-in.h'' &    ``init.c'' &    ``opncls.c'' \\
    ``libbfd.c'' &    ``bfdio.c'' &   ``bfdwin.c'' \\
    ``section.c'' &   ``archures.c'' &``reloc.c''  \\
    ``syms.c'' &      ``bfd.c'' &     ``archive.c'' \\
    ``corefile.c'' &  ``targets.c'' & ``format.c'' \\
    ``linker.c'' &    ``simple.c''  &  ``compress.c''\\
    \hline 
  \end{tabular}

  Run ``\texttt{make headers}'' in your build \texttt{bfd/} to regenerate.
\end{minipage} \\
    \hline
bfd/bfd-in.h (m)         &  Main header file for the bfd library:  portable access   to   object   files. \\
bfd/bfd.m4 (m)           &  This file was derived from acinclude.m4. \\
    \hline
bfd/config.bfd (y)       &  
\begin{minipage}[h]{\linewidth}
  Convert  a canonical  host type  into a  BFD host  type.   Set shell
  variable \texttt{targ} to canonical target
  name, and run using ``\texttt{. config.bfd}''. \\
  Sets the following shell variables: \\
  \begin{tabular}[h]{|l|l|}
    \hline
  \texttt{targ\_defvec} &	Default vector for this target \\
  \texttt{targ\_selvecs} &	Vectors to build for this target \\
  \texttt{targ64\_selvecs} &	\parbox{.65\linewidth}{Vectors     to
    build    if \texttt{--enable-64-bit-bfd}  is  given or if host is
    64 bit.} \\ 
  \texttt{targ\_archs} &	Architectures for this target \\
  \texttt{targ\_cflags} &	\texttt{\$(CFLAGS)} for this target (FIXME: pretty bogus) \\
  \texttt{targ\_underscore} &	Whether underscores are used: yes or no \\
    \hline
\end{tabular}
\end{minipage}\\
    \hline
bfd/configure (m)        &  Guess values for system-dependent variables and create Makefiles. \\
bfd/configure.in (m)     &  Process this file with autoconf to produce a configure script. \\
bfd/cpu-sparc.c (y)      &  BFD support for the SPARC architecture. \\
bfd/elf32-sparc.c (y)    &  SPARC-specific support for 32-bit ELF. \\
bfd/elf64-ajit.c (m)     &  SPARC-specific support for 64-bit ELF \\
bfd/elf64-sparc.c (m)    &  SPARC-specific support for 64-bit ELF \\
bfd/elf-bfd.h (y)        &  BFD back-end data structures for ELF files. \\
bfd/elf.c (y)            &  ELF executable support for BFD. \\
bfd/elfcode.h (y)        &  ELF executable support for BFD. \\
bfd/elfxx-sparc.c (y)    &  SPARC-specific support for ELF. \\
bfd/elfxx-sparc.h (y)    &  SPARC ELF specific backend routines. \\
bfd/freebsd.h (m)        &  BFD back-end definitions used by all FreeBSD targets. \\
  \hline
bfd/libbfd.h (m)         &  
\begin{minipage}[h]{\linewidth}
  Declarations  used by  bfd library  *implementation*.  (This include
  file is not  for users of the library.)   This file is automatically
  generated from
  \begin{tabular}[h]{|l|l|l|}
  \hline
  ``libbfd-in.h'' &    ``init.c'' &     ``libbfd.c'' \\
  ``bfdio.c'' &        ``bfdwin.c'' &   ``cache.c''  \\
  ``reloc.c'' &        ``archures.c'' &  ``elf.c'' \\
  \hline
\end{tabular}
  Run  ``\texttt{make  headers}''  in  your  build  \texttt{bfd/}  to
  regenerate.
\end{minipage} \\
\hline
bfd/Makefile.am (m)      &  Process this file with automake to generate Makefile.in \\
bfd/Makefile.in (m)      &  Makefile.in generated by automake 1.11.1 from Makefile.am. \\
bfd/nlm32-sparc.c (m)    &  Support for 32-bit SPARC NLM (NetWare Loadable Module) \\
bfd/reloc.c (m)          &  BFD support for handling relocation entries. \\
bfd/targets.c (y)     &  Generic target-file-type support for the BFD library. \\
\hline
\end{tabular}
\caption[List of files 4]{List of files in \texttt{binutils/bfd} that
  contain the word ``\texttt{sparc}'' and are possible candidate files
  for the AJIT port.} 
\label{tab:bfd:sparc:files:4}
\end{table}

\begin{table}[ht]
  \centering
  \begin{tabular}[h]{|l|c|p{.63\linewidth}|l|}
\hline
bfd/elf-bfd.h      &01&  BFD back-end data structures for ELF files.                                               & yes   \\
bfd/elf.c          &02&  ELF executable support for BFD.                                                           & yes   \\
bfd/elfcode.h      &03&  ELF executable support for BFD. External to internal conversions.                         & yes   \\
bfd/elfxx-sparc.c  &04&  SPARC-specific support for ELF.                                                           & yes   \\
bfd/elfxx-sparc.h  &05&  SPARC ELF specific backend routines.                                                      & yes   \\
bfd/libbfd.h       &06&  Declarations used by bfd library; generated file, see source files                        & yes   \\
bfd/targets.c      &07&  Generic target-file-type support for the BFD library.                                     & yes   \\
bfd/reloc.c        &08&  BFD support for handling relocation entries.                                              & yes   \\
bfd/elf32-sparc.c  &09&  SPARC-specific support for 32-bit ELF.                                                    & yes   \\
bfd/cpu-sparc.c    &10&  BFD support for the SPARC architecture.                                                   & yes   \\
bfd/archures.c     &11&  BFD library support routines for architectures.                                           & yes   \\
bfd/config.bfd     &12&  Convert a canonical host type into a BFD host type.                                       & yes   \\
bfd/bfd-in2.h      &13&  Main header file for the bfd library: portable \& generated                               & yes   \\
bfd/bfd-in.h       &14&  Main header file for the bfd library:  portable                                           & yes   \\
bfd/bfd.m4         &21&  This file was derived from acinclude.m4. \textbf{To Check}                                & maybe \\
bfd/configure.in   &22&  Process this file with autoconf to produce a configure script.                            & maybe \\
bfd/Makefile.am    &22&  Process this file with automake to generate Makefile.in                                   & no    \\
bfd/elf64-ajit.c   &22&  SPARC-specific support for 64-bit ELF - to DENY 64 bit support                            & maybe \\
bfd/nlm32-sparc.c  &22&  Support for 32-bit SPARC NLM (NetWare Loadable Module)                                    & no    \\
bfd/freebsd.h      &23&  BFD back-end definitions used by all FreeBSD targets.                                     & no    \\
bfd/configure      &23&  Guess values for system-dependent variables and                                           & no    \\
bfd/Makefile.in    &23&  Makefile.in generated by automake 1.11.1 from Makefile.am.                                & no    \\
bfd/elf64-sparc.c  &23&  SPARC-specific support for 64-bit ELF                                                     & no    \\
\hline
\end{tabular}
\caption[List of files 5]{List of files in \texttt{binutils/bfd} that
  contain the word ``\texttt{sparc}'' and that are possible candidate
  files for the AJIT port along with their number in the sequence of
  modifications.} 
\label{tab:bfd:sparc:files:5}
\end{table}

In  the following  sections we  look at  each file  in  detail.  These
sections are written  by trying to guess the  most probable next file,
and then  going back and forth across  the other files to  fill in the
information.

\subsection{\texttt{bfd/elf-bfd.h}: Yes}
\label{sec:bfd:elf-bfd.h}

We take this  as the first file to examine.  Among  the first files it
includes  is: \texttt{include/elf/common.h}  which has  the  basic ELF
definitions implemented  for the GNU BFD system.   For the definitions
of ELF format,  refer to the standard reference  [ELF REFERENCE HERE].
Among  the  important fields  is  the  \texttt{e\_machine} field.   We
reproduce the relevant comment in \texttt{include/elf/common.h} below:
\begin{verbatim}
Values  for  e_machine,  which  identifies  the  architecture.   These
numbers are officially assigned  by registry@sco.com.  See below for a
list of ad-hoc numbers used during initial development.

If it is  necessary to assign new unofficial  EM_* values, please pick
large random numbers (0x8523, 0xa7f2, etc.) to minimize the chances of
collision with official or non-GNU unofficial values.

NOTE: Do not  just increment the most recent  number by one.  Somebody
else somewhere  will do exactly  the same thing,  and you will  have a
collision.  Instead, pick a random number.

Normally, each entity or maintainer  responsible for a machine with an
unofficial e_machine number should eventually ask registry@sco.com for
an officially blessed number to be added to the list above.
\end{verbatim}
We will  assign a temporary value  0xABCD as of now  (i.e. 2020) until
the AJIT architecture matures  for a more global standardised support.
At that point, we should follow the instructions in the comment above.
We add the following to \texttt{include/elf/common.h}:

\verb|#define  EM_AJIT  0xABCD   /* The IITB AJIT Processor */|

Note that  this change  will reflect only  after the support  is fully
implemented.

The two other  files included are: \texttt{include/elf/internal.h} and
\texttt{include/elf/external.h}.  These  respectively describe the ELF
format within the BFD system when in-memory and in-file.

An enum, ``\texttt{enum elf\_target\_id}''  is used to identify target
specific    extensions    to    the    \texttt{elf\_obj\_tdata}    and
\texttt{elf\_link\_hash\_table}    structures.    Both    the   latter
structures are in this file too.   Since AJIT has no extensions, we do
not seem to need adding an AJIT identifier to this enum.  If, however,
we do need to add then there are two main issues to consider given
that the enum constants are lexicographically ordered:
\begin{enumerate}
\item \label{elf-bfd:h:lexical:order} The name ``AJIT'' will appear as
  the first enum in the  lexicographical order.  That will offset each
  subsequent enum  value by  +1 relative to  its previous  value.  One
  value  from this  enum  gets built  into  the tools  for a  specific
  system; in particular the GNU BFD library on that system.  This is a
  potential problem.   If, for a system  (say the i386),  we build two
  GNU BFD  library versions, one  using the standard binutils  and the
  other using binutils with AJIT  support, then the libraries will use
  different indices internally.  If these indices result in different,
  but legal,  ELF processing  then we have  a problem.   Our resulting
  system is fragile and will break easily.
\item \label{elf-bfd:h:no:lexical:order} The other choice to place the
  ``AJIT'' enum  is at the  second last position.  This  will preserve
  the enums of all the other  supported targets.  We will not have the
  problem in \ref{elf-bfd:h:lexical:order} above.  However, some other
  development effort might  add another target to binutils  and at the
  same  place!  Unless  it so  happens that  the tools  for  these two
  non-standard targets come together  we will not have problems.  This
  is a low probability event, and we ignore it.  If both these targets
  become standard then the development effort will have to ensure that
  these have distinct enum values.
\end{enumerate}


\subsection{\texttt{bfd/archures.c}: Yes}
\label{sec:bfd:archures.c}

From the comments in the file we summarize:

About Architectures

The BFD approach keeps one atom in a BFD describing the architecture
of the data attached to the BFD: a pointer to a
``\texttt{bfd\_arch\_info\_type}''.

Pointers to structures can be requested independently of a BFD
	so that an architecture's information can be interrogated
	without access to an open BFD.

	The architecture information is provided by each architecture package.
	The set of default architectures is selected by the macro
	``\texttt{SELECT\_ARCHITECTURES}''.  This is normally set up in the
	@file{config/@var{target}.mt} file of your choice.  If the name is not
	defined, then all the architectures supported are included.

	When BFD starts up, all the architectures are called with an
	initialize method.  It is up to the architecture back end to
	insert as many items into the list of architectures as it wants to;
	generally this would be one for each machine and one for the
	default case (an item with a machine field of 0).

	BFD's idea of an architecture is implemented in	@file{archures.c}.
*/

/*

SUBSECTION
	bfd\_architecture

DESCRIPTION
	This enum gives the object file's CPU architecture, in a
	global sense---i.e., what processor family does it belong to?
	Another field indicates which processor within
	the family is in use.  The machine gives a number which
	distinguishes different versions of the architecture,
	containing, for example, 2 and 3 for Intel i960 KA and i960 KB,
	and 68020 and 68030 for Motorola 68020 and 68030.



\subsection{\texttt{bfd/config.bfd}: Yes}
\label{sec:bfd:config.bfd}


\subsection{\texttt{bfd/cpu-sparc.c}: Yes}
\label{sec:bfd:cpu-sparc.c}


\subsection{\texttt{bfd/elf32-sparc.c}: Yes}
\label{sec:bfd:elf32-sparc.c}


\subsection{\texttt{bfd/elf.c}: Yes}
\label{sec:bfd:elf.c}

The only  place needed  is the  routine that groks  the core  file for
NETBSD.  Since AJIT is not ported  to any other OS except GNU/Linux we
do not need to add or change any code in this file.

\subsection{\texttt{bfd/elfcode.h}: Yes}
\label{sec:bfd:elfcode.h}

This    file    returns    a    \texttt{bfd\_target}    (defined    in
\texttt{bfd/bfd-in2.h}) object from the ELF file.  Also \texttt{struct
  bfd} is in the same file.

% /* Check to see if the file associated with ABFD matches the target vector
%    that ABFD points to.

%    Note that we may be called several times with the same ABFD, but different
%    target vectors, most of which will not match.  We have to avoid leaving
%    any side effects in ABFD, or any data it points to (like tdata), if the
%    file does not match the target vector.  */

% const bfd_target *
% elf_object_p (bfd *abfd)


\subsection{\texttt{bfd/elfxx-sparc.c}: Yes}
\label{sec:bfd:elfxx-sparc.c}


\subsection{\texttt{bfd/elfxx-sparc.h}: Yes}
\label{sec:bfd:elfxx-sparc.h}


\subsection{\texttt{bfd/targets.c}: Yes}
\label{sec:bfd:targets.c}


\subsection{\texttt{bfd/bfd-in2.h}: Maybe}
\label{sec:bfd:bfd-in2.h}


\subsection{\texttt{bfd/bfd-in.h}: Maybe}
\label{sec:bfd:bfd-in.h}


\subsection{\texttt{bfd/bfd.m4}: Maybe}
\label{sec:bfd:bfd.m4}


\subsection{\texttt{bfd/configure}: Maybe}
\label{sec:bfd:configure}


\subsection{\texttt{bfd/configure.in}: Maybe}
\label{sec:bfd:configure.in}


\subsection{\texttt{bfd/elf64-ajit.c}: Maybe}
\label{sec:bfd:elf64-ajit.c}


\subsection{\texttt{bfd/elf64-sparc.c}: Maybe}
\label{sec:bfd:elf64-sparc.c}


\subsection{\texttt{bfd/freebsd.h}: Maybe}
\label{sec:bfd:freebsd.h}


\subsection{\texttt{bfd/libbfd.h}: Maybe}
\label{sec:bfd:libbfd.h}


\subsection{\texttt{bfd/Makefile.am}: Maybe}
\label{sec:bfd:Makefile.am}


\subsection{\texttt{bfd/Makefile.in}: Maybe}
\label{sec:bfd:Makefile.in}


\subsection{\texttt{bfd/nlm32-sparc.c}: Maybe}
\label{sec:bfd:nlm32-sparc.c}


\subsection{\texttt{bfd/reloc.c}: Maybe}
\label{sec:bfd:reloc.c}

\section{Studying the Build Process}
\label{sec:build:study}

To study the build process we build the GNU binutils-2.22 for at least
two targets: a native  and a cross.  Our host machine is  a 64 bit x86
as we write this.  So we use  the x86 or i386 as a the native machine,
and the  SPARC as the  cross machine.\footnote{For a cross  target our
  build generates binaries that run \textbf{on} the host machine (i386
  in this  case) and generate  output \textbf{for} the  target machine
  (SPARC  in this  case).}   We install  the  binaries built  on to  a
separate  directory  hierarchy  for   each  target  than  the  default
\texttt{/usr/local}.  We will refer to  the i386 install folder as the
\texttt{\$X86INSTALLDIR}   and  the  SPARC   install  folder   as  the
\texttt{\$SPARCINSTALLDIR}.

The  build   follows  the  usual  steps  to   building  GNU  software:
\texttt{configure} followed by  \texttt{make} followed by \texttt{make
  install}.  The standard  output and the standard error  of each step
is  individually  redirected into  files.   This  allows a  systematic
exploration  of  the sequence  of  the  build  process that  has  been
actually followed.  Here are the commands used to capture the details
of the i386 build:
\begin{enumerate}[noitemsep]
\item  \texttt{cd  BINUTILS-SOURCES-FOR-i386}:  change to  the  folder
  where  we have the  pristine binutils  sources to  be built  for the
  i386.
\item           \texttt{./configure           --prefix=\$X86INSTALLDIR
    --target=i386-pc-linux-gnu > i386-configure.stdout 2>
    i386-configure.stderr} \\
  This command sets  up the build for the i386  target.  The target is
  specified using the  \texttt{--target} option to \texttt{configure}.
  The  specification  follows  the  GNU  rules;  the  i386  target  is
  specified as: \texttt{i386-pc-linux-gnu}.  The build process is also
  informed    that    the   installation    is    to    be   in    the
  \texttt{\$X86INSTALLDIR}  folder.  Apart from  checking if  the host
  has all  the support required for the  build, the \texttt{configure}
  may also set up some variables sensitive to the target, and may even
  generate  some files that  are target  specific.  In  particular the
  \texttt{Makefile}  that  will   build  the  entire  target  specific
  binutils is generated towards the end of its run.
\item \texttt{make > i386-make.stdout 2> i386-make.stderr} \\
  Using the \texttt{Makefile} generated  by configure, this command is
  the  main workhorse  that builds  the software  system.   Usually it
  compiles and links the programs.   At times it also generates target
  specific files.  Both these are critical to our study below.
\item \texttt{make install > i386-install.stdout 2>
    i386-install.stderr} \\
  This command  installs the binaries, libraries and  any header files
  generated by the build in the required directory hierarchy below the
  \texttt{\$X86INSTALLDIR} folder.
\end{enumerate}
Similarly for the SPARC build we have in summary:
\begin{enumerate}[noitemsep]
\item  \texttt{cd BINUTILS-SOURCES-FOR-SPARC}:  change  to the  folder
  where  we have the  pristine binutils  sources to  be built  for the
  SPARC.
\item \texttt{./configure --prefix=\$SPARCINSTALLDIR
    --target=sparc-linux-gnu > sparc-configure.stdout 2> sparc-configure.stderr} 
\item \texttt{make > sparc-make.stdout 2> sparc-make.stderr} 
\item \texttt{make install > sparc-install.stdout 2> sparc-install.stderr} 
\end{enumerate}

The \texttt{configure} sequence of configuring over the folders in binutils is:
\begin{enumerate}[noitemsep]
\item Configuring in ./intl
\item Configuring in ./libiberty
\item Configuring in ./bfd
\item Configuring in ./opcodes
\item Configuring in ./binutils
\item Configuring in ./etc
\item Configuring in ./gas
\item Configuring in ./gprof
\item Configuring in ./ld
\end{enumerate}

Assuming  a successful build  our main  source of  study of  the build
process are the \texttt{*.stdout}  files.  Study of these files yields
the following sequence:
\begin{enumerate}[noitemsep]
\item Configuring in ./intl
\item Configuring in ./libiberty
\item Configuring in ./bfd
\item The files created during \texttt{configure} in \texttt{bfd/} are:
  \begin{enumerate}
  \item config.status: creating Makefile
  \item config.status: creating doc/Makefile
  \item config.status: creating bfd-in3.h:
  \item config.status: creating po/Makefile.in
  \item config.status: creating config.h
  \end{enumerate}
\item Building the libiberty library.  The build process configures in
  \texttt{bfd} \emph{before} it enters the \texttt{libiberty} to build
  this   library.    C   files    that   contribute   to   this   are:

  regex.c,   cplus-dem.c,  cp-demangle.c,  md5.c,   sha1.c,  alloca.c,
  argv.c, choose-temp.c, concat.c, cp-demint.c, crc32.c, dyn-string.c,
  fdmatch.c,  fibheap.c, filename$\_$cmp.c,  floatformat.c, fnmatch.c,
  fopen$\_$unlocked.c,  getopt.c,  getopt1.c, getpwd.c,  getruntime.c,
  hashtab.c, hex.c,  lbasename.c, lrealpath.c, make-relative-prefix.c,
  make-temp-file.c,  objalloc.c,  obstack.c, partition.c,  pexecute.c,
  physmem.c,  pex-common.c,   ne.c,  ne.c,  pex-unix.c,  safe-ctype.c,
  bject.c,    bject.c,   bject-coff.c,    bject-coff.c,   bject-elf.c,
  bject-elf.c, bject-mach, bject-mach, sort.c, spaces.c, splay-tree.c,
  stack-limit.c,   strerror.c,   strsignal.c,  rdinary.c,   rdinary.c,
  xatexit.c,  xexit.c, xmalloc.c,  xmemdup.c,  xstrdup.c, xstrerror.c,
  xstrndup.c, setproctitle.c

  % \texttt{regex.c},    \texttt{cplus-dem.c},   \texttt{cp-demangle.c},
  % \texttt{md5.c}, \texttt{sha1.c}, \texttt{alloca.c}, \texttt{argv.c},
  % \texttt{choose-temp.c},   \texttt{concat.c},   \texttt{cp-demint.c},
  % \texttt{crc32.c},     \texttt{dyn-string.c},     \texttt{fdmatch.c},
  % \texttt{fibheap.c},                         \texttt{filename\_cmp.c},
  % \texttt{floatformat.c},                           \texttt{fnmatch.c},
  % \texttt{fopen\_unlocked.c},  \texttt{getopt.c},  \texttt{getopt1.c},
  % \texttt{getpwd.c},     \texttt{getruntime.c},    \texttt{hashtab.c},
  % \texttt{hex.c},      \texttt{lbasename.c},     \texttt{lrealpath.c},
  % \texttt{make-relative-prefix.c},           \texttt{make-temp-file.c},
  % \texttt{objalloc.c},    \texttt{obstack.c},    \texttt{partition.c},
  % \texttt{pexecute.c},    \texttt{physmem.c},   \texttt{pex-common.c},
  % \texttt{ne.c},          \texttt{ne.c},          \texttt{pex-unix.c},
  % \texttt{safe-ctype.c},      \texttt{bject.c},      \texttt{bject.c},
  % \texttt{bject-coff.c},  \texttt{bject-coff.c}, \texttt{bject-elf.c},
  % \texttt{bject-elf.c},    \texttt{bject-mach},   \texttt{bject-mach},
  % \texttt{sort.c},      \texttt{spaces.c},      \texttt{splay-tree.c},
  % \texttt{stack-limit.c},  \texttt{strerror.c},  \texttt{strsignal.c},
  % \texttt{rdinary.c},      \texttt{rdinary.c},     \texttt{xatexit.c},
  % \texttt{xexit.c},       \texttt{xmalloc.c},      \texttt{xmemdup.c},
  % \texttt{xstrdup.c}, \texttt{xstrerror.c}, \texttt{xstrndup.c}.

% regex.c  cplus-dem.c  cp-demangle.c  md5.c  sha1.c  alloca.c  argv.c
% choose-temp.c  concat.c  cp-demint.c  crc32.c  dyn-string.c
% fdmatch.c  fibheap.c  filename_cmp.c  floatformat.c  fnmatch.c
% fopen_unlocked.c  getopt.c  getopt1.c  getpwd.c  getruntime.c
% hashtab.c  hex.c  lbasename.c  lrealpath.c  make-relative-prefix.c
% make-temp-file.c  objalloc.c  obstack.c  partition.c  pexecute.c
% physmem.c  pex-common.c ne.c  ne.c   pex-unix.c  safe-ctype.c
% bject.c  bject.c  bject-coff.c  bject-coff.c  bject-elf.c
% bject-elf.c  bject-mach bject-mach  sort.c  spaces.c  splay-tree.c
% stack-limit.c  strerror.c  strsignal.c rdinary.c  rdinary.c
% xatexit.c  xexit.c  xmalloc.c  xmemdup.c  xstrdup.c  xstrerror.c
% xstrndup.c  setproctitle.c

  After removing  any previous libiberty library  files, the libiberty
  library is built afresh using \texttt{ar} and \texttt{ranlib}.  Also
  a   list  of   these  object   files  is   collected  in   the  file
  \texttt{required-list}.
\item Building in \texttt{bfd}.

  The sequence here is:
  \begin{enumerate}[noitemsep]
  \item Create \texttt{bfdver.h}
  \item Create \texttt{elf32-target.h}: Commands sequence is: 
    \texttt{rm -f elf32-target.h}, 
    \texttt{sed -e s/NN/32/g < ./elfxx-target.h > elf32-target.new}, 
    \texttt{mv -f elf32-target.new elf32-target.h}.
  \item Create \texttt{elf64-target.h}: Commands sequence is: 
    \texttt{rm -f elf64-target.h}, 
    \texttt{sed -e s/NN/64/g < ./elfxx-target.h > elf64-target.new}, 
    \texttt{mv -f elf64-target.new elf64-target.h}.
  \item Create \texttt{targmatch.h}: Commands sequence is: 
    \texttt{rm -f targmatch.h}, 
    \texttt{sed -f ./targmatch.sed < ./config.bfd > targmatch.new}, 
    \texttt{mv -f targmatch.new targmatch.h}.
  \item Build the BFD specific documentation.  We skip these details.
  \item Create \texttt{bfd.h}.  Commands sequence is: 
    \texttt{rm -f bfd-tmp.h}, 
    \texttt{cp bfd-in3.h bfd-tmp.h}, 
    \texttt{/bin/bash ./../move-if-change bfd-tmp.h bfd.h}, 
    \texttt{rm -f bfd-tmp.h}, 
    \texttt{touch stmp-bfd-h}, 
  \end{enumerate}
\end{enumerate}






\end{document}
