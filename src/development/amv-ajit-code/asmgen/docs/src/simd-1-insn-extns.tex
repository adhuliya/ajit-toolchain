\subsubsection{SIMD I instructions:}
\label{sec:simd:1:insn:impl}

The  first   set  of  SIMD  instructions  are   the  three  arithmetic
instructions: add,  sub, and mul.  The ``mul''  instruction has signed
and unsigned variations.  Each of the three instructions have 8 bit (1
byte),  16 bit  (1 half  word) and  32 bit  (1 word)  versions.  These
versions  are encoded  as  shown in  table~\ref{tab:types:for:simd:1},
where the first column denotes the  bit numbers.  We list all the SIMD
I instructions version wise below.
\begin{table}[h]
  \centering
  \begin{tabular}[p]{|l|l|l|}
  \hline
  \textbf{987} & \textbf{Type} & \textbf{Example}\\
  \hline
  001 & Byte & e.g. VADDD8\\
  010 & Half-word (16-bits) & e.g. VADDD16\\
  100 & Word (32-bits) & e.g. VADDD32\\
  \hline
\end{tabular}
\caption{Data type encoding for SIMD I instructions.}
\label{tab:types:for:simd:1}
\end{table}
\begin{enumerate}
\item \textbf{8 bit} (\textbf{1 Byte})
  \begin{enumerate}
  \item \textbf{VADDD8}:\\
    \begin{center}
      \begin{tabular}[p]{|c|c|l|l|}
        \hline
        \textbf{Start} & \textbf{End} & \textbf{Range} & \textbf{Meaning} \\
        \hline
        0 & 4 & 32 & Source register 2, rs2 \\
        5 & 6 & 4 & \emph{Always} 2, i.e. insn[6:5] = 10$_b$ \\
        7 & 9 & 8 & \textbf{Data type} specifier:  \emph{Always} 0x1\\
        10 & 12 & -- & \textbf{unused} \\
        13 & 13 & 0,1 & The \textbf{i} bit. \emph{Always} 0. \\
        14 & 18 & 32 & Source register 1, rs1 \\
        19 & 24 & 000000 & ``\textbf{op3}'' \\
        25 & 29 & 32 & Destination register, rd \\
        30 & 31 & 4 & Always ``10'' \\
        \hline
      \end{tabular}
    \end{center}
    \begin{itemize}
    \item []\textbf{VADDD8}: same as  ADD, but with Instr[13]=0 (i=0),
      and  Instr[6:5]=2.  Bits Instr[9:7]  are  a  3-bit field,  which
      specify the data type
    \item []\textbf{Syntax}: ``\texttt{vaddd8 SrcReg1, SrcReg2,
        DestReg}''.
    \item []\textbf{Semantics}: \emph{not given}
    \end{itemize}
    Bits layout:
\begin{verbatim}
    Offsets      : 31       24 23       16  15        8   7        0
    Bit layout   :  XXXX  XXXX  XXXX  XXXX   XXXX  XXXX   XXXX  XXXX
    Insn Bits    :  10       0  0000  0        0     00   110       
    Destination  :    DD  DDD                                       
    Source 1     :                     SSS   SS
    Source 2     :                                           S  SSSS
    Unused (0)   :                              U  UU               
    Final layout :  10DD  DDD0  0000  0SSS   SS0U  UU00   110S  SSSS
    To match     :  ^^       ^  ^^^^  ^        ^     ^^   ^^^
    Bitfield name:  OP          OP3            i     9-   765
\end{verbatim}

    To  set  up  bits  5  and  6, we  use  an  already  defined  macro
    \texttt{OP\_AJIT\_BIT\_5\_AND\_6}.  The  value to be set  in these
    two bits is 0x2.   To set bits 7 through 9, we  define a new macro
    \texttt{OP\_AJIT\_BIT\_7\_THRU\_9}.  The value  set in these three
    bits  decides the  \emph{type}, byte,  half word  or word,  of the
    instruction.  For  \textbf{vaddd8} instruction,  bits 7  through 9
    are  set  to the  value  0x1.   Both  these macros  influence  the
    \emph{unused} bits of  the SPARC V8 architecture.  So  we define a
    macro \texttt{OP\_AJIT\_SET\_UNUSED} that uses the previous two to
    set these bits unused by the SPARC V8, but used by AJIT.

    \verb|#define OP_AJIT_BIT_7_THRU_9(x)   ((x) << 0x7)|

    \verb+#define OP_AJIT_SET_UNUSED        (OP_AJIT_BIT_5_AND_6(0x2) | \\+

    \verb+                                   OP_AJIT_BIT_7_THRU_9(0x1))+

    We can  now define the final  macro \texttt{F6(x, y, z,  b, a)} to
    set the match bits for this instruction.
\begin{verbatim}
#define OP_AJIT_BIT_5(x)          (((x) & 0x1) << 5)
#define F4(x, y, z, b)            (F3(x, y, z) | OP_AJIT_BIT_5(b))
#define OP_AJIT_BIT_5_AND_6(x)    (((x) & 0x3) << 6)
#define F5(x, y, z, b)            (F3(x, y, z) | OP_AJIT_BIT_5_AND_6 (b))
#define OP_AJIT_BIT_7_THRU_9(x)   (((x) & 0x3) << 9)
#define F6(x, y, z, b, a)         (F5 (x, y, z, b) | OP_AJIT_BIT_7_THRU_9(a))
\end{verbatim}
    Hence the SPARC bit layout of this instruction is:

  \begin{tabular}[h]{lclcl}
    Macro to set  &=& \texttt{F4(x, y, z)} &in& \texttt{sparc.h}     \\
    Macro to reset  &=& \texttt{INVF4(x, y, z)} &in& \texttt{sparc.h}     \\
    x &=& 0x2      &in& \texttt{OP(x)  /* ((x) \& 0x3)  $<<$ 30 */} \\
    y &=& 0x00     &in& \texttt{OP3(y) /* ((y) \& 0x3f) $<<$ 19 */} \\
    z &=& 0x0      &in& \texttt{F3I(z) /* ((z) \& 0x1)  $<<$ 13 */} \\
    a &=& 0x1      &in& \texttt{OP\_AJIT\_BIT(a) /* ((a) \& 0x1)  $<<$ 5 */}
  \end{tabular}

  The AJIT bit (insn[5]) is set internally by \texttt{F4}, and hence
  there are only three arguments.
\end{enumerate}
\item \textbf{1 Half word} (\textbf{16 bit})
\item \textbf{1 Word} (\textbf{32 bit})
\end{enumerate}

