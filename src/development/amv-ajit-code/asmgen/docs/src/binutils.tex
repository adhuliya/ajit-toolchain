\chapter{AJIT Support for the GNU Binutils Toolchain}
\label{chap:amv:work}

\section{Towards a GNU Binutils Toolchain}
\label{sec:binutils:support}

This section describes the details  of adding the AJIT instructions to
SPARC v8 part of GNU Binutils 2.22.  We use the SPARC v8 manual to get
the details of  the sparc instruction.  It's bit  pattern is described
\emph{again}, and  the new  bit pattern  required for  AJIT is  set up
alongside.  Bit layouts to determine  the ``match'' etc.  of the sparc
port  are  also  laid  out.    The  SPARC  manual  also  contains  the
``suggested asm syntax''  that we adapt for the  new AJIT instruction.
The sections below follow the sections in chapter~\ref{sec:isa:extns}.
For each  instruction, we  need to  define its  bitfields in  terms of
macros  in  \texttt{\$BINUTILSHOME/include/opcode/sparc.h} and  define
the opcodes table in \texttt{\$BINUTILSHOME/opcodes/sparc-opc.c}.

The AJIT  instructions are  variations of  the corresponding  SPARC V8
instructions.  Please refer to the SPARC V8 manual for details of such
corresponding SPARC instructions. For  example, the \texttt{ADD} insn,
pg. 108 (pg.  130 in PDF  sequence) of the manual.  Other instructions
can be similarly found, and will not be mentioned.

\subsection{Integer-Unit Extensions: Arithmetic-Logic Instructions}
\label{sec:integer-unit-extns:arith-logic-insns:impl}

The  integer  unit extensions  of  AJIT  are  based  on the  SPARC  V8
instructions.    See:  SPArc   v8  architecture   manual.   SPARC   v8
instructions  are  32   bits  long.   The  GNU   Binutils  2.22  SPARC
implementation defines a  set of macros to capture the  bits set by an
instruction.  These are the so called ``match'' masks.  Please see the
code     in     \texttt{\$BINUTILSHOME/include/opcode/sparc.h}     and
\texttt{\$BINUTILSHOME/opcodes/sparc-opc.c}.
\vfill
\newpage
\subsubsection{Addition and subtraction instructions:}
\label{sec:add:sub:insn:impl}
\begin{enumerate}
\item \textbf{ADDD}:\\
  \begin{center}
    \begin{tabular}[p]{|c|c|l|l|l|}
      \hline
      \textbf{Start} & \textbf{End} & \textbf{Range} & \textbf{Meaning} &
                                                                          \textbf{New Meaning}\\
      \hline
      0 & 4 & 32 & Source register 2, rs2 & No change \\
      5 & 12 & -- & \textbf{unused} & \textbf{Set bit 5 to ``1''} \\
      13 & 13 & 0,1 & The \textbf{i} bit & \textbf{Set i to ``0''} \\
      14 & 18 & 32 & Source register 1, rs1 & No change \\
      19 & 24 & 000000 & ``\textbf{op3}'' & No change \\
      25 & 29 & 32 & Destination register, rd & No change \\
      30 & 31 & 4 & Always ``10'' & No change \\
      \hline
    \end{tabular}
  \end{center}
  \begin{itemize}
  \item []\textbf{ADDD}: same as ADD, but with Instr[13]=0 (i=0), and
    Instr[5]=1.
  \item []\textbf{Syntax}: ``\texttt{addd  SrcReg1, SrcReg2, DestReg}''.
  \item []\textbf{Semantics}: rd(pair) $\leftarrow$ rs1(pair) + rs2(pair).
  \end{itemize}
  Bits layout:
\begin{verbatim}
    Offsets      : 31       24 23       16  15        8   7        0
    Bit layout   :  XXXX  XXXX  XXXX  XXXX   XXXX  XXXX   XXXX  XXXX
    Insn Bits    :  10       0  0000  0        0            1       
    Destination  :    DD  DDD                                       
    Source 1     :                     SSS   SS
    Source 2     :                                           S  SSSS
    Unused (0)   :                              U  UUUU   UU        
    Final layout :  10DD  DDD0  0000  0SSS   SS0U  UUUU   UU1S  SSSS
\end{verbatim}

  Hence the SPARC bit layout of this instruction is:

  \begin{tabular}[h]{lclcl}
    Macro to set  &=& \texttt{F4(x, y, z)} &in& \texttt{sparc.h}     \\
    Macro to reset  &=& \texttt{INVF4(x, y, z)} &in& \texttt{sparc.h}     \\
    x &=& 0x2      &in& \texttt{OP(x)  /* ((x) \& 0x3)  $<<$ 30 */} \\
    y &=& 0x00     &in& \texttt{OP3(y) /* ((y) \& 0x3f) $<<$ 19 */} \\
    z &=& 0x0      &in& \texttt{F3I(z) /* ((z) \& 0x1)  $<<$ 13 */} \\
    a &=& 0x1      &in& \texttt{OP\_AJIT\_BIT(a) /* ((a) \& 0x1)  $<<$ 5 */}
  \end{tabular}

  The AJIT bit  (insn[5]) is set internally by  \texttt{F4}, and hence
  there are only three arguments.

\item \textbf{ADDDCC}:\\
  \begin{center}
    \begin{tabular}[p]{|c|c|l|l|l|}
      \hline
      \textbf{Start} & \textbf{End} & \textbf{Range} & \textbf{Meaning} &
                                                                          \textbf{New Meaning}\\
      \hline
      0 & 4 & 32 & Source register 2, rs2 & No change \\
      5 & 12 & -- & \textbf{unused} & \textbf{Set bit 5 to ``1''} \\
      13 & 13 & 0,1 & The \textbf{i} bit & \textbf{Set i to ``0''} \\
      14 & 18 & 32 & Source register 1, rs1 & No change \\
      19 & 24 & 010000 & ``\textbf{op3}'' & No change \\
      25 & 29 & 32 & Destination register, rd & No change \\
      30 & 31 & 4 & Always ``10'' & No change \\
      \hline
    \end{tabular}
  \end{center}
  New addition:
  \begin{itemize}
  \item []\textbf{ADDDCC}: same as ADDCC, but with Instr[13]=0 (i=0), and
    Instr[5]=1.
  \item []\textbf{Syntax}: ``\texttt{adddcc  SrcReg1, SrcReg2, DestReg}''.
  \item []\textbf{Semantics}: rd(pair) $\leftarrow$ rs1(pair) + rs2(pair), set Z,N
  \end{itemize}
  Bits layout:
\begin{verbatim}
    Offsets      : 31       24 23       16  15        8   7        0
    Bit layout   :  XXXX  XXXX  XXXX  XXXX   XXXX  XXXX   XXXX  XXXX
    Insn Bits    :  10       0  1000  0        0            1       
    Destination  :    DD  DDD                                       
    Source 1     :                     SSS   SS
    Source 2     :                                           S  SSSS
    Unused (0)   :                              U  UUUU   UU        
    Final layout :  10DD  DDD0  1000  0SSS   SS0U  UUUU   UU1S  SSSS
\end{verbatim}

  Hence the SPARC bit layout of this instruction is:

  \begin{tabular}[h]{lclcl}
    Macro to set  &=& \texttt{F4(x, y, z)} &in& \texttt{sparc.h}     \\
    Macro to reset  &=& \texttt{INVF4(x, y, z)} &in& \texttt{sparc.h}     \\
    x &=& 0x2      &in& \texttt{OP(x)  /* ((x) \& 0x3)  $<<$ 30 */} \\
    y &=& 0x10     &in& \texttt{OP3(y) /* ((y) \& 0x3f) $<<$ 19 */} \\
    z &=& 0x0      &in& \texttt{F3I(z) /* ((z) \& 0x1)  $<<$ 13 */} \\
    a &=& 0x1      &in& \texttt{OP\_AJIT\_BIT(a) /* ((a) \& 0x1)  $<<$ 5 */}
  \end{tabular}

  The AJIT bit  (insn[5]) is set internally by  \texttt{F4}, and hence
  there are only three arguments.

\item \textbf{SUBD}:\\
  \begin{center}
    \begin{tabular}[p]{|c|c|l|l|l|}
      \hline
      \textbf{Start} & \textbf{End} & \textbf{Range} & \textbf{Meaning} &
                                                                          \textbf{New Meaning}\\
      \hline
      0 & 4 & 32 & Source register 2, rs2 & No change \\
      5 & 12 & -- & \textbf{unused} & \textbf{Set bit 5 to ``1''} \\
      13 & 13 & 0,1 & The \textbf{i} bit & \textbf{Set i to ``0''} \\
      14 & 18 & 32 & Source register 1, rs1 & No change \\
      19 & 24 & 000100 & ``\textbf{op3}'' & No change \\
      25 & 29 & 32 & Destination register, rd & No change \\
      30 & 31 & 4 & Always ``10'' & No change \\
      \hline
    \end{tabular}
  \end{center}
  New addition:
  \begin{itemize}
  \item []\textbf{SUBD}: same as SUB, but with Instr[13]=0 (i=0), and
    Instr[5]=1.
  \item []\textbf{Syntax}: ``\texttt{subd  SrcReg1, SrcReg2, DestReg}''.
  \item []\textbf{Semantics}: rd(pair) $\leftarrow$ rs1(pair) - rs2(pair).
  \end{itemize}
  Bits layout:
\begin{verbatim}
    Offsets      : 31       24 23       16  15        8   7        0
    Bit layout   :  XXXX  XXXX  XXXX  XXXX   XXXX  XXXX   XXXX  XXXX
    Insn Bits    :  10       0  0010  0        0            1       
    Destination  :    DD  DDD                                       
    Source 1     :                     SSS   SS
    Source 2     :                                           S  SSSS
    Unused (0)   :                              U  UUUU   UU        
    Final layout :  10DD  DDD0  0010  0SSS   SS0U  UUUU   UU1S  SSSS
\end{verbatim}

  Hence the SPARC bit layout of this instruction is:

  \begin{tabular}[h]{lclcl}
    Macro to set  &=& \texttt{F4(x, y, z)} &in& \texttt{sparc.h}     \\
    Macro to reset  &=& \texttt{INVF4(x, y, z)} &in& \texttt{sparc.h}     \\
    x &=& 0x2      &in& \texttt{OP(x)  /* ((x) \& 0x3)  $<<$ 30 */} \\
    y &=& 0x04     &in& \texttt{OP3(y) /* ((y) \& 0x3f) $<<$ 19 */} \\
    z &=& 0x0      &in& \texttt{F3I(z) /* ((z) \& 0x1)  $<<$ 13 */} \\
    a &=& 0x1      &in& \texttt{OP\_AJIT\_BIT(a) /* ((a) \& 0x1)  $<<$ 5 */}
  \end{tabular}

  The AJIT bit  (insn[5]) is set internally by  \texttt{F4}, and hence
  there are only three arguments.

\item \textbf{SUBDCC}:\\
  \begin{center}
    \begin{tabular}[p]{|c|c|l|l|l|}
      \hline
      \textbf{Start} & \textbf{End} & \textbf{Range} & \textbf{Meaning} &
                                                                          \textbf{New Meaning}\\
      \hline
      0 & 4 & 32 & Source register 2, rs2 & No change \\
      5 & 12 & -- & \textbf{unused} & \textbf{Set bit 5 to ``1''} \\
      13 & 13 & 0,1 & The \textbf{i} bit & \textbf{Set i to ``0''} \\
      14 & 18 & 32 & Source register 1, rs1 & No change \\
      19 & 24 & 010100 & ``\textbf{op3}'' & No change \\
      25 & 29 & 32 & Destination register, rd & No change \\
      30 & 31 & 4 & Always ``10'' & No change \\
      \hline
    \end{tabular}
  \end{center}
  New addition:
  \begin{itemize}
  \item []\textbf{SUBDCC}: same as SUBCC, but with Instr[13]=0 (i=0), and
    Instr[5]=1.
  \item []\textbf{Syntax}: ``\texttt{subdcc  SrcReg1, SrcReg2, DestReg}''.
  \item []\textbf{Semantics}: rd(pair) $\leftarrow$ rs1(pair) - rs2(pair), set Z,N
  \end{itemize}
  Bits layout:
\begin{verbatim}
    Offsets      : 31       24 23       16  15        8   7        0
    Bit layout   :  XXXX  XXXX  XXXX  XXXX   XXXX  XXXX   XXXX  XXXX
    Insn Bits    :  10       0  1010  0        0            1       
    Destination  :    DD  DDD                                       
    Source 1     :                     SSS   SS
    Source 2     :                                           S  SSSS
    Unused (0)   :                              U  UUUU   UU        
    Final layout :  10DD  DDD0  1010  0SSS   SS0U  UUUU   UU1S  SSSS
\end{verbatim}

  Hence the SPARC bit layout of this instruction is:

  \begin{tabular}[h]{lclcl}
    Macro to set  &=& \texttt{F4(x, y, z)} &in& \texttt{sparc.h}     \\
    Macro to reset  &=& \texttt{INVF4(x, y, z)} &in& \texttt{sparc.h}     \\
    x &=& 0x2      &in& \texttt{OP(x)  /* ((x) \& 0x3)  $<<$ 30 */} \\
    y &=& 0x14     &in& \texttt{OP3(y) /* ((y) \& 0x3f) $<<$ 19 */} \\
    z &=& 0x0      &in& \texttt{F3I(z) /* ((z) \& 0x1)  $<<$ 13 */} \\
    a &=& 0x1      &in& \texttt{OP\_AJIT\_BIT(a) /* ((a) \& 0x1)  $<<$ 5 */}
  \end{tabular}

  The AJIT bit  (insn[5]) is set internally by  \texttt{F4}, and hence
  there are only three arguments.
\end{enumerate}

\subsubsection{Shift instructions:}
\label{sec:shift:insn:impl}
The shift  family of instructions  of AJIT  may each be  considered to
have  two versions:  a direct  count version  and a  register indirect
count version.  In the direct count  version the shift count is a part
of the  instruction bits.   In the indirect  count version,  the shift
count is  found on the  register specified by  the bit pattern  in the
instruction  bits.   The direct  count  version  is specified  by  the
14$^{th}$  bit, i.e.  insn[13]  (bit  number 13  in  the  0 based  bit
numbering scheme), being set to 1.  If insn[13] is 0 then the register
indirect version is specified.

Similar to the addition and subtraction instructions, the shift family
of instructions of  SPARC V8 also do  not use bits from 5  to 12 (both
inclusive).  The AJIT processor uses bits  5 and 6.  In particular bit
6 is always 1.   Bit 5 may be used in the direct  version giving a set
of 6 bits  available for specifying the shift count.   The shift count
can have  a maximum  value of  64.  Bit  5 is  unused in  the register
indirect version, and is always 0 in that case.

These instructions  are therefore  worked out  below in  two different
sets: the direct and the register indirect ones.
\begin{enumerate}
\item The direct versions  are given by insn[13] = 1.  The 6 bit shift
  count  is directly  specified  in the  instruction bits.   Therefore
  insn[5:0] specify the  shift count.  insn[6] =  1, distinguishes the
  AJIT version from the SPARC V8 version.
  \begin{enumerate}
  \item \textbf{SLLD}:\\
    \begin{center}
      \begin{tabular}[p]{|c|c|l|p{.25\textwidth}|p{.3\textwidth}|}
        \hline
        \textbf{Start} & \textbf{End} & \textbf{Range} & \textbf{Meaning} & \textbf{New Meaning}\\
        \hline
        0 & 4 & 32 & Source register 2, rs2 & Lowest 5 bits of shift count \\
        \hline
        5 & 12 & -- & \textbf{Unused. Set to 0 by software.} &
                                        \begin{minipage}[h]{1.0\linewidth}
                                          \begin{itemize}
                                          \item \textbf{Use bit 5
                                              to specify the msb of
                                              shift count.}
                                          \item \textbf{Use bit 6 to
                                              distinguish AJIT from
                                              SPARC V8.}
                                          \item \textbf{Set bits 7:12
                                              to 0.}
                                          \end{itemize}
                                        \end{minipage}
        \\
        \hline
        13 & 13 & 0,1 & The \textbf{i} bit & \textbf{Set i to ``1''} \\
        14 & 18 & 32 & Source register 1, rs1 & No change \\
        19 & 24 & 100101 & ``\textbf{op3}'' & No change \\
        25 & 29 & 32 & Destination register, rd & No change \\
        30 & 31 & 4 & Always ``10'' & No change \\
        \hline
      \end{tabular}
    \end{center}
    \begin{itemize}
    \item []\textbf{SLLD}: same as SLL, but with Instr[13]=0 (i=0),
      and Instr[5]=1.
    \item []\textbf{Syntax}: ``\texttt{slld SrcReg1, 6BitShiftCnt,
        DestReg}''. \\
      (\textbf{Note:} In an assembly language program, when the second
      argument is a number, we have direct mode.  A register number is
      prefixed with  ``r'', and hence the  syntax itself distinguished
      between   direct  and   register   indirect   version  of   this
      instruction.)
    \item []\textbf{Semantics}: rd(pair) $\leftarrow$ rs1(pair) $<<$
      shift count.
    \end{itemize}
    Bits layout:
\begin{verbatim}
    Offsets      : 31       24 23       16  15        8   7        0
    Bit layout   :  XXXX  XXXX  XXXX  XXXX   XXXX  XXXX   XXXX  XXXX
    Insn Bits    :  10       1  0010  1        1           1        
    Destination  :    DD  DDD                                       
    Source 1     :                     SSS   SS
    Source 2     :                                           S  SSSS
    Unused (0)   :                              U  UUUU   UU        
    Final layout :  10DD  DDD1  0010  1SSS   SS1U  UUUU   U1II  IIII
\end{verbatim}

    This will need another macro that sets bits 5 and 6. Let's call it
    \texttt{OP\_AJIT\_BITS\_5\_AND\_6}.   Hence the  SPARC bit  layout of  this
    instruction is:

    \begin{tabular}[h]{lclcl}
      Macro to set  &=& \texttt{F5(x, y, z)} &in& \texttt{sparc.h}     \\
      Macro to reset  &=& \texttt{INVF5(x, y, z)} &in& \texttt{sparc.h}     \\
      x &=& 0x2      &in& \texttt{OP(x)  /* ((x) \& 0x3)  $<<$ 30 */} \\
      y &=& 0x25     &in& \texttt{OP3(y) /* ((y) \& 0x3f) $<<$ 19 */} \\
      z &=& 0x1      &in& \texttt{F3I(z) /* ((z) \& 0x1)  $<<$ 13 */} \\
      a &=& 0x2      &in& \texttt{OP\_AJIT\_BITS\_5\_AND\_6(a) /* ((a) \& 0x3  $<<$ 6 */}
    \end{tabular}

    The AJIT bits (insn[6:5]) is  set or reset internally by \texttt{F5}
    (just  like  in  \texttt{F4}),  and   hence  there  are  only  three
    arguments.

  \item \textbf{SRLD}:\\
    \begin{center}
      \begin{tabular}[p]{|c|c|l|l|p{.35\textwidth}|}
        \hline
        \textbf{Start} & \textbf{End} & \textbf{Range} & \textbf{Meaning} & \textbf{New Meaning}\\
        \hline
        0 & 4 & 32 & Source register 2, rs2 & Lowest 5 bits of shift count \\
        \hline
        5 & 12 & -- & \textbf{unused} &
                                        \begin{minipage}[h]{1.0\linewidth}
                                          \begin{itemize}
                                          \item \textbf{Use bit 5
                                              to specify the msb of
                                              shift count.}
                                          \item \textbf{Use bit 6 to
                                              distinguish AJIT from
                                              SPARC V8.}
                                          \end{itemize}
                                        \end{minipage}
        \\
        \hline
        13 & 13 & 0,1 & The \textbf{i} bit & \textbf{Set i to ``1''} \\
        14 & 18 & 32 & Source register 1, rs1 & No change \\
        19 & 24 & 100110 & ``\textbf{op3}'' & No change \\
        25 & 29 & 32 & Destination register, rd & No change \\
        30 & 31 & 4 & Always ``10'' & No change \\
        \hline
      \end{tabular}
    \end{center}
    \begin{itemize}
    \item []\textbf{SRLD}: same as SRL, but with Instr[13]=0 (i=0),
      and Instr[5]=1.
    \item []\textbf{Syntax}: ``\texttt{sral SrcReg1, 6BitShiftCnt,
        DestReg}''. \\
      (\textbf{Note:} In an assembly language program, when the second
      argument is a number, we have direct mode.  A register number is
      prefixed with  ``r'', and hence the  syntax itself distinguished
      between   direct  and   register   indirect   version  of   this
      instruction.)
    \item []\textbf{Semantics}: rd(pair) $\leftarrow$ rs1(pair) $>>$
      shift count.
    \end{itemize}
    Bits layout:
\begin{verbatim}
    Offsets      : 31       24 23       16  15        8   7        0
    Bit layout   :  XXXX  XXXX  XXXX  XXXX   XXXX  XXXX   XXXX  XXXX
    Insn Bits    :  10       1  0011  0        1           1        
    Destination  :    DD  DDD                                       
    Source 1     :                     SSS   SS
    Source 2     :                                           S  SSSS
    Unused (0)   :                              U  UUUU   UU        
    Final layout :  10DD  DDD1  0011  0SSS   SS1U  UUUU   U1II  IIII
\end{verbatim}

    This will need another macro that sets bits 5 and 6. Let's call it
    \texttt{OP\_AJIT\_BITS\_5\_AND\_6}.   Hence the  SPARC bit  layout of  this
    instruction is:

    \begin{tabular}[h]{lclcl}
      Macro to set  &=& \texttt{F5(x, y, z)} &in& \texttt{sparc.h}     \\
      Macro to reset  &=& \texttt{INVF5(x, y, z)} &in& \texttt{sparc.h}     \\
      x &=& 0x2      &in& \texttt{OP(x)  /* ((x) \& 0x3)  $<<$ 30 */} \\
      y &=& 0x26     &in& \texttt{OP3(y) /* ((y) \& 0x3f) $<<$ 19 */} \\
      z &=& 0x1      &in& \texttt{F3I(z) /* ((z) \& 0x1)  $<<$ 13 */} \\
      a &=& 0x2      &in& \texttt{OP\_AJIT\_BITS\_5\_AND\_6(a) /* ((a) \& 0x3  $<<$ 6 */}
    \end{tabular}

    The AJIT bits (insn[6:5]) is  set or reset internally by \texttt{F5}
    (just  like  in  \texttt{F4}),  and   hence  there  are  only  three
    arguments.
    
  \item \textbf{SRAD}:\\
    \begin{center}
      \begin{tabular}[p]{|c|c|l|l|p{.35\textwidth}|}
        \hline
        \textbf{Start} & \textbf{End} & \textbf{Range} & \textbf{Meaning} & \textbf{New Meaning}\\
        \hline
        0 & 4 & 32 & Source register 2, rs2 & Lowest 5 bits of shift count \\
        \hline
        5 & 12 & -- & \textbf{unused} &
                                        \begin{minipage}[h]{1.0\linewidth}
                                          \begin{itemize}
                                          \item \textbf{Use bit 5
                                              to specify the msb of
                                              shift count.}
                                          \item \textbf{Use bit 6 to
                                              distinguish AJIT from
                                              SPARC V8.}
                                          \end{itemize}
                                        \end{minipage}
        \\
        \hline
        13 & 13 & 0,1 & The \textbf{i} bit & \textbf{Set i to ``1''} \\
        14 & 18 & 32 & Source register 1, rs1 & No change \\
        19 & 24 & 100111 & ``\textbf{op3}'' & No change \\
        25 & 29 & 32 & Destination register, rd & No change \\
        30 & 31 & 4 & Always ``10'' & No change \\
        \hline
      \end{tabular}
    \end{center}
    \begin{itemize}
    \item []\textbf{SRAD}: same as SRA, but with Instr[13]=0 (i=0),
      and Instr[5]=1.
    \item []\textbf{Syntax}: ``\texttt{srad SrcReg1, 6BitShiftCnt,
        DestReg}''. \\
      (\textbf{Note:} In an assembly language program, when the second
      argument is a number, we have direct mode.  A register number is
      prefixed with  ``r'', and hence the  syntax itself distinguished
      between   direct  and   register   indirect   version  of   this
      instruction.)
    \item []\textbf{Semantics}: rd(pair) $\leftarrow$ rs1(pair) $>>$
      shift count (with sign extension).
    \end{itemize}
    Bits layout:
\begin{verbatim}
    Offsets      : 31       24 23       16  15        8   7        0
    Bit layout   :  XXXX  XXXX  XXXX  XXXX   XXXX  XXXX   XXXX  XXXX
    Insn Bits    :  10       1  0011  1        1           1        
    Destination  :    DD  DDD                                       
    Source 1     :                     SSS   SS
    Source 2     :                                           S  SSSS
    Unused (0)   :                              U  UUUU   UU        
    Final layout :  10DD  DDD1  0011  1SSS   SS1U  UUUU   U1II  IIII
\end{verbatim}

    This will need another macro that sets bits 5 and 6. Let's call it
    \texttt{OP\_AJIT\_BITS\_5\_AND\_6}.   Hence the  SPARC bit  layout of  this
    instruction is:

    \begin{tabular}[h]{lclcl}
      Macro to set  &=& \texttt{F5(x, y, z)} &in& \texttt{sparc.h}     \\
      Macro to reset  &=& \texttt{INVF5(x, y, z)} &in& \texttt{sparc.h}     \\
      x &=& 0x2      &in& \texttt{OP(x)  /* ((x) \& 0x3)  $<<$ 30 */} \\
      y &=& 0x27     &in& \texttt{OP3(y) /* ((y) \& 0x3f) $<<$ 19 */} \\
      z &=& 0x1      &in& \texttt{F3I(z) /* ((z) \& 0x1)  $<<$ 13 */} \\
      a &=& 0x2      &in& \texttt{OP\_AJIT\_BITS\_5\_AND\_6(a) /* ((a) \& 0x3  $<<$ 6 */}
    \end{tabular}

    The AJIT bits (insn[6:5]) is  set or reset internally by \texttt{F5}
    (just  like  in  \texttt{F4}),  and   hence  there  are  only  three
    arguments.

  \end{enumerate}
\item The register  indirect versions are given by insn[13]  = 0.  The
  shift count is indirectly specified in the 32 bit register specified
  in instruction bits.  Therefore  insn[4:0] specify the register that
  has the  shift count.  insn[6]  = 1, distinguishes the  AJIT version
  from the SPARC V8 version.  In this case, insn[5] = 0, necessarily.
  \begin{enumerate}
  \item \textbf{SLLD}:\\
    \begin{center}
      \begin{tabular}[p]{|c|c|l|l|p{.35\textwidth}|}
        \hline
        \textbf{Start} & \textbf{End} & \textbf{Range} & \textbf{Meaning} &
                                                                            \textbf{New Meaning}\\
        \hline
        0 & 4 & 32 & Source register 2, rs2 & Register number \\
        \hline
        5 & 12 & -- & \textbf{unused} &
                                        \begin{minipage}[h]{1.0\linewidth}
                                          \begin{itemize}
                                          \item \textbf{Set bit 5 to 0.}
                                          \item \textbf{Use bit 6 to
                                              distinguish AJIT from
                                              SPARC V8.}
                                          \end{itemize}
                                        \end{minipage}
        \\
        \hline
        13 & 13 & 0,1 & The \textbf{i} bit & \textbf{Set i to ``0''} \\
        14 & 18 & 32 & Source register 1, rs1 & No change \\
        19 & 24 & 100101 & ``\textbf{op3}'' & No change \\
        25 & 29 & 32 & Destination register, rd & No change \\
        30 & 31 & 4 & Always ``10'' & No change \\
        \hline
      \end{tabular}
    \end{center}
    \begin{itemize}
    \item []\textbf{SLLD}: same as SLL, but with Instr[13]=0 (i=0),
      and Instr[5]=1.
    \item []\textbf{Syntax}: ``\texttt{slld SrcReg1, SrcReg2,
        DestReg}''.
    \item []\textbf{Semantics}: rd(pair) $\leftarrow$ rs1(pair) $<<$
      shift count register rs2.
    \end{itemize}
    Bits layout:
\begin{verbatim}
    Offsets      : 31       24 23       16  15        8   7        0
    Bit layout   :  XXXX  XXXX  XXXX  XXXX   XXXX  XXXX   XXXX  XXXX
    Insn Bits    :  10       1  0010  1        0           10        
    Destination  :    DD  DDD                                       
    Source 1     :                     SSS   SS
    Source 2     :                                           S  SSSS
    Unused (0)   :                              U  UUUU   UU        
    Final layout :  10DD  DDD1  0010  1SSS   SS0U  UUUU   U10I  IIII
\end{verbatim}

    This will need another macro that sets bits 5 and 6. Let's call it
    \texttt{OP\_AJIT\_BITS\_5\_AND\_6}.   Hence the  SPARC bit  layout of  this
    instruction is:

    \begin{tabular}[h]{lclcl}
      Macro to set  &=& \texttt{F5(x, y, z)} &in& \texttt{sparc.h}     \\
      Macro to reset  &=& \texttt{INVF5(x, y, z)} &in& \texttt{sparc.h}     \\
      x &=& 0x2      &in& \texttt{OP(x)  /* ((x) \& 0x3)  $<<$ 30 */} \\
      y &=& 0x25     &in& \texttt{OP3(y) /* ((y) \& 0x3f) $<<$ 19 */} \\
      z &=& 0x0      &in& \texttt{F3I(z) /* ((z) \& 0x1)  $<<$ 13 */} \\
      a &=& 0x2      &in& \texttt{OP\_AJIT\_BITS\_5\_AND\_6(a) /* ((a) \& 0x3  $<<$ 6 */}
    \end{tabular}

    The AJIT bits (insn[6:5]) is  set or reset internally by \texttt{F5}
    (just  like  in  \texttt{F4}),  and   hence  there  are  only  three
    arguments.

  \item \textbf{SRLD}:\\
    \begin{center}
      \begin{tabular}[p]{|c|c|l|l|p{.35\textwidth}|}
        \hline
        \textbf{Start} & \textbf{End} & \textbf{Range} & \textbf{Meaning} &
                                                                            \textbf{New Meaning}\\
        \hline
        0 & 4 & 32 & Source register 2, rs2 & Register number \\
        \hline
        5 & 12 & -- & \textbf{unused} &
                                        \begin{minipage}[h]{1.0\linewidth}
                                          \begin{itemize}
                                          \item \textbf{Set bit 5 to 0.}
                                          \item \textbf{Use bit 6 to
                                              distinguish AJIT from
                                              SPARC V8.}
                                          \end{itemize}
                                        \end{minipage}
        \\
        \hline
        13 & 13 & 0,1 & The \textbf{i} bit & \textbf{Set i to ``0''} \\
        14 & 18 & 32 & Source register 1, rs1 & No change \\
        19 & 24 & 100110 & ``\textbf{op3}'' & No change \\
        25 & 29 & 32 & Destination register, rd & No change \\
        30 & 31 & 4 & Always ``10'' & No change \\
        \hline
      \end{tabular}
    \end{center}
    \begin{itemize}
    \item []\textbf{SRLD}: same as SRL, but with Instr[13]=0 (i=0),
      and Instr[5]=1.
    \item []\textbf{Syntax}: ``\texttt{slld SrcReg1, SrcReg2,
        DestReg}''.
    \item []\textbf{Semantics}: rd(pair) $\leftarrow$ rs1(pair) $>>$
      shift count register rs2.
    \end{itemize}
    Bits layout:
\begin{verbatim}
    Offsets      : 31       24 23       16  15        8   7        0
    Bit layout   :  XXXX  XXXX  XXXX  XXXX   XXXX  XXXX   XXXX  XXXX
    Insn Bits    :  10       1  0011  0        0           10        
    Destination  :    DD  DDD                                       
    Source 1     :                     SSS   SS
    Source 2     :                                           S  SSSS
    Unused (0)   :                              U  UUUU   UU        
    Final layout :  10DD  DDD1  0011  0SSS   SS0U  UUUU   U10I  IIII
\end{verbatim}

    This will need another macro that sets bits 5 and 6. Let's call it
    \texttt{OP\_AJIT\_BITS\_5\_AND\_6}.   Hence the  SPARC bit  layout of  this
    instruction is:

    \begin{tabular}[h]{lclcl}
      Macro to set  &=& \texttt{F5(x, y, z)} &in& \texttt{sparc.h}     \\
      Macro to reset  &=& \texttt{INVF5(x, y, z)} &in& \texttt{sparc.h}     \\
      x &=& 0x2      &in& \texttt{OP(x)  /* ((x) \& 0x3)  $<<$ 30 */} \\
      y &=& 0x26     &in& \texttt{OP3(y) /* ((y) \& 0x3f) $<<$ 19 */} \\
      z &=& 0x0      &in& \texttt{F3I(z) /* ((z) \& 0x1)  $<<$ 13 */} \\
      a &=& 0x2      &in& \texttt{OP\_AJIT\_BITS\_5\_AND\_6(a) /* ((a) \& 0x3  $<<$ 6 */}
    \end{tabular}

    The AJIT bits (insn[6:5]) is  set or reset internally by \texttt{F5}
    (just  like  in  \texttt{F4}),  and   hence  there  are  only  three
    arguments.

  \item \textbf{SRAD}:\\
    \begin{center}
      \begin{tabular}[p]{|c|c|l|l|p{.35\textwidth}|}
        \hline
        \textbf{Start} & \textbf{End} & \textbf{Range} & \textbf{Meaning} &
                                                                            \textbf{New Meaning}\\
        \hline
        0 & 4 & 32 & Source register 2, rs2 & Register number \\
        \hline
        5 & 12 & -- & \textbf{unused} &
                                        \begin{minipage}[h]{1.0\linewidth}
                                          \begin{itemize}
                                          \item \textbf{Set bit 5 to 0.}
                                          \item \textbf{Use bit 6 to
                                              distinguish AJIT from
                                              SPARC V8.}
                                          \end{itemize}
                                        \end{minipage}
        \\
        \hline
        13 & 13 & 0,1 & The \textbf{i} bit & \textbf{Set i to ``0''} \\
        14 & 18 & 32 & Source register 1, rs1 & No change \\
        19 & 24 & 100101 & ``\textbf{op3}'' & No change \\
        25 & 29 & 32 & Destination register, rd & No change \\
        30 & 31 & 4 & Always ``10'' & No change \\
        \hline
      \end{tabular}
    \end{center}
    \begin{itemize}
    \item []\textbf{SRAD}: same as SRA, but with Instr[13]=0 (i=0),
      and Instr[5]=1.
    \item []\textbf{Syntax}: ``\texttt{slld SrcReg1, SrcReg2,
        DestReg}''.
    \item []\textbf{Semantics}: rd(pair) $\leftarrow$ rs1(pair) $>>$
      shift count register rs2 (with sign extension).
    \end{itemize}
    Bits layout:
\begin{verbatim}
    Offsets      : 31       24 23       16  15        8   7        0
    Bit layout   :  XXXX  XXXX  XXXX  XXXX   XXXX  XXXX   XXXX  XXXX
    Insn Bits    :  10       1  0011  1        0           10        
    Destination  :    DD  DDD                                       
    Source 1     :                     SSS   SS
    Source 2     :                                           S  SSSS
    Unused (0)   :                              U  UUUU   UU        
    Final layout :  10DD  DDD1  0011  1SSS   SS0U  UUUU   U10I  IIII
\end{verbatim}

    This will need another macro that sets bits 5 and 6. Let's call it
    \texttt{OP\_AJIT\_BITS\_5\_AND\_6}.   Hence the  SPARC bit  layout of  this
    instruction is:

    \begin{tabular}[h]{lclcl}
      Macro to set  &=& \texttt{F5(x, y, z)} &in& \texttt{sparc.h}     \\
      Macro to reset  &=& \texttt{INVF5(x, y, z)} &in& \texttt{sparc.h}     \\
      x &=& 0x2      &in& \texttt{OP(x)  /* ((x) \& 0x3)  $<<$ 30 */} \\
      y &=& 0x27     &in& \texttt{OP3(y) /* ((y) \& 0x3f) $<<$ 19 */} \\
      z &=& 0x0      &in& \texttt{F3I(z) /* ((z) \& 0x1)  $<<$ 13 */} \\
      a &=& 0x2      &in& \texttt{OP\_AJIT\_BITS\_5\_AND\_6(a) /* ((a) \& 0x3  $<<$ 6 */}
    \end{tabular}

    The AJIT bits (insn[6:5]) is  set or reset internally by \texttt{F5}
    (just  like  in  \texttt{F4}),  and   hence  there  are  only  three
    arguments.
  \end{enumerate}
\end{enumerate}

\subsubsection{Multiplication and division instructions:}
\label{sec:mul:div:insn:impl}
\begin{enumerate}
\item \textbf{UMULD}: Unsigned Integer Multiply AJIT, no immediate
  version (i.e. i is always 0).\\
	\textbf{NOTE:} The \emph{suggested} mnemonic ``umuld'' conflicts with a mnemonic of the same name for another sparc architecture (other than v8).   Hence we change it to: ``\textbf{umuldaj}'' in the implementation, but not in the documentation below.

 This conflict occurs despite forcing the GNU assembler to assemble for v8 only via the command line switch ``-Av8''! It appears that forcing the assembler to use v8 is not universally applied throughout the assembler code. 
  \begin{center}
    \begin{tabular}[p]{|c|c|l|l|l|}
      \hline
      \textbf{Start} & \textbf{End} & \textbf{Range} & \textbf{Meaning} &
                                                                          \textbf{New Meaning}\\
      \hline
      0 & 4 & 32 & Source register 2, rs2 & No change \\
      5 & 12 & -- & \textbf{unused} & \textbf{Set bit 5 to ``1''} \\
      13 & 13 & 0,1 & The \textbf{i} bit & \textbf{Set i to ``0''} \\
      14 & 18 & 32 & Source register 1, rs1 & No change \\
      19 & 24 & 001010 & ``\textbf{op3}'' & No change \\
      25 & 29 & 32 & Destination register, rd & No change \\
      30 & 31 & 4 & Always ``10'' & No change \\
      \hline
    \end{tabular}
  \end{center}
  \begin{itemize}
  \item []\textbf{UMULD}: same as UMUL, but with Instr[13]=0 (i=0), and
    Instr[5]=1.
  \item []\textbf{Syntax}: ``\texttt{umuld  SrcReg1, SrcReg2, DestReg}''.
  \item []\textbf{Semantics}: rd(pair) $\leftarrow$ rs1(pair) * rs2(pair).
  \end{itemize}
  Bits layout:
\begin{verbatim}
    Offsets      : 31       24 23       16  15        8   7        0
    Bit layout   :  XXXX  XXXX  XXXX  XXXX   XXXX  XXXX   XXXX  XXXX
    Insn Bits    :  10       0  0101  0        0            1       
    Destination  :    DD  DDD                                       
    Source 1     :                     SSS   SS
    Source 2     :                                           S  SSSS
    Unused (0)   :                              U  UUUU   UU        
    Final layout :  10DD  DDD0  0101  0SSS   SS0U  UUUU   UU1S  SSSS
\end{verbatim}

  Hence the SPARC bit layout of this instruction is:

  \begin{tabular}[h]{lclcl}
    Macro to set  &=& \texttt{F4(x, y, z)} &in& \texttt{sparc.h}     \\
    Macro to reset  &=& \texttt{INVF4(x, y, z)} &in& \texttt{sparc.h}     \\
    x &=& 0x2      &in& \texttt{OP(x)  /* ((x) \& 0x3)  $<<$ 30 */} \\
    y &=& 0x0A     &in& \texttt{OP3(y) /* ((y) \& 0x3f) $<<$ 19 */} \\
    z &=& 0x0      &in& \texttt{F3I(z) /* ((z) \& 0x1)  $<<$ 13 */} \\
    a &=& 0x1      &in& \texttt{OP\_AJIT\_BIT(a) /* ((a) \& 0x1)  $<<$ 5 */}
  \end{tabular}

  The AJIT bit  (insn[5]) is set internally by  \texttt{F4}, and hence
  there are only three arguments.

\item \textbf{UMULDCC}:\\
  \begin{center}
    \begin{tabular}[p]{|c|c|l|l|l|}
      \hline
      \textbf{Start} & \textbf{End} & \textbf{Range} & \textbf{Meaning} &
                                                                          \textbf{New Meaning}\\
      \hline
      0 & 4 & 32 & Source register 2, rs2 & No change \\
      5 & 12 & -- & \textbf{unused} & \textbf{Set bit 5 to ``1''} \\
      13 & 13 & 0,1 & The \textbf{i} bit & \textbf{Set i to ``0''} \\
      14 & 18 & 32 & Source register 1, rs1 & No change \\
      19 & 24 & 011010 & ``\textbf{op3}'' & No change \\
      25 & 29 & 32 & Destination register, rd & No change \\
      30 & 31 & 4 & Always ``10'' & No change \\
      \hline
    \end{tabular}
  \end{center}
  New addition:
  \begin{itemize}
  \item []\textbf{UMULDCC}: same as UMULCC, but with Instr[13]=0 (i=0), and
    Instr[5]=1.
  \item []\textbf{Syntax}: ``\texttt{umuldcc  SrcReg1, SrcReg2, DestReg}''.
  \item []\textbf{Semantics}: rd(pair) $\leftarrow$ rs1(pair) * rs2(pair), set Z
  \end{itemize}
  Bits layout:
\begin{verbatim}
    Offsets      : 31       24 23       16  15        8   7        0
    Bit layout   :  XXXX  XXXX  XXXX  XXXX   XXXX  XXXX   XXXX  XXXX
    Insn Bits    :  10       0  1101  0        0            1       
    Destination  :    DD  DDD                                       
    Source 1     :                     SSS   SS
    Source 2     :                                           S  SSSS
    Unused (0)   :                              U  UUUU   UU        
    Final layout :  10DD  DDD0  1101  0SSS   SS0U  UUUU   UU1S  SSSS
\end{verbatim}

  Hence the SPARC bit layout of this instruction is:

  \begin{tabular}[h]{lclcl}
    Macro to set  &=& \texttt{F4(x, y, z)} &in& \texttt{sparc.h}     \\
    Macro to reset  &=& \texttt{INVF4(x, y, z)} &in& \texttt{sparc.h}     \\
    x &=& 0x2      &in& \texttt{OP(x)  /* ((x) \& 0x3)  $<<$ 30 */} \\
    y &=& 0x1A     &in& \texttt{OP3(y) /* ((y) \& 0x3f) $<<$ 19 */} \\
    z &=& 0x0      &in& \texttt{F3I(z) /* ((z) \& 0x1)  $<<$ 13 */} \\
    a &=& 0x1      &in& \texttt{OP\_AJIT\_BIT(a) /* ((a) \& 0x1)  $<<$ 5 */}
  \end{tabular}

  The AJIT bit  (insn[5]) is set internally by  \texttt{F4}, and hence
  there are only three arguments.

\item \textbf{SMULD}: Unsigned Integer Multiply AJIT, no immediate
  version (i.e. i is always 0).\\
  \begin{center}
    \begin{tabular}[p]{|c|c|l|l|l|}
      \hline
      \textbf{Start} & \textbf{End} & \textbf{Range} & \textbf{Meaning} &
                                                                          \textbf{New Meaning}\\
      \hline
      0 & 4 & 32 & Source register 2, rs2 & No change \\
      5 & 12 & -- & \textbf{unused} & \textbf{Set bit 5 to ``1''} \\
      13 & 13 & 0,1 & The \textbf{i} bit & \textbf{Set i to ``0''} \\
      14 & 18 & 32 & Source register 1, rs1 & No change \\
      19 & 24 & 001011 & ``\textbf{op3}'' & No change \\
      25 & 29 & 32 & Destination register, rd & No change \\
      30 & 31 & 4 & Always ``10'' & No change \\
      \hline
    \end{tabular}
  \end{center}
  \begin{itemize}
  \item []\textbf{SMULD}: same as SMUL, but with Instr[13]=0 (i=0), and
    Instr[5]=1.
  \item []\textbf{Syntax}: ``\texttt{smuld  SrcReg1, SrcReg2, DestReg}''.
  \item []\textbf{Semantics}: rd(pair) $\leftarrow$ rs1(pair) *
    rs2(pair) (signed).
  \end{itemize}
  Bits layout:
\begin{verbatim}
    Offsets      : 31       24 23       16  15        8   7        0
    Bit layout   :  XXXX  XXXX  XXXX  XXXX   XXXX  XXXX   XXXX  XXXX
    Insn Bits    :  10       0  0101  1        0            1       
    Destination  :    DD  DDD                                       
    Source 1     :                     SSS   SS
    Source 2     :                                           S  SSSS
    Unused (0)   :                              U  UUUU   UU        
    Final layout :  10DD  DDD0  0101  1SSS   SS0U  UUUU   UU1S  SSSS
\end{verbatim}

  Hence the SPARC bit layout of this instruction is:

  \begin{tabular}[h]{lclcl}
    Macro to set  &=& \texttt{F4(x, y, z)} &in& \texttt{sparc.h}     \\
    Macro to reset  &=& \texttt{INVF4(x, y, z)} &in& \texttt{sparc.h}     \\
    x &=& 0x2      &in& \texttt{OP(x)  /* ((x) \& 0x3)  $<<$ 30 */} \\
    y &=& 0x0B     &in& \texttt{OP3(y) /* ((y) \& 0x3f) $<<$ 19 */} \\
    z &=& 0x0      &in& \texttt{F3I(z) /* ((z) \& 0x1)  $<<$ 13 */} \\
    a &=& 0x1      &in& \texttt{OP\_AJIT\_BIT(a) /* ((a) \& 0x1)  $<<$ 5 */}
  \end{tabular}

  The AJIT bit  (insn[5]) is set internally by  \texttt{F4}, and hence
  there are only three arguments.

\item \textbf{SMULDCC}:\\
  \begin{center}
    \begin{tabular}[p]{|c|c|l|l|l|}
      \hline
      \textbf{Start} & \textbf{End} & \textbf{Range} & \textbf{Meaning} &
                                                                          \textbf{New Meaning}\\
      \hline
      0 & 4 & 32 & Source register 2, rs2 & No change \\
      5 & 12 & -- & \textbf{unused} & \textbf{Set bit 5 to ``1''} \\
      13 & 13 & 0,1 & The \textbf{i} bit & \textbf{Set i to ``0''} \\
      14 & 18 & 32 & Source register 1, rs1 & No change \\
      19 & 24 & 011011 & ``\textbf{op3}'' & No change \\
      25 & 29 & 32 & Destination register, rd & No change \\
      30 & 31 & 4 & Always ``10'' & No change \\
      \hline
    \end{tabular}
  \end{center}
  New addition:
  \begin{itemize}
  \item []\textbf{SMULDCC}: same as SMULCC, but with Instr[13]=0 (i=0), and
    Instr[5]=1.
  \item []\textbf{Syntax}: ``\texttt{smuldcc  SrcReg1, SrcReg2, DestReg}''.
  \item []\textbf{Semantics}: rd(pair) $\leftarrow$ rs1(pair) *
    rs2(pair) (signed), set Z,N,O
  \end{itemize}
  Bits layout:
\begin{verbatim}
    Offsets      : 31       24 23       16  15        8   7        0
    Bit layout   :  XXXX  XXXX  XXXX  XXXX   XXXX  XXXX   XXXX  XXXX
    Insn Bits    :  10       0  1101  1        0            1       
    Destination  :    DD  DDD                                       
    Source 1     :                     SSS   SS
    Source 2     :                                           S  SSSS
    Unused (0)   :                              U  UUUU   UU        
    Final layout :  10DD  DDD0  1101  1SSS   SS0U  UUUU   UU1S  SSSS
\end{verbatim}

  Hence the SPARC bit layout of this instruction is:

  \begin{tabular}[h]{lclcl}
    Macro to set  &=& \texttt{F4(x, y, z)} &in& \texttt{sparc.h}     \\
    Macro to reset  &=& \texttt{INVF4(x, y, z)} &in& \texttt{sparc.h}     \\
    x &=& 0x2      &in& \texttt{OP(x)  /* ((x) \& 0x3)  $<<$ 30 */} \\
    y &=& 0x1B     &in& \texttt{OP3(y) /* ((y) \& 0x3f) $<<$ 19 */} \\
    z &=& 0x0      &in& \texttt{F3I(z) /* ((z) \& 0x1)  $<<$ 13 */} \\
    a &=& 0x1      &in& \texttt{OP\_AJIT\_BIT(a) /* ((a) \& 0x1)  $<<$ 5 */}
  \end{tabular}

  The AJIT bit  (insn[5]) is set internally by  \texttt{F4}, and hence
  there are only three arguments.

\item \textbf{UDIVD}:\\
  \begin{center}
    \begin{tabular}[p]{|c|c|l|l|l|}
      \hline
      \textbf{Start} & \textbf{End} & \textbf{Range} & \textbf{Meaning} &
                                                                          \textbf{New Meaning}\\
      \hline
      0 & 4 & 32 & Source register 2, rs2 & No change \\
      5 & 12 & -- & \textbf{unused} & \textbf{Set bit 5 to ``1''} \\
      13 & 13 & 0,1 & The \textbf{i} bit & \textbf{Set i to ``0''} \\
      14 & 18 & 32 & Source register 1, rs1 & No change \\
      19 & 24 & 001110 & ``\textbf{op3}'' & No change \\
      25 & 29 & 32 & Destination register, rd & No change \\
      30 & 31 & 4 & Always ``10'' & No change \\
      \hline
    \end{tabular}
  \end{center}
  New addition:
  \begin{itemize}
  \item []\textbf{UDIVD}: same as UDIV, but with Instr[13]=0 (i=0), and
    Instr[5]=1.
  \item []\textbf{Syntax}: ``\texttt{udivd  SrcReg1, SrcReg2, DestReg}''.
  \item []\textbf{Semantics}: rd(pair) $\leftarrow$ rs1(pair) / rs2(pair).
  \end{itemize}
  Bits layout:
\begin{verbatim}
    Offsets      : 31       24 23       16  15        8   7        0
    Bit layout   :  XXXX  XXXX  XXXX  XXXX   XXXX  XXXX   XXXX  XXXX
    Insn Bits    :  10       0  0111  0        0            1       
    Destination  :    DD  DDD                                       
    Source 1     :                     SSS   SS
    Source 2     :                                           S  SSSS
    Unused (0)   :                              U  UUUU   UU        
    Final layout :  10DD  DDD0  0111  0SSS   SS0U  UUUU   UU1S  SSSS
\end{verbatim}

  Hence the SPARC bit layout of this instruction is:

  \begin{tabular}[h]{lclcl}
    Macro to set  &=& \texttt{F4(x, y, z)} &in& \texttt{sparc.h}     \\
    Macro to reset  &=& \texttt{INVF4(x, y, z)} &in& \texttt{sparc.h}     \\
    x &=& 0x2      &in& \texttt{OP(x)  /* ((x) \& 0x3)  $<<$ 30 */} \\
    y &=& 0x0E     &in& \texttt{OP3(y) /* ((y) \& 0x3f) $<<$ 19 */} \\
    z &=& 0x0      &in& \texttt{F3I(z) /* ((z) \& 0x1)  $<<$ 13 */} \\
    a &=& 0x1      &in& \texttt{OP\_AJIT\_BIT(a) /* ((a) \& 0x1)  $<<$ 5 */}
  \end{tabular}

  The AJIT bit  (insn[5]) is set internally by  \texttt{F4}, and hence
  there are only three arguments.

\item \textbf{UDIVDCC}:\\
  \begin{center}
    \begin{tabular}[p]{|c|c|l|l|l|}
      \hline
      \textbf{Start} & \textbf{End} & \textbf{Range} & \textbf{Meaning} &
                                                                          \textbf{New Meaning}\\
      \hline
      0 & 4 & 32 & Source register 2, rs2 & No change \\
      5 & 12 & -- & \textbf{unused} & \textbf{Set bit 5 to ``1''} \\
      13 & 13 & 0,1 & The \textbf{i} bit & \textbf{Set i to ``0''} \\
      14 & 18 & 32 & Source register 1, rs1 & No change \\
      19 & 24 & 011110 & ``\textbf{op3}'' & No change \\
      25 & 29 & 32 & Destination register, rd & No change \\
      30 & 31 & 4 & Always ``10'' & No change \\
      \hline
    \end{tabular}
  \end{center}
  New addition:
  \begin{itemize}
  \item []\textbf{UDIVDCC}: same as UDIVCC, but with Instr[13]=0 (i=0), and
    Instr[5]=1.
  \item []\textbf{Syntax}: ``\texttt{udivdcc  SrcReg1, SrcReg2, DestReg}''.
  \item []\textbf{Semantics}: rd(pair) $\leftarrow$ rs1(pair) / rs2(pair), set Z,O
  \end{itemize}
  Bits layout:
\begin{verbatim}
    Offsets      : 31       24 23       16  15        8   7        0
    Bit layout   :  XXXX  XXXX  XXXX  XXXX   XXXX  XXXX   XXXX  XXXX
    Insn Bits    :  10       0  1111  0        0            1       
    Destination  :    DD  DDD                                       
    Source 1     :                     SSS   SS
    Source 2     :                                           S  SSSS
    Unused (0)   :                              U  UUUU   UU        
    Final layout :  10DD  DDD0  1111  0SSS   SS0U  UUUU   UU1S  SSSS
\end{verbatim}

  Hence the SPARC bit layout of this instruction is:

  \begin{tabular}[h]{lclcl}
    Macro to set  &=& \texttt{F4(x, y, z)} &in& \texttt{sparc.h}     \\
    Macro to reset  &=& \texttt{INVF4(x, y, z)} &in& \texttt{sparc.h}     \\
    x &=& 0x2      &in& \texttt{OP(x)  /* ((x) \& 0x3)  $<<$ 30 */} \\
    y &=& 0x1E     &in& \texttt{OP3(y) /* ((y) \& 0x3f) $<<$ 19 */} \\
    z &=& 0x0      &in& \texttt{F3I(z) /* ((z) \& 0x1)  $<<$ 13 */} \\
    a &=& 0x1      &in& \texttt{OP\_AJIT\_BIT(a) /* ((a) \& 0x1)  $<<$ 5 */}
  \end{tabular}

  The AJIT bit  (insn[5]) is set internally by  \texttt{F4}, and hence
  there are only three arguments.

\item \textbf{SDIVD}:\\
  \begin{center}
    \begin{tabular}[p]{|c|c|l|l|l|}
      \hline
      \textbf{Start} & \textbf{End} & \textbf{Range} & \textbf{Meaning} &
                                                                          \textbf{New Meaning}\\
      \hline
      0 & 4 & 32 & Source register 2, rs2 & No change \\
      5 & 12 & -- & \textbf{unused} & \textbf{Set bit 5 to ``1''} \\
      13 & 13 & 0,1 & The \textbf{i} bit & \textbf{Set i to ``0''} \\
      14 & 18 & 32 & Source register 1, rs1 & No change \\
      19 & 24 & 001111 & ``\textbf{op3}'' & No change \\
      25 & 29 & 32 & Destination register, rd & No change \\
      30 & 31 & 4 & Always ``10'' & No change \\
      \hline
    \end{tabular}
  \end{center}
  New addition:
  \begin{itemize}
  \item []\textbf{SDIVD}: same as SDIV, but with Instr[13]=0 (i=0), and
    Instr[5]=1.
  \item []\textbf{Syntax}: ``\texttt{sdivd  SrcReg1, SrcReg2, DestReg}''.
  \item []\textbf{Semantics}: rd(pair) $\leftarrow$ rs1(pair) /
    rs2(pair) (signed).
  \end{itemize}
  Bits layout:
\begin{verbatim}
    Offsets      : 31       24 23       16  15        8   7        0
    Bit layout   :  XXXX  XXXX  XXXX  XXXX   XXXX  XXXX   XXXX  XXXX
    Insn Bits    :  10       0  0111  1        0            1       
    Destination  :    DD  DDD                                       
    Source 1     :                     SSS   SS
    Source 2     :                                           S  SSSS
    Unused (0)   :                              U  UUUU   UU        
    Final layout :  10DD  DDD0  0111  1SSS   SS0U  UUUU   UU1S  SSSS
\end{verbatim}

  Hence the SPARC bit layout of this instruction is:

  \begin{tabular}[h]{lclcl}
    Macro to set  &=& \texttt{F4(x, y, z)} &in& \texttt{sparc.h}     \\
    Macro to reset  &=& \texttt{INVF4(x, y, z)} &in& \texttt{sparc.h}     \\
    x &=& 0x2      &in& \texttt{OP(x)  /* ((x) \& 0x3)  $<<$ 30 */} \\
    y &=& 0x0F     &in& \texttt{OP3(y) /* ((y) \& 0x3f) $<<$ 19 */} \\
    z &=& 0x0      &in& \texttt{F3I(z) /* ((z) \& 0x1)  $<<$ 13 */} \\
    a &=& 0x1      &in& \texttt{OP\_AJIT\_BIT(a) /* ((a) \& 0x1)  $<<$ 5 */}
  \end{tabular}

  The AJIT bit  (insn[5]) is set internally by  \texttt{F4}, and hence
  there are only three arguments.

\item \textbf{SDIVDCC}:\\
  \begin{center}
    \begin{tabular}[p]{|c|c|l|l|l|}
      \hline
      \textbf{Start} & \textbf{End} & \textbf{Range} & \textbf{Meaning} &
                                                                          \textbf{New Meaning}\\
      \hline
      0 & 4 & 32 & Source register 2, rs2 & No change \\
      5 & 12 & -- & \textbf{unused} & \textbf{Set bit 5 to ``1''} \\
      13 & 13 & 0,1 & The \textbf{i} bit & \textbf{Set i to ``0''} \\
      14 & 18 & 32 & Source register 1, rs1 & No change \\
      19 & 24 & 011111 & ``\textbf{op3}'' & No change \\
      25 & 29 & 32 & Destination register, rd & No change \\
      30 & 31 & 4 & Always ``10'' & No change \\
      \hline
    \end{tabular}
  \end{center}
  New addition:
  \begin{itemize}
  \item []\textbf{SDIVDCC}: same as SDIVCC, but with Instr[13]=0 (i=0), and
    Instr[5]=1.
  \item []\textbf{Syntax}: ``\texttt{sdivdcc  SrcReg1, SrcReg2, DestReg}''.
  \item []\textbf{Semantics}: rd(pair) $\leftarrow$ rs1(pair) /
    rs2(pair) (signed), set Z,N,O
  \end{itemize}
  Bits layout:
\begin{verbatim}
    Offsets      : 31       24 23       16  15        8   7        0
    Bit layout   :  XXXX  XXXX  XXXX  XXXX   XXXX  XXXX   XXXX  XXXX
    Insn Bits    :  10       0  1111  1        0            1       
    Destination  :    DD  DDD                                       
    Source 1     :                     SSS   SS
    Source 2     :                                           S  SSSS
    Unused (0)   :                              U  UUUU   UU        
    Final layout :  10DD  DDD0  1111  1SSS   SS0U  UUUU   UU1S  SSSS
\end{verbatim}

  Hence the SPARC bit layout of this instruction is:

  \begin{tabular}[h]{lclcl}
    Macro to set  &=& \texttt{F4(x, y, z)} &in& \texttt{sparc.h}     \\
    Macro to reset  &=& \texttt{INVF4(x, y, z)} &in& \texttt{sparc.h}     \\
    x &=& 0x2      &in& \texttt{OP(x)  /* ((x) \& 0x3)  $<<$ 30 */} \\
    y &=& 0x1F     &in& \texttt{OP3(y) /* ((y) \& 0x3f) $<<$ 19 */} \\
    z &=& 0x0      &in& \texttt{F3I(z) /* ((z) \& 0x1)  $<<$ 13 */} \\
    a &=& 0x1      &in& \texttt{OP\_AJIT\_BIT(a) /* ((a) \& 0x1)  $<<$ 5 */}
  \end{tabular}

  The AJIT bit  (insn[5]) is set internally by  \texttt{F4}, and hence
  there are only three arguments.
\end{enumerate}

%%% Local Variables:
%%% mode: latex
%%% TeX-master: t
%%% End:

\subsubsection{64 Bit Logical Instructions:}
\label{sec:64:bit:logical:insn:impl}

No immediate mode, i.e. insn[5] $\equiv$ i = 0, always.

\begin{enumerate}
\item \textbf{ORD}:\\
  \begin{center}
    \begin{figure}[h]
      \centering
      \epsfxsize=.8\linewidth
      \epsffile{../figs/ord-ajit-insn-32-bit-layout.eps}
      \caption{The AJIT ORD instruction  with register operands.}
      \label{fig:ajit:ord:insn}
    \end{figure}
  \end{center}
  \begin{itemize}
  \item []\textbf{ORD}: same as OR, but with Instr[13]=0 (i=0), and
    Instr[5]=1.
  \item []\textbf{Syntax}: ``\texttt{ord  SrcReg1, SrcReg2, DestReg}''.
  \item []\textbf{Semantics}: rd(pair) $\leftarrow$ rs1(pair) $\vert$ rs2(pair).
  \end{itemize}

  Hence the SPARC bit layout of this instruction is:

  \begin{tabular}[h]{lclcl}
    Macro to set   &=&  \verb|F4(x, y, z, b)|     &in& \verb|sparc.h|     \\
    Macro to reset &=&  \verb|F4(~x, ~y, ~z, ~b)| &in& \verb|sparc.h|     \\
    x              &=& 0x2                        &in& \verb|OP(x) | \\
    y              &=& 0x02                       &in& \verb|OP3(y) | \\
    z              &=& 0x0                        &in& \verb|F3I(z) | \\
    b              &=& 0x1                        &in& \verb|OP_AJIT_BIT_5(a) |
  \end{tabular}

\item \textbf{ORDCC}:\\
  \begin{center}
    \begin{figure}[h]
      \centering
      \epsfxsize=.8\linewidth
      \epsffile{../figs/ordcc-ajit-insn-32-bit-layout.eps}
      \caption{The AJIT ORDCC instruction  with register operands.}
      \label{fig:ajit:ordcc:insn}
    \end{figure}
  \end{center}
  \begin{itemize}
  \item []\textbf{ORDCC}: same as ORCC, but with Instr[13]=0 (i=0), and
    Instr[5]=1.
  \item []\textbf{Syntax}: ``\texttt{ordcc  SrcReg1, SrcReg2, DestReg}''.
  \item []\textbf{Semantics}: rd(pair) $\leftarrow$ rs1(pair) $\vert$
    rs2(pair), sets Z.
  \end{itemize}

  Hence the SPARC bit layout of this instruction is:

  \begin{tabular}[h]{lclcl}
    Macro to set   &=&  \verb|F4(x, y, z, b)|     &in& \verb|sparc.h|     \\
    Macro to reset &=&  \verb|F4(~x, ~y, ~z, ~b)| &in& \verb|sparc.h|     \\
    x              &=& 0x2                        &in& \verb|OP(x) | \\
    y              &=& 0x12                       &in& \verb|OP3(y) | \\
    z              &=& 0x0                        &in& \verb|F3I(z) | \\
    b              &=& 0x1                        &in& \verb|OP_AJIT_BIT_5(a) |
  \end{tabular}

\item \textbf{ORDN}:\\
  \begin{center}
    \begin{figure}[h]
      \centering
      \epsfxsize=.8\linewidth
      \epsffile{../figs/ordn-ajit-insn-32-bit-layout.eps}
      \caption{The AJIT ORDN instruction  with register operands.}
      \label{fig:ajit:ordn:insn}
    \end{figure}
  \end{center}
  \begin{itemize}
  \item []\textbf{ORDN}: same as ORN, but with Instr[13]=0 (i=0), and
    Instr[5]=1.
  \item []\textbf{Syntax}: ``\texttt{ordn  SrcReg1, SrcReg2, DestReg}''.
  \item []\textbf{Semantics}: rd(pair) $\leftarrow$ rs1(pair) $\vert$ ($\sim$rs2(pair)).
  \end{itemize}

  Hence the SPARC bit layout of this instruction is:

  \begin{tabular}[h]{lclcl}
    Macro to set   &=&  \verb|F4(x, y, z, b)|     &in& \verb|sparc.h|     \\
    Macro to reset &=&  \verb|F4(~x, ~y, ~z, ~b)| &in& \verb|sparc.h|     \\
    x              &=& 0x2                        &in& \verb|OP(x) | \\
    y              &=& 0x06                       &in& \verb|OP3(y) | \\
    z              &=& 0x0                        &in& \verb|F3I(z) | \\
    b              &=& 0x1                        &in& \verb|OP_AJIT_BIT_5(a) |
  \end{tabular}

\item \textbf{ORDNCC}:\\
  \begin{center}
    \begin{figure}[h]
      \centering
      \epsfxsize=.8\linewidth
      \epsffile{../figs/ordncc-ajit-insn-32-bit-layout.eps}
      \caption{The AJIT ORDNCC instruction  with register operands.}
      \label{fig:ajit:ordncc:insn}
    \end{figure}
  \end{center}
  \begin{itemize}
  \item []\textbf{ORDNCC}: same as ORN, but with Instr[13]=0 (i=0), and
    Instr[5]=1.
  \item []\textbf{Syntax}: ``\texttt{ordncc  SrcReg1, SrcReg2, DestReg}''.
  \item []\textbf{Semantics}: rd(pair) $\leftarrow$ rs1(pair) $\vert$
    ($\sim$rs2(pair)), sets Z.
  \end{itemize}

  Hence the SPARC bit layout of this instruction is:

  \begin{tabular}[h]{lclcl}
    Macro to set   &=&  \verb|F4(x, y, z, b)|     &in& \verb|sparc.h|     \\
    Macro to reset &=&  \verb|F4(~x, ~y, ~z, ~b)| &in& \verb|sparc.h|     \\
    x              &=& 0x2                        &in& \verb|OP(x) | \\
    y              &=& 0x16                       &in& \verb|OP3(y) | \\
    z              &=& 0x0                        &in& \verb|F3I(z) | \\
    b              &=& 0x1                        &in& \verb|OP_AJIT_BIT_5(a) |
  \end{tabular}

\item \textbf{XORDCC}:\\
  \begin{center}
    \begin{figure}[h]
      \centering
      \epsfxsize=.8\linewidth
      \epsffile{../figs/xordcc-ajit-insn-32-bit-layout.eps}
      \caption{The AJIT XORDCC instruction  with register operands.}
      \label{fig:ajit:xordcc:insn}
    \end{figure}
  \end{center}
  \begin{itemize}
  \item []\textbf{XORDCC}: same as XORCC, but with Instr[13]=0 (i=0), and
    Instr[5]=1.
  \item []\textbf{Syntax}: ``\texttt{xordcc  SrcReg1, SrcReg2, DestReg}''.
  \item []\textbf{Semantics}: rd(pair) $\leftarrow$ rs1(pair) $\hat{~}$
    rs2(pair), sets Z.
  \end{itemize}

  Hence the SPARC bit layout of this instruction is:

  \begin{tabular}[h]{lclcl}
    Macro to set   &=&  \verb|F4(x, y, z, b)|     &in& \verb|sparc.h|     \\
    Macro to reset &=&  \verb|F4(~x, ~y, ~z, ~b)| &in& \verb|sparc.h|     \\
    x              &=& 0x2                        &in& \verb|OP(x) | \\
    y              &=& 0x13                       &in& \verb|OP3(y) | \\
    z              &=& 0x0                        &in& \verb|F3I(z) | \\
    b              &=& 0x1                        &in& \verb|OP_AJIT_BIT_5(a) |
  \end{tabular}

  The AJIT bit  (insn[5]) is set internally by  \texttt{F4}, and hence
  there are only three arguments.

\item \textbf{XNORD}:\\
  \begin{center}
    \begin{figure}[h]
      \centering
      \epsfxsize=.8\linewidth
      \epsffile{../figs/xnord-ajit-insn-32-bit-layout.eps}
      \caption{The AJIT XNORD instruction  with register operands.}
      \label{fig:ajit:xnord:insn}
    \end{figure}
  \end{center}
  \begin{itemize}
  \item []\textbf{XNORD}: same as XNOR, but with Instr[13]=0 (i=0), and
    Instr[5]=1.
  \item []\textbf{Syntax}: ``\texttt{xnordcc  SrcReg1, SrcReg2, DestReg}''.
  \item []\textbf{Semantics}: rd(pair) $\leftarrow$ rs1(pair) $\hat{~}$
    rs2(pair).
  \end{itemize}

  Hence the SPARC bit layout of this instruction is:

  \begin{tabular}[h]{lclcl}
    Macro to set   &=&  \verb|F4(x, y, z, b)|     &in& \verb|sparc.h|     \\
    Macro to reset &=&  \verb|F4(~x, ~y, ~z, ~b)| &in& \verb|sparc.h|     \\
    x              &=& 0x2                        &in& \verb|OP(x) | \\
    y              &=& 0x07                       &in& \verb|OP3(y) | \\
    z              &=& 0x0                        &in& \verb|F3I(z) | \\
    b              &=& 0x1                        &in& \verb|OP_AJIT_BIT_5(a) |
  \end{tabular}

\item \textbf{XNORDCC}:\\
  \begin{center}
    \begin{figure}[h]
      \centering
      \epsfxsize=.8\linewidth
      \epsffile{../figs/xnordcc-ajit-insn-32-bit-layout.eps}
      \caption{The AJIT XNORDCC instruction  with register operands.}
      \label{fig:ajit:xnordcc:insn}
    \end{figure}
  \end{center}
  \begin{itemize}
  \item []\textbf{XNORDCC}: same as XNORD, but with Instr[13]=0 (i=0), and
    Instr[5]=1.
  \item []\textbf{Syntax}: ``\texttt{xnordcc  SrcReg1, SrcReg2, DestReg}''.
  \item []\textbf{Semantics}: rd(pair) $\leftarrow$ rs1(pair) $\hat{~}$
    rs2(pair), sets Z.
  \end{itemize}

  Hence the SPARC bit layout of this instruction is:

  \begin{tabular}[h]{lclcl}
    Macro to set   &=&  \verb|F4(x, y, z, b)|     &in& \verb|sparc.h|     \\
    Macro to reset &=&  \verb|F4(~x, ~y, ~z, ~b)| &in& \verb|sparc.h|     \\
    x              &=& 0x2                        &in& \verb|OP(x) | \\
    y              &=& 0x17                       &in& \verb|OP3(y) | \\
    z              &=& 0x0                        &in& \verb|F3I(z) | \\
    b              &=& 0x1                        &in& \verb|OP_AJIT_BIT_5(a) |
  \end{tabular}

\item \textbf{ANDD}:\\
  \begin{center}
    \begin{figure}[h]
      \centering
      \epsfxsize=.8\linewidth
      \epsffile{../figs/andd-ajit-insn-32-bit-layout.eps}
      \caption{The AJIT ANDD instruction  with register operands.}
      \label{fig:ajit:andd:insn}
    \end{figure}
  \end{center}
  \begin{itemize}
  \item []\textbf{ANDD}: same as AND, but with Instr[13]=0 (i=0), and
    Instr[5]=1.
  \item []\textbf{Syntax}: ``\texttt{andd  SrcReg1, SrcReg2, DestReg}''.
  \item []\textbf{Semantics}: rd(pair) $\leftarrow$ rs1(pair) $\cdot$ rs2(pair).
  \end{itemize}

  Hence the SPARC bit layout of this instruction is:

  \begin{tabular}[h]{lclcl}
    Macro to set   &=&  \verb|F4(x, y, z, b)|     &in& \verb|sparc.h|     \\
    Macro to reset &=&  \verb|F4(~x, ~y, ~z, ~b)| &in& \verb|sparc.h|     \\
    x              &=& 0x2                        &in& \verb|OP(x) | \\
    y              &=& 0x01                       &in& \verb|OP3(y) | \\
    z              &=& 0x0                        &in& \verb|F3I(z) | \\
    b              &=& 0x1                        &in& \verb|OP_AJIT_BIT_5(a) |
  \end{tabular}

\item \textbf{ANDDCC}:\\
  \begin{center}
    \begin{figure}[h]
      \centering
      \epsfxsize=.8\linewidth
      \epsffile{../figs/anddcc-ajit-insn-32-bit-layout.eps}
      \caption{The AJIT ANDDCC instruction  with register operands.}
      \label{fig:ajit:anddcc:insn}
    \end{figure}
  \end{center}
  \begin{itemize}
  \item []\textbf{ANDDCC}: same as ANDCC, but with Instr[13]=0 (i=0), and
    Instr[5]=1.
  \item []\textbf{Syntax}: ``\texttt{anddcc  SrcReg1, SrcReg2, DestReg}''.
  \item []\textbf{Semantics}: rd(pair) $\leftarrow$ rs1(pair) $\cdot$
    rs2(pair), sets Z.
  \end{itemize}

  Hence the SPARC bit layout of this instruction is:

  \begin{tabular}[h]{lclcl}
    Macro to set   &=&  \verb|F4(x, y, z, b)|     &in& \verb|sparc.h|     \\
    Macro to reset &=&  \verb|F4(~x, ~y, ~z, ~b)| &in& \verb|sparc.h|     \\
    x              &=& 0x2                        &in& \verb|OP(x) | \\
    y              &=& 0x11                       &in& \verb|OP3(y) | \\
    z              &=& 0x0                        &in& \verb|F3I(z) | \\
    b              &=& 0x1                        &in& \verb|OP_AJIT_BIT_5(a) |
  \end{tabular}

\item \textbf{ANDDN}:\\
  \begin{center}
    \begin{figure}[h]
      \centering
      \epsfxsize=.8\linewidth
      \epsffile{../figs/anddn-ajit-insn-32-bit-layout.eps}
      \caption{The AJIT ANDDN instruction  with register operands.}
      \label{fig:ajit:anddn:insn}
    \end{figure}
  \end{center}
  \begin{itemize}
  \item []\textbf{ANDDN}: same as ANDN, but with Instr[13]=0 (i=0), and
    Instr[5]=1.
  \item []\textbf{Syntax}: ``\texttt{anddn  SrcReg1, SrcReg2, DestReg}''.
  \item []\textbf{Semantics}: rd(pair) $\leftarrow$ rs1(pair) $\cdot$ ($\sim$rs2(pair)).
  \end{itemize}

  Hence the SPARC bit layout of this instruction is:

  \begin{tabular}[h]{lclcl}
    Macro to set   &=&  \verb|F4(x, y, z, b)|     &in& \verb|sparc.h|     \\
    Macro to reset &=&  \verb|F4(~x, ~y, ~z, ~b)| &in& \verb|sparc.h|     \\
    x              &=& 0x2                        &in& \verb|OP(x) | \\
    y              &=& 0x05                       &in& \verb|OP3(y) | \\
    z              &=& 0x0                        &in& \verb|F3I(z) | \\
    b              &=& 0x1                        &in& \verb|OP_AJIT_BIT_5(a) |
  \end{tabular}

\item \textbf{ANDDNCC}:\\
  \begin{center}
    \begin{figure}[h]
      \centering
      \epsfxsize=.8\linewidth
      \epsffile{../figs/anddncc-ajit-insn-32-bit-layout.eps}
      \caption{The AJIT ANDDNCC instruction  with register operands.}
      \label{fig:ajit:anddncc:insn}
    \end{figure}
  \end{center}
  \begin{itemize}
  \item []\textbf{ANDDNCC}: same as ANDN, but with Instr[13]=0 (i=0), and
    Instr[5]=1.
  \item []\textbf{Syntax}: ``\texttt{anddncc  SrcReg1, SrcReg2, DestReg}''.
  \item []\textbf{Semantics}: rd(pair) $\leftarrow$ rs1(pair) $\cdot$
    ($\sim$rs2(pair)), sets Z.
  \end{itemize}

  Hence the SPARC bit layout of this instruction is:

  \begin{tabular}[h]{lclcl}
    Macro to set   &=&  \verb|F4(x, y, z, b)|     &in& \verb|sparc.h|     \\
    Macro to reset &=&  \verb|F4(~x, ~y, ~z, ~b)| &in& \verb|sparc.h|     \\
    x              &=& 0x2                        &in& \verb|OP(x) | \\
    y              &=& 0x15                       &in& \verb|OP3(y) | \\
    z              &=& 0x0                        &in& \verb|F3I(z) | \\
    b              &=& 0x1                        &in& \verb|OP_AJIT_BIT_5(a) |
  \end{tabular}

  The AJIT bit  (insn[5]) is set internally by  \texttt{F4}, and hence
  there are only three arguments.

\end{enumerate}

%%% Local Variables:
%%% mode: latex
%%% TeX-master: t
%%% End:

\subsubsection{Integer Unit Extensions Summary}
\label{sec:int:unit:extns:summary}

\begin{itemize}
  % \subsubsection{Addition and subtraction instructions:}
\label{sec:add:sub:insn:impl}
\begin{enumerate}
\item \textbf{ADDD}:\\
  \begin{center}
    \begin{tabular}[p]{|c|c|l|l|l|}
      \hline
      \textbf{Start} & \textbf{End} & \textbf{Range} & \textbf{Meaning} &
                                                                          \textbf{New Meaning}\\
      \hline
      0 & 4 & 32 & Source register 2, rs2 & No change \\
      5 & 12 & -- & \textbf{unused} & \textbf{Set bit 5 to ``1''} \\
      13 & 13 & 0,1 & The \textbf{i} bit & \textbf{Set i to ``0''} \\
      14 & 18 & 32 & Source register 1, rs1 & No change \\
      19 & 24 & 000000 & ``\textbf{op3}'' & No change \\
      25 & 29 & 32 & Destination register, rd & No change \\
      30 & 31 & 4 & Always ``10'' & No change \\
      \hline
    \end{tabular}
  \end{center}
  \begin{itemize}
  \item []\textbf{ADDD}: same as ADD, but with Instr[13]=0 (i=0), and
    Instr[5]=1.
  \item []\textbf{Syntax}: ``\texttt{addd  SrcReg1, SrcReg2, DestReg}''.
  \item []\textbf{Semantics}: rd(pair) $\leftarrow$ rs1(pair) + rs2(pair).
  \end{itemize}
  Bits layout:
\begin{verbatim}
    Offsets      : 31       24 23       16  15        8   7        0
    Bit layout   :  XXXX  XXXX  XXXX  XXXX   XXXX  XXXX   XXXX  XXXX
    Insn Bits    :  10       0  0000  0        0            1       
    Destination  :    DD  DDD                                       
    Source 1     :                     SSS   SS
    Source 2     :                                           S  SSSS
    Unused (0)   :                              U  UUUU   UU        
    Final layout :  10DD  DDD0  0000  0SSS   SS0U  UUUU   UU1S  SSSS
\end{verbatim}

  Hence the SPARC bit layout of this instruction is:

  \begin{tabular}[h]{lclcl}
    Macro to set  &=& \texttt{F4(x, y, z)} &in& \texttt{sparc.h}     \\
    Macro to reset  &=& \texttt{INVF4(x, y, z)} &in& \texttt{sparc.h}     \\
    x &=& 0x2      &in& \texttt{OP(x)  /* ((x) \& 0x3)  $<<$ 30 */} \\
    y &=& 0x00     &in& \texttt{OP3(y) /* ((y) \& 0x3f) $<<$ 19 */} \\
    z &=& 0x0      &in& \texttt{F3I(z) /* ((z) \& 0x1)  $<<$ 13 */} \\
    a &=& 0x1      &in& \texttt{OP\_AJIT\_BIT(a) /* ((a) \& 0x1)  $<<$ 5 */}
  \end{tabular}

  The AJIT bit  (insn[5]) is set internally by  \texttt{F4}, and hence
  there are only three arguments.

\item \textbf{ADDDCC}:\\
  \begin{center}
    \begin{tabular}[p]{|c|c|l|l|l|}
      \hline
      \textbf{Start} & \textbf{End} & \textbf{Range} & \textbf{Meaning} &
                                                                          \textbf{New Meaning}\\
      \hline
      0 & 4 & 32 & Source register 2, rs2 & No change \\
      5 & 12 & -- & \textbf{unused} & \textbf{Set bit 5 to ``1''} \\
      13 & 13 & 0,1 & The \textbf{i} bit & \textbf{Set i to ``0''} \\
      14 & 18 & 32 & Source register 1, rs1 & No change \\
      19 & 24 & 010000 & ``\textbf{op3}'' & No change \\
      25 & 29 & 32 & Destination register, rd & No change \\
      30 & 31 & 4 & Always ``10'' & No change \\
      \hline
    \end{tabular}
  \end{center}
  New addition:
  \begin{itemize}
  \item []\textbf{ADDDCC}: same as ADDCC, but with Instr[13]=0 (i=0), and
    Instr[5]=1.
  \item []\textbf{Syntax}: ``\texttt{adddcc  SrcReg1, SrcReg2, DestReg}''.
  \item []\textbf{Semantics}: rd(pair) $\leftarrow$ rs1(pair) + rs2(pair), set Z,N
  \end{itemize}
  Bits layout:
\begin{verbatim}
    Offsets      : 31       24 23       16  15        8   7        0
    Bit layout   :  XXXX  XXXX  XXXX  XXXX   XXXX  XXXX   XXXX  XXXX
    Insn Bits    :  10       0  1000  0        0            1       
    Destination  :    DD  DDD                                       
    Source 1     :                     SSS   SS
    Source 2     :                                           S  SSSS
    Unused (0)   :                              U  UUUU   UU        
    Final layout :  10DD  DDD0  1000  0SSS   SS0U  UUUU   UU1S  SSSS
\end{verbatim}

  Hence the SPARC bit layout of this instruction is:

  \begin{tabular}[h]{lclcl}
    Macro to set  &=& \texttt{F4(x, y, z)} &in& \texttt{sparc.h}     \\
    Macro to reset  &=& \texttt{INVF4(x, y, z)} &in& \texttt{sparc.h}     \\
    x &=& 0x2      &in& \texttt{OP(x)  /* ((x) \& 0x3)  $<<$ 30 */} \\
    y &=& 0x10     &in& \texttt{OP3(y) /* ((y) \& 0x3f) $<<$ 19 */} \\
    z &=& 0x0      &in& \texttt{F3I(z) /* ((z) \& 0x1)  $<<$ 13 */} \\
    a &=& 0x1      &in& \texttt{OP\_AJIT\_BIT(a) /* ((a) \& 0x1)  $<<$ 5 */}
  \end{tabular}

  The AJIT bit  (insn[5]) is set internally by  \texttt{F4}, and hence
  there are only three arguments.

\item \textbf{SUBD}:\\
  \begin{center}
    \begin{tabular}[p]{|c|c|l|l|l|}
      \hline
      \textbf{Start} & \textbf{End} & \textbf{Range} & \textbf{Meaning} &
                                                                          \textbf{New Meaning}\\
      \hline
      0 & 4 & 32 & Source register 2, rs2 & No change \\
      5 & 12 & -- & \textbf{unused} & \textbf{Set bit 5 to ``1''} \\
      13 & 13 & 0,1 & The \textbf{i} bit & \textbf{Set i to ``0''} \\
      14 & 18 & 32 & Source register 1, rs1 & No change \\
      19 & 24 & 000100 & ``\textbf{op3}'' & No change \\
      25 & 29 & 32 & Destination register, rd & No change \\
      30 & 31 & 4 & Always ``10'' & No change \\
      \hline
    \end{tabular}
  \end{center}
  New addition:
  \begin{itemize}
  \item []\textbf{SUBD}: same as SUB, but with Instr[13]=0 (i=0), and
    Instr[5]=1.
  \item []\textbf{Syntax}: ``\texttt{subd  SrcReg1, SrcReg2, DestReg}''.
  \item []\textbf{Semantics}: rd(pair) $\leftarrow$ rs1(pair) - rs2(pair).
  \end{itemize}
  Bits layout:
\begin{verbatim}
    Offsets      : 31       24 23       16  15        8   7        0
    Bit layout   :  XXXX  XXXX  XXXX  XXXX   XXXX  XXXX   XXXX  XXXX
    Insn Bits    :  10       0  0010  0        0            1       
    Destination  :    DD  DDD                                       
    Source 1     :                     SSS   SS
    Source 2     :                                           S  SSSS
    Unused (0)   :                              U  UUUU   UU        
    Final layout :  10DD  DDD0  0010  0SSS   SS0U  UUUU   UU1S  SSSS
\end{verbatim}

  Hence the SPARC bit layout of this instruction is:

  \begin{tabular}[h]{lclcl}
    Macro to set  &=& \texttt{F4(x, y, z)} &in& \texttt{sparc.h}     \\
    Macro to reset  &=& \texttt{INVF4(x, y, z)} &in& \texttt{sparc.h}     \\
    x &=& 0x2      &in& \texttt{OP(x)  /* ((x) \& 0x3)  $<<$ 30 */} \\
    y &=& 0x04     &in& \texttt{OP3(y) /* ((y) \& 0x3f) $<<$ 19 */} \\
    z &=& 0x0      &in& \texttt{F3I(z) /* ((z) \& 0x1)  $<<$ 13 */} \\
    a &=& 0x1      &in& \texttt{OP\_AJIT\_BIT(a) /* ((a) \& 0x1)  $<<$ 5 */}
  \end{tabular}

  The AJIT bit  (insn[5]) is set internally by  \texttt{F4}, and hence
  there are only three arguments.

\item \textbf{SUBDCC}:\\
  \begin{center}
    \begin{tabular}[p]{|c|c|l|l|l|}
      \hline
      \textbf{Start} & \textbf{End} & \textbf{Range} & \textbf{Meaning} &
                                                                          \textbf{New Meaning}\\
      \hline
      0 & 4 & 32 & Source register 2, rs2 & No change \\
      5 & 12 & -- & \textbf{unused} & \textbf{Set bit 5 to ``1''} \\
      13 & 13 & 0,1 & The \textbf{i} bit & \textbf{Set i to ``0''} \\
      14 & 18 & 32 & Source register 1, rs1 & No change \\
      19 & 24 & 010100 & ``\textbf{op3}'' & No change \\
      25 & 29 & 32 & Destination register, rd & No change \\
      30 & 31 & 4 & Always ``10'' & No change \\
      \hline
    \end{tabular}
  \end{center}
  New addition:
  \begin{itemize}
  \item []\textbf{SUBDCC}: same as SUBCC, but with Instr[13]=0 (i=0), and
    Instr[5]=1.
  \item []\textbf{Syntax}: ``\texttt{subdcc  SrcReg1, SrcReg2, DestReg}''.
  \item []\textbf{Semantics}: rd(pair) $\leftarrow$ rs1(pair) - rs2(pair), set Z,N
  \end{itemize}
  Bits layout:
\begin{verbatim}
    Offsets      : 31       24 23       16  15        8   7        0
    Bit layout   :  XXXX  XXXX  XXXX  XXXX   XXXX  XXXX   XXXX  XXXX
    Insn Bits    :  10       0  1010  0        0            1       
    Destination  :    DD  DDD                                       
    Source 1     :                     SSS   SS
    Source 2     :                                           S  SSSS
    Unused (0)   :                              U  UUUU   UU        
    Final layout :  10DD  DDD0  1010  0SSS   SS0U  UUUU   UU1S  SSSS
\end{verbatim}

  Hence the SPARC bit layout of this instruction is:

  \begin{tabular}[h]{lclcl}
    Macro to set  &=& \texttt{F4(x, y, z)} &in& \texttt{sparc.h}     \\
    Macro to reset  &=& \texttt{INVF4(x, y, z)} &in& \texttt{sparc.h}     \\
    x &=& 0x2      &in& \texttt{OP(x)  /* ((x) \& 0x3)  $<<$ 30 */} \\
    y &=& 0x14     &in& \texttt{OP3(y) /* ((y) \& 0x3f) $<<$ 19 */} \\
    z &=& 0x0      &in& \texttt{F3I(z) /* ((z) \& 0x1)  $<<$ 13 */} \\
    a &=& 0x1      &in& \texttt{OP\_AJIT\_BIT(a) /* ((a) \& 0x1)  $<<$ 5 */}
  \end{tabular}

  The AJIT bit  (insn[5]) is set internally by  \texttt{F4}, and hence
  there are only three arguments.
\end{enumerate}

\item {Addition and subtraction instructions:}\\
  \begin{enumerate}
  \item \textbf{ADDD}:\\
    \begin{tabular}[h]{lclcl}
      Macro to set  &=& \texttt{F4(x, y, z)} &in& \texttt{sparc.h}     \\
      Macro to reset  &=& \texttt{INVF4(x, y, z)} &in& \texttt{sparc.h}     \\
      x &=& 0x2      &in& \texttt{OP(x)  /* ((x) \& 0x3)  $<<$ 30 */} \\
      y &=& 0x00     &in& \texttt{OP3(y) /* ((y) \& 0x3f) $<<$ 19 */} \\
      z &=& 0x0      &in& \texttt{F3I(z) /* ((z) \& 0x1)  $<<$ 13 */} \\
      a &=& 0x1      &in& \texttt{OP\_AJIT\_BIT(a) /* ((a) \& 0x1)  $<<$ 5 */}
    \end{tabular}

    The AJIT bit  (insn[5]) is set internally by  \texttt{F4}, and hence
    there are only three arguments.

  \item \textbf{ADDDCC}:\\
    \begin{tabular}[h]{lclcl}
      Macro to set  &=& \texttt{F4(x, y, z)} &in& \texttt{sparc.h}     \\
      Macro to reset  &=& \texttt{INVF4(x, y, z)} &in& \texttt{sparc.h}     \\
      x &=& 0x2      &in& \texttt{OP(x)  /* ((x) \& 0x3)  $<<$ 30 */} \\
      y &=& 0x10     &in& \texttt{OP3(y) /* ((y) \& 0x3f) $<<$ 19 */} \\
      z &=& 0x0      &in& \texttt{F3I(z) /* ((z) \& 0x1)  $<<$ 13 */} \\
      a &=& 0x1      &in& \texttt{OP\_AJIT\_BIT(a) /* ((a) \& 0x1)  $<<$ 5 */}
    \end{tabular}

    The AJIT bit  (insn[5]) is set internally by  \texttt{F4}, and hence
    there are only three arguments.

  \item \textbf{SUBD}:\\
    \begin{tabular}[h]{lclcl}
      Macro to set  &=& \texttt{F4(x, y, z)} &in& \texttt{sparc.h}     \\
      Macro to reset  &=& \texttt{INVF4(x, y, z)} &in& \texttt{sparc.h}     \\
      x &=& 0x2      &in& \texttt{OP(x)  /* ((x) \& 0x3)  $<<$ 30 */} \\
      y &=& 0x04     &in& \texttt{OP3(y) /* ((y) \& 0x3f) $<<$ 19 */} \\
      z &=& 0x0      &in& \texttt{F3I(z) /* ((z) \& 0x1)  $<<$ 13 */} \\
      a &=& 0x1      &in& \texttt{OP\_AJIT\_BIT(a) /* ((a) \& 0x1)  $<<$ 5 */}
    \end{tabular}

    The AJIT bit  (insn[5]) is set internally by  \texttt{F4}, and hence
    there are only three arguments.

  \item \textbf{SUBDCC}:\\
    \begin{tabular}[h]{lclcl}
      Macro to set  &=& \texttt{F4(x, y, z)} &in& \texttt{sparc.h}     \\
      Macro to reset  &=& \texttt{INVF4(x, y, z)} &in& \texttt{sparc.h}     \\
      x &=& 0x2      &in& \texttt{OP(x)  /* ((x) \& 0x3)  $<<$ 30 */} \\
      y &=& 0x14     &in& \texttt{OP3(y) /* ((y) \& 0x3f) $<<$ 19 */} \\
      z &=& 0x0      &in& \texttt{F3I(z) /* ((z) \& 0x1)  $<<$ 13 */} \\
      a &=& 0x1      &in& \texttt{OP\_AJIT\_BIT(a) /* ((a) \& 0x1)  $<<$ 5 */}
    \end{tabular}

    The AJIT bit  (insn[5]) is set internally by  \texttt{F4}, and hence
    there are only three arguments.
  \end{enumerate}

  % \subsubsection{Multiplication and division instructions:}
\label{sec:mul:div:insn:impl}
\begin{enumerate}
\item \textbf{UMULD}: Unsigned Integer Multiply AJIT, no immediate
  version (i.e. i is always 0).\\
	\textbf{NOTE:} The \emph{suggested} mnemonic ``umuld'' conflicts with a mnemonic of the same name for another sparc architecture (other than v8).   Hence we change it to: ``\textbf{umuldaj}'' in the implementation, but not in the documentation below.

 This conflict occurs despite forcing the GNU assembler to assemble for v8 only via the command line switch ``-Av8''! It appears that forcing the assembler to use v8 is not universally applied throughout the assembler code. 
  \begin{center}
    \begin{tabular}[p]{|c|c|l|l|l|}
      \hline
      \textbf{Start} & \textbf{End} & \textbf{Range} & \textbf{Meaning} &
                                                                          \textbf{New Meaning}\\
      \hline
      0 & 4 & 32 & Source register 2, rs2 & No change \\
      5 & 12 & -- & \textbf{unused} & \textbf{Set bit 5 to ``1''} \\
      13 & 13 & 0,1 & The \textbf{i} bit & \textbf{Set i to ``0''} \\
      14 & 18 & 32 & Source register 1, rs1 & No change \\
      19 & 24 & 001010 & ``\textbf{op3}'' & No change \\
      25 & 29 & 32 & Destination register, rd & No change \\
      30 & 31 & 4 & Always ``10'' & No change \\
      \hline
    \end{tabular}
  \end{center}
  \begin{itemize}
  \item []\textbf{UMULD}: same as UMUL, but with Instr[13]=0 (i=0), and
    Instr[5]=1.
  \item []\textbf{Syntax}: ``\texttt{umuld  SrcReg1, SrcReg2, DestReg}''.
  \item []\textbf{Semantics}: rd(pair) $\leftarrow$ rs1(pair) * rs2(pair).
  \end{itemize}
  Bits layout:
\begin{verbatim}
    Offsets      : 31       24 23       16  15        8   7        0
    Bit layout   :  XXXX  XXXX  XXXX  XXXX   XXXX  XXXX   XXXX  XXXX
    Insn Bits    :  10       0  0101  0        0            1       
    Destination  :    DD  DDD                                       
    Source 1     :                     SSS   SS
    Source 2     :                                           S  SSSS
    Unused (0)   :                              U  UUUU   UU        
    Final layout :  10DD  DDD0  0101  0SSS   SS0U  UUUU   UU1S  SSSS
\end{verbatim}

  Hence the SPARC bit layout of this instruction is:

  \begin{tabular}[h]{lclcl}
    Macro to set  &=& \texttt{F4(x, y, z)} &in& \texttt{sparc.h}     \\
    Macro to reset  &=& \texttt{INVF4(x, y, z)} &in& \texttt{sparc.h}     \\
    x &=& 0x2      &in& \texttt{OP(x)  /* ((x) \& 0x3)  $<<$ 30 */} \\
    y &=& 0x0A     &in& \texttt{OP3(y) /* ((y) \& 0x3f) $<<$ 19 */} \\
    z &=& 0x0      &in& \texttt{F3I(z) /* ((z) \& 0x1)  $<<$ 13 */} \\
    a &=& 0x1      &in& \texttt{OP\_AJIT\_BIT(a) /* ((a) \& 0x1)  $<<$ 5 */}
  \end{tabular}

  The AJIT bit  (insn[5]) is set internally by  \texttt{F4}, and hence
  there are only three arguments.

\item \textbf{UMULDCC}:\\
  \begin{center}
    \begin{tabular}[p]{|c|c|l|l|l|}
      \hline
      \textbf{Start} & \textbf{End} & \textbf{Range} & \textbf{Meaning} &
                                                                          \textbf{New Meaning}\\
      \hline
      0 & 4 & 32 & Source register 2, rs2 & No change \\
      5 & 12 & -- & \textbf{unused} & \textbf{Set bit 5 to ``1''} \\
      13 & 13 & 0,1 & The \textbf{i} bit & \textbf{Set i to ``0''} \\
      14 & 18 & 32 & Source register 1, rs1 & No change \\
      19 & 24 & 011010 & ``\textbf{op3}'' & No change \\
      25 & 29 & 32 & Destination register, rd & No change \\
      30 & 31 & 4 & Always ``10'' & No change \\
      \hline
    \end{tabular}
  \end{center}
  New addition:
  \begin{itemize}
  \item []\textbf{UMULDCC}: same as UMULCC, but with Instr[13]=0 (i=0), and
    Instr[5]=1.
  \item []\textbf{Syntax}: ``\texttt{umuldcc  SrcReg1, SrcReg2, DestReg}''.
  \item []\textbf{Semantics}: rd(pair) $\leftarrow$ rs1(pair) * rs2(pair), set Z
  \end{itemize}
  Bits layout:
\begin{verbatim}
    Offsets      : 31       24 23       16  15        8   7        0
    Bit layout   :  XXXX  XXXX  XXXX  XXXX   XXXX  XXXX   XXXX  XXXX
    Insn Bits    :  10       0  1101  0        0            1       
    Destination  :    DD  DDD                                       
    Source 1     :                     SSS   SS
    Source 2     :                                           S  SSSS
    Unused (0)   :                              U  UUUU   UU        
    Final layout :  10DD  DDD0  1101  0SSS   SS0U  UUUU   UU1S  SSSS
\end{verbatim}

  Hence the SPARC bit layout of this instruction is:

  \begin{tabular}[h]{lclcl}
    Macro to set  &=& \texttt{F4(x, y, z)} &in& \texttt{sparc.h}     \\
    Macro to reset  &=& \texttt{INVF4(x, y, z)} &in& \texttt{sparc.h}     \\
    x &=& 0x2      &in& \texttt{OP(x)  /* ((x) \& 0x3)  $<<$ 30 */} \\
    y &=& 0x1A     &in& \texttt{OP3(y) /* ((y) \& 0x3f) $<<$ 19 */} \\
    z &=& 0x0      &in& \texttt{F3I(z) /* ((z) \& 0x1)  $<<$ 13 */} \\
    a &=& 0x1      &in& \texttt{OP\_AJIT\_BIT(a) /* ((a) \& 0x1)  $<<$ 5 */}
  \end{tabular}

  The AJIT bit  (insn[5]) is set internally by  \texttt{F4}, and hence
  there are only three arguments.

\item \textbf{SMULD}: Unsigned Integer Multiply AJIT, no immediate
  version (i.e. i is always 0).\\
  \begin{center}
    \begin{tabular}[p]{|c|c|l|l|l|}
      \hline
      \textbf{Start} & \textbf{End} & \textbf{Range} & \textbf{Meaning} &
                                                                          \textbf{New Meaning}\\
      \hline
      0 & 4 & 32 & Source register 2, rs2 & No change \\
      5 & 12 & -- & \textbf{unused} & \textbf{Set bit 5 to ``1''} \\
      13 & 13 & 0,1 & The \textbf{i} bit & \textbf{Set i to ``0''} \\
      14 & 18 & 32 & Source register 1, rs1 & No change \\
      19 & 24 & 001011 & ``\textbf{op3}'' & No change \\
      25 & 29 & 32 & Destination register, rd & No change \\
      30 & 31 & 4 & Always ``10'' & No change \\
      \hline
    \end{tabular}
  \end{center}
  \begin{itemize}
  \item []\textbf{SMULD}: same as SMUL, but with Instr[13]=0 (i=0), and
    Instr[5]=1.
  \item []\textbf{Syntax}: ``\texttt{smuld  SrcReg1, SrcReg2, DestReg}''.
  \item []\textbf{Semantics}: rd(pair) $\leftarrow$ rs1(pair) *
    rs2(pair) (signed).
  \end{itemize}
  Bits layout:
\begin{verbatim}
    Offsets      : 31       24 23       16  15        8   7        0
    Bit layout   :  XXXX  XXXX  XXXX  XXXX   XXXX  XXXX   XXXX  XXXX
    Insn Bits    :  10       0  0101  1        0            1       
    Destination  :    DD  DDD                                       
    Source 1     :                     SSS   SS
    Source 2     :                                           S  SSSS
    Unused (0)   :                              U  UUUU   UU        
    Final layout :  10DD  DDD0  0101  1SSS   SS0U  UUUU   UU1S  SSSS
\end{verbatim}

  Hence the SPARC bit layout of this instruction is:

  \begin{tabular}[h]{lclcl}
    Macro to set  &=& \texttt{F4(x, y, z)} &in& \texttt{sparc.h}     \\
    Macro to reset  &=& \texttt{INVF4(x, y, z)} &in& \texttt{sparc.h}     \\
    x &=& 0x2      &in& \texttt{OP(x)  /* ((x) \& 0x3)  $<<$ 30 */} \\
    y &=& 0x0B     &in& \texttt{OP3(y) /* ((y) \& 0x3f) $<<$ 19 */} \\
    z &=& 0x0      &in& \texttt{F3I(z) /* ((z) \& 0x1)  $<<$ 13 */} \\
    a &=& 0x1      &in& \texttt{OP\_AJIT\_BIT(a) /* ((a) \& 0x1)  $<<$ 5 */}
  \end{tabular}

  The AJIT bit  (insn[5]) is set internally by  \texttt{F4}, and hence
  there are only three arguments.

\item \textbf{SMULDCC}:\\
  \begin{center}
    \begin{tabular}[p]{|c|c|l|l|l|}
      \hline
      \textbf{Start} & \textbf{End} & \textbf{Range} & \textbf{Meaning} &
                                                                          \textbf{New Meaning}\\
      \hline
      0 & 4 & 32 & Source register 2, rs2 & No change \\
      5 & 12 & -- & \textbf{unused} & \textbf{Set bit 5 to ``1''} \\
      13 & 13 & 0,1 & The \textbf{i} bit & \textbf{Set i to ``0''} \\
      14 & 18 & 32 & Source register 1, rs1 & No change \\
      19 & 24 & 011011 & ``\textbf{op3}'' & No change \\
      25 & 29 & 32 & Destination register, rd & No change \\
      30 & 31 & 4 & Always ``10'' & No change \\
      \hline
    \end{tabular}
  \end{center}
  New addition:
  \begin{itemize}
  \item []\textbf{SMULDCC}: same as SMULCC, but with Instr[13]=0 (i=0), and
    Instr[5]=1.
  \item []\textbf{Syntax}: ``\texttt{smuldcc  SrcReg1, SrcReg2, DestReg}''.
  \item []\textbf{Semantics}: rd(pair) $\leftarrow$ rs1(pair) *
    rs2(pair) (signed), set Z,N,O
  \end{itemize}
  Bits layout:
\begin{verbatim}
    Offsets      : 31       24 23       16  15        8   7        0
    Bit layout   :  XXXX  XXXX  XXXX  XXXX   XXXX  XXXX   XXXX  XXXX
    Insn Bits    :  10       0  1101  1        0            1       
    Destination  :    DD  DDD                                       
    Source 1     :                     SSS   SS
    Source 2     :                                           S  SSSS
    Unused (0)   :                              U  UUUU   UU        
    Final layout :  10DD  DDD0  1101  1SSS   SS0U  UUUU   UU1S  SSSS
\end{verbatim}

  Hence the SPARC bit layout of this instruction is:

  \begin{tabular}[h]{lclcl}
    Macro to set  &=& \texttt{F4(x, y, z)} &in& \texttt{sparc.h}     \\
    Macro to reset  &=& \texttt{INVF4(x, y, z)} &in& \texttt{sparc.h}     \\
    x &=& 0x2      &in& \texttt{OP(x)  /* ((x) \& 0x3)  $<<$ 30 */} \\
    y &=& 0x1B     &in& \texttt{OP3(y) /* ((y) \& 0x3f) $<<$ 19 */} \\
    z &=& 0x0      &in& \texttt{F3I(z) /* ((z) \& 0x1)  $<<$ 13 */} \\
    a &=& 0x1      &in& \texttt{OP\_AJIT\_BIT(a) /* ((a) \& 0x1)  $<<$ 5 */}
  \end{tabular}

  The AJIT bit  (insn[5]) is set internally by  \texttt{F4}, and hence
  there are only three arguments.

\item \textbf{UDIVD}:\\
  \begin{center}
    \begin{tabular}[p]{|c|c|l|l|l|}
      \hline
      \textbf{Start} & \textbf{End} & \textbf{Range} & \textbf{Meaning} &
                                                                          \textbf{New Meaning}\\
      \hline
      0 & 4 & 32 & Source register 2, rs2 & No change \\
      5 & 12 & -- & \textbf{unused} & \textbf{Set bit 5 to ``1''} \\
      13 & 13 & 0,1 & The \textbf{i} bit & \textbf{Set i to ``0''} \\
      14 & 18 & 32 & Source register 1, rs1 & No change \\
      19 & 24 & 001110 & ``\textbf{op3}'' & No change \\
      25 & 29 & 32 & Destination register, rd & No change \\
      30 & 31 & 4 & Always ``10'' & No change \\
      \hline
    \end{tabular}
  \end{center}
  New addition:
  \begin{itemize}
  \item []\textbf{UDIVD}: same as UDIV, but with Instr[13]=0 (i=0), and
    Instr[5]=1.
  \item []\textbf{Syntax}: ``\texttt{udivd  SrcReg1, SrcReg2, DestReg}''.
  \item []\textbf{Semantics}: rd(pair) $\leftarrow$ rs1(pair) / rs2(pair).
  \end{itemize}
  Bits layout:
\begin{verbatim}
    Offsets      : 31       24 23       16  15        8   7        0
    Bit layout   :  XXXX  XXXX  XXXX  XXXX   XXXX  XXXX   XXXX  XXXX
    Insn Bits    :  10       0  0111  0        0            1       
    Destination  :    DD  DDD                                       
    Source 1     :                     SSS   SS
    Source 2     :                                           S  SSSS
    Unused (0)   :                              U  UUUU   UU        
    Final layout :  10DD  DDD0  0111  0SSS   SS0U  UUUU   UU1S  SSSS
\end{verbatim}

  Hence the SPARC bit layout of this instruction is:

  \begin{tabular}[h]{lclcl}
    Macro to set  &=& \texttt{F4(x, y, z)} &in& \texttt{sparc.h}     \\
    Macro to reset  &=& \texttt{INVF4(x, y, z)} &in& \texttt{sparc.h}     \\
    x &=& 0x2      &in& \texttt{OP(x)  /* ((x) \& 0x3)  $<<$ 30 */} \\
    y &=& 0x0E     &in& \texttt{OP3(y) /* ((y) \& 0x3f) $<<$ 19 */} \\
    z &=& 0x0      &in& \texttt{F3I(z) /* ((z) \& 0x1)  $<<$ 13 */} \\
    a &=& 0x1      &in& \texttt{OP\_AJIT\_BIT(a) /* ((a) \& 0x1)  $<<$ 5 */}
  \end{tabular}

  The AJIT bit  (insn[5]) is set internally by  \texttt{F4}, and hence
  there are only three arguments.

\item \textbf{UDIVDCC}:\\
  \begin{center}
    \begin{tabular}[p]{|c|c|l|l|l|}
      \hline
      \textbf{Start} & \textbf{End} & \textbf{Range} & \textbf{Meaning} &
                                                                          \textbf{New Meaning}\\
      \hline
      0 & 4 & 32 & Source register 2, rs2 & No change \\
      5 & 12 & -- & \textbf{unused} & \textbf{Set bit 5 to ``1''} \\
      13 & 13 & 0,1 & The \textbf{i} bit & \textbf{Set i to ``0''} \\
      14 & 18 & 32 & Source register 1, rs1 & No change \\
      19 & 24 & 011110 & ``\textbf{op3}'' & No change \\
      25 & 29 & 32 & Destination register, rd & No change \\
      30 & 31 & 4 & Always ``10'' & No change \\
      \hline
    \end{tabular}
  \end{center}
  New addition:
  \begin{itemize}
  \item []\textbf{UDIVDCC}: same as UDIVCC, but with Instr[13]=0 (i=0), and
    Instr[5]=1.
  \item []\textbf{Syntax}: ``\texttt{udivdcc  SrcReg1, SrcReg2, DestReg}''.
  \item []\textbf{Semantics}: rd(pair) $\leftarrow$ rs1(pair) / rs2(pair), set Z,O
  \end{itemize}
  Bits layout:
\begin{verbatim}
    Offsets      : 31       24 23       16  15        8   7        0
    Bit layout   :  XXXX  XXXX  XXXX  XXXX   XXXX  XXXX   XXXX  XXXX
    Insn Bits    :  10       0  1111  0        0            1       
    Destination  :    DD  DDD                                       
    Source 1     :                     SSS   SS
    Source 2     :                                           S  SSSS
    Unused (0)   :                              U  UUUU   UU        
    Final layout :  10DD  DDD0  1111  0SSS   SS0U  UUUU   UU1S  SSSS
\end{verbatim}

  Hence the SPARC bit layout of this instruction is:

  \begin{tabular}[h]{lclcl}
    Macro to set  &=& \texttt{F4(x, y, z)} &in& \texttt{sparc.h}     \\
    Macro to reset  &=& \texttt{INVF4(x, y, z)} &in& \texttt{sparc.h}     \\
    x &=& 0x2      &in& \texttt{OP(x)  /* ((x) \& 0x3)  $<<$ 30 */} \\
    y &=& 0x1E     &in& \texttt{OP3(y) /* ((y) \& 0x3f) $<<$ 19 */} \\
    z &=& 0x0      &in& \texttt{F3I(z) /* ((z) \& 0x1)  $<<$ 13 */} \\
    a &=& 0x1      &in& \texttt{OP\_AJIT\_BIT(a) /* ((a) \& 0x1)  $<<$ 5 */}
  \end{tabular}

  The AJIT bit  (insn[5]) is set internally by  \texttt{F4}, and hence
  there are only three arguments.

\item \textbf{SDIVD}:\\
  \begin{center}
    \begin{tabular}[p]{|c|c|l|l|l|}
      \hline
      \textbf{Start} & \textbf{End} & \textbf{Range} & \textbf{Meaning} &
                                                                          \textbf{New Meaning}\\
      \hline
      0 & 4 & 32 & Source register 2, rs2 & No change \\
      5 & 12 & -- & \textbf{unused} & \textbf{Set bit 5 to ``1''} \\
      13 & 13 & 0,1 & The \textbf{i} bit & \textbf{Set i to ``0''} \\
      14 & 18 & 32 & Source register 1, rs1 & No change \\
      19 & 24 & 001111 & ``\textbf{op3}'' & No change \\
      25 & 29 & 32 & Destination register, rd & No change \\
      30 & 31 & 4 & Always ``10'' & No change \\
      \hline
    \end{tabular}
  \end{center}
  New addition:
  \begin{itemize}
  \item []\textbf{SDIVD}: same as SDIV, but with Instr[13]=0 (i=0), and
    Instr[5]=1.
  \item []\textbf{Syntax}: ``\texttt{sdivd  SrcReg1, SrcReg2, DestReg}''.
  \item []\textbf{Semantics}: rd(pair) $\leftarrow$ rs1(pair) /
    rs2(pair) (signed).
  \end{itemize}
  Bits layout:
\begin{verbatim}
    Offsets      : 31       24 23       16  15        8   7        0
    Bit layout   :  XXXX  XXXX  XXXX  XXXX   XXXX  XXXX   XXXX  XXXX
    Insn Bits    :  10       0  0111  1        0            1       
    Destination  :    DD  DDD                                       
    Source 1     :                     SSS   SS
    Source 2     :                                           S  SSSS
    Unused (0)   :                              U  UUUU   UU        
    Final layout :  10DD  DDD0  0111  1SSS   SS0U  UUUU   UU1S  SSSS
\end{verbatim}

  Hence the SPARC bit layout of this instruction is:

  \begin{tabular}[h]{lclcl}
    Macro to set  &=& \texttt{F4(x, y, z)} &in& \texttt{sparc.h}     \\
    Macro to reset  &=& \texttt{INVF4(x, y, z)} &in& \texttt{sparc.h}     \\
    x &=& 0x2      &in& \texttt{OP(x)  /* ((x) \& 0x3)  $<<$ 30 */} \\
    y &=& 0x0F     &in& \texttt{OP3(y) /* ((y) \& 0x3f) $<<$ 19 */} \\
    z &=& 0x0      &in& \texttt{F3I(z) /* ((z) \& 0x1)  $<<$ 13 */} \\
    a &=& 0x1      &in& \texttt{OP\_AJIT\_BIT(a) /* ((a) \& 0x1)  $<<$ 5 */}
  \end{tabular}

  The AJIT bit  (insn[5]) is set internally by  \texttt{F4}, and hence
  there are only three arguments.

\item \textbf{SDIVDCC}:\\
  \begin{center}
    \begin{tabular}[p]{|c|c|l|l|l|}
      \hline
      \textbf{Start} & \textbf{End} & \textbf{Range} & \textbf{Meaning} &
                                                                          \textbf{New Meaning}\\
      \hline
      0 & 4 & 32 & Source register 2, rs2 & No change \\
      5 & 12 & -- & \textbf{unused} & \textbf{Set bit 5 to ``1''} \\
      13 & 13 & 0,1 & The \textbf{i} bit & \textbf{Set i to ``0''} \\
      14 & 18 & 32 & Source register 1, rs1 & No change \\
      19 & 24 & 011111 & ``\textbf{op3}'' & No change \\
      25 & 29 & 32 & Destination register, rd & No change \\
      30 & 31 & 4 & Always ``10'' & No change \\
      \hline
    \end{tabular}
  \end{center}
  New addition:
  \begin{itemize}
  \item []\textbf{SDIVDCC}: same as SDIVCC, but with Instr[13]=0 (i=0), and
    Instr[5]=1.
  \item []\textbf{Syntax}: ``\texttt{sdivdcc  SrcReg1, SrcReg2, DestReg}''.
  \item []\textbf{Semantics}: rd(pair) $\leftarrow$ rs1(pair) /
    rs2(pair) (signed), set Z,N,O
  \end{itemize}
  Bits layout:
\begin{verbatim}
    Offsets      : 31       24 23       16  15        8   7        0
    Bit layout   :  XXXX  XXXX  XXXX  XXXX   XXXX  XXXX   XXXX  XXXX
    Insn Bits    :  10       0  1111  1        0            1       
    Destination  :    DD  DDD                                       
    Source 1     :                     SSS   SS
    Source 2     :                                           S  SSSS
    Unused (0)   :                              U  UUUU   UU        
    Final layout :  10DD  DDD0  1111  1SSS   SS0U  UUUU   UU1S  SSSS
\end{verbatim}

  Hence the SPARC bit layout of this instruction is:

  \begin{tabular}[h]{lclcl}
    Macro to set  &=& \texttt{F4(x, y, z)} &in& \texttt{sparc.h}     \\
    Macro to reset  &=& \texttt{INVF4(x, y, z)} &in& \texttt{sparc.h}     \\
    x &=& 0x2      &in& \texttt{OP(x)  /* ((x) \& 0x3)  $<<$ 30 */} \\
    y &=& 0x1F     &in& \texttt{OP3(y) /* ((y) \& 0x3f) $<<$ 19 */} \\
    z &=& 0x0      &in& \texttt{F3I(z) /* ((z) \& 0x1)  $<<$ 13 */} \\
    a &=& 0x1      &in& \texttt{OP\_AJIT\_BIT(a) /* ((a) \& 0x1)  $<<$ 5 */}
  \end{tabular}

  The AJIT bit  (insn[5]) is set internally by  \texttt{F4}, and hence
  there are only three arguments.
\end{enumerate}

%%% Local Variables:
%%% mode: latex
%%% TeX-master: t
%%% End:

\item {Multiplication and division instructions:} \\
  \begin{enumerate}
  \item \textbf{UMULD}: Unsigned Integer Multiply AJIT, no immediate
    version (i.e. i is always 0).\\
    \begin{tabular}[h]{lclcl}
      Macro to set  &=& \texttt{F4(x, y, z)} &in& \texttt{sparc.h}     \\
      Macro to reset  &=& \texttt{INVF4(x, y, z)} &in& \texttt{sparc.h}     \\
      x &=& 0x2      &in& \texttt{OP(x)  /* ((x) \& 0x3)  $<<$ 30 */} \\
      y &=& 0x0A     &in& \texttt{OP3(y) /* ((y) \& 0x3f) $<<$ 19 */} \\
      z &=& 0x0      &in& \texttt{F3I(z) /* ((z) \& 0x1)  $<<$ 13 */} \\
      a &=& 0x1      &in& \texttt{OP\_AJIT\_BIT(a) /* ((a) \& 0x1)  $<<$ 5 */}
    \end{tabular}

    The AJIT bit  (insn[5]) is set internally by  \texttt{F4}, and hence
    there are only three arguments.

  \item \textbf{UMULDCC}:\\
    \begin{tabular}[h]{lclcl}
      Macro to set  &=& \texttt{F4(x, y, z)} &in& \texttt{sparc.h}     \\
      Macro to reset  &=& \texttt{INVF4(x, y, z)} &in& \texttt{sparc.h}     \\
      x &=& 0x2      &in& \texttt{OP(x)  /* ((x) \& 0x3)  $<<$ 30 */} \\
      y &=& 0x1A     &in& \texttt{OP3(y) /* ((y) \& 0x3f) $<<$ 19 */} \\
      z &=& 0x0      &in& \texttt{F3I(z) /* ((z) \& 0x1)  $<<$ 13 */} \\
      a &=& 0x1      &in& \texttt{OP\_AJIT\_BIT(a) /* ((a) \& 0x1)  $<<$ 5 */}
    \end{tabular}

    The AJIT bit  (insn[5]) is set internally by  \texttt{F4}, and hence
    there are only three arguments.

  \item \textbf{SMULD}: Unsigned Integer Multiply AJIT, no immediate
    version (i.e. i is always 0).\\
    \begin{tabular}[h]{lclcl}
      Macro to set  &=& \texttt{F4(x, y, z)} &in& \texttt{sparc.h}     \\
      Macro to reset  &=& \texttt{INVF4(x, y, z)} &in& \texttt{sparc.h}     \\
      x &=& 0x2      &in& \texttt{OP(x)  /* ((x) \& 0x3)  $<<$ 30 */} \\
      y &=& 0x0B     &in& \texttt{OP3(y) /* ((y) \& 0x3f) $<<$ 19 */} \\
      z &=& 0x0      &in& \texttt{F3I(z) /* ((z) \& 0x1)  $<<$ 13 */} \\
      a &=& 0x1      &in& \texttt{OP\_AJIT\_BIT(a) /* ((a) \& 0x1)  $<<$ 5 */}
    \end{tabular}

    The AJIT bit  (insn[5]) is set internally by  \texttt{F4}, and hence
    there are only three arguments.

  \item \textbf{SMULDCC}:\\
    \begin{tabular}[h]{lclcl}
      Macro to set  &=& \texttt{F4(x, y, z)} &in& \texttt{sparc.h}     \\
      Macro to reset  &=& \texttt{INVF4(x, y, z)} &in& \texttt{sparc.h}     \\
      x &=& 0x2      &in& \texttt{OP(x)  /* ((x) \& 0x3)  $<<$ 30 */} \\
      y &=& 0x1B     &in& \texttt{OP3(y) /* ((y) \& 0x3f) $<<$ 19 */} \\
      z &=& 0x0      &in& \texttt{F3I(z) /* ((z) \& 0x1)  $<<$ 13 */} \\
      a &=& 0x1      &in& \texttt{OP\_AJIT\_BIT(a) /* ((a) \& 0x1)  $<<$ 5 */}
    \end{tabular}

    The AJIT bit  (insn[5]) is set internally by  \texttt{F4}, and hence
    there are only three arguments.

  \item \textbf{UDIVD}:\\
    \begin{tabular}[h]{lclcl}
      Macro to set  &=& \texttt{F4(x, y, z)} &in& \texttt{sparc.h}     \\
      Macro to reset  &=& \texttt{INVF4(x, y, z)} &in& \texttt{sparc.h}     \\
      x &=& 0x2      &in& \texttt{OP(x)  /* ((x) \& 0x3)  $<<$ 30 */} \\
      y &=& 0x0E     &in& \texttt{OP3(y) /* ((y) \& 0x3f) $<<$ 19 */} \\
      z &=& 0x0      &in& \texttt{F3I(z) /* ((z) \& 0x1)  $<<$ 13 */} \\
      a &=& 0x1      &in& \texttt{OP\_AJIT\_BIT(a) /* ((a) \& 0x1)  $<<$ 5 */}
    \end{tabular}

    The AJIT bit  (insn[5]) is set internally by  \texttt{F4}, and hence
    there are only three arguments.

  \item \textbf{UDIVDCC}:\\
    \begin{tabular}[h]{lclcl}
      Macro to set  &=& \texttt{F4(x, y, z)} &in& \texttt{sparc.h}     \\
      Macro to reset  &=& \texttt{INVF4(x, y, z)} &in& \texttt{sparc.h}     \\
      x &=& 0x2      &in& \texttt{OP(x)  /* ((x) \& 0x3)  $<<$ 30 */} \\
      y &=& 0x1E     &in& \texttt{OP3(y) /* ((y) \& 0x3f) $<<$ 19 */} \\
      z &=& 0x0      &in& \texttt{F3I(z) /* ((z) \& 0x1)  $<<$ 13 */} \\
      a &=& 0x1      &in& \texttt{OP\_AJIT\_BIT(a) /* ((a) \& 0x1)  $<<$ 5 */}
    \end{tabular}

    The AJIT bit  (insn[5]) is set internally by  \texttt{F4}, and hence
    there are only three arguments.

  \item \textbf{SDIVD}:\\
    \begin{tabular}[h]{lclcl}
      Macro to set  &=& \texttt{F4(x, y, z)} &in& \texttt{sparc.h}     \\
      Macro to reset  &=& \texttt{INVF4(x, y, z)} &in& \texttt{sparc.h}     \\
      x &=& 0x2      &in& \texttt{OP(x)  /* ((x) \& 0x3)  $<<$ 30 */} \\
      y &=& 0x0F     &in& \texttt{OP3(y) /* ((y) \& 0x3f) $<<$ 19 */} \\
      z &=& 0x0      &in& \texttt{F3I(z) /* ((z) \& 0x1)  $<<$ 13 */} \\
      a &=& 0x1      &in& \texttt{OP\_AJIT\_BIT(a) /* ((a) \& 0x1)  $<<$ 5 */}
    \end{tabular}

    The AJIT bit  (insn[5]) is set internally by  \texttt{F4}, and hence
    there are only three arguments.

  \item \textbf{SDIVDCC}:\\
    \begin{tabular}[h]{lclcl}
      Macro to set  &=& \texttt{F4(x, y, z)} &in& \texttt{sparc.h}     \\
      Macro to reset  &=& \texttt{INVF4(x, y, z)} &in& \texttt{sparc.h}     \\
      x &=& 0x2      &in& \texttt{OP(x)  /* ((x) \& 0x3)  $<<$ 30 */} \\
      y &=& 0x1F     &in& \texttt{OP3(y) /* ((y) \& 0x3f) $<<$ 19 */} \\
      z &=& 0x0      &in& \texttt{F3I(z) /* ((z) \& 0x1)  $<<$ 13 */} \\
      a &=& 0x1      &in& \texttt{OP\_AJIT\_BIT(a) /* ((a) \& 0x1)  $<<$ 5 */}
    \end{tabular}

    The AJIT bit  (insn[5]) is set internally by  \texttt{F4}, and hence
    there are only three arguments.
  \end{enumerate}

  % \subsubsection{64 Bit Logical Instructions:}
\label{sec:64:bit:logical:insn:impl}

No immediate mode, i.e. insn[5] $\equiv$ i = 0, always.

\begin{enumerate}
\item \textbf{ORD}:\\
  \begin{center}
    \begin{figure}[h]
      \centering
      \epsfxsize=.8\linewidth
      \epsffile{../figs/ord-ajit-insn-32-bit-layout.eps}
      \caption{The AJIT ORD instruction  with register operands.}
      \label{fig:ajit:ord:insn}
    \end{figure}
  \end{center}
  \begin{itemize}
  \item []\textbf{ORD}: same as OR, but with Instr[13]=0 (i=0), and
    Instr[5]=1.
  \item []\textbf{Syntax}: ``\texttt{ord  SrcReg1, SrcReg2, DestReg}''.
  \item []\textbf{Semantics}: rd(pair) $\leftarrow$ rs1(pair) $\vert$ rs2(pair).
  \end{itemize}

  Hence the SPARC bit layout of this instruction is:

  \begin{tabular}[h]{lclcl}
    Macro to set   &=&  \verb|F4(x, y, z, b)|     &in& \verb|sparc.h|     \\
    Macro to reset &=&  \verb|F4(~x, ~y, ~z, ~b)| &in& \verb|sparc.h|     \\
    x              &=& 0x2                        &in& \verb|OP(x) | \\
    y              &=& 0x02                       &in& \verb|OP3(y) | \\
    z              &=& 0x0                        &in& \verb|F3I(z) | \\
    b              &=& 0x1                        &in& \verb|OP_AJIT_BIT_5(a) |
  \end{tabular}

\item \textbf{ORDCC}:\\
  \begin{center}
    \begin{figure}[h]
      \centering
      \epsfxsize=.8\linewidth
      \epsffile{../figs/ordcc-ajit-insn-32-bit-layout.eps}
      \caption{The AJIT ORDCC instruction  with register operands.}
      \label{fig:ajit:ordcc:insn}
    \end{figure}
  \end{center}
  \begin{itemize}
  \item []\textbf{ORDCC}: same as ORCC, but with Instr[13]=0 (i=0), and
    Instr[5]=1.
  \item []\textbf{Syntax}: ``\texttt{ordcc  SrcReg1, SrcReg2, DestReg}''.
  \item []\textbf{Semantics}: rd(pair) $\leftarrow$ rs1(pair) $\vert$
    rs2(pair), sets Z.
  \end{itemize}

  Hence the SPARC bit layout of this instruction is:

  \begin{tabular}[h]{lclcl}
    Macro to set   &=&  \verb|F4(x, y, z, b)|     &in& \verb|sparc.h|     \\
    Macro to reset &=&  \verb|F4(~x, ~y, ~z, ~b)| &in& \verb|sparc.h|     \\
    x              &=& 0x2                        &in& \verb|OP(x) | \\
    y              &=& 0x12                       &in& \verb|OP3(y) | \\
    z              &=& 0x0                        &in& \verb|F3I(z) | \\
    b              &=& 0x1                        &in& \verb|OP_AJIT_BIT_5(a) |
  \end{tabular}

\item \textbf{ORDN}:\\
  \begin{center}
    \begin{figure}[h]
      \centering
      \epsfxsize=.8\linewidth
      \epsffile{../figs/ordn-ajit-insn-32-bit-layout.eps}
      \caption{The AJIT ORDN instruction  with register operands.}
      \label{fig:ajit:ordn:insn}
    \end{figure}
  \end{center}
  \begin{itemize}
  \item []\textbf{ORDN}: same as ORN, but with Instr[13]=0 (i=0), and
    Instr[5]=1.
  \item []\textbf{Syntax}: ``\texttt{ordn  SrcReg1, SrcReg2, DestReg}''.
  \item []\textbf{Semantics}: rd(pair) $\leftarrow$ rs1(pair) $\vert$ ($\sim$rs2(pair)).
  \end{itemize}

  Hence the SPARC bit layout of this instruction is:

  \begin{tabular}[h]{lclcl}
    Macro to set   &=&  \verb|F4(x, y, z, b)|     &in& \verb|sparc.h|     \\
    Macro to reset &=&  \verb|F4(~x, ~y, ~z, ~b)| &in& \verb|sparc.h|     \\
    x              &=& 0x2                        &in& \verb|OP(x) | \\
    y              &=& 0x06                       &in& \verb|OP3(y) | \\
    z              &=& 0x0                        &in& \verb|F3I(z) | \\
    b              &=& 0x1                        &in& \verb|OP_AJIT_BIT_5(a) |
  \end{tabular}

\item \textbf{ORDNCC}:\\
  \begin{center}
    \begin{figure}[h]
      \centering
      \epsfxsize=.8\linewidth
      \epsffile{../figs/ordncc-ajit-insn-32-bit-layout.eps}
      \caption{The AJIT ORDNCC instruction  with register operands.}
      \label{fig:ajit:ordncc:insn}
    \end{figure}
  \end{center}
  \begin{itemize}
  \item []\textbf{ORDNCC}: same as ORN, but with Instr[13]=0 (i=0), and
    Instr[5]=1.
  \item []\textbf{Syntax}: ``\texttt{ordncc  SrcReg1, SrcReg2, DestReg}''.
  \item []\textbf{Semantics}: rd(pair) $\leftarrow$ rs1(pair) $\vert$
    ($\sim$rs2(pair)), sets Z.
  \end{itemize}

  Hence the SPARC bit layout of this instruction is:

  \begin{tabular}[h]{lclcl}
    Macro to set   &=&  \verb|F4(x, y, z, b)|     &in& \verb|sparc.h|     \\
    Macro to reset &=&  \verb|F4(~x, ~y, ~z, ~b)| &in& \verb|sparc.h|     \\
    x              &=& 0x2                        &in& \verb|OP(x) | \\
    y              &=& 0x16                       &in& \verb|OP3(y) | \\
    z              &=& 0x0                        &in& \verb|F3I(z) | \\
    b              &=& 0x1                        &in& \verb|OP_AJIT_BIT_5(a) |
  \end{tabular}

\item \textbf{XORDCC}:\\
  \begin{center}
    \begin{figure}[h]
      \centering
      \epsfxsize=.8\linewidth
      \epsffile{../figs/xordcc-ajit-insn-32-bit-layout.eps}
      \caption{The AJIT XORDCC instruction  with register operands.}
      \label{fig:ajit:xordcc:insn}
    \end{figure}
  \end{center}
  \begin{itemize}
  \item []\textbf{XORDCC}: same as XORCC, but with Instr[13]=0 (i=0), and
    Instr[5]=1.
  \item []\textbf{Syntax}: ``\texttt{xordcc  SrcReg1, SrcReg2, DestReg}''.
  \item []\textbf{Semantics}: rd(pair) $\leftarrow$ rs1(pair) $\hat{~}$
    rs2(pair), sets Z.
  \end{itemize}

  Hence the SPARC bit layout of this instruction is:

  \begin{tabular}[h]{lclcl}
    Macro to set   &=&  \verb|F4(x, y, z, b)|     &in& \verb|sparc.h|     \\
    Macro to reset &=&  \verb|F4(~x, ~y, ~z, ~b)| &in& \verb|sparc.h|     \\
    x              &=& 0x2                        &in& \verb|OP(x) | \\
    y              &=& 0x13                       &in& \verb|OP3(y) | \\
    z              &=& 0x0                        &in& \verb|F3I(z) | \\
    b              &=& 0x1                        &in& \verb|OP_AJIT_BIT_5(a) |
  \end{tabular}

  The AJIT bit  (insn[5]) is set internally by  \texttt{F4}, and hence
  there are only three arguments.

\item \textbf{XNORD}:\\
  \begin{center}
    \begin{figure}[h]
      \centering
      \epsfxsize=.8\linewidth
      \epsffile{../figs/xnord-ajit-insn-32-bit-layout.eps}
      \caption{The AJIT XNORD instruction  with register operands.}
      \label{fig:ajit:xnord:insn}
    \end{figure}
  \end{center}
  \begin{itemize}
  \item []\textbf{XNORD}: same as XNOR, but with Instr[13]=0 (i=0), and
    Instr[5]=1.
  \item []\textbf{Syntax}: ``\texttt{xnordcc  SrcReg1, SrcReg2, DestReg}''.
  \item []\textbf{Semantics}: rd(pair) $\leftarrow$ rs1(pair) $\hat{~}$
    rs2(pair).
  \end{itemize}

  Hence the SPARC bit layout of this instruction is:

  \begin{tabular}[h]{lclcl}
    Macro to set   &=&  \verb|F4(x, y, z, b)|     &in& \verb|sparc.h|     \\
    Macro to reset &=&  \verb|F4(~x, ~y, ~z, ~b)| &in& \verb|sparc.h|     \\
    x              &=& 0x2                        &in& \verb|OP(x) | \\
    y              &=& 0x07                       &in& \verb|OP3(y) | \\
    z              &=& 0x0                        &in& \verb|F3I(z) | \\
    b              &=& 0x1                        &in& \verb|OP_AJIT_BIT_5(a) |
  \end{tabular}

\item \textbf{XNORDCC}:\\
  \begin{center}
    \begin{figure}[h]
      \centering
      \epsfxsize=.8\linewidth
      \epsffile{../figs/xnordcc-ajit-insn-32-bit-layout.eps}
      \caption{The AJIT XNORDCC instruction  with register operands.}
      \label{fig:ajit:xnordcc:insn}
    \end{figure}
  \end{center}
  \begin{itemize}
  \item []\textbf{XNORDCC}: same as XNORD, but with Instr[13]=0 (i=0), and
    Instr[5]=1.
  \item []\textbf{Syntax}: ``\texttt{xnordcc  SrcReg1, SrcReg2, DestReg}''.
  \item []\textbf{Semantics}: rd(pair) $\leftarrow$ rs1(pair) $\hat{~}$
    rs2(pair), sets Z.
  \end{itemize}

  Hence the SPARC bit layout of this instruction is:

  \begin{tabular}[h]{lclcl}
    Macro to set   &=&  \verb|F4(x, y, z, b)|     &in& \verb|sparc.h|     \\
    Macro to reset &=&  \verb|F4(~x, ~y, ~z, ~b)| &in& \verb|sparc.h|     \\
    x              &=& 0x2                        &in& \verb|OP(x) | \\
    y              &=& 0x17                       &in& \verb|OP3(y) | \\
    z              &=& 0x0                        &in& \verb|F3I(z) | \\
    b              &=& 0x1                        &in& \verb|OP_AJIT_BIT_5(a) |
  \end{tabular}

\item \textbf{ANDD}:\\
  \begin{center}
    \begin{figure}[h]
      \centering
      \epsfxsize=.8\linewidth
      \epsffile{../figs/andd-ajit-insn-32-bit-layout.eps}
      \caption{The AJIT ANDD instruction  with register operands.}
      \label{fig:ajit:andd:insn}
    \end{figure}
  \end{center}
  \begin{itemize}
  \item []\textbf{ANDD}: same as AND, but with Instr[13]=0 (i=0), and
    Instr[5]=1.
  \item []\textbf{Syntax}: ``\texttt{andd  SrcReg1, SrcReg2, DestReg}''.
  \item []\textbf{Semantics}: rd(pair) $\leftarrow$ rs1(pair) $\cdot$ rs2(pair).
  \end{itemize}

  Hence the SPARC bit layout of this instruction is:

  \begin{tabular}[h]{lclcl}
    Macro to set   &=&  \verb|F4(x, y, z, b)|     &in& \verb|sparc.h|     \\
    Macro to reset &=&  \verb|F4(~x, ~y, ~z, ~b)| &in& \verb|sparc.h|     \\
    x              &=& 0x2                        &in& \verb|OP(x) | \\
    y              &=& 0x01                       &in& \verb|OP3(y) | \\
    z              &=& 0x0                        &in& \verb|F3I(z) | \\
    b              &=& 0x1                        &in& \verb|OP_AJIT_BIT_5(a) |
  \end{tabular}

\item \textbf{ANDDCC}:\\
  \begin{center}
    \begin{figure}[h]
      \centering
      \epsfxsize=.8\linewidth
      \epsffile{../figs/anddcc-ajit-insn-32-bit-layout.eps}
      \caption{The AJIT ANDDCC instruction  with register operands.}
      \label{fig:ajit:anddcc:insn}
    \end{figure}
  \end{center}
  \begin{itemize}
  \item []\textbf{ANDDCC}: same as ANDCC, but with Instr[13]=0 (i=0), and
    Instr[5]=1.
  \item []\textbf{Syntax}: ``\texttt{anddcc  SrcReg1, SrcReg2, DestReg}''.
  \item []\textbf{Semantics}: rd(pair) $\leftarrow$ rs1(pair) $\cdot$
    rs2(pair), sets Z.
  \end{itemize}

  Hence the SPARC bit layout of this instruction is:

  \begin{tabular}[h]{lclcl}
    Macro to set   &=&  \verb|F4(x, y, z, b)|     &in& \verb|sparc.h|     \\
    Macro to reset &=&  \verb|F4(~x, ~y, ~z, ~b)| &in& \verb|sparc.h|     \\
    x              &=& 0x2                        &in& \verb|OP(x) | \\
    y              &=& 0x11                       &in& \verb|OP3(y) | \\
    z              &=& 0x0                        &in& \verb|F3I(z) | \\
    b              &=& 0x1                        &in& \verb|OP_AJIT_BIT_5(a) |
  \end{tabular}

\item \textbf{ANDDN}:\\
  \begin{center}
    \begin{figure}[h]
      \centering
      \epsfxsize=.8\linewidth
      \epsffile{../figs/anddn-ajit-insn-32-bit-layout.eps}
      \caption{The AJIT ANDDN instruction  with register operands.}
      \label{fig:ajit:anddn:insn}
    \end{figure}
  \end{center}
  \begin{itemize}
  \item []\textbf{ANDDN}: same as ANDN, but with Instr[13]=0 (i=0), and
    Instr[5]=1.
  \item []\textbf{Syntax}: ``\texttt{anddn  SrcReg1, SrcReg2, DestReg}''.
  \item []\textbf{Semantics}: rd(pair) $\leftarrow$ rs1(pair) $\cdot$ ($\sim$rs2(pair)).
  \end{itemize}

  Hence the SPARC bit layout of this instruction is:

  \begin{tabular}[h]{lclcl}
    Macro to set   &=&  \verb|F4(x, y, z, b)|     &in& \verb|sparc.h|     \\
    Macro to reset &=&  \verb|F4(~x, ~y, ~z, ~b)| &in& \verb|sparc.h|     \\
    x              &=& 0x2                        &in& \verb|OP(x) | \\
    y              &=& 0x05                       &in& \verb|OP3(y) | \\
    z              &=& 0x0                        &in& \verb|F3I(z) | \\
    b              &=& 0x1                        &in& \verb|OP_AJIT_BIT_5(a) |
  \end{tabular}

\item \textbf{ANDDNCC}:\\
  \begin{center}
    \begin{figure}[h]
      \centering
      \epsfxsize=.8\linewidth
      \epsffile{../figs/anddncc-ajit-insn-32-bit-layout.eps}
      \caption{The AJIT ANDDNCC instruction  with register operands.}
      \label{fig:ajit:anddncc:insn}
    \end{figure}
  \end{center}
  \begin{itemize}
  \item []\textbf{ANDDNCC}: same as ANDN, but with Instr[13]=0 (i=0), and
    Instr[5]=1.
  \item []\textbf{Syntax}: ``\texttt{anddncc  SrcReg1, SrcReg2, DestReg}''.
  \item []\textbf{Semantics}: rd(pair) $\leftarrow$ rs1(pair) $\cdot$
    ($\sim$rs2(pair)), sets Z.
  \end{itemize}

  Hence the SPARC bit layout of this instruction is:

  \begin{tabular}[h]{lclcl}
    Macro to set   &=&  \verb|F4(x, y, z, b)|     &in& \verb|sparc.h|     \\
    Macro to reset &=&  \verb|F4(~x, ~y, ~z, ~b)| &in& \verb|sparc.h|     \\
    x              &=& 0x2                        &in& \verb|OP(x) | \\
    y              &=& 0x15                       &in& \verb|OP3(y) | \\
    z              &=& 0x0                        &in& \verb|F3I(z) | \\
    b              &=& 0x1                        &in& \verb|OP_AJIT_BIT_5(a) |
  \end{tabular}

  The AJIT bit  (insn[5]) is set internally by  \texttt{F4}, and hence
  there are only three arguments.

\end{enumerate}

%%% Local Variables:
%%% mode: latex
%%% TeX-master: t
%%% End:

\item {64 Bit Logical Instructions:}\\

  No immediate mode, i.e. insn[5] $\equiv$ i = 0, always.

  \begin{enumerate}
  \item \textbf{ORD}:\\
    \begin{tabular}[h]{lclcl}
      Macro to set  &=& \texttt{F4(x, y, z)} &in& \texttt{sparc.h}     \\
      Macro to reset  &=& \texttt{INVF4(x, y, z)} &in& \texttt{sparc.h}     \\
      x &=& 0x2      &in& \texttt{OP(x)  /* ((x) \& 0x3)  $<<$ 30 */} \\
      y &=& 0x02     &in& \texttt{OP3(y) /* ((y) \& 0x3f) $<<$ 19 */} \\
      z &=& 0x0      &in& \texttt{F3I(z) /* ((z) \& 0x1)  $<<$ 13 */} \\
      a &=& 0x1      &in& \texttt{OP\_AJIT\_BIT(a) /* ((a) \& 0x1)  $<<$ 5 */}
    \end{tabular}

    The AJIT bit  (insn[5]) is set internally by  \texttt{F4}, and hence
    there are only three arguments.

  \item \textbf{ORDCC}:\\
    \begin{tabular}[h]{lclcl}
      Macro to set  &=& \texttt{F4(x, y, z)} &in& \texttt{sparc.h}     \\
      Macro to reset  &=& \texttt{INVF4(x, y, z)} &in& \texttt{sparc.h}     \\
      x &=& 0x2      &in& \texttt{OP(x)  /* ((x) \& 0x3)  $<<$ 30 */} \\
      y &=& 0x12     &in& \texttt{OP3(y) /* ((y) \& 0x3f) $<<$ 19 */} \\
      z &=& 0x0      &in& \texttt{F3I(z) /* ((z) \& 0x1)  $<<$ 13 */} \\
      a &=& 0x1      &in& \texttt{OP\_AJIT\_BIT(a) /* ((a) \& 0x1)  $<<$ 5 */}
    \end{tabular}

    The AJIT bit  (insn[5]) is set internally by  \texttt{F4}, and hence
    there are only three arguments.

  \item \textbf{ORDN}:\\
    \begin{tabular}[h]{lclcl}
      Macro to set  &=& \texttt{F4(x, y, z)} &in& \texttt{sparc.h}     \\
      Macro to reset  &=& \texttt{INVF4(x, y, z)} &in& \texttt{sparc.h}     \\
      x &=& 0x2      &in& \texttt{OP(x)  /* ((x) \& 0x3)  $<<$ 30 */} \\
      y &=& 0x06     &in& \texttt{OP3(y) /* ((y) \& 0x3f) $<<$ 19 */} \\
      z &=& 0x0      &in& \texttt{F3I(z) /* ((z) \& 0x1)  $<<$ 13 */} \\
      a &=& 0x1      &in& \texttt{OP\_AJIT\_BIT(a) /* ((a) \& 0x1)  $<<$ 5 */}
    \end{tabular}

    The AJIT bit  (insn[5]) is set internally by  \texttt{F4}, and hence
    there are only three arguments.

  \item \textbf{ORDNCC}:\\
    \begin{tabular}[h]{lclcl}
      Macro to set  &=& \texttt{F4(x, y, z)} &in& \texttt{sparc.h}     \\
      Macro to reset  &=& \texttt{INVF4(x, y, z)} &in& \texttt{sparc.h}     \\
      x &=& 0x2      &in& \texttt{OP(x)  /* ((x) \& 0x3)  $<<$ 30 */} \\
      y &=& 0x16     &in& \texttt{OP3(y) /* ((y) \& 0x3f) $<<$ 19 */} \\
      z &=& 0x0      &in& \texttt{F3I(z) /* ((z) \& 0x1)  $<<$ 13 */} \\
      a &=& 0x1      &in& \texttt{OP\_AJIT\_BIT(a) /* ((a) \& 0x1)  $<<$ 5 */}
    \end{tabular}

    The AJIT bit  (insn[5]) is set internally by  \texttt{F4}, and hence
    there are only three arguments.

  \item \textbf{XORDCC}:\\
    \begin{tabular}[h]{lclcl}
      Macro to set  &=& \texttt{F4(x, y, z)} &in& \texttt{sparc.h}     \\
      Macro to reset  &=& \texttt{INVF4(x, y, z)} &in& \texttt{sparc.h}     \\
      x &=& 0x2      &in& \texttt{OP(x)  /* ((x) \& 0x3)  $<<$ 30 */} \\
      y &=& 0x13     &in& \texttt{OP3(y) /* ((y) \& 0x3f) $<<$ 19 */} \\
      z &=& 0x0      &in& \texttt{F3I(z) /* ((z) \& 0x1)  $<<$ 13 */} \\
      a &=& 0x1      &in& \texttt{OP\_AJIT\_BIT(a) /* ((a) \& 0x1)  $<<$ 5 */}
    \end{tabular}

    The AJIT bit  (insn[5]) is set internally by  \texttt{F4}, and hence
    there are only three arguments.

  \item \textbf{XNORD}:\\
    \begin{tabular}[h]{lclcl}
      Macro to set  &=& \texttt{F4(x, y, z)} &in& \texttt{sparc.h}     \\
      Macro to reset  &=& \texttt{INVF4(x, y, z)} &in& \texttt{sparc.h}     \\
      x &=& 0x2      &in& \texttt{OP(x)  /* ((x) \& 0x3)  $<<$ 30 */} \\
      y &=& 0x07     &in& \texttt{OP3(y) /* ((y) \& 0x3f) $<<$ 19 */} \\
      z &=& 0x0      &in& \texttt{F3I(z) /* ((z) \& 0x1)  $<<$ 13 */} \\
      a &=& 0x1      &in& \texttt{OP\_AJIT\_BIT(a) /* ((a) \& 0x1)  $<<$ 5 */}
    \end{tabular}

    The AJIT bit  (insn[5]) is set internally by  \texttt{F4}, and hence
    there are only three arguments.
    
  \item \textbf{XNORDCC}:\\
    \begin{tabular}[h]{lclcl}
      Macro to set  &=& \texttt{F4(x, y, z)} &in& \texttt{sparc.h}     \\
      Macro to reset  &=& \texttt{INVF4(x, y, z)} &in& \texttt{sparc.h}     \\
      x &=& 0x2      &in& \texttt{OP(x)  /* ((x) \& 0x3)  $<<$ 30 */} \\
      y &=& 0x07     &in& \texttt{OP3(y) /* ((y) \& 0x3f) $<<$ 19 */} \\
      z &=& 0x0      &in& \texttt{F3I(z) /* ((z) \& 0x1)  $<<$ 13 */} \\
      a &=& 0x1      &in& \texttt{OP\_AJIT\_BIT(a) /* ((a) \& 0x1)  $<<$ 5 */}
    \end{tabular}

    The AJIT bit  (insn[5]) is set internally by  \texttt{F4}, and hence
    there are only three arguments.
    
  \item \textbf{ANDD}:\\
    \begin{tabular}[h]{lclcl}
      Macro to set  &=& \texttt{F4(x, y, z)} &in& \texttt{sparc.h}     \\
      Macro to reset  &=& \texttt{INVF4(x, y, z)} &in& \texttt{sparc.h}     \\
      x &=& 0x2      &in& \texttt{OP(x)  /* ((x) \& 0x3)  $<<$ 30 */} \\
      y &=& 0x01     &in& \texttt{OP3(y) /* ((y) \& 0x3f) $<<$ 19 */} \\
      z &=& 0x0      &in& \texttt{F3I(z) /* ((z) \& 0x1)  $<<$ 13 */} \\
      a &=& 0x1      &in& \texttt{OP\_AJIT\_BIT(a) /* ((a) \& 0x1)  $<<$ 5 */}
    \end{tabular}

    The AJIT bit  (insn[5]) is set internally by  \texttt{F4}, and hence
    there are only three arguments.

  \item \textbf{ANDDCC}:\\
    \begin{tabular}[h]{lclcl}
      Macro to set  &=& \texttt{F4(x, y, z)} &in& \texttt{sparc.h}     \\
      Macro to reset  &=& \texttt{INVF4(x, y, z)} &in& \texttt{sparc.h}     \\
      x &=& 0x2      &in& \texttt{OP(x)  /* ((x) \& 0x3)  $<<$ 30 */} \\
      y &=& 0x11     &in& \texttt{OP3(y) /* ((y) \& 0x3f) $<<$ 19 */} \\
      z &=& 0x0      &in& \texttt{F3I(z) /* ((z) \& 0x1)  $<<$ 13 */} \\
      a &=& 0x1      &in& \texttt{OP\_AJIT\_BIT(a) /* ((a) \& 0x1)  $<<$ 5 */}
    \end{tabular}

    The AJIT bit  (insn[5]) is set internally by  \texttt{F4}, and hence
    there are only three arguments.

  \item \textbf{ANDDN}:\\
    \begin{tabular}[h]{lclcl}
      Macro to set  &=& \texttt{F4(x, y, z)} &in& \texttt{sparc.h}     \\
      Macro to reset  &=& \texttt{INVF4(x, y, z)} &in& \texttt{sparc.h}     \\
      x &=& 0x2      &in& \texttt{OP(x)  /* ((x) \& 0x3)  $<<$ 30 */} \\
      y &=& 0x05     &in& \texttt{OP3(y) /* ((y) \& 0x3f) $<<$ 19 */} \\
      z &=& 0x0      &in& \texttt{F3I(z) /* ((z) \& 0x1)  $<<$ 13 */} \\
      a &=& 0x1      &in& \texttt{OP\_AJIT\_BIT(a) /* ((a) \& 0x1)  $<<$ 5 */}
    \end{tabular}

    The AJIT bit  (insn[5]) is set internally by  \texttt{F4}, and hence
    there are only three arguments.

  \item \textbf{ANDDNCC}:\\
    \begin{tabular}[h]{lclcl}
      Macro to set  &=& \texttt{F4(x, y, z)} &in& \texttt{sparc.h}     \\
      Macro to reset  &=& \texttt{INVF4(x, y, z)} &in& \texttt{sparc.h}     \\
      x &=& 0x2      &in& \texttt{OP(x)  /* ((x) \& 0x3)  $<<$ 30 */} \\
      y &=& 0x15     &in& \texttt{OP3(y) /* ((y) \& 0x3f) $<<$ 19 */} \\
      z &=& 0x0      &in& \texttt{F3I(z) /* ((z) \& 0x1)  $<<$ 13 */} \\
      a &=& 0x1      &in& \texttt{OP\_AJIT\_BIT(a) /* ((a) \& 0x1)  $<<$ 5 */}
    \end{tabular}

    The AJIT bit  (insn[5]) is set internally by  \texttt{F4}, and hence
    there are only three arguments.

  \end{enumerate}
% \subsubsection{Shift instructions:}
\label{sec:shift:insn:impl}
The shift  family of instructions  of AJIT  may each be  considered to
have  two versions:  a direct  count version  and a  register indirect
count version.  In the direct count  version the shift count is a part
of the  instruction bits.   In the indirect  count version,  the shift
count is  found on the  register specified by  the bit pattern  in the
instruction  bits.   The direct  count  version  is specified  by  the
14$^{th}$  bit, i.e.  insn[13]  (bit  number 13  in  the  0 based  bit
numbering scheme), being set to 1.  If insn[13] is 0 then the register
indirect version is specified.

Similar to the addition and subtraction instructions, the shift family
of instructions of  SPARC V8 also do  not use bits from 5  to 12 (both
inclusive).  The AJIT processor uses bits  5 and 6.  In particular bit
6 is always 1.   Bit 5 may be used in the direct  version giving a set
of 6 bits  available for specifying the shift count.   The shift count
can have  a maximum  value of  64.  Bit  5 is  unused in  the register
indirect version, and is always 0 in that case.

These instructions  are therefore  worked out  below in  two different
sets: the direct and the register indirect ones.
\begin{enumerate}
\item The direct versions  are given by insn[13] = 1.  The 6 bit shift
  count  is directly  specified  in the  instruction bits.   Therefore
  insn[5:0] specify the  shift count.  insn[6] =  1, distinguishes the
  AJIT version from the SPARC V8 version.
  \begin{enumerate}
  \item \textbf{SLLD}:\\
    \begin{center}
      \begin{tabular}[p]{|c|c|l|p{.25\textwidth}|p{.3\textwidth}|}
        \hline
        \textbf{Start} & \textbf{End} & \textbf{Range} & \textbf{Meaning} & \textbf{New Meaning}\\
        \hline
        0 & 4 & 32 & Source register 2, rs2 & Lowest 5 bits of shift count \\
        \hline
        5 & 12 & -- & \textbf{Unused. Set to 0 by software.} &
                                        \begin{minipage}[h]{1.0\linewidth}
                                          \begin{itemize}
                                          \item \textbf{Use bit 5
                                              to specify the msb of
                                              shift count.}
                                          \item \textbf{Use bit 6 to
                                              distinguish AJIT from
                                              SPARC V8.}
                                          \item \textbf{Set bits 7:12
                                              to 0.}
                                          \end{itemize}
                                        \end{minipage}
        \\
        \hline
        13 & 13 & 0,1 & The \textbf{i} bit & \textbf{Set i to ``1''} \\
        14 & 18 & 32 & Source register 1, rs1 & No change \\
        19 & 24 & 100101 & ``\textbf{op3}'' & No change \\
        25 & 29 & 32 & Destination register, rd & No change \\
        30 & 31 & 4 & Always ``10'' & No change \\
        \hline
      \end{tabular}
    \end{center}
    \begin{itemize}
    \item []\textbf{SLLD}: same as SLL, but with Instr[13]=0 (i=0),
      and Instr[5]=1.
    \item []\textbf{Syntax}: ``\texttt{slld SrcReg1, 6BitShiftCnt,
        DestReg}''. \\
      (\textbf{Note:} In an assembly language program, when the second
      argument is a number, we have direct mode.  A register number is
      prefixed with  ``r'', and hence the  syntax itself distinguished
      between   direct  and   register   indirect   version  of   this
      instruction.)
    \item []\textbf{Semantics}: rd(pair) $\leftarrow$ rs1(pair) $<<$
      shift count.
    \end{itemize}
    Bits layout:
\begin{verbatim}
    Offsets      : 31       24 23       16  15        8   7        0
    Bit layout   :  XXXX  XXXX  XXXX  XXXX   XXXX  XXXX   XXXX  XXXX
    Insn Bits    :  10       1  0010  1        1           1        
    Destination  :    DD  DDD                                       
    Source 1     :                     SSS   SS
    Source 2     :                                           S  SSSS
    Unused (0)   :                              U  UUUU   UU        
    Final layout :  10DD  DDD1  0010  1SSS   SS1U  UUUU   U1II  IIII
\end{verbatim}

    This will need another macro that sets bits 5 and 6. Let's call it
    \texttt{OP\_AJIT\_BITS\_5\_AND\_6}.   Hence the  SPARC bit  layout of  this
    instruction is:

    \begin{tabular}[h]{lclcl}
      Macro to set  &=& \texttt{F5(x, y, z)} &in& \texttt{sparc.h}     \\
      Macro to reset  &=& \texttt{INVF5(x, y, z)} &in& \texttt{sparc.h}     \\
      x &=& 0x2      &in& \texttt{OP(x)  /* ((x) \& 0x3)  $<<$ 30 */} \\
      y &=& 0x25     &in& \texttt{OP3(y) /* ((y) \& 0x3f) $<<$ 19 */} \\
      z &=& 0x1      &in& \texttt{F3I(z) /* ((z) \& 0x1)  $<<$ 13 */} \\
      a &=& 0x2      &in& \texttt{OP\_AJIT\_BITS\_5\_AND\_6(a) /* ((a) \& 0x3  $<<$ 6 */}
    \end{tabular}

    The AJIT bits (insn[6:5]) is  set or reset internally by \texttt{F5}
    (just  like  in  \texttt{F4}),  and   hence  there  are  only  three
    arguments.

  \item \textbf{SRLD}:\\
    \begin{center}
      \begin{tabular}[p]{|c|c|l|l|p{.35\textwidth}|}
        \hline
        \textbf{Start} & \textbf{End} & \textbf{Range} & \textbf{Meaning} & \textbf{New Meaning}\\
        \hline
        0 & 4 & 32 & Source register 2, rs2 & Lowest 5 bits of shift count \\
        \hline
        5 & 12 & -- & \textbf{unused} &
                                        \begin{minipage}[h]{1.0\linewidth}
                                          \begin{itemize}
                                          \item \textbf{Use bit 5
                                              to specify the msb of
                                              shift count.}
                                          \item \textbf{Use bit 6 to
                                              distinguish AJIT from
                                              SPARC V8.}
                                          \end{itemize}
                                        \end{minipage}
        \\
        \hline
        13 & 13 & 0,1 & The \textbf{i} bit & \textbf{Set i to ``1''} \\
        14 & 18 & 32 & Source register 1, rs1 & No change \\
        19 & 24 & 100110 & ``\textbf{op3}'' & No change \\
        25 & 29 & 32 & Destination register, rd & No change \\
        30 & 31 & 4 & Always ``10'' & No change \\
        \hline
      \end{tabular}
    \end{center}
    \begin{itemize}
    \item []\textbf{SRLD}: same as SRL, but with Instr[13]=0 (i=0),
      and Instr[5]=1.
    \item []\textbf{Syntax}: ``\texttt{sral SrcReg1, 6BitShiftCnt,
        DestReg}''. \\
      (\textbf{Note:} In an assembly language program, when the second
      argument is a number, we have direct mode.  A register number is
      prefixed with  ``r'', and hence the  syntax itself distinguished
      between   direct  and   register   indirect   version  of   this
      instruction.)
    \item []\textbf{Semantics}: rd(pair) $\leftarrow$ rs1(pair) $>>$
      shift count.
    \end{itemize}
    Bits layout:
\begin{verbatim}
    Offsets      : 31       24 23       16  15        8   7        0
    Bit layout   :  XXXX  XXXX  XXXX  XXXX   XXXX  XXXX   XXXX  XXXX
    Insn Bits    :  10       1  0011  0        1           1        
    Destination  :    DD  DDD                                       
    Source 1     :                     SSS   SS
    Source 2     :                                           S  SSSS
    Unused (0)   :                              U  UUUU   UU        
    Final layout :  10DD  DDD1  0011  0SSS   SS1U  UUUU   U1II  IIII
\end{verbatim}

    This will need another macro that sets bits 5 and 6. Let's call it
    \texttt{OP\_AJIT\_BITS\_5\_AND\_6}.   Hence the  SPARC bit  layout of  this
    instruction is:

    \begin{tabular}[h]{lclcl}
      Macro to set  &=& \texttt{F5(x, y, z)} &in& \texttt{sparc.h}     \\
      Macro to reset  &=& \texttt{INVF5(x, y, z)} &in& \texttt{sparc.h}     \\
      x &=& 0x2      &in& \texttt{OP(x)  /* ((x) \& 0x3)  $<<$ 30 */} \\
      y &=& 0x26     &in& \texttt{OP3(y) /* ((y) \& 0x3f) $<<$ 19 */} \\
      z &=& 0x1      &in& \texttt{F3I(z) /* ((z) \& 0x1)  $<<$ 13 */} \\
      a &=& 0x2      &in& \texttt{OP\_AJIT\_BITS\_5\_AND\_6(a) /* ((a) \& 0x3  $<<$ 6 */}
    \end{tabular}

    The AJIT bits (insn[6:5]) is  set or reset internally by \texttt{F5}
    (just  like  in  \texttt{F4}),  and   hence  there  are  only  three
    arguments.
    
  \item \textbf{SRAD}:\\
    \begin{center}
      \begin{tabular}[p]{|c|c|l|l|p{.35\textwidth}|}
        \hline
        \textbf{Start} & \textbf{End} & \textbf{Range} & \textbf{Meaning} & \textbf{New Meaning}\\
        \hline
        0 & 4 & 32 & Source register 2, rs2 & Lowest 5 bits of shift count \\
        \hline
        5 & 12 & -- & \textbf{unused} &
                                        \begin{minipage}[h]{1.0\linewidth}
                                          \begin{itemize}
                                          \item \textbf{Use bit 5
                                              to specify the msb of
                                              shift count.}
                                          \item \textbf{Use bit 6 to
                                              distinguish AJIT from
                                              SPARC V8.}
                                          \end{itemize}
                                        \end{minipage}
        \\
        \hline
        13 & 13 & 0,1 & The \textbf{i} bit & \textbf{Set i to ``1''} \\
        14 & 18 & 32 & Source register 1, rs1 & No change \\
        19 & 24 & 100111 & ``\textbf{op3}'' & No change \\
        25 & 29 & 32 & Destination register, rd & No change \\
        30 & 31 & 4 & Always ``10'' & No change \\
        \hline
      \end{tabular}
    \end{center}
    \begin{itemize}
    \item []\textbf{SRAD}: same as SRA, but with Instr[13]=0 (i=0),
      and Instr[5]=1.
    \item []\textbf{Syntax}: ``\texttt{srad SrcReg1, 6BitShiftCnt,
        DestReg}''. \\
      (\textbf{Note:} In an assembly language program, when the second
      argument is a number, we have direct mode.  A register number is
      prefixed with  ``r'', and hence the  syntax itself distinguished
      between   direct  and   register   indirect   version  of   this
      instruction.)
    \item []\textbf{Semantics}: rd(pair) $\leftarrow$ rs1(pair) $>>$
      shift count (with sign extension).
    \end{itemize}
    Bits layout:
\begin{verbatim}
    Offsets      : 31       24 23       16  15        8   7        0
    Bit layout   :  XXXX  XXXX  XXXX  XXXX   XXXX  XXXX   XXXX  XXXX
    Insn Bits    :  10       1  0011  1        1           1        
    Destination  :    DD  DDD                                       
    Source 1     :                     SSS   SS
    Source 2     :                                           S  SSSS
    Unused (0)   :                              U  UUUU   UU        
    Final layout :  10DD  DDD1  0011  1SSS   SS1U  UUUU   U1II  IIII
\end{verbatim}

    This will need another macro that sets bits 5 and 6. Let's call it
    \texttt{OP\_AJIT\_BITS\_5\_AND\_6}.   Hence the  SPARC bit  layout of  this
    instruction is:

    \begin{tabular}[h]{lclcl}
      Macro to set  &=& \texttt{F5(x, y, z)} &in& \texttt{sparc.h}     \\
      Macro to reset  &=& \texttt{INVF5(x, y, z)} &in& \texttt{sparc.h}     \\
      x &=& 0x2      &in& \texttt{OP(x)  /* ((x) \& 0x3)  $<<$ 30 */} \\
      y &=& 0x27     &in& \texttt{OP3(y) /* ((y) \& 0x3f) $<<$ 19 */} \\
      z &=& 0x1      &in& \texttt{F3I(z) /* ((z) \& 0x1)  $<<$ 13 */} \\
      a &=& 0x2      &in& \texttt{OP\_AJIT\_BITS\_5\_AND\_6(a) /* ((a) \& 0x3  $<<$ 6 */}
    \end{tabular}

    The AJIT bits (insn[6:5]) is  set or reset internally by \texttt{F5}
    (just  like  in  \texttt{F4}),  and   hence  there  are  only  three
    arguments.

  \end{enumerate}
\item The register  indirect versions are given by insn[13]  = 0.  The
  shift count is indirectly specified in the 32 bit register specified
  in instruction bits.  Therefore  insn[4:0] specify the register that
  has the  shift count.  insn[6]  = 1, distinguishes the  AJIT version
  from the SPARC V8 version.  In this case, insn[5] = 0, necessarily.
  \begin{enumerate}
  \item \textbf{SLLD}:\\
    \begin{center}
      \begin{tabular}[p]{|c|c|l|l|p{.35\textwidth}|}
        \hline
        \textbf{Start} & \textbf{End} & \textbf{Range} & \textbf{Meaning} &
                                                                            \textbf{New Meaning}\\
        \hline
        0 & 4 & 32 & Source register 2, rs2 & Register number \\
        \hline
        5 & 12 & -- & \textbf{unused} &
                                        \begin{minipage}[h]{1.0\linewidth}
                                          \begin{itemize}
                                          \item \textbf{Set bit 5 to 0.}
                                          \item \textbf{Use bit 6 to
                                              distinguish AJIT from
                                              SPARC V8.}
                                          \end{itemize}
                                        \end{minipage}
        \\
        \hline
        13 & 13 & 0,1 & The \textbf{i} bit & \textbf{Set i to ``0''} \\
        14 & 18 & 32 & Source register 1, rs1 & No change \\
        19 & 24 & 100101 & ``\textbf{op3}'' & No change \\
        25 & 29 & 32 & Destination register, rd & No change \\
        30 & 31 & 4 & Always ``10'' & No change \\
        \hline
      \end{tabular}
    \end{center}
    \begin{itemize}
    \item []\textbf{SLLD}: same as SLL, but with Instr[13]=0 (i=0),
      and Instr[5]=1.
    \item []\textbf{Syntax}: ``\texttt{slld SrcReg1, SrcReg2,
        DestReg}''.
    \item []\textbf{Semantics}: rd(pair) $\leftarrow$ rs1(pair) $<<$
      shift count register rs2.
    \end{itemize}
    Bits layout:
\begin{verbatim}
    Offsets      : 31       24 23       16  15        8   7        0
    Bit layout   :  XXXX  XXXX  XXXX  XXXX   XXXX  XXXX   XXXX  XXXX
    Insn Bits    :  10       1  0010  1        0           10        
    Destination  :    DD  DDD                                       
    Source 1     :                     SSS   SS
    Source 2     :                                           S  SSSS
    Unused (0)   :                              U  UUUU   UU        
    Final layout :  10DD  DDD1  0010  1SSS   SS0U  UUUU   U10I  IIII
\end{verbatim}

    This will need another macro that sets bits 5 and 6. Let's call it
    \texttt{OP\_AJIT\_BITS\_5\_AND\_6}.   Hence the  SPARC bit  layout of  this
    instruction is:

    \begin{tabular}[h]{lclcl}
      Macro to set  &=& \texttt{F5(x, y, z)} &in& \texttt{sparc.h}     \\
      Macro to reset  &=& \texttt{INVF5(x, y, z)} &in& \texttt{sparc.h}     \\
      x &=& 0x2      &in& \texttt{OP(x)  /* ((x) \& 0x3)  $<<$ 30 */} \\
      y &=& 0x25     &in& \texttt{OP3(y) /* ((y) \& 0x3f) $<<$ 19 */} \\
      z &=& 0x0      &in& \texttt{F3I(z) /* ((z) \& 0x1)  $<<$ 13 */} \\
      a &=& 0x2      &in& \texttt{OP\_AJIT\_BITS\_5\_AND\_6(a) /* ((a) \& 0x3  $<<$ 6 */}
    \end{tabular}

    The AJIT bits (insn[6:5]) is  set or reset internally by \texttt{F5}
    (just  like  in  \texttt{F4}),  and   hence  there  are  only  three
    arguments.

  \item \textbf{SRLD}:\\
    \begin{center}
      \begin{tabular}[p]{|c|c|l|l|p{.35\textwidth}|}
        \hline
        \textbf{Start} & \textbf{End} & \textbf{Range} & \textbf{Meaning} &
                                                                            \textbf{New Meaning}\\
        \hline
        0 & 4 & 32 & Source register 2, rs2 & Register number \\
        \hline
        5 & 12 & -- & \textbf{unused} &
                                        \begin{minipage}[h]{1.0\linewidth}
                                          \begin{itemize}
                                          \item \textbf{Set bit 5 to 0.}
                                          \item \textbf{Use bit 6 to
                                              distinguish AJIT from
                                              SPARC V8.}
                                          \end{itemize}
                                        \end{minipage}
        \\
        \hline
        13 & 13 & 0,1 & The \textbf{i} bit & \textbf{Set i to ``0''} \\
        14 & 18 & 32 & Source register 1, rs1 & No change \\
        19 & 24 & 100110 & ``\textbf{op3}'' & No change \\
        25 & 29 & 32 & Destination register, rd & No change \\
        30 & 31 & 4 & Always ``10'' & No change \\
        \hline
      \end{tabular}
    \end{center}
    \begin{itemize}
    \item []\textbf{SRLD}: same as SRL, but with Instr[13]=0 (i=0),
      and Instr[5]=1.
    \item []\textbf{Syntax}: ``\texttt{slld SrcReg1, SrcReg2,
        DestReg}''.
    \item []\textbf{Semantics}: rd(pair) $\leftarrow$ rs1(pair) $>>$
      shift count register rs2.
    \end{itemize}
    Bits layout:
\begin{verbatim}
    Offsets      : 31       24 23       16  15        8   7        0
    Bit layout   :  XXXX  XXXX  XXXX  XXXX   XXXX  XXXX   XXXX  XXXX
    Insn Bits    :  10       1  0011  0        0           10        
    Destination  :    DD  DDD                                       
    Source 1     :                     SSS   SS
    Source 2     :                                           S  SSSS
    Unused (0)   :                              U  UUUU   UU        
    Final layout :  10DD  DDD1  0011  0SSS   SS0U  UUUU   U10I  IIII
\end{verbatim}

    This will need another macro that sets bits 5 and 6. Let's call it
    \texttt{OP\_AJIT\_BITS\_5\_AND\_6}.   Hence the  SPARC bit  layout of  this
    instruction is:

    \begin{tabular}[h]{lclcl}
      Macro to set  &=& \texttt{F5(x, y, z)} &in& \texttt{sparc.h}     \\
      Macro to reset  &=& \texttt{INVF5(x, y, z)} &in& \texttt{sparc.h}     \\
      x &=& 0x2      &in& \texttt{OP(x)  /* ((x) \& 0x3)  $<<$ 30 */} \\
      y &=& 0x26     &in& \texttt{OP3(y) /* ((y) \& 0x3f) $<<$ 19 */} \\
      z &=& 0x0      &in& \texttt{F3I(z) /* ((z) \& 0x1)  $<<$ 13 */} \\
      a &=& 0x2      &in& \texttt{OP\_AJIT\_BITS\_5\_AND\_6(a) /* ((a) \& 0x3  $<<$ 6 */}
    \end{tabular}

    The AJIT bits (insn[6:5]) is  set or reset internally by \texttt{F5}
    (just  like  in  \texttt{F4}),  and   hence  there  are  only  three
    arguments.

  \item \textbf{SRAD}:\\
    \begin{center}
      \begin{tabular}[p]{|c|c|l|l|p{.35\textwidth}|}
        \hline
        \textbf{Start} & \textbf{End} & \textbf{Range} & \textbf{Meaning} &
                                                                            \textbf{New Meaning}\\
        \hline
        0 & 4 & 32 & Source register 2, rs2 & Register number \\
        \hline
        5 & 12 & -- & \textbf{unused} &
                                        \begin{minipage}[h]{1.0\linewidth}
                                          \begin{itemize}
                                          \item \textbf{Set bit 5 to 0.}
                                          \item \textbf{Use bit 6 to
                                              distinguish AJIT from
                                              SPARC V8.}
                                          \end{itemize}
                                        \end{minipage}
        \\
        \hline
        13 & 13 & 0,1 & The \textbf{i} bit & \textbf{Set i to ``0''} \\
        14 & 18 & 32 & Source register 1, rs1 & No change \\
        19 & 24 & 100101 & ``\textbf{op3}'' & No change \\
        25 & 29 & 32 & Destination register, rd & No change \\
        30 & 31 & 4 & Always ``10'' & No change \\
        \hline
      \end{tabular}
    \end{center}
    \begin{itemize}
    \item []\textbf{SRAD}: same as SRA, but with Instr[13]=0 (i=0),
      and Instr[5]=1.
    \item []\textbf{Syntax}: ``\texttt{slld SrcReg1, SrcReg2,
        DestReg}''.
    \item []\textbf{Semantics}: rd(pair) $\leftarrow$ rs1(pair) $>>$
      shift count register rs2 (with sign extension).
    \end{itemize}
    Bits layout:
\begin{verbatim}
    Offsets      : 31       24 23       16  15        8   7        0
    Bit layout   :  XXXX  XXXX  XXXX  XXXX   XXXX  XXXX   XXXX  XXXX
    Insn Bits    :  10       1  0011  1        0           10        
    Destination  :    DD  DDD                                       
    Source 1     :                     SSS   SS
    Source 2     :                                           S  SSSS
    Unused (0)   :                              U  UUUU   UU        
    Final layout :  10DD  DDD1  0011  1SSS   SS0U  UUUU   U10I  IIII
\end{verbatim}

    This will need another macro that sets bits 5 and 6. Let's call it
    \texttt{OP\_AJIT\_BITS\_5\_AND\_6}.   Hence the  SPARC bit  layout of  this
    instruction is:

    \begin{tabular}[h]{lclcl}
      Macro to set  &=& \texttt{F5(x, y, z)} &in& \texttt{sparc.h}     \\
      Macro to reset  &=& \texttt{INVF5(x, y, z)} &in& \texttt{sparc.h}     \\
      x &=& 0x2      &in& \texttt{OP(x)  /* ((x) \& 0x3)  $<<$ 30 */} \\
      y &=& 0x27     &in& \texttt{OP3(y) /* ((y) \& 0x3f) $<<$ 19 */} \\
      z &=& 0x0      &in& \texttt{F3I(z) /* ((z) \& 0x1)  $<<$ 13 */} \\
      a &=& 0x2      &in& \texttt{OP\_AJIT\_BITS\_5\_AND\_6(a) /* ((a) \& 0x3  $<<$ 6 */}
    \end{tabular}

    The AJIT bits (insn[6:5]) is  set or reset internally by \texttt{F5}
    (just  like  in  \texttt{F4}),  and   hence  there  are  only  three
    arguments.
  \end{enumerate}
\end{enumerate}

\item {Shift instructions:} \\

  The shift  family of instructions  of AJIT  may each be  considered to
  have  two versions:  a direct  count version  and a  register indirect
  count version.  In the direct count  version the shift count is a part
  of the  instruction bits.   In the indirect  count version,  the shift
  count is  found on the  register specified by  the bit pattern  in the
  instruction  bits.   The direct  count  version  is specified  by  the
  14$^{th}$  bit, i.e.  insn[13]  (bit  number 13  in  the  0 based  bit
  numbering scheme), being set to 1.  If insn[13] is 0 then the register
  indirect version is specified.

  Similar to the addition and subtraction instructions, the shift family
  of instructions of  SPARC V8 also do  not use bits from 5  to 12 (both
  inclusive).  The AJIT processor uses bits  5 and 6.  In particular bit
  6 is always 1.   Bit 5 may be used in the direct  version giving a set
  of 6 bits  available for specifying the shift count.   The shift count
  can have  a maximum  value of  64.  Bit  5 is  unused in  the register
  indirect version, and is always 0 in that case.

  These instructions  are therefore  worked out  below in  two different
  sets: the direct and the register indirect ones.
  \begin{enumerate}
  \item The direct versions  are given by insn[13] = 1.  The 6 bit shift
    count  is directly  specified  in the  instruction bits.   Therefore
    insn[5:0] specify the  shift count.  insn[6] =  1, distinguishes the
    AJIT version from the SPARC V8 version.
    \begin{enumerate}
    \item \textbf{SLLD}:\\
      This will need another macro that sets bits 5 and 6. Let's call it
      \texttt{OP\_AJIT\_BIT\_2}.   Hence the  SPARC bit  layout of  this
      instruction is:

      \begin{tabular}[h]{lclcl}
        Macro to set  &=& \texttt{F5(x, y, z)} &in& \texttt{sparc.h}     \\
        Macro to reset  &=& \texttt{INVF5(x, y, z)} &in& \texttt{sparc.h}     \\
        x &=& 0x2      &in& \texttt{OP(x)  /* ((x) \& 0x3)  $<<$ 30 */} \\
        y &=& 0x25     &in& \texttt{OP3(y) /* ((y) \& 0x3f) $<<$ 19 */} \\
        z &=& 0x1      &in& \texttt{F3I(z) /* ((z) \& 0x1)  $<<$ 13 */} \\
        a &=& 0x2      &in& \texttt{OP\_AJIT\_BIT\_2(a) /* ((a) \& 0x3  $<<$ 6 */}
      \end{tabular}

      The AJIT bits (insn[6:5]) is  set or reset internally by \texttt{F5}
      (just  like  in  \texttt{F4}),  and   hence  there  are  only  three
      arguments.

    \item \textbf{SRLD}:\\
      This will need another macro that sets bits 5 and 6. Let's call it
      \texttt{OP\_AJIT\_BIT\_2}.   Hence the  SPARC bit  layout of  this
      instruction is:

      \begin{tabular}[h]{lclcl}
        Macro to set  &=& \texttt{F5(x, y, z)} &in& \texttt{sparc.h}     \\
        Macro to reset  &=& \texttt{INVF5(x, y, z)} &in& \texttt{sparc.h}     \\
        x &=& 0x2      &in& \texttt{OP(x)  /* ((x) \& 0x3)  $<<$ 30 */} \\
        y &=& 0x26     &in& \texttt{OP3(y) /* ((y) \& 0x3f) $<<$ 19 */} \\
        z &=& 0x1      &in& \texttt{F3I(z) /* ((z) \& 0x1)  $<<$ 13 */} \\
        a &=& 0x2      &in& \texttt{OP\_AJIT\_BIT\_2(a) /* ((a) \& 0x3  $<<$ 6 */}
      \end{tabular}

      The AJIT bits (insn[6:5]) is  set or reset internally by \texttt{F5}
      (just  like  in  \texttt{F4}),  and   hence  there  are  only  three
      arguments.
      
    \item \textbf{SRAD}:\\
      This will need another macro that sets bits 5 and 6. Let's call it
      \texttt{OP\_AJIT\_BIT\_2}.   Hence the  SPARC bit  layout of  this
      instruction is:

      \begin{tabular}[h]{lclcl}
        Macro to set  &=& \texttt{F5(x, y, z)} &in& \texttt{sparc.h}     \\
        Macro to reset  &=& \texttt{INVF5(x, y, z)} &in& \texttt{sparc.h}     \\
        x &=& 0x2      &in& \texttt{OP(x)  /* ((x) \& 0x3)  $<<$ 30 */} \\
        y &=& 0x27     &in& \texttt{OP3(y) /* ((y) \& 0x3f) $<<$ 19 */} \\
        z &=& 0x1      &in& \texttt{F3I(z) /* ((z) \& 0x1)  $<<$ 13 */} \\
        a &=& 0x2      &in& \texttt{OP\_AJIT\_BIT\_2(a) /* ((a) \& 0x3  $<<$ 6 */}
      \end{tabular}

      The AJIT bits (insn[6:5]) is  set or reset internally by \texttt{F5}
      (just  like  in  \texttt{F4}),  and   hence  there  are  only  three
      arguments.

    \end{enumerate}
  \item The register  indirect versions are given by insn[13]  = 0.  The
    shift count is indirectly specified in the 32 bit register specified
    in instruction bits.  Therefore  insn[4:0] specify the register that
    has the  shift count.  insn[6]  = 1, distinguishes the  AJIT version
    from the SPARC V8 version.  In this case, insn[5] = 0, necessarily.
    \begin{enumerate}
    \item \textbf{SLLD}:\\
      This will need another macro that sets bits 5 and 6. Let's call it
      \texttt{OP\_AJIT\_BIT\_2}.   Hence the  SPARC bit  layout of  this
      instruction is:

      \begin{tabular}[h]{lclcl}
        Macro to set  &=& \texttt{F5(x, y, z)} &in& \texttt{sparc.h}     \\
        Macro to reset  &=& \texttt{INVF5(x, y, z)} &in& \texttt{sparc.h}     \\
        x &=& 0x2      &in& \texttt{OP(x)  /* ((x) \& 0x3)  $<<$ 30 */} \\
        y &=& 0x25     &in& \texttt{OP3(y) /* ((y) \& 0x3f) $<<$ 19 */} \\
        z &=& 0x0      &in& \texttt{F3I(z) /* ((z) \& 0x1)  $<<$ 13 */} \\
        a &=& 0x2      &in& \texttt{OP\_AJIT\_BIT\_2(a) /* ((a) \& 0x3  $<<$ 6 */}
      \end{tabular}

      The AJIT bits (insn[6:5]) is  set or reset internally by \texttt{F5}
      (just  like  in  \texttt{F4}),  and   hence  there  are  only  three
      arguments.

    \item \textbf{SRLD}:\\
      This will need another macro that sets bits 5 and 6. Let's call it
      \texttt{OP\_AJIT\_BIT\_2}.   Hence the  SPARC bit  layout of  this
      instruction is:

      \begin{tabular}[h]{lclcl}
        Macro to set  &=& \texttt{F5(x, y, z)} &in& \texttt{sparc.h}     \\
        Macro to reset  &=& \texttt{INVF5(x, y, z)} &in& \texttt{sparc.h}     \\
        x &=& 0x2      &in& \texttt{OP(x)  /* ((x) \& 0x3)  $<<$ 30 */} \\
        y &=& 0x26     &in& \texttt{OP3(y) /* ((y) \& 0x3f) $<<$ 19 */} \\
        z &=& 0x0      &in& \texttt{F3I(z) /* ((z) \& 0x1)  $<<$ 13 */} \\
        a &=& 0x2      &in& \texttt{OP\_AJIT\_BIT\_2(a) /* ((a) \& 0x3  $<<$ 6 */}
      \end{tabular}

      The AJIT bits (insn[6:5]) is  set or reset internally by \texttt{F5}
      (just  like  in  \texttt{F4}),  and   hence  there  are  only  three
      arguments.

    \item \textbf{SRAD}:\\
      This will need another macro that sets bits 5 and 6. Let's call it
      \texttt{OP\_AJIT\_BIT\_2}.   Hence the  SPARC bit  layout of  this
      instruction is:

      \begin{tabular}[h]{lclcl}
        Macro to set  &=& \texttt{F5(x, y, z)} &in& \texttt{sparc.h}     \\
        Macro to reset  &=& \texttt{INVF5(x, y, z)} &in& \texttt{sparc.h}     \\
        x &=& 0x2      &in& \texttt{OP(x)  /* ((x) \& 0x3)  $<<$ 30 */} \\
        y &=& 0x27     &in& \texttt{OP3(y) /* ((y) \& 0x3f) $<<$ 19 */} \\
        z &=& 0x0      &in& \texttt{F3I(z) /* ((z) \& 0x1)  $<<$ 13 */} \\
        a &=& 0x2      &in& \texttt{OP\_AJIT\_BIT\_2(a) /* ((a) \& 0x3  $<<$ 6 */}
      \end{tabular}

      The AJIT bits (insn[6:5]) is  set or reset internally by \texttt{F5}
      (just  like  in  \texttt{F4}),  and   hence  there  are  only  three
      arguments.
    \end{enumerate}
  \end{enumerate}
\end{itemize}

%%% Local Variables:
%%% mode: latex
%%% TeX-master: t
%%% End:


\subsection{Integer-Unit Extensions: SIMD Instructions}
\label{sec:integer-unit-extns:simd-instructions:impl}

\subsubsection{SIMD I instructions:}
\label{sec:simd:1:insn:impl}

The  first   set  of  SIMD  instructions  are   the  three  arithmetic
instructions: add,  sub, and mul.  The ``mul''  instruction has signed
and unsigned variations.  Each of the three instructions have 8 bit (1
byte),  16 bit  (1 half  word) and  32 bit  (1 word)  versions.  These
versions  are encoded  as  shown in  table~\ref{tab:types:for:simd:1},
where the first column denotes the  bit numbers.  We list all the SIMD
I instructions version wise below.
\begin{table}[h]
  \centering
  \begin{tabular}[p]{|l|l|l|}
  \hline
  \textbf{987} & \textbf{Type} & \textbf{Example}\\
  \hline
  001 & Byte & e.g. VADDD8\\
  010 & Half-word (16-bits) & e.g. VADDD16\\
  100 & Word (32-bits) & e.g. VADDD32\\
  \hline
\end{tabular}
\caption{Data type encoding for SIMD I instructions.}
\label{tab:types:for:simd:1}
\end{table}
\begin{enumerate}
\item \textbf{8 bit} (\textbf{1 Byte})
  \begin{enumerate}
  \item \textbf{VADDD8}:\\
    \begin{center}
      \begin{tabular}[p]{|c|c|l|l|}
        \hline
        \textbf{Start} & \textbf{End} & \textbf{Range} & \textbf{Meaning} \\
        \hline
        0 & 4 & 32 & Source register 2, rs2 \\
        5 & 6 & 4 & \emph{Always} 2, i.e. insn[6:5] = 10$_b$ \\
        7 & 9 & 8 & \textbf{Data type} specifier:  \emph{Always} 0x1\\
        10 & 12 & -- & \textbf{unused} \\
        13 & 13 & 0,1 & The \textbf{i} bit. \emph{Always} 0. \\
        14 & 18 & 32 & Source register 1, rs1 \\
        19 & 24 & 000000 & ``\textbf{op3}'' \\
        25 & 29 & 32 & Destination register, rd \\
        30 & 31 & 4 & Always ``10'' \\
        \hline
      \end{tabular}
    \end{center}
    \begin{itemize}
    \item []\textbf{VADDD8}: same as  ADD, but with Instr[13]=0 (i=0),
      and  Instr[6:5]=2.  Bits Instr[9:7]  are  a  3-bit field,  which
      specify the data type
    \item []\textbf{Syntax}: ``\texttt{vaddd8 SrcReg1, SrcReg2,
        DestReg}''.
    \item []\textbf{Semantics}: \emph{not given}
    \end{itemize}
    Bits layout:
\begin{verbatim}
    Offsets      : 31       24 23       16  15        8   7        0
    Bit layout   :  XXXX  XXXX  XXXX  XXXX   XXXX  XXXX   XXXX  XXXX
    Insn Bits    :  10       0  0000  0        0     00   110       
    Destination  :    DD  DDD                                       
    Source 1     :                     SSS   SS
    Source 2     :                                           S  SSSS
    Unused (0)   :                              U  UU               
    Final layout :  10DD  DDD0  0000  0SSS   SS0U  UU00   110S  SSSS
    To match     :  ^^       ^  ^^^^  ^        ^     ^^   ^^^
    Bitfield name:  OP          OP3            i     9-   765
\end{verbatim}

    To  set  up  bits  5  and  6, we  use  an  already  defined  macro
    \texttt{OP\_AJIT\_BIT\_5\_AND\_6}.  The  value to be set  in these
    two bits is 0x2.   To set bits 7 through 9, we  define a new macro
    \texttt{OP\_AJIT\_BIT\_7\_THRU\_9}.  The value  set in these three
    bits  decides the  \emph{type}, byte,  half word  or word,  of the
    instruction.  For  \textbf{vaddd8} instruction,  bits 7  through 9
    are  set  to the  value  0x1.   Both  these macros  influence  the
    \emph{unused} bits of  the SPARC V8 architecture.  So  we define a
    macro \texttt{OP\_AJIT\_SET\_UNUSED} that uses the previous two to
    set these bits unused by the SPARC V8, but used by AJIT.

    \verb|#define OP_AJIT_BIT_7_THRU_9(x)   ((x) << 0x7)|

    \verb+#define OP_AJIT_SET_UNUSED        (OP_AJIT_BIT_5_AND_6(0x2) | \\+

    \verb+                                   OP_AJIT_BIT_7_THRU_9(0x1))+

    We can  now define the final  macro \texttt{F6(x, y, z,  b, a)} to
    set the match bits for this instruction.
\begin{verbatim}
#define OP_AJIT_BIT_5(x)          (((x) & 0x1) << 5)
#define F4(x, y, z, b)            (F3(x, y, z) | OP_AJIT_BIT_5(b))
#define OP_AJIT_BIT_5_AND_6(x)    (((x) & 0x3) << 6)
#define F5(x, y, z, b)            (F3(x, y, z) | OP_AJIT_BIT_5_AND_6 (b))
#define OP_AJIT_BIT_7_THRU_9(x)   (((x) & 0x3) << 9)
#define F6(x, y, z, b, a)         (F5 (x, y, z, b) | OP_AJIT_BIT_7_THRU_9(a))
\end{verbatim}
    Hence the SPARC bit layout of this instruction is:

  \begin{tabular}[h]{lclcl}
    Macro to set  &=& \texttt{F4(x, y, z)} &in& \texttt{sparc.h}     \\
    Macro to reset  &=& \texttt{INVF4(x, y, z)} &in& \texttt{sparc.h}     \\
    x &=& 0x2      &in& \texttt{OP(x)  /* ((x) \& 0x3)  $<<$ 30 */} \\
    y &=& 0x00     &in& \texttt{OP3(y) /* ((y) \& 0x3f) $<<$ 19 */} \\
    z &=& 0x0      &in& \texttt{F3I(z) /* ((z) \& 0x1)  $<<$ 13 */} \\
    a &=& 0x1      &in& \texttt{OP\_AJIT\_BIT(a) /* ((a) \& 0x1)  $<<$ 5 */}
  \end{tabular}

  The AJIT bit (insn[5]) is set internally by \texttt{F4}, and hence
  there are only three arguments.
\end{enumerate}
\item \textbf{1 Half word} (\textbf{16 bit})
\item \textbf{1 Word} (\textbf{32 bit})
\end{enumerate}



\subsection{Integer-Unit Extensions: SIMD Instructions II}
\label{sec:integer-unit-extns:simd-instructions:2:impl}

\subsection{Vector Floating Point Instructions}
\label{sec:vector-floating-point-instructions:impl}

\subsection{CSWAP instructions}
\label{sec:cswap-instructions:impl}

