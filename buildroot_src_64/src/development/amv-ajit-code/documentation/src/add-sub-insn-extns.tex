\subsubsection{Addition and subtraction instructions:}
\label{sec:add:sub:insn:impl}
\begin{enumerate}
\item \textbf{ADDD}:\\
  \begin{center}
    \begin{figure}[h]
      \centering
      \epsfxsize=.8\linewidth
      \epsffile{../figs/addd-ajit-insn-32-bit-layout.eps}
      \caption{The AJIT ADDD instruction  with register operands.  The
        {\darkgreen {\textbf{green}}}  regions need  to be set  in the
        GNU \texttt{as} code.  % (The upper ``template of 32 bits'' will
        % be skipped in the rest of the figures.)
      }
      \label{fig:ajit:addd:insn}
    \end{figure}
    % \begin{tabular}[p]{|c|c|l|l|l|}
    %   \hline
    %   \textbf{Start} & \textbf{End} & \textbf{Range} & \textbf{Meaning} &
    %                                                                       \textbf{New Meaning}\\
    %   \hline
    %   0 & 4 & 32 & Source register 2, rs2 & No change \\
    %   5 & 12 & -- & \textbf{unused} & \textbf{Set bit 5 to ``1''} \\
    %   13 & 13 & 0,1 & The \textbf{i} bit & \textbf{Set i to ``0''} \\
    %   14 & 18 & 32 & Source register 1, rs1 & No change \\
    %   19 & 24 & 000000 & ``\textbf{op3}'' & No change \\
    %   25 & 29 & 32 & Destination register, rd & No change \\
    %   30 & 31 & 4 & Always ``10'' & No change \\
    %   \hline
    % \end{tabular}
  \end{center}

  \begin{itemize}
  \item []\textbf{ADDD}: same as ADD, but with Instr[13]=0 (i=0), and
    Instr[5]=1.
  \item []\textbf{Syntax}: ``\texttt{addd  SrcReg1, SrcReg2, DestReg}''.
  \item []\textbf{Semantics}: rd(pair) $\leftarrow$ rs1(pair) + rs2(pair).
  \end{itemize}
%   Bits layout:
% \begin{verbatim}
%     Offsets      : 31       24 23       16  15        8   7        0
%     Bit layout   :  XXXX  XXXX  XXXX  XXXX   XXXX  XXXX   XXXX  XXXX
%     Insn Bits    :  10       0  0000  0        0            1       
%     Destination  :    DD  DDD                                       
%     Source 1     :                     SSS   SS
%     Source 2     :                                           S  SSSS
%     Unused (0)   :                              U  UUUU   UU        
%     Final layout :  10DD  DDD0  0000  0SSS   SS0U  UUUU   UU1S  SSSS
% \end{verbatim}

  The SPARC bit layout of this instruction is:

  \begin{tabular}[h]{lclcl}
    Macro to set   &=&  \verb|F4(x, y, z, b)|     &in& \verb|sparc.h|     \\
    Macro to reset &=&  \verb|F4(~x, ~y, ~z, ~b)| &in& \verb|sparc.h|     \\
    x &=& 0x2      &in& \verb|OP(x)          /* ((x) \& 0x3)  $<<$ 30 */| \\
    y &=& 0x00     &in& \verb|OP3(y)         /* ((y) \& 0x3f) $<<$ 19 */| \\
    z &=& 0x0      &in& \verb|F3I(z)         /* ((z) \& 0x1)  $<<$ 13 */| \\
    b &=& 0x1      &in& \verb|OP_AJIT_BIT(b) /* ((b) \& 0x1)  $<<$ 5  */|
  \end{tabular}

\item \textbf{ADDDCC}:\\
  \begin{center}
    \begin{figure}[h]
      \centering
      \epsfxsize=.8\linewidth
      \epsffile{../figs/adddcc-ajit-insn-32-bit-layout.eps}
      \caption{The   AJIT   ADDDCC   instruction.    The   {\darkgreen
          {\textbf{green}}}  regions  need  to   be  set  in  the  GNU
        \texttt{as} code.}
      \label{fig:ajit:adddcc:insn}
    \end{figure}
    % \begin{tabular}[p]{|c|c|l|l|l|}
    %   \hline
    %   \textbf{Start} & \textbf{End} & \textbf{Range} & \textbf{Meaning} &
    %                                                                       \textbf{New Meaning}\\
    %   \hline
    %   0 & 4 & 32 & Source register 2, rs2 & No change \\
    %   5 & 12 & -- & \textbf{unused} & \textbf{Set bit 5 to ``1''} \\
    %   13 & 13 & 0,1 & The \textbf{i} bit & \textbf{Set i to ``0''} \\
    %   14 & 18 & 32 & Source register 1, rs1 & No change \\
    %   19 & 24 & 010000 & ``\textbf{op3}'' & No change \\
    %   25 & 29 & 32 & Destination register, rd & No change \\
    %   30 & 31 & 4 & Always ``10'' & No change \\
    %   \hline
    % \end{tabular}
  \end{center}

  \begin{itemize}
  \item []\textbf{ADDDCC}: same as ADDCC, but with Instr[13]=0 (i=0), and
    Instr[5]=1.
  \item []\textbf{Syntax}: ``\texttt{adddcc  SrcReg1, SrcReg2, DestReg}''.
  \item []\textbf{Semantics}: rd(pair) $\leftarrow$ rs1(pair) + rs2(pair), set Z,N
  \end{itemize}
%   Bits layout:
% \begin{verbatim}
%     Offsets      : 31       24 23       16  15        8   7        0
%     Bit layout   :  XXXX  XXXX  XXXX  XXXX   XXXX  XXXX   XXXX  XXXX
%     Insn Bits    :  10       0  1000  0        0            1       
%     Destination  :    DD  DDD                                       
%     Source 1     :                     SSS   SS
%     Source 2     :                                           S  SSSS
%     Unused (0)   :                              U  UUUU   UU        
%     Final layout :  10DD  DDD0  1000  0SSS   SS0U  UUUU   UU1S  SSSS
% \end{verbatim}

  The SPARC bit layout of this instruction is:

  \begin{tabular}[h]{lclcl}
    Macro to set   &=&  \verb|F4(x, y, z, b)|     &in& \verb|sparc.h|     \\
    Macro to reset &=&  \verb|F4(~x, ~y, ~z, ~b)| &in& \verb|sparc.h|     \\
    x &=& 0x2      &in& \verb|OP(x)          /* ((x) \& 0x3)  $<<$ 30 */| \\
    y &=& 0x10     &in& \verb|OP3(y)         /* ((y) \& 0x3f) $<<$ 19 */| \\
    z &=& 0x0      &in& \verb|F3I(z)         /* ((z) \& 0x1)  $<<$ 13 */| \\
    b &=& 0x1      &in& \verb|OP_AJIT_BIT(b) /* ((b) \& 0x1)  $<<$ 5  */|
  \end{tabular}

\item \textbf{SUBD}:\\
  \begin{center}
    \begin{figure}[h]
      \centering
      \epsfxsize=.8\linewidth
      \epsffile{../figs/subd-ajit-insn-32-bit-layout.eps}
      \caption{The   AJIT   SUBD   instruction.    The   {\darkgreen
          {\textbf{green}}}  regions  need  to   be  set  in  the  GNU
        \texttt{as} code.}
      \label{fig:ajit:subd:insn}
    \end{figure}
    % \begin{tabular}[p]{|c|c|l|l|l|}
    %   \hline
    %   \textbf{Start} & \textbf{End} & \textbf{Range} & \textbf{Meaning} &
    %                                                                       \textbf{New Meaning}\\
    %   \hline
    %   0 & 4 & 32 & Source register 2, rs2 & No change \\
    %   5 & 12 & -- & \textbf{unused} & \textbf{Set bit 5 to ``1''} \\
    %   13 & 13 & 0,1 & The \textbf{i} bit & \textbf{Set i to ``0''} \\
    %   14 & 18 & 32 & Source register 1, rs1 & No change \\
    %   19 & 24 & 000100 & ``\textbf{op3}'' & No change \\
    %   25 & 29 & 32 & Destination register, rd & No change \\
    %   30 & 31 & 4 & Always ``10'' & No change \\
    %   \hline
    % \end{tabular}
  \end{center}
  \begin{itemize}
  \item []\textbf{SUBD}: same as SUB, but with Instr[13]=0 (i=0), and
    Instr[5]=1.
  \item []\textbf{Syntax}: ``\texttt{subd  SrcReg1, SrcReg2, DestReg}''.
  \item []\textbf{Semantics}: rd(pair) $\leftarrow$ rs1(pair) - rs2(pair).
  \end{itemize}
%   Bits layout:
% \begin{verbatim}
%     Offsets      : 31       24 23       16  15        8   7        0
%     Bit layout   :  XXXX  XXXX  XXXX  XXXX   XXXX  XXXX   XXXX  XXXX
%     Insn Bits    :  10       0  0010  0        0            1       
%     Destination  :    DD  DDD                                       
%     Source 1     :                     SSS   SS
%     Source 2     :                                           S  SSSS
%     Unused (0)   :                              U  UUUU   UU        
%     Final layout :  10DD  DDD0  0010  0SSS   SS0U  UUUU   UU1S  SSSS
% \end{verbatim}

  The SPARC bit layout of this instruction is:

  \begin{tabular}[h]{lclcl}
    Macro to set   &=&  \verb|F4(x, y, z, b)| &in& \verb|sparc.h|     \\
    Macro to reset &=&  \verb|F4(~x, ~y, ~z, ~b)| &in& \verb|sparc.h|     \\
    x &=& 0x2      &in& \verb|OP(x)          /* ((x) \& 0x3)  $<<$ 30 */| \\
    y &=& 0x04     &in& \verb|OP3(y)         /* ((y) \& 0x3f) $<<$ 19 */| \\
    z &=& 0x0      &in& \verb|F3I(z)         /* ((z) \& 0x1)  $<<$ 13 */| \\
    b &=& 0x1      &in& \verb|OP_AJIT_BIT(b) /* ((b) \& 0x1)  $<<$ 5  */|
  \end{tabular}

\item \textbf{SUBDCC}:\\
  \begin{center}
    \begin{figure}[h]
      \centering
      \epsfxsize=.8\linewidth
      \epsffile{../figs/subdcc-ajit-insn-32-bit-layout.eps}
      \caption{The   AJIT   SUBDCC   instruction.}
      \label{fig:ajit:subdcc:insn}
    \end{figure}
    % \begin{tabular}[p]{|c|c|l|l|l|}
    %   \hline
    %   \textbf{Start} & \textbf{End} & \textbf{Range} & \textbf{Meaning} &
    %                                                                       \textbf{New Meaning}\\
    %   \hline
    %   0 & 4 & 32 & Source register 2, rs2 & No change \\
    %   5 & 12 & -- & \textbf{unused} & \textbf{Set bit 5 to ``1''} \\
    %   13 & 13 & 0,1 & The \textbf{i} bit & \textbf{Set i to ``0''} \\
    %   14 & 18 & 32 & Source register 1, rs1 & No change \\
    %   19 & 24 & 010100 & ``\textbf{op3}'' & No change \\
    %   25 & 29 & 32 & Destination register, rd & No change \\
    %   30 & 31 & 4 & Always ``10'' & No change \\
    %   \hline
    % \end{tabular}
  \end{center}

  \begin{itemize}
  \item []\textbf{SUBDCC}: same as SUBCC, but with Instr[13]=0 (i=0), and
    Instr[5]=1.
  \item []\textbf{Syntax}: ``\texttt{subdcc  SrcReg1, SrcReg2, DestReg}''.
  \item []\textbf{Semantics}: rd(pair) $\leftarrow$ rs1(pair) - rs2(pair), set Z,N
  \end{itemize}
%   Bits layout:
% \begin{verbatim}
%     Offsets      : 31       24 23       16  15        8   7        0
%     Bit layout   :  XXXX  XXXX  XXXX  XXXX   XXXX  XXXX   XXXX  XXXX
%     Insn Bits    :  10       0  1010  0        0            1       
%     Destination  :    DD  DDD                                       
%     Source 1     :                     SSS   SS
%     Source 2     :                                           S  SSSS
%     Unused (0)   :                              U  UUUU   UU        
%     Final layout :  10DD  DDD0  1010  0SSS   SS0U  UUUU   UU1S  SSSS
% \end{verbatim}

  The SPARC bit layout of this instruction is:

  \begin{tabular}[h]{lclcl}
    Macro to set   &=&  \verb|F4(x, y, z, b)|     &in& \verb|sparc.h|     \\
    Macro to reset &=&  \verb|F4(~x, ~y, ~z, ~b)| &in& \verb|sparc.h|     \\
    x &=& 0x2      &in& \verb|OP(x)          /* ((x) \& 0x3)  $<<$ 30 */| \\
    y &=& 0x14     &in& \verb|OP3(y)         /* ((y) \& 0x3f) $<<$ 19 */| \\
    z &=& 0x0      &in& \verb|F3I(z)         /* ((z) \& 0x1)  $<<$ 13 */| \\
    b &=& 0x1      &in& \verb|OP_AJIT_BIT(b) /* ((b) \& 0x1)  $<<$ 5  */|
  \end{tabular}
\end{enumerate}
